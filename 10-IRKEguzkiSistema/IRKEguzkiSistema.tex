\chapter{IRK: Eguzki-sistema.}


\section{Sarrera.}
  

Kapitulu honetan, eguzki-sistemaren ekuazio diferentzialei Kepler-en fluxuan oinarritutako aldagai aldaketa aplikatzea proposatuko dugu. \ref{chap:IRK-PF}. kapituluan puntu-finkoaren iterazioan oinarrituz eta \ref{chap:IRK-NEW}. kapituluan Newton sinplifikatuaren iterazioan oinarrituz IRK inplementazioak garatu ditugu; eguzki-sistemaren problemaren integraziorako, bi inplementazioen artean, puntu-finkoarena eraginkorragoa dela baieztatu dugu. Hortaz, puntu-finkoaren iterazioan oinarritutako IRK inplementazioa erabiliko dugu eta  ekuazio diferentzialetako aldagaiei eragingo diegun aldagai aldaketaren bidez, integrazio eraginkorra lortzea espero dugu.  

Aplikatzen dugun integrazio metodoa sinplektikoa eta simetrikoa da: neurri batean, Splitting metodoen baliokidea. Aldagai berriekiko ekuazio diferentzialak, magnitude txikiko balioak hartzen dituzte eta honek, hiru abantaila eragingo ditu. Lehenik, eguzki-sistemaren problemaren trunkatze errore nagusiena ezabatzen dugunez, urrats luzera handiagoak erabili ahal izango ditugu. Bigarrenik, batura konpensatuaren konputazioan, informazio gutxiago galduko dugu. Jacobiarraren balioa txikia denez, puntu-finkoaren iterazioek konbergentzia azkarra izango dute. 

Lehenengo, Kepler-en fluxuaren inplementazioa azalduko dugu. Bigarrenik, aldagai aldaketa definitu eta metodoa integratzeko zehaztapenak emango ditugu. Hirugarrenik, eguzki-sistemaren problemaren zenbakizko integrazioak egingo ditugu: inplementazio honen eta doitasun altuko beste metodo sinplektikoen eraginkortasunak, alderatuko ditugu.     

 

\section{Kepler-en fluxua.}
   
   
Kepler problema bi gorputzen problemaren kasu partikularra da eta  honako Hamiltondarra dagokio,
\begin{equation}
H(q,p)=\frac{p^2}{2m}-\frac{\mu}{\|q\|},
\end{equation}
non $m$ eta $\mu$ konstanteen balioak, formulazioaren araberakoak diren.

Koordenatu sistema $q=q_2-q_1$ duen formulazioa aukeratzen badugu, konstanteen balioak hauek dira,  
\begin{equation*}
m=(1/m_1+1/m_2)^{-1},\ \ \mu=Gm_1m_2,
\end{equation*} 
%
eta ekuazio diferentzialak era honetan definitzen dira,
\begin{equation}
\label{eq:kode}
\dot{q}=p, \ \ \dot{p}= - \frac{k \ q}{\|q\|^3} ,
\end{equation}
non $k= \mu / m$ eta  $q,p \in \mathbb{R}^3$.

Kepler problemaren soluzio zehatza kalkula daiteke: une bateko kokapen eta abiadurak emanik, $\Delta t$ denbora tarte bat igarotakoan (positiboa ala negatiboa), kokapen eta abiadura zehatzak konputatu daitezke. Eguzki-sistemaren integrazio metodoentzat, Kepler problema doitasun handian eta era eraginkorrean kalkulatzea, funtsezkoa da. Kepler problemaren erreferentziazko inplementazioak, Danby \cite{Danby1992} eta J.Wisdom-enak  \cite{Wisdom2015} ditugu. 

Kepler-en fluxua, era honetan kalkulatzen da. Lehenik, koordenatu cartesiarretatik ($q,p\in \mathbb{R}^3$), koordenatu eliptikoetara $(a,e,i,\Omega,E)$ itzulpena egingo dugu. Koordenatu eliptikoetan, $E$ (\emph{eccentric anomaly}) aldagaia izan ezik, beste aldagaiak konstante mantentzen dira: beraz $E_0$ balioa emanda, $\Delta t$ denbora tartea aurrera egin eta $E_1$ balio berria kalkulatuko dugu. Azkenik, koordenatu eliptikoetatik koordenatu cartesiarretara itzulpena eginez, kokapen eta abiadura berriak eskuratuko ditugu. 

\begin{equation*}
(q_0,v_0) \in \mathbb{R}^6 \ \ \ \longrightarrow \ \ \  (a,e,i,\Omega,E_0) \in \mathbb{R}^6 
\end{equation*}
\begin{equation*}
\quad \quad \quad \quad \quad \quad \quad \quad \downarrow \Delta t
\end{equation*}
\begin{equation*}
(q_1,v_1) \in \mathbb{R}^6 \ \ \ \longleftarrow \ \ \  (a,e,i,\Omega,E_1) \in \mathbb{R}^6 
\end{equation*}

Gorputz baten orbita Kepleriarra hiru motakoa izan daiteke: $H(q_0,p_0)<0$ denean orbita eliptikoa da, $H(q_0,p_0)>0$ orbita hiperbolikoa eta $H(q_0,p_0)=0$ orbita  parabolikoa. Kepler fluxuaren C inplementazioa, orbita eliptikoetarako garatu dugu eta zehaztasunak, \ref{erans:B1} eranskinean eman ditugu. (\ref{eq:kode}) problemari dagokion fluxua, era honetan defini daiteke,
\begin{align*}
\varphi_{\Delta t}^k:&  \quad \mathbb{R}^{6} \quad  \longrightarrow \quad \mathbb{R}^6,  \\
&  \quad u_0 \ \  \rightsquigarrow \ \ u_1. 
\end{align*} 
non $u=(q,v) \in \mathbb{R}^6$  den.

\section{Inplementazioa.}

\subsection*{Aldagai aldaketa.}

Kontutan hartuko ditugun sistemak Hamiltondarrak dira, $H: \mathbb{R} \times \mathbb{R}^{2d} \longrightarrow \mathbb{R}$, gainera, Hamiltondarra bi zatitan bana dakieken sistemak hartuko ditugu kontuan:
\begin{align}
\begin{split}
&H(q,p,t)=H_K(q,p)+H_I(q,p,t)
\end{split}
\end{align} 
non $H_K$ mugimendu Kepleriarrari dagokion Hamiltondarraren aldea den eta $H_I$ perturbazioei dagokien Hamiltondarraren aldea den.

Problema horri dagozkion ekuazioetan Keplerren fluxuan oinarritutako aldagai aldaketa bat egingo dugu, baina horretarako notazioa finkatuko dugu: jatorrizko aldagaiak $u=(q,p) \in \mathbb{R}^{2d}$ izango dira eta aldagai berriak $U=(Q,P) \in \mathbb{R}^{2d}$ letra larriz adieraziko ditugu. Jatorrizko aldagaien bidez adierazitako problema, alegia, ebatzi beharreko hasierako baliodun problema, honakoa da:

\begin{align}
\begin{split}
\label{eq: HamEDA}
&\frac{du}{dt} = k(u) + g(u,t),\ \ \ u(0) = u_0
\end{split}
\end{align} 
non $k(u)$ alde kepleriarrari dagokion eta $g(u,t)$ perturbazioari. 
Problema horretan honako aldagai aldaketa egingo dugu, kontuan izan urrats bakoitzean egingo dugula aldagai aldaketa, hau da $j=0, 1, 2 \ldots$ indizeak $j$. urratsean aplikatu beharreko aldaketa adierazten du:

\begin{align}
\begin{split}
\label{eq: uUaldaketa}
&u(t) = \varphi_{t-(j+\frac{1}{2})h}^k\left(U_j^{j+\frac{1}{2}}(t)\right)
\end{split}
\end{align} 
$\varphi_{\Delta t}^k$ fluxua $\Delta t>0$ eta $\Delta t <0$ balioentzat definitzen da, eta  $u= \varphi_{-t}(\varphi_{t}(u))$ betetzen dela kontutan hartuz honako alderantzizko aldaketa ere egin dezakegu:
\begin{align}
\begin{split}
\label{eq: Uualdaketa}
U^{j+\frac{1}{2}}(t) = \varphi^k_{-t+(j+\frac{1}{2})h} \left( u(t) \right)
\end{split}
\end{align} 

Aldaketa hauekin asmoa da $i+1$ urratsa emateko $u_i \approx u(hi)$ zenbakizko soluzioan oinarrituz, aldagai aldaketaren bidez $U_i^{i+\frac{1}{2}}=\varphi^k_{\frac{h}{2}}(u_i)$ lortu, hau da, fluxuan $\frac{h}{2}$ aurrera egin aldagai berriak lortzeko, aldagai berri hauetan ebatzi jatorrizko problemaren urrats bati dagokion zenbakizko soluzioa (ikusiko dugun bezala, aldagai berrietan alde kepleriarrari dagokion espresioak ez du eraginik eta, azken finean perturbazioari dgokion aldaketa da hemen kalkulatuko dena) eta azkenik, aldagai berri hauen balio berriak jatorrizko aldagaietara itzuli behar dira, baina $i+1$ urratsari dagozkion unera pasa behar dira aldagaiak, hau da, fluxuan aurrera $\frac{h}{2}$ egin behar da. Atzera egingo bagenu urratsaren hasierako balioei perturbazioak zein aldaketa eragiten dien kalkulatuko baikenuke, baina guk urratsaren bukaerako balioak nahi ditugu. Laburbilduz:


\begin{align*}
   &   \quad U_0^{\frac{1}{2}} \quad \quad \quad \quad \Longrightarrow  & U_1^{\frac{1}{2}}&  &\\
  & \nearrow \varphi_{\frac{h}{2}}(u_0) &            & \searrow \varphi_{\frac{h}{2}}(U_1)& \\
u_0 &                  &    &\quad \quad \quad  u_1
\end{align*}

Aldagai aldaketak fluxuan aurrera egiten du urratsaren luzeraren erdia. Hori horrela egiteak badu arrazoi bat: urratsa bere osotasunean simetrikoa da. Aurrera $h$ luzerako urratsa ematea $-h$ luzerako urratsa ematearekin desegiten baita. 

%\begin{figure}[h!]
%\centering
%\subfloat[Aldagai aldaketa.]{
%\includegraphics[width=.400\textwidth]{Aldagaialdaketa1}
%}
%\subfloat[Aldagai berrien integrazioa.]{
%\includegraphics[width=.400\textwidth]{Aldagaialdaketa2}
%}
%\caption[Atalen hasieraketa.]
%        {\small (a)irudian, aldagai aldaketa irudikatu dugu eta (b) irudian, perturbatutako gorputza baten orbitaren integrazioak erakutsi ditugu. Bi irudietan, $(Q,P)$ balioen aldaketa txikiak gorriz nabarmendu ditugu          
%        }
%\label{fig:Aldg}
%\end{figure}   

\subsection*{Aldagai berrietan ekuazio diferentzialak.}

(\ref{eq: uUaldaketa}) aldagai aldaketa kontuan hartuz, bi aldeak denborarekiko deribatuz, 
\begin{align}
\begin{split}
&\frac{d}{dt}u = \frac{d}{dt}\left(\varphi(U)\right),
\end{split}
\end{align}
($\varphi(U)$ irakur erraztasunagatik indizerik gabe idatzi dugu)

 Eskuin aldeari katearen erregla aplikatuz, lortuko dugu
\begin{align}
\begin{split}
&\dot{u} = \dot{\varphi}(U) + \varphi'(U) 
\frac{d}{dt}U.
\end{split}
\end{align}
$\varphi$ fluxua $\dot{u} = k(u)$ problemaren fluxua da eta fluxuaren definizioz $\dot{\varphi}(U) = k(\varphi(U))$ da, eta (\ref{eq: uUaldaketa}) ekuazioa kontuan hartuz, lortuko dugu
\begin{align}
\begin{split}
&k(u) + g(u,t) = k(u) + \varphi'(U) \dot{U}.
\end{split}
\end{align}
Bi aldeetan $k(u)$  kenduz, $U$ aldagaiekiko ebatzi beharreko ekuazio diferentziala lortuko dugu:
\begin{align}
\begin{split}
\label{eq:hamEDAU}
&\dot{U} = \left(\varphi'(U)\right)^{-1} g(u,t).
\end{split}
\end{align}
Alderantzizko matrizeak kalkulatu beharrik gabe idatz ditzakegu (\ref{eq:hamEDAU}) ekuazioak. Horretarako $\varphi$ fluxuaren izaera sinplektikoa erabiliko dugu, hau da, $(\varphi')^tJ\varphi'= J$ propietatea betetzen du fluxuak, ondorioz,
%
\begin{align}
\begin{split}
\label{eq:hamEDAU2}
&\dot{U} = J^{-1}(\varphi')^{t}(U)J g(u,t).
\end{split}
\end{align}
(\ref{eq:hamEDAU2}) ekuazioak $\varphi'(U)$ kalkulatzea eskatzen du, eta horretarako deribazio automatikoko teknikak erabil ditzakegu


\paragraph*{Algoritmoa.}
$U$ aldagaietan oinarritutako ekuazio diferentzialenn integrazioa, hiru urratsetan egingo dugu:
\begin{enumerate}
\item $\{u,aux\} \leftarrow KeplerFlowGen (t,U,mu)$.

Kepler-en fluxua $(q,v)= \varphi_t(U)$ aplikatuko dugu eta fluxuaren kalkulutarako erabilitako tarteko balioak, ~$aux\in \mathbb{R}^{16}$ aldagaian itzuliko ditugu. 

\item $g \leftarrow g(u,t)$.

Jatorrizko problemako ekuazio diferentzialetan perturbazioei dagokien espresiioa kalkulatuko dugu.

\item $KeplerFlowGFcnaux(aux,U,t,g)$.

Lehenik, $\varphi'_t())$ modu eraginkorrean kalkulatu behar da. Deribazio automatikoaren teknikaren bidez, Kepler fluxuaren $U$ aldagaiekiko deribatuaren konputazio eraginkorra definitu dugu. 

Ekuazio diferentzialen (\ref{eq:hamEDAU}) espresioaren konputazioa egingo dugu,   
\begin{align*}
&KeplerFlowGFcnaux(aux,U,t,g)= ( \varphi'_t(U))^{-1} \ g.
\end{align*}

\end{enumerate} 



\subsubsection*{Alde Kepleriar bat baino gehiago.}

Alde Kepleriar bat baino gehiago dugun problemen azterketa egingo dugu. Problemaren alde Kepleriarren kopurua $k$ bada, era honetako ekuazio diferentzialak ditugu,
\begin{equation*}
\frac{d}{dt}\left(\begin{array}{c}
                q  \\
                v  \\
\end{array}\right)=
\left(\begin{array}{c}
                q_1  \\
                v_1  \\
                q_2  \\
                v_2  \\
                \vdots \\
                q_k    \\
                v_k    \\
                w      \\
\end{array}\right)=
\left(\begin{array}{c}
                v_1  \\
                -\mu_1 \ q_1/\|q_1\|^3  \\
                v_2  \\
                -\mu_2 \ q_2/\|q_2\|^3  \\
                \vdots \\
                v_k    \\
                -\mu_k \ q_k/\|q_k\|^3  \\
                0      \\
\end{array}\right)+
g(t,q_1,v_1,\dots, q_k,v_k,w),
\end{equation*} 
non 
\begin{equation*}
g(t,q_1,v_1,\dots, q_k,v_k,w)=
\left(\begin{array}{c}
                g_1(t,q_1,v_1,\dots, q_k,v_k,w)      \\
                g_2 (t,q_1,v_1,\dots, q_k,v_k,w)     \\
                \vdots   \\
                g_k (t,q_1,v_1,\dots, q_k,v_k,w)      \\
                g_{k+1}(t,q_1,v_1,\dots, q_k,v_k,w)    \\
\end{array}\right).
\end{equation*}

Gorputz bakoitzari dagokion aldagai aldaketa lokala da,
\begin{align}
\label{eq:aldfl2}
\begin{split}
\left(\begin{array}{c}
                q_j  \\
                v_j  \\
\end{array}\right)&= \varphi_t^{\mu_j}(Q_j,V_j), \ \ \ j=1,\dots,k, \\
w&=W.
\end{split}
\end{align}

Aldagaia berriekiko ekuazio diferentzialak honakoak dira (ikus \ref{erans:B4} eranskinean garapena),
\begin{equation*}
\frac{d}{dt}
\left(\begin{array}{c}
                Q_1  \\
                V_1  \\
                \vdots \\
                Q_k    \\
                V_k    \\
                W      \\
\end{array}\right)=
\left(\begin{array}{c}
               \varphi_t'^{\ \mu_1}(Q_1,V_1)^{-1} \ g_1 \\ 
               \varphi_t'^{\ \mu_2}(Q_2,V_2)^{-1} \ g_2 \\
               \vdots \\
               \varphi_t'^{\ \mu_k}(Q_k,V_k)^{-1} \ g_k \\
               g_{k+1}
\end{array}\right).
\end{equation*}



\subsection*{Metodo simetrikoa.}

Lehenik, azpimarratu behar dugu aldagai aldaketa lokala izan behar duela eta horretako, integraziorako egoera aldagai berri bat ($\tau$) gehitu behar dugula,
\begin{equation}
(U,\tau).
\end{equation}

Metodoa simetriko izateko, integrazio eskema orokorra \ref{fig:proiekzioa0}~irudian laburtu dugu,
\begin{figure} [h!]
\centerline{\includegraphics [width=16cm, height=4cm] {proiekzioa11}}
\caption{}
\label{fig:proiekzioa0}
\end{figure} 

Integrazioaren urrats guztietan ez baditugu emaitzak itzuli behar, bi urratsen arteko, $\varphi_{h/2}$ fluxuaren bi konputazioak, $\varphi_{h}$ fluxuaren konputazio bakarrarekin konputatuko dugu. Horretarako, proiekzio kontzeptua sortuko dugu (\ref{fig:proiekzioa2}~irudia).

\begin{figure} [h!]
\centerline{\includegraphics [width=14cm, height=4cm] {proiekzioa12}}
\caption{\small Proiekzioa: bi urratsen arteko, $\varphi_{h/2}$ fluxuaren bi konputazioak, $\varphi_{h}$ fluxuaren konputazio bakarrarekin konputatuko dugu}
\label{fig:proiekzioa2}
\end{figure} 


 Azkenik, emaitzak behar ditugun urratsetarako fluxua $\varphi_{-h/2}$ aplikatuko dugu (\ref{fig:proiekzioa1}~irudia). 

\begin{figure} [h!]
\centerline{\includegraphics [width=14cm, height=4cm] {proiekzioa1}}
\caption{\small $u_i$ jatorrizko aldagaiak eta $U_i$ aldagai berriak adierazten dute. Lehenengo, $u_0$ jatorrizko aldagaien hasierako baliotik abiatuta, aldagai berriei dagokion hasierako balioa finkatuko dugu $(U_0,-h/2)$. Urrats bakoitza, integrazio eta proiekzioaren konposaketa da eta \ref{fig:proiekzioa2} irudian zehaztu dugu. Erabiltzaileak definitutako urratsetarako, $u_n$ jatorrizko aldagaietann zenbakizko soluzioa itzuliko dugu}
\label{fig:proiekzioa1}
\end{figure} 


$u_i$ jatorrizko aldagaiak eta $U_i$ aldagai berriak adierazten duten notazioa erabiliko dugu. Hauek dira, integratzeko emango ditugun urratsak:
\begin{enumerate}
\item \emph{Startfun} funtzioa.

Lehenengo, $u_0$ jatorrizko aldagaien hasierako baliotik abiatuta, $\varphi_{h/2}$ fluxuaren konputazioaren bidez, aldagai berrietan dagokion hasierako balioa lortuko dugu.
\begin{align*}
u_0 \ \rightarrow \ (U_0,-h/2).
\end{align*}

\item \emph{Urratsa}.

Urratsa integrazio eta proiekzioaren konposaketa da; \ref{fig:proiekzioa2} irudian zehaztapenak eman ditugu. Aldagai aldaketa urrats bakoitzean aplikatzen dugu.  
\begin{align*}
(U_0,-h/2) \ \rightarrow \ (U_1,-h/2).
\end{align*}

Biribiltze errorea txikitzeko, proiekzioa doitasun altuan konputatzea garrantzitsua da. Modu honetan, batura konpensatua aplikatzerakoan zifra batzuk irabaziko ditugu. 

\item \emph{Outputfun} funtzioa.

Erabiltzaileak definitutako urratsetarako, $\varphi_{-h/2}$ fluxuaren konputazioaren bidez, $u_n$ zenbakizko soluzioa jatorrizko aldagaietan  itzuliko dugu.
\begin{align*}
(U_n,-h/2) \ \rightarrow \ u_n.
\end{align*}


\end{enumerate}


Gauss metodoa, neurri batean  Splitting eta konposizio metodoen baliokideak dira. 
\begin{align*}
&\text{Konposizio metodoa} \ \ \Leftrightarrow \ \ \text{Gauss metodoa aldagai aldaketa gabe}.\\
&\text{Splitting metodoa}  \ \ \Leftrightarrow \ \  \text{Gauss metodoa aldagai aldaketarekin}.
\end{align*}

Splitting metodoekiko antzekotasuna azaltzeko, (\ref{eq:stverlet})~\emph{Störmer-Verlet} Splitting metodoarekin konparatuko dugu. \emph{Störmer-Verlet} metodoa, era honetan aplikatzen da: $h/2$ fluxua aplikatu, perturbazioak kalkulatu eta berriz  $h/2$ fluxua aplikatu. Fluxuaren aldagai aldaketarekin, gauza bera egiten ari gara: $h/2$ fluxua aurreratu, perturbazioak kalkulatu (aldagai berrietan eta beraz, hobeto kalkulatzen dugu), $h/2$ fluxua aurreratu. 


\section{Zenbakizko esperimentuak.}


Atal honetan, puntu-finkoaren iterazioan oinarritutako Gauss metodoaren inplementazioa erabili dugu eta eguzki-sistemaren ekuazio diferentzialei, Kepler-en fluxuan oinarritutako aldagai aldaketa aplikatu diegu. $s=6,8,9,16$ ataletako Gauss metodoak exekutatu ditugu eta metodo eraginkorrena aukeratu dugu, ordena altuko beste metodo sinplektikoekin konparatzeko. 


\subsection{Problemak.}


9-planeten problema (\ref{sss:9body}~atala) erabili dugu integrazioetarako. Hasierako balioak \emph{DE-430} efemerideen artikulutik hartu ditugu: planeten masak  \ref{tab:9bodymas}~taulan laburtu ditugu; hasierako kokapen eta abiadurak \ref{tab:9bodyhas}~taulan aurki daitezke. Integratzeko, koordenatu heliozentrikoei dagokien (\ref{eq:nbodyHel}) Hamiltondarrean  oinarrituko gara. 

Integrazioaren tartea, $t_{end}=10^6$ egunetakoa izan da eta zenbakizko integrazioetan, $h$-ren balio ezberdinak erabili ditugu. $s=6$ metodoarentzat urrats luzerak aukeratu ditugu eta gainontzeko metodoentzat, $s$-atalen araberako urrats luzera proportzionalak finkatu ditugu:
\begin{align*}
&s=6: \quad  \ \ h=2^{k/4}, \ k=4,\dots,28, \\
&s=8: \quad  \ \ (8/6)h, \\
&s=9: \quad  \ \ (9/6)h, \\
&s=16: \quad (16/6)h. \\
\end{align*} 

Zenbakizko esperimentuetarako, aldagai aldaketa planeta guziei aplikatzea erabaki dugu. $9$-planeten probleman, gorputz kopurua txikia denez,  Kepler fluxuaren gainkarga esanguratsua da eta  barne-planetei bakarrik aplikatzea, eraginkorragoa izan daiteke. Baina, gorputz gehiago kontsideratzen baditugu (esaterako ilargia eta asteroide nagusienak) edo eguzki-sistemaren eredu konplexuagoetan (esaterako erlatibitate efektua gehitzerakoan), perturbazio aldearen konputazioa nagusituko da eta Kepler fluxuaren kalkuluak pisua galduko du. 


\subsection*{Lehen esperimentua.}


\ref{fig:esp81s}~irudian, $s=6,8,9,16$ ataletako metodoen, bi eraginkortasun grafiko irudikatu ditugu. Eraginkortasuna, energia errore erlatibo maximoaren arabera neurtu dugu: lehen kasuan, \emph{CPU}-denborarekiko eta bigarren kasuan, ekuazio diferentzialen ebaluazio kopuruarekiko (\emph{FCN}). Eraginkortasuna \emph{CPU}-rekiko, problema zehatz honetarako gertatzen dena azaltzen digu eta eraginkortasuna \emph{FCN}-rekiko, metodoak problema erreal batean nola jokatuko luke erakusten digu.


\begin{figure}[h!]
\centering
\begin{tabular}{c c}
\subfloat[ Exekuzioa serian. $\max (|\Delta E|)$-CPU.]
{\includegraphics[width=.5\textwidth]{esperimentua811}}
&
\subfloat[ Exekuzioa serian. $\max (|\Delta E|)$-FCN.]
{\includegraphics[width=.5\textwidth]{esperimentua812}}\\
\subfloat[Exekuzio paraleloa. $\max (|\Delta E|)$-CPU.]
{\includegraphics[width=.5\textwidth]{esperimentua813}}
&
\subfloat[Exekuzio paraleloa. $\max (|\Delta E|)$-FCN.]
{\includegraphics[width=.5\textwidth]{esperimentua814}}
\end{tabular}
\caption{\small 
Eraginkortasun grafikoak eskala logaritmiko bikoitza erabiliz irudikatu ditugu. Batetik ardatz bertikalean, energiaren errore erlatibo maximoa eman dugu. Bestetik, ardatz horizontalean, ezkerreko grafikoan CPU denbora eta eskuineko grafikoan, ekuazio diferentzialen ebaluazio kopurua (FCN) erakutsi dugu. Irudi bakoitzean, Gauss metodoaren lau integrazio konparatu ditugu: $s=6$  urdinez, $s=8$ gorriz, $s=9$ berdez, eta $s=16$ grisez. (a) eta (b) konputazioak modu sekuentzialean egin ditugu; (c) eta (d) modu paraleloan, lehenak hari kopurua $2$ eta bigarrenak hari kopurua $4$}
\label{fig:esp81s}
\end{figure}

Gure helburua, bai exekuzio sekuentzialak bai exekuzio paraleloak aztertuz, doitasun altuko integrazioetarako Gauss metodo eraginkorrena aukeratzea da. Horretarako, biribiltze errorea nagusitzen hasten den inguruko unean gertatutakoa aztertu dugu: $s=8,9,16$ metodoak, $s=6$ metodoa baino eraginkorragoak azaldu zaizkigu eta hirurak oso antzekoak dira. 

\subsection*{Bigarren esperimentua.}


$s=8$ metodoarentzat, birbiltze errorea hasten den uneko urrats luzera hartu dut: $k=12, \ h=10,667$. Kokapen errore erlatiboaren estimazioa, $h/2$ integrazioarekiko diferentzia gisa kalkulatu ditugu.


\subsubsection*{Energiaren eboluzioa}

\begin{figure}[h!]
\centering
\begin{tabular}{c c}
\subfloat[Energia errorea: $h=10,667$.]
{\includegraphics[width=.5\textwidth]{esperimentua831}}
&
\subfloat[Energia errore: $h=10,667/2$.]
{\includegraphics[width=.5\textwidth]{esperimentua832}}
\end{tabular}
\caption{\small Energia errorearen eboluzioa. }
\label{fig:esp83}
\end{figure}


\subsubsection*{Errore globalak}

\begin{figure}[h!]
\centering
\begin{tabular}{c c}
\subfloat[Kokapen errorea.]
{\includegraphics[width=.5\textwidth]{esperimentua841}}
&
\subfloat[Abiadura errorea.]
{\includegraphics[width=.5\textwidth]{esperimentua842}}
\end{tabular}
\caption{\small }
\label{fig:esp84}
\end{figure}
 

\subsection*{Hirugarren esperimentua.}


Biribiltze errorearen azterketa (momentu angeluarra). $s=8$ metodoa eta urrats luzera handiak ($h=42.667$ eta $h=42.667/2$) erabiliz egindako integrazioak. Momentu angeluarraren trunkatze errorea beti zero da, metodo sinplektikoek inbariante koadratikoak zehazki mantentzen dituzte.

\begin{figure}[h!]
\centering
\begin{tabular}{c c}
\subfloat[Momentu angeluarra $h=42.667$.]
{\includegraphics[width=.5\textwidth]{esperimentua851}}
&
\subfloat[Momentu angeluarra $h=42.667/2$]
{\includegraphics[width=.5\textwidth]{esperimentua852}}
\end{tabular}
\caption{\small }
\label{fig:esp85}
\end{figure}



\subsection*{Laugarren esperimentua.}


Beste metodo sinplektikoekiko konparaketa. Gauss metodoa modu paraleloan exekutatu dugu: $s=16$ eta $s=8$ metodoak, splitting/konposizio metodoekin konparatu ditugu.

\begin{figure}[h!]
\centering
\begin{tabular}{c c}
\subfloat[$s=16$ Exekuzio sekuentziala: CPU-denbora.]
{\includegraphics[width=.4\textwidth]{esperimentua821}}
&
\subfloat[$s=16$ Exekuzio sekuentziala:: FCN.]
{\includegraphics[width=.4\textwidth]{esperimentua822}}\\
\subfloat[$s=16$ Exekuzio paraleloa: hariak=$2$.]
{\includegraphics[width=.4\textwidth]{esperimentua823}}
&
\subfloat[$s=16$ Exekuzio paraleloa: hariak=$4$.]
{\includegraphics[width=.4\textwidth]{esperimentua824}}
\end{tabular}
\caption{\small 
Eraginkortasun grafikoak irudikatu ditugu: ezkerrean energiaren errore maximoa, CPU denborarekiko; eskuinean ekuazio diferentzialen ebaluazio kopuruarekiko (FCN). Lau integrazio metodo konparatu ditugu: $ABAH1064$  urdinez, $CO1035$ gorriz,  eta \emph{IRKFLUXU} grisez}
\label{fig:esp82}
\end{figure}

\begin{figure}[h!]
\centering
\begin{tabular}{c c}
\subfloat[$s=16$ Exekuzio sekuentziala: CPU-denbora.]
{\includegraphics[width=.5\textwidth]{esperimentua821}}
&
\subfloat[$s=16$ Exekuzio sekuentziala:: FCN.]
{\includegraphics[width=.5\textwidth]{esperimentua822}}\\
\subfloat[$s=16$ Exekuzio paraleloa: hariak=$2$.]
{\includegraphics[width=.5\textwidth]{esperimentua823}}
&
\subfloat[$s=16$ Exekuzio paraleloa: hariak=$4$.]
{\includegraphics[width=.5\textwidth]{esperimentua824}}
\end{tabular}
\caption{\small 
Eraginkortasun grafikoak irudikatu ditugu: ezkerrean energiaren errore maximoa, CPU denborarekiko; eskuinean ekuazio diferentzialen ebaluazio kopuruarekiko (FCN). Lau integrazio metodo konparatu ditugu: $ABAH1064$  urdinez, $CO1035$ gorriz,  eta \emph{IRKFLUXU} grisez}
\label{fig:esp82}
\end{figure}


\section{Laburpena.}