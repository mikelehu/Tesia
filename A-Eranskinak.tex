\chapter{A-Eranskinak}.

\section{Kepler ekuazioak eta definizioak.}

\paragraph*{\textbf{Kepler ekuazioak (definizioa)}},
\begin{equation*}
E_0-e\sin E_0=n (t_0-t_p)
\end{equation*}
\begin{equation*}
E_1-e\sin E_1=n (t_1-t_p)
\end{equation*}
non $n=\frac{2\pi}{P}$, $P$ periodoa den.

\paragraph*{} Honako garapen hau egingo dugu,
\begin{equation*}
E_1-E_0-e(\sin(E_1)-\sin(E_0))=n \triangle t \ \ \ \longrightarrow \ \triangle E - e(\sin(E_0+\triangle E)-\sin(E_0))=n \triangle t
\end{equation*}
\begin{equation*}
E_1=E_0+\triangle E  
\end{equation*}

\begin{equation*}
\triangle E - ce \ \sin(\triangle E- se \ (\cos(\triangle E)-1)=n \triangle t
\end{equation*}
non, $ce=e \ \cos(E_0)$ eta $se=e \ \sin(E_0)$

\paragraph*{\textbf{Newton metodoa}},
\begin{equation*}
f(\triangle E)=\triangle E - ce \sin(\triangle E)- se (\cos(\triangle E)-1)-n \triangle t=0
\end{equation*}
\begin{equation*}
f'(\triangle E)=1-ce \cos(\triangle E)+ se \sin(\triangle E)
\end{equation*}
\begin{equation}
\triangle E^{[k+1]}=\triangle E^{[k]}- \frac{f(\triangle E^{[k]})}{f'(\triangle E^{[k]})}
\end{equation}

\paragraph*{\textbf{$\triangle E^{[0]}$ hasierako balioa}}, finkatzea da dugun zailtasun handiena. Horretarako honako garapena egingo dugu,
\begin{equation*}
\triangle E - ce \ \sin(\triangle E- se \ (\cos(\triangle E)-1)=n \triangle t
\end{equation*} 
\begin{equation*}
x=\triangle E- n \triangle t
\end{equation*}

eta beraz,
\begin{equation*}
x-ce \sin(n\triangle t+x)-se(cos(n \triangle t+x)-1)=0
\end{equation*}

Honako baliokidetasun trigonometrikoak aplikatuz,
\begin{equation*}
\cos(A+B)=\cos(A)\cos(B)-\sin(A)\cos(B)
\end{equation*}
\begin{equation*}
\sin(A+b)=\cos(A)\sin(B)+\sin(A)\cos(B)
\end{equation*}

berdintza hau lortzen dugu,
\begin{equation*}
x- (se \ \cos(n \triangle t)+ ce \ \sin(n \triangle t)) \cos(x)+ (se \ \sin(n \triangle t)-ce \ \cos(n \triangle t)) \sin(x)+se =0
\end{equation*}

$x$ txikia denean honako hurbilpenak ordezkatuz,
\begin{equation*}
x \approx \sin(x), \ \cos(x) \approx 1- \frac{x^2}{2}
\end{equation*}
\begin{equation}
(se \ \cos(n \triangle t)+ce \ \sin(n \triangle t)) \frac{x^2}{2}+ (1+se \ \sin(n \triangle t)-ce \cos(n \triangle t))x-(se )=0
\end{equation}

Goiko ekuazio hau askatuz ($Ax^2+Bx+C=0, \ \rightarrow x=\frac{-B\pm \sqrt{B^2-4AC}}{2A}$) lortuko dugu $\triangle E^{[0]}=x+n\triangle t$.

\paragraph*{\textbf{Koordenatu kartesiarren}}, kalkulua modu ekuazio hauen bidez egingo dugu,

\begin{equation*}
(q_1,v_1)=(q_0,v_0)+ (q_0,v_0) \left(\begin{array}{cc}
                                       b_{11} & b_{12} \\
                                       b_{21} & b_{22} \\
                                \end{array}\right)
\end{equation*}

\begin{equation*}
b_{11}=(C-1) \frac{a}{\|q\|}
\end{equation*}

\begin{equation*}
b_{21}=\triangle t+(S-\triangle E) \frac{a^{\frac{3}{2}}}{\mu^{\frac{1}{2}}}
\end{equation*}

\begin{equation*}
b_{12}=\frac{s}{\|q\| \sqrt{a} (1-ce \ C +se \ S)}
\end{equation*}

\begin{equation*}
b_{22}=\frac{C-1}{1-ce \ C+ se \ S}
\end{equation*}

Eta osagai bakoitzaren definizioa,
\begin{equation*}
C=\cos(\triangle E), \ S=\sin(\triangle E)
\end{equation*}

\begin{equation*}
ce=e \ \cos(E_0) = \|q\| \|v\|^2-1
\end{equation*}

\begin{equation*}
se= e \ \sin(E_0)=\frac{(q \cdot v)}{\sqrt{\mu \ a}}
\end{equation*}

\begin{equation*}
a= \frac{\mu \|q\|}{2\mu-\|q\|\|v\|^2}
\end{equation*}

\begin{equation*}
n= \frac{\mu^{\frac{1}{2}}}{a^{\frac{3}{2}}}
\end{equation*}

\paragraph*{\textbf{Biribiltze errorea}}. $\triangle E$ txikia denean, $\cos(\triangle E)-1$ espresioaren kalkuluaren ezabapen arazoak biribiltze errore handia eragin dezake. Hori konpontzeko baliokidetasun hau erabiliko dugu,

\begin{equation*}
\cos(\triangle E)-1=-\frac{(\sin^2(\triangle E)}{1+\cos(\triangle E)}
\end{equation*}  

Eta beraz, kepler-en ekuazioak hauek izango dira,
\begin{equation*}
f(\triangle E)=\triangle E - ce \sin(\triangle E)+ se \bigg(\frac{(\sin^2(\triangle E)}{1+\cos(\triangle E)}\bigg)-n \triangle t=0
\end{equation*}

Eta $(q_1,v_1)$ balioak kalkulatzeko,

\begin{equation*}
b_{11}=(C-1) \frac{a}{\|q\|}, \longrightarrow b_{11}=-\frac{(\sin^2(\triangle E)}{1+\cos(\triangle E)} \frac{a}{\|q\|}
\end{equation*}

\section{Ekuazio Heliozentrikoak.}

\subsection*{Sarrera.}

\paragraph*{} Lehenik koordenatu barizentrikoei,

\begin{equation*}
\mathbf{q_i}, \mathbf{p_i} \in \mathbb{R}^3, \ i=0,\dots,N
\end{equation*}

dagokien Hamiltondarra gogoratuko dugu, 
\begin{equation}
H(\mathbf{q},\mathbf{p})=\frac{1}{2}\ \sum^N_{i=0}{\ \frac{{\|\mathbf{p_i}\|}^2}{m_i}}-G\ \sum^N_{0\le i<j\le N}{\frac{m_im_j}{\|\mathbf{q_i}-\mathbf{q_j}\|}}.
\end{equation}

\paragraph*{} Koordenatu barizentrikoetatik abiatuta eta aldagai aldaketa bat aplikatuz ekuazio koordenatu heliozentrikoen

\begin{equation*}
\mathbf{Q_i}, \mathbf{P_i} \in \mathbb{R}^3, \ i=0,\dots,N
\end{equation*}
arabera berridatziko ditugu. 

\subsection*{Aldagai aldaketa.}

Lehenik honako aldagai aldaketa aplikatuko dugu,

\begin{equation*}
\mathbf{Q_0}=\mathbf{q_0}, \ \ \mathbf{Q_i}=\mathbf{q_i}-\mathbf{q_0}, \ \ i=1,\dots{,N}
\end{equation*}

\begin{equation*}
\mathbf{P_0}=\sum\limits_{i=0}^{N}\mathbf{p_i}, \ \ \mathbf{P_i}=\mathbf{p_i}, \ \ i=1,\dots{,N}.
\end{equation*}

\paragraph*{} Hamiltondarra era honetan deskonposatu daiteke $H=H_k+T_1+U_1$,

\begin{equation*}
H_k=\sum\limits_{i=1}^{N}\bigg(\frac{\|\mathbf{P_i}\|^2}{2} \big(\frac{m_0+m_i}{m_0m_i}\big)-G \frac{m_0m_i}{\|\mathbf{Q_i}\|}\bigg)
\end{equation*} 

\begin{equation*}
T_1=\sum\limits_{0<i<j\le N}^{N} \frac{\mathbf{P_i}\mathbf{P_j}}{m_0}.
\end{equation*}

\begin{equation*}
U_1= -G \sum\limits_{0\le i<j\le N}^{N} \frac{m_im_j}{\|\mathbf{Q_i}-\mathbf{Q_j}\|}
\end{equation*}

\paragraph*{}Hamiltondarraren alde bakoitza bere aldetik deribatuz,

\begin{enumerate}
\item $H_k$.

\begin{equation*}
\dot{\mathbf{Q_i}}= \triangledown_p H_k \ \ \Rightarrow \ \  \dot{\mathbf{Q_i}}=\mathbf{P_i}\big(\frac{m_0+m_i}{m_0m_i}\big), \ \ i=1,\dots, N.
\end{equation*}

\begin{equation*}
\dot{\mathbf{P_i}}= -\triangledown_q H_k \ \ \Rightarrow \ \  \dot{\mathbf{P_i}}= - G  \ \frac{m_0m_i}{\|\mathbf{Q_i}\|^3 }\ \mathbf{Q_i}, \ \ i=1,\dots, N.
\end{equation*}

\item $T_1$.

\begin{equation*}
\dot{\mathbf{Q_i}}= \triangledown_p T_1 \ \ \Rightarrow \ \  \dot{\mathbf{Q_i}}=\sum\limits_{j\ne i,\ j=1}^{N} \frac{\mathbf{P_i}}{m_0}, \ \ i=1,\dots, N.
\end{equation*}

\begin{equation*}
\dot{\mathbf{P_i}}= -\triangledown_q T_1 \ \ \Rightarrow \ \  \dot{\mathbf{P_i}}= 0, \ \ i=1,\dots, N.
\end{equation*}

\item $U_1$.

\begin{equation*}
\dot{\mathbf{Q_i}}= \triangledown_p U_1 \ \ \Rightarrow \ \  \dot{\mathbf{Q_i}}=0, \ \ i=1,\dots, N.
\end{equation*}

\begin{equation*}
\dot{\mathbf{P_i}}= -\triangledown_q U_1 \ \ \Rightarrow \ \ 
 \dot{\mathbf{P_i}}= -G \ \sum\limits_{j \ne i , \ j=1}^{N} \bigg(\frac{m_im_j}{\|\mathbf{Q_i}-\mathbf{Q_j}\|^3} \ (\mathbf{Q_i-\mathbf{Q_j}}) \bigg), \ \ i=1,\dots, N.
\end{equation*}

\end{enumerate}

%\subsection*{Kepler ekuazioak.}

%Bigarrenik, $H_k$ aldeari dagokion ekuazioak kepler ekuazioak idazteko ohiko moduan lortzeko,

%\begin{enumerate}
%\item Bigarren ordeneko ekuazio diferentzial bezala idatzi $\mathbf{P_i}$ gaia kenduz,

%\begin{equation*}
%\frac{d^2}{dt^2} \dot{\mathbf{Q_i}}=- G  \ \frac{(m_0+m_i)}{\|\mathbf{Q_i}\|^3 }\ \mathbf{Q_i}, \ \ i=1,\dots, N. 
%\end{equation*}

%\item Berriz ere, lehen ordenako ekuazio diferentzial gisa idatziko dugu baina orain komeni zaigun \emph{aldagai artifiziala} $\mathbf{V_i}=\mathbf{P_i} (\frac{m_0+m_i}{m_0m_i})$ erabiliz,

%\begin{equation*}
%\dot{\mathbf{Q_i}}=\mathbf{V_i},
%\end{equation*}

%\begin{equation*}
%\dot{\mathbf{V_i}}=- G  \ \frac{\mu_i}{\|\mathbf{Q_i}\|^3 }\ \mathbf{Q_i}, \ \ \mu_i=G(m_0+m_i),  \ \ i=1,\dots, N.  
%\end{equation*}

%\end{enumerate}  

\subsection*{Ekuazio diferentzialak ($Q_i,V_i$).}

Ekuazio diferentzialak ($\mathbf{Q_i},\mathbf{P_i}$) aldagai berriekiko idatziko ditugu, honako definizioa kontutan harturik,

\begin{equation*}
\mathbf{V_i}=\frac{\mathbf{P_i}}{(m_0+m_i)}, \ \ \mu_i=\frac{m_0m_i}{(m_0+m_i)}.
\end{equation*}

\begin{enumerate}
\item $H_k$.

\begin{equation*}
\dot{\mathbf{Q_i}}=\mathbf{V_i} \ \ i=1,\dots, N.
\end{equation*}

\begin{equation*}
\dot{\mathbf{V_i}}= - G  \ \frac{(m_0+m_i)}{\|\mathbf{Q_i}\|^3 }\ \mathbf{Q_i},  \ \ i=1,\dots, N.
\end{equation*}

\item $T_1$.

\begin{equation*}
\dot{\mathbf{Q_i}}=\sum\limits_{j\ne i,\ j=1}^{N} \frac{\mathbf{V_j} \ m_j}{(m_0+m_j)}, \ \ i=1,\dots, N.
\end{equation*}

\begin{equation*}
\dot{\mathbf{V_i}}= 0, \ \ i=1,\dots, N.
\end{equation*}

\item $U_1$.

\begin{equation*}
\dot{\mathbf{Q_i}}=0, \ \ i=1,\dots, N.
\end{equation*}

\begin{equation*}
\dot{\mathbf{V_i}}= -G \ \frac{(m_0+m_i)}{m_0}
                    \sum\limits_{j \ne i , \ j=1}^{N} \bigg( \frac{m_j}{\|\mathbf{Q_i}-\mathbf{Q_j}\|^3} (\mathbf{Q_i-\mathbf{Q_j}})     \bigg), \ \ i=1,\dots, N.
\end{equation*}

\end{enumerate}


\subsection*{Energia kalkulua.}

Koordenatu heliozentrikoetan integrazioak egiten ditugunean, sistemaren energia kalkulatzeko koordenatu barizentrikoeetara bihurtuko dugu soluzioa.

\paragraph*{} Hauek dira koordenatu heliozentrikoetatik abiatuta ($\mathbf{Q_i},\mathbf{V_i}$) , koordenatu barizentrikoak ($\mathbf{q_i},\mathbf{v_i}$) kalkulatzeko ekuazioak,

\begin{enumerate}

\item $\mathbf{q_i}, \ \ i=0,\dots,N$. 

\begin{equation*}
\mathbf{q_0}=-\sum\limits_{i=1}^{M} \frac{m_i \mathbf{Q_i}}{M}, \ \ M= \sum\limits_{i=0}^{N} m_i.
\end{equation*} 

\begin{equation*}
\mathbf{q_i}=\mathbf{q_0}+\mathbf{Q_i}, \ \ i=1,\dots,N.
\end{equation*}

\item $\mathbf{v_i}, \ \ i=0,\dots,N$. 

\begin{equation*}
\mathbf{v_i}=\frac{m_0}{m_0+m_i} \ \mathbf{V_i}, \ \ i=1,\dots,N.
\end{equation*}

\begin{equation*}
\mathbf{P_0}=\sum\limits_{i=0}^{N} \mathbf{p_i}=\sum\limits_{i=0}^{N} m_i \mathbf{v_i}=0 \Rightarrow \ \  m_0\mathbf{v_0}+ \sum\limits_{i=1}^{N} m_i \mathbf{v_i}=0 \Rightarrow \mathbf{v_0}=-\frac{1}{m_0} \sum\limits_{i=1}^{N} m_i \mathbf{v_i}. 
\end{equation*}

\end{enumerate}





 
