\chapter{Review.}

\section{Sarrera.}

\section{Efemerideak.}

Hiru dira planetak efemerideak,

\begin{enumerate}
\item Jet Propulsion Laboratory, DE (Development Ephemerides).

      Integrazio tartea: $1550-2650$.

      Zenbkizko metodoa: "DIVA" (Krogh,1997).A variable order Adams method.
      
      Doitasuna: "QIVA" a quaruple precision of "DIVA" : the equations of motion , the newtonian part is computed in quadruple precision; all of the rest are computed in double precision.

\item Paris Observatory, INPOP (Intégrateur Númerique Planétaire de l'Observatoire de Paris).
      
      Integrazio tartea: .
      
	  Zenbakizko metodoa: The integrator is an Adams-Cowell method with fixed step-size.
	  
	  Doitasuna:  the programming is done in C language, thus allowing to use the extended precision (80 bits). 
	  Integrating in quadruple precision would of course reduce the round off error in a very large amount, but the
	  CPU time is about 15 time larger than for double precision arithmetic (or extended arithmetic) on our machine
	  (Itanium II with Intel C++ compiler). Nevertheless, it was possible to obtain an additional order of magnitude im-
	  provement by using a single addition in simulated quadruple precision in the corrector step with a very small over-
	  head.
	  
	  Hardware: Intel Itanium II processors.
	  
\item St. Petersburg, EPM (Ephemerides Planets-Moon).
      
      Integrazio tartea: .
      
      Zenbakizko metodoa: Everhart. Implicit RK method (Gauss-Radau).
      (An efficient integrator that uses Gauss-Radau Spacings)
      
      Doitasuna: double precision. The change of ERA system integrator (19 decimal digits instead of 15 ones) with the aim to reduce the round-off error. (Extended precision).
      
\end{enumerate}

\paragraph{Influence of the methods of constructing ephemerides...}
It is obvious that such ephemerides in themselves must have 128 bits, that is, comprise the coefficients calculated with quadruple precision.


\section{Eguzki-sistemaren integrazio luzeak.}

Wisdomek eta Holmanenek bere lanean \ycite[1991]{Sussman1992},  eguzki-sistemaren epe luzeko simulazioetarako integratzaile  sinplektikoen erabilerak arrakasta izan zuen. N-planeta eta masa nagusiko gorputza bat dugula kontsideratuta, problemaren Hamiltondarra bitan banatu zuten:  Hamiltondar Kleperiarra eta interakzioen Hamiltondarra. Metodo honetan, Hamiltondar bakoitzaren soluzioa tartekatuz, problema osoaren ebazpena kalkulatuko da. 

Wisdom eta Holmanen inplementazioak ez ditu kolisio gertuko egoerak onartzen. Arazo hau gainditzeko, urteetan zehar algoritmo honen hainbat aldaera proposatu dira: Levinson eta Duncan-ek \ycite[1994]{Levison1994}  \emph{SWIFT} softwarea garatu zuten;  Duncan, Levinson eta Lee-k \ycite[1998]{Duncan1998} \emph{SYMBA} softwarea garatu zuten; Chambers-ek \cite{Chambers1999} \emph{MERCURY} softwarea garatu zuen. Berriki, Hernandez eta Bertschinger-ek \ycite[2015]{Hernandez2015} garapen berri bat proposatu dute.

Koordenatu sistema aukera ezberdinak erabili dira Hamiltondarraren banaketa lortzeko. Jacobi koordenatuak eta koordenatu Heliozentrikoak erabili ohi dira bakoitzak bere abantaila eta desbaintailekin.

Problema integratzeko oinarrizko metodoa \emph{leapfrog} metodoa dugu. Metodo hau 2 ordeneko da. Orden altuagoko splitting eskemak : McLachlan \ycite[1995]{McLachlan1995}, Laskar eta Robutel \cite[2001]{Laskar2001}, Blanes \cite{Blanes2013}.

   
       
