\chapter{Inplementazioak.}

Lan honetan, ekarpen bakoitzari dagokion C lengoaiako inplementazio bat garatu dugu eta \href{https://github.com/mikelehu/}{kode bilgunean} eskuragarri jarri dugu. Kodea eskuragarri jartzeari, lehentasuna eman diogu \cite{Atmanspacher2016,Wilson2014}. Batetik, garapenak modu argian dokumentatzea behartu gaitu; bertan zehaztasun guztiak aztertu daitezke eta gure lanaren baliagarritasuna baieztatu daiteke. Bestetik, ikerlariei gure lana erabiltzeko eta hobetzeko aukera eskaintzen diegu.

Konputazio zientzian, idatzitako kodearen $200$ lerro oro, errore bat izaten dela \cite{Frenkel2001} estimatzen da. Gure inplementazioak ondo frogatu baditugu ere, errorerik ez denik izango ezin dugu ziurtatu. Akatsen bat izatekotan, larria ez izatea espero dugu eta edozein kasutan  erabiltzaileak jakinaraztea eskertu genuke.   

Inplementazioen kodea antolatzeko, irizpide berdinak erabili ditugu. Lehenengo, inplementazioen egitura orokorra azaldu dugu eta ondoren, garapen bakoitzaren argibideak emango ditugu.      

\section{Egitura orokorra.}

Inplementazioen edukia, direktorio hauetan banatu dugu:
\begin{enumerate}

\item CoefficcientsData.

Mathematican, IRK Gauss kolokazio metodoaren koefizienteak sortzeko aplikazio bat garatu dugu. Gure esperimentuetarako $s=6,s=8$ eta $s=16$ ataletako Gaussen metodoak aplikatu ditugu eta metodo bakoitzari dagokien doitasun bikoitzeko ($64$-bit) koefizienteak, azpidirektorio batean bildu ditugu.

\item PerturbationsData.

Errorearen azterketa estatistikoak egiteko, hasierako balio ezagun baten $P=1000$ hasierako balio perturbatuak sortu ditugu; hasierako balio originalaren osagai bakoitza ausaz perturbatu dugu ($\mathcal{O}(10^{-6})$ tamainako errore erlatiboarekin). Problema bakoitzari dagokion, $P=1000$ hasierako balio perturbatuak, fitxategi bitar batean idatzi dugu eta horrela, fitxategi hauetatik hasierako balioak  irakurrita, esperimentu bera errepikatu daiteke.

\item Packages.

Esperimentuak, Mathematica ingurunetik exekutatu ditugu eta direktorio honetan, soluzioen azterketak egiteko hainbat funtzio garatu ditugu. Batetik, problema bakoitzari dagokion ekuazio diferentzialak eta Hamiltondarra inplementatu ditugu. Bestetik, esperimentuen grafikoak irudikatzeko funtzioak garatu ditugu. 

\item Examples.

Mathematica erabiliz egindako zenbakizko integrazioen adibideak eman ditugu. Adibide bakoitza, azpidirektorio batean bildu dugu: zenbakizko integrazioa eta honen analisia, modu independentean exekutatu daitezke. Horretarako, integrazioaren soluzioak fitxategi bitarretan idazten dira eta analisiak, fitxategi hauek irakurriko ditu.       

\item Code.

C lengoaian garatutako gure inplementazioa. 

\end{enumerate}

\subsection*{Exekuzioa.}

Code/Readme.txt fitxategian, inplementazioaren argibideak eman ditugu. Inplementazioa bi modutan exekutatu daiteke.

\begin{enumerate}
\item Linux terminala.

Exekuzioa Linux terminaletik exekutatu daiteke eta exekutatzeko adibidea bat  "terminal.c" fitxategian eman dugu. 

\item Mathematica ingurunetik.

Bestalde, inplementazioa Mathematicatik exekutatzeko prest dago eta "Examples" direktorioan hainbat adibide eman ditugu. Mathematicatik gure C inplementazioaren funtzio nagusia deitu ahal izateko, 'math-IRK.tm' eta 'math-IRK.c' fitxategiak definitu ditugu. Exekutagarria, Mathematicatik bateragarria izan dadin, konpilazio eta esteka egiteko moduak \emph{makefile} fitxategian kontsultatu daitezke.    

\end{enumerate}

Erabiltzaileak, bere problemaren ekuazio diferentzialak eta integrazioen emaitzak definituko dituela espero da. 

\subsection*{Paralelizazioa.}

IRK metodoen s-ataletako funtzioen konputazioak ($f(Y_i), \ i=1,\dots,s$), paraleloan exekutatu daitezke. Gure inplementazioan, OpenMP erabili dugu:  \emph{PARALLEL} konpilazio parametroarekin aktibatu daiteke eta orduan, prozesadore kopurua 'threadcount' aldagaian, zehaztu behar da. 


\section{IRK puntu-finkoa.}

Puntu-finkoaren iterazioan oinarritutako IRK metodoaren inplementazioa, \href{https://github.com/mikelehu/IRK-FixedPoint}{kodea} helbidean eskuragarri dago. Bi modutara exekutatu daiteke: puntu-finkoaren iterazio estandarra edo puntu-finkoaren iterazio partizionatua. 

Code/Readme.txt fitxategian, IRK puntu-finkoaren inplementazioaren argibide guztiak eman ditugu. Integrazioaren funtzio nagusiaren deia honakoa da:   
\begin{lstlisting}
IRKFP (t0,t1,h,&method,&u,&system,&options,&thestat);
\end{lstlisting}

Exekuzioa Linux terminaletik exekutatu daiteke eta exekutatzeko adibidea bat  "terminal.c" fitxategian eman dugu. 
Bestalde, inplementazioa Mathematicatik exekutatzeko prest dago eta "Examples" direktorioan hainbat adibide eman ditugu.    

Inplementazio honetan, ez dugu \emph{BLAS} liburutegia erabili. 

\section{IRK Newton.}

Newton sinplifikatuaren iterazioan oinarritutako IRK metodoaren inplementazioa, \href{https://github.com/mikelehu/IRK-Newton}{kodea} helbidean eskuragarri dago. Bi inplementazio exekutatu daitezke: Newton sinplifikatuaren inplementazio eraginkorra eta artikuluan, proposatutako Newton iterazioan oinarritutako IRK inplementazio berria.

Code/Readme.txt fitxategian, IRK puntu-finkoaren inplementazioaren argibide guztiak eman ditugu. Integrazioaren funtzio nagusiaren deia honakoa da:   
\begin{lstlisting}
IRKNEWTON (t0,t1,h,&method,&u,&system,&options,&thestat);
\end{lstlisting}

Exekuzioa Linux terminaletik exekutatu daiteke eta exekutatzeko adibidea bat  "terminal.c" fitxategian eman dugu. 
Bestalde, inplementazioa Mathematicatik exekutatzeko prest dago eta "Examples" direktorioan hainbat adibide eman ditugu. 

Inplementazio honetan, \emph{BLAS} eta \emph{LAPACK} liburutegiak erabili ditugu.

\section{IRK Eguzki-sistema.}

\section{Konposizio-Splitting metodoak.}

Newton sinplifikatuaren iterazioan oinarritutako IRK metodoaren inplementazioa, \href{https://github.com/mikelehu/Composition}{kodea} helbidean eskuragarri dago. Bi inplementazio exekutatu daitezke: konposizio metodoak (CO1035) eta splitting metodoak (ABAH1064).

Code/Readme.txt fitxategian, konposizio/Splitting inplementazioaren argibide guztiak eman ditugu. Integrazioaren funtzio nagusiaren deia honakoa da:   
\begin{lstlisting}
Solve_Comp (t0,t1,h,&method,basic,&system,&options,&u,&thestat);
\end{lstlisting}

Exekuzioa Linux terminaletik exekutatu daiteke eta exekutatzeko adibidea bat  "terminal.c" fitxategian eman dugu. 
Bestalde, inplementazioa Mathematicatik exekutatzeko prest dago eta "Examples" direktorioan hainbat adibide eman ditugu. 

Eguzki-sistemari egokitutako Splitting metodoak aplikatzeko, Kepler fluxuaren inplementazio berri bat garatu dugu. Inplementazio honen C kodean, 'Kepler.c' fitxategian aurkitzen da. 
