\chapter{IRK: Newton.}

\epigraph{The top 10 algorithms in Applied Mathematics: 1.Newton and quasi-Newton methods.}
{\textit {Nick Higham (2016)}}

\section{Sarrera.}

Runge-Kutta metodo inplizituen inplementazio eraginkorren arazo handiena, ekuazio sistema ez-lineala metodo iteratibo baten bidez askatzea da. Problema zurruna denean, puntu finkoaren iterazio ez da eraginkorra eta Newton interazioa aplikatu behar da. Gainera problema ez-zurruna izanik ere, Newton iterazioak interesgarriak izan daitezke; bereziki doitasun altuko (doitasun laukoitza) konputazioetan iterazio metodoaren konbergentzi ezaugarri onak direla eta. 
 
Newton metodo iteratiboa, ekuazio sistema ez-linealen zenbakizko soluzioa aurkitzeko metodoa da. $u\in \mathbb{R}^{n}$ eta $F: \mathbb{R}^n \ \longrightarrow {\mathbb{R}}^n$ emanik, $F(u)=0$ betetzen duen $u^{[*]}$ soluzioa aurkitu nahi dugu. Hasierako soluzioaren $u^{[0]}$ estimazioa  emanik,  Newton metodoa era honetan definituko dugu (Algoritmoa \ref{alg:801}).

\begin{algorithm}[H]
  $ \text{Hasieratu} \ u^{[0]}$\;
  \For{ (k=1,2,\dots \text{konbergentzia lortu arte})}
  {
   \BlankLine
   $F^{[k]}=F(u^{[k-1]})$\;
   $\text{Askatu} \ \ J(u^{[k-1]}) \ \Delta u^{[k]}=- F^{[k]}$\;
   \BlankLine
   $u^{[k]}=u^{[k-1]}+\Delta u^{[k]}$\;
  }
 \caption{Newton metodoa.}
 \label{alg:801}
\end{algorithm}

non $J(u^{[k-1]})$, \ $n \times n$ tamainako matrize jacobiarra den ($J_{ij}(u)=\partial f_i/\partial u_j (u), \ \ 1 \leq i,j \leq n$).  Newton metodoak, $u^{[*]}$ soluzioarekiko konbergentzia koadratikoa du,
\begin{equation*}
\label{eq:801}
\|u^{[k+1]}-u^{[*]}\| \le C \ \|u^{[k]}-u^{[*]}\|^2.
\end{equation*}

Newton metodo interesgarria da, baina konputazionalki garestia; iterazio bakoitzean $J(u^{[k-1]})$ jacobiarraren ebaluazioa eta $n \times n$ tamainako matrizearen \emph{LU} deskonposaketa kalkulatu behar da. Horregatik, konputazionalki merkeagoak diren Newton metodoaren aldaerak erabili ohi dira. Newton metodoaren aldaera horien artean, Newton sinplifikatua (Algoritmoa \ref{alg:802}) dugu aukera nagusienetako bat.

\begin{algorithm}[H]
  $ \text{Hasieratu} \ \ u^{[0]}   \quad \quad \quad \quad \quad \quad \quad \quad\quad \quad \quad \quad \quad \quad \quad \quad \quad    (64-bit)$\;
  $J^{[0]}=\frac{\partial F}{\partial u}(u^{[0]})  \ \ \quad \quad \quad \quad \quad \quad \quad \quad\quad \quad \quad \quad \quad \quad \quad \quad \quad    (32-bit)$\;
  $M=LU(J^{[0]}) \ \ \quad \quad \quad \quad \quad \quad \quad \quad\quad \quad \quad \quad \quad \quad \quad \quad \quad    (32-bit)$\;
  \For{ (k=1,2,\dots \text{konbergentzia lortu arte})}
  {
   \BlankLine
   $F^{[k]}=F(u^{[k-1]}) \ \ \quad \ \quad \quad \quad \quad \quad \quad \quad \quad \quad \quad \quad \quad \ \   (64-bit)$\;
   $\text{Askatu} \ \ M \ \Delta u^{[k]}=- F^{[k]} \ \quad \ \quad \quad \quad \quad \quad \quad \quad \quad \ \ \  (32-bit)$\;
   \BlankLine
   $u^{[k]}=u^{[k-1]}+\Delta u^{[k]}  \ \ \ \quad \quad \quad \quad\quad \quad \quad \quad \quad \quad \quad \ \     (64-bit)$\;
  }
 \caption{Newton sinplifikatua.}
 \label{alg:802}
\end{algorithm}

Argia da Newton sinplifikatua konputazionalki merkeagoa dela: urrats bakoitzean, jacobiarraren ebaluazio $J^{[0]}=\frac{\partial F}{\partial u}(u^{[0]})$ eta dagokion \emph{LU} deskonposaketa behin bakarrik kalkulatu behar dira. 

Azpimarratzekoa da ere, Newton metodoaren eragiketa konplexuenak doitasun txikiagoan kalkula daitezkeela \cite{Baboulin20092526} eta honek, konputazionalki abantaila interesgarria suposatzen duela. Algoritmoaren (Algoritmoa \ref{alg:802}) eskubi aldean, doitasun bikoitzeko ($64$-bit) inplementazioa balitz, eragiketa bakoitzaren doitasuna zehaztu dugu: jacobiarraren balioztapena eta algebra linealeko eragiketak, doitasun arruntean ($32$-bit) kalkulatu daitezke.

$S$-ataletako IRK metodoa, Newton iterazioaren bidez $d$-dimentsioko ODE sistemari aplikatzeko, urrats bakoitzean $sd \times sd$ tamaineko hainbat ekuazio sistema (iterazio bakoitzeko bat) askatu behar dira. Atal honetan, jatorrizko $sd$-dimentsioko ekuazio sistema, $(s+1)d$ sistema baliokide moduan berridatziko dugu eta sistema baliokidea,  $d \times d$ tamaineko $[s/2]+1$ matrize errealen \emph{LU}-deskonposaketa bidez askatuko dugu. Tamaina txikiko matrizeen LU-deskonposaketa azkarra denez, konputazionalki eraginkorragoa izatea espero dugu.   

\section{IRK-Newton estandarra.}

\paragraph*{}Demagun honako hasierako baliodun problema,
\begin{equation}
\label{eq:802}
\dot{y}=f(t,y),\ \ \ y(t_0)=y_0, 
\end{equation}
non  $y=[q^1,\dots,q^n,p^1,\dots,p^n]^T \in \mathbb{R}^{d=2n}$  eta $f: \  {\mathbb{R}}^{d+1} \ \longrightarrow {\mathbb{R}}^d$ diren. 

\paragraph*{}S-ataletako IRK metodoa gogoratuz, hasierako baliodun problemaren $y(t)$ soluzioaren $y_{n}\approx y(t_{n})$ hurbilpena, era honetan kalkulatzen da,  
\begin{equation}  
\label{eq:803}
y_{n+1}=y_n+h\sum^s_{i=1}{b_i \ f(t_n+c_ih,Y_{n,i})\ \ },\
\end{equation} 

non $Y_{n,i}$ atalak, era honetan inplizituki  definitzen diren,
\begin{equation}
\label{eq:804}
Y_{n,i}=y_n+\ h\ \sum^s_{j=1}{a_{ij}\ f(t_n+c_jh,Y_{n,j})}\ \ \ \ \ i=1 ,\dots, s.\
\end{equation} 

$Y_{n,i} \in \mathbb{R}^d,\ i=1,\ldots,s$ ezezagunen eta $sd$ tamainako ekuazio sistema ez-lineala askatzeko (\ref{eq:803}), iterazio metodo bat aplikatu behar dugu. Aurreko atalean puntu-finkoaren iterazioa aztertu genuen eta atal honetan,  Newton metodoa modu eraginkorrean aplikatzeko bidea ikertuko dugu.

\subsection*{Newton iterazioa.}

Newton iterazioa, $k=1,2,\dots$  $Y_i^{[k]}$ atalen hurbilpenak kalkulatzeko algoritmoa, modu honetan definituko dugu,

\begin{align}
\label{eq:(1)Newton_iteration}
1) & \quad r_i^{[k]} := -Y_{i}^{[k-1]} + y + h \sum_{j=1}^{s}\, a_{ij}\, f(t + c_j h,Y_{j}^{[k-1]}), \quad  i=1 ,\ldots, s, \\
\label{eq:(2)Newton_iteration}
\begin{split}
2) & \quad \mathrm{Askatu \ } \Delta Y_{i}^{[k]},\\
& \quad \Delta Y_{i}^{[k]}  - h \sum_{j=1}^{s}\, a_{ij}\, J_j^{[k]} \Delta Y_{j}^{[k]} = r_i^{[k]} \quad  i=1 ,\ldots, s, \\
& \mbox{non} \quad  J_i^{[k]}=\frac{\partial f}{\partial y}(t + c_i h,Y_{i}^{[k]}) \quad \quad  i=1,\ldots, s, 
\end{split} \\
\label{eq:(3)Newton_iteration}
3)& \quad Y_i^{[k]} := Y_i^{[k-1]} + \Delta Y_i^{[k]}, \quad  i=1 ,\ldots, s,
\end{align}

\paragraph*{}Azpimarratu behar da iterazio bakoitzeko,  $J_i^{[k]}$ jakobiarraren $s$-ebaluazio eta $sd \times sd$ tamainako matrizearen LU-deskonposaketa kalkulatu behar ditugula. Eragiketa hauek konplexuak dira eta beraz, Newton osoaren inplementazioa konputazionalki garestia da. 

\subsection*{Newton sinplifikatuaren iterazioa.}

Newton sinplifikatuaren iterazioa aplikatzerakoan, $J_i^{[k]}$ jacobiarra $J_i^{[0]}=\partial f / \partial y \ (t+c_ih, Y_i^{[0]}) \ \ i=1,\cdots,s$ jacobiarraz ordezkatzen da eta askatu beharreko ekuazio sistema honakoa da,

\begin{equation*}
\Delta Y_{i}^{[k]}  - h \sum_{j=1}^{s}\, a_{ij}\, J_j^{[0]} \Delta Y_{j}^{[k]} = r_i^{[k]}, \quad  i=1 ,\ldots, s,
\end{equation*}
$\mbox{non} \quad  J_i^{[0]}=\frac{\partial f}{\partial y}(t + c_i h,Y_{i}^{[0]}) \quad \quad  i=1,\ldots, s$ den.

Lehen sinplifikazio honetan, integrazioaren urrats bakoitzeko,  $J_i^{[0]}$ jakobiarraren s-ebaluazio eta $sd \times sd$ tamainako matrizearen LU-deskonposaketa behin bakarrik kalkulatu behar ditugu. Modu baliokidean, ekuazio lineala notazio matriziala erabiliz laburtu daiteke,
\begin{equation*}
\label{eq:805}
\left (I_s \otimes I_d - h  
\begin{bmatrix}
a_{11}  J_1^{[0]} & \dots & a_{1s}  J_s^{[0]} \\
a_{21}  J_1^{[0]} & \dots & a_{2s}  J_s^{[0]} \\
\dots            & \ddots & \dots \\
a_{s1}  J_1^{[0]} & \dots & a_{ss}  J_s^{[0]} \\ 
\end{bmatrix} \right) \Delta Y^{[k]} =r^{[k]}.
\end{equation*}

non,
\begin{equation*}
\label{eq:806}
Y^{[k]}=\begin{bmatrix}
Y_1^{[k]} \\
\vdots \\
Y_s^{[k]}
\end{bmatrix} \in \mathbb{R}^{sd}, \ \ \
r^{[k]}=\begin{bmatrix}
r_1^{[k]} \\
\vdots \\
r_s^{[k]}
\end{bmatrix} \in \mathbb{R}^{sd},
\end{equation*}

\begin{equation*}
\label{eq:807}
J_{is}(y)=\left(\partial f^i/\partial y^j (y)\right)_{i,j}^d=
\begin{bmatrix}
    \frac{\partial f^1}{\partial y^1} & \cdots & \frac{\partial f^1}{\partial y^d}\\    
    \vdots & \ddots & \vdots \\    
    \frac{\partial f^d}{\partial y^1} & \cdots & \frac{\partial f^d}{\partial y^d}\\    
\end{bmatrix} \in \mathbb{R}^{d \times d},\ is=1,\cdots,s.
\end{equation*}

\subsection*{Newton super-sinplifikatuaren iterazioa.}

Bigarren sinplifikazioa bat aplika daiteke, $J_i^{[0]}=\partial f / \partial y \ (t+c_ih, Y_i^{[0]}), \ \  i=1,\dots,s$ matrizeak,  $J_i^{[0]} \approx J, \ i=1,\cdots,s$ hurbilpen bakarrekin ordezkatuz. Era honetako ekuazio sistema lortuko dugu,   

\begin{equation}
\label{eq:808}
(I_s \otimes I_d - h \ A \otimes J) \Delta Y^{[k]} = r^{[k]}.
\end{equation}
non $I_s,I_d$ identitate eta $A=(a_{ij})_{i,j}^s$ koefiziente matrizeak diren.

Ohikoa da $J=\partial f / \partial y \ (t+h/2, y_n)$ hurbilpena erabiltzea. Problema zurruna denean, atalen hasieraketa $Y_i^{[0]}=y_n$, ($i=1,\dots,s$)  erabili ohi da, eta jakobiarraren hurbilpen honekin ekuazio lineala askatzeak zentzua izango du.

Newton iterazio bertsio honi super-sinplifikatua deitu diogu. Iterazio bakoitzean $f$ funtzioaren $s$ ebaluazio eta $sd$ tamainako ekuazio-sistema lineala askatu behar da. $(I_s \otimes I_d - h \ A \otimes J)$ matrizea iterazio guztietarako berdina da,  bere LU-deskonposaketa behin bakarrik egin behar da eta konputazionalki garestia da \cite{Butcher1976} \cite{Hairer1996},
\begin{align*}
&\text{LU-deskonposaketa}, \ \ 2s^3d^3/3+\mathcal{O}(d^3), \\
&\text{back substitution}, \ \ 2s^2d^2+\mathcal{O}(d).
\end{align*}

Jarraian, Newton super-sinplifikatuaren inplementazioaren algoritmo orokorra laburtu dugu (Algoritmoa \ref{alg:nss}).

\subsection*{Algoritmoa.}

\begin{algorithm}[H]
 \BlankLine
  $\tilde{y}_0=fl(y_0)$\;
  $e_0=fl(y_0-\tilde{y}_0)$\;
  \For{$n\leftarrow 0$ \KwTo ($endstep-1$)}
  {
   \BlankLine
   $k=0$\;
   \text{Hasieratu} $Y_{n,i}^{[0]} \ \ , \ \ i=1,\dots,s $\;
   \BlankLine
   $J=\frac{\partial f}{\partial y}(t+h/2,y_n) $\; 
   $M=LU(I_s \otimes I_d - h \ A \otimes J)$\;
   \BlankLine
   \While{ (\text{not konbergentzia})}
   {
    \BlankLine 
    $k=k+1$\;
    $r_i^{[k]}=-Y_i^{[k-1]}+y+h \sum\limits_{j=1}^{s} a_{ij} f(t+c_jh,Y_j^{[k-1]}) $\;
    Solve $(M \Delta Y^{[k]}=r^{[k]})$\;
    $Y^{[k]}=Y^{[k-1]}+\Delta Y^{[k]}$\;
    $\text{konbergentzia} \leftarrow \text{GeratzeErizpidea}(Y^{[k]},Y^{[k-1]},\Delta_{min}) $\;
   }
   \BlankLine
   \If{($\exists j \ \text{non} \ \Delta_j^{[K]}\neq 0$)}
   {
    \If{$(NormalizeDistance(Y^{[k]},Y^{[k-1]})>1$}
     {$\text{fail convergence}$\;}
   }
   {$(\tilde y_{n+1},e_{n+1})\leftarrow \text{baturakonpensatua}(y_n,e_n,Y_n^{[k]})$\;}    
 }
 \caption{IRK (Newton super-sinplifikatua).}
 \label{alg:nss}
\end{algorithm}



\section{IRK-Newton eraginkorra.}
\label{sec:s74}

\subsection*{Sarrera.}

Atal honetan, honako ekuazio lineala modu eraginkorrean askatzeko inplementazioa proposatuko,
\begin{equation*}
(I_s \otimes I_d - h \ A \otimes J) \ \Delta Y = r,
\end{equation*}
$J \in \mathbb{R}^{d \times d}$  eta $r \in \mathbb{R}^{s \times d}$ emandako matrizeak izanik.

\paragraph*{}$S$-ataletako IRK metodoa, Newton iterazioaren bidez $d$-dimentsioko ODE sistemari aplikatzeko, urrats bakoitzean $sd \times sd$ tamaineko hainbat ekuazio sistema (iterazio bakoitzeko bat) askatu behar dira. Atal honetan, jatorrizko $sd$-dimentsioko ekuazio sistema, $(s+1)d$ sistema baliokide moduan berridatziko dugu eta sistema baliokidea,  $d \times d$ tamaineko $[s/2]+1$ matrize errealen \emph{LU}-deskonposaketa bidez askatuko dugu. Tamaina txikiko matrizeen LU-deskonposaketa azkarra denez, konputazionalki eraginkorragoa izatea espero dugu.      

\paragraph*{}Gauss nodoetan oinarritutako \emph{IRK} metodoak sinplektikoak eta simetrikoak dira. 
\begin{enumerate}
\item Sinplektikoa.
\begin{equation} 
\label{eq:sympl}
b_{i}a_{ij}+b_{j}a_{ji}-b_{i}b_{j}=0, \ \ 1 \leqslant i,j \leqslant s.
\end{equation} 
\item Simetrikoa.
\begin{align}
\label{eq:simm}
b_{s+1-i}=b_i, \ \ {c}_{s+1-i}=1-{c}_i,& \quad \ 1\leqslant i,j \leqslant s, \\
b_j={a}_{s+1-i,s+1-j}+a_{i,j},& \quad 1\leqslant i,j \leqslant s. 
\end{align} 
\end{enumerate}

Proposamen berria, bi propietate hauetan oinarrituz garatuko dugu.

\subsection*{Notazio alternatiboa.}

Koefiziente notazioa berri bat finkatuta,
\begin{equation*}
\bar{c}_i=c_i-\frac{1}{2}, \ \ \bar{a}_{ij}=a_{ij}-\frac{b_j}{2}, \ \ 1\leqslant i,j \leqslant s,
\end{equation*}
sinplektikotasun (\ref{eq:sympl}) eta simetrikotasun (\ref{eq:simm}) propietateak modu baliokidean era honetan adieraziko ditugu,
\begin{enumerate}
\item {Sinplektikoa.}
\begin{equation}
\label{eq:eqlineala}
(B \bar{A}) \ \ \mbox{antisimetrikoa da},
\end{equation}
non $\bar{A}=(\bar{a}_{ij})_{i,j=1}^s$ eta $B$, ($b_1,b_2,\dots,b_s$) balioen  matrize diagonala diren.

\item {Simetrikoa.}
\begin{align}
\label{eq:simm2}
b_{s+1-i}=b_i, \ \ \bar{c}_{s+1-i}=-\bar{c}_i,& \quad 1\leq i \leq s, \\
\bar{a}_{s+1-i,s+1-j}=-\bar{a}_{ij},& \quad 1\leq i,j \leq s. \\
\end{align} 

\end{enumerate}

Inplementazio berri honen matrizeen dimentsioak zehazteko, balio berri hauetan oinarrituko gara,
$m=[(s+1)/2]$, eta $s-m =[s/2]$. Beraz metodoaren $s$-atal kopurua bikoiti ala bakoiti den arabera,
\begin{itemize}
\item $s$ bikoitia.

Adibidea $s=6 \rightarrow m=3,s-m=3$.

\item $s$ bakoitia.

Adibidea $s=7 \rightarrow m=4, s-m=3$.
\end{itemize}


\subsubsection*{Sinplektikoa.}

$(B \bar{A})$ antisimetrikoa bada, $B^{1/2}\bar{A}B^{-1/2}$ antisimetrikoa da. Era berean, honek $\bar{A}$ diagonalizagarria dela eta irudikari balio propio puruak dituela suposatzen du. Eta beraz,  $Q$ ($s \times s$) tamaineko matrize ortogonala existitzen da,
\begin{align}
\label{eq:syb}
Q^{-1}\bar{A}Q=
\left(
\begin{matrix}
0 & D \\
-D^T & 0
\end{matrix}
\right)
\end{align}
non $D$, ($m \times (s-m)$) tamainako eta balio erreal positiboen matrize diagonala den. 

\subsubsection*{Simetrikoa.}

Sinplektikotasun eta simetri propietate hauetan oinarrituz jarraiko matrizeak definituko ditugu.

\begin{enumerate}
\item P matrizea.

Kontsideratu ($s \times s$) tamainako $P=(P_1 \ P_2)$ matrize ortogonala. $x=(x_1,\dots,x_s)^T \in \mathbb{R}^s$, $P_1^Tx=(y_1,\dots,y_m)^T$, eta $P_2^Tx=(y_{m+1},\dots,y_s)^T$ non,
\begin{align*}
&y_i = \frac{\sqrt{2}}{2} (x_{s+1-i}+x_i), \ \ i=1,\dots,[s/2], \\
&y_i =\frac{\sqrt{2}}{2} (x_{s+1-i}-x_{i}), \ \ i=m+1,\dots,s, \\
&y_{m} = x_{m}, \ \ s \ \ \mbox{bakoitia bada}.
\end{align*}  

Matrizearen dimentsioak laburtuz, $P=(P_1 \ P_2) \in \mathbb{R}^{s \times s}$, $P_1 \in \mathbb{R}^{s \times m}$ eta $P_2 \in \mathbb{R}^{s \times (s-m)}$.

\item K matrizea.

Batetik, simetri propietateak (\ref{eq:simm2}),  $P_i^TB^{\frac{1}{2}}\bar{A}B^{-\frac{1}{2}}P_i=0, \ i=1,2$  eta bestetik, propietate sinplektikoak $B^{1/2}\bar{A}B^{-1/2}$ antisimetrikoa dela ziurtatzen duenez, $\bar{A}$ matrizea honako matrizearen antzekoa dela ondorioztatu daiteke,
\begin{align}
P^TB^{\frac{1}{2}}\bar{A}B^{-\frac{1}{2}}P=
\left(
\begin{matrix}
0 & K \\
-K^T & 0
\end{matrix}
\right)
\end{align}
non $K=P_1^TB^{\frac{1}{2}}\bar{A}B^{-\frac{1}{2}}P_2 \ \in \mathbb{R}^{m \times (s-m)}$ den.

\item D matrizea.

$K=UDV^T$ balio singularraren deskonposaketa izanik, non $U \in \mathbb{R}^{m \times m}$, $V \in \mathbb{R}^{(s-m) \times (s-m)}$ matrize ortonormalak diren eta $D \in \mathbb{R}^{m \times (s-m)}$, $K$ matrizearen balio singularren ($\sigma_1, \dots, \sigma_{s-m}$) matrize diagonala da,
\begin{align}
\label{eq:Dmat}
D=
\left(
\begin{matrix}
\sigma_1 & 0 & \dots & 0 \\
0 & \sigma_2 & \dots & 0 \\
0 & 0 & \dots & 0 \\
0 & 0 & \dots & \sigma_{s-m}
\end{matrix}
\right), \ \ \
D=
\left(
\begin{matrix}
\sigma_1 & 0 & \dots & 0 \\
0 & \sigma_2 & \dots & 0 \\
0 & 0 & \dots & 0 \\
0 & 0 & \dots & \sigma_{s-m} \\
0 & 0 & \dots & 0
\end{matrix}
\right).
\end{align}
s bikoitia bada, $D$ matrizea ezkerrean eta s bakoitia bada, $D$ matrizea eskubian ($\sigma_m=0$) irudikatu dugu. 

\item Q matrizea.

Simetrikoa den (\ref{eq:syb}) propietateari esker,
\begin{align*}
&Q=(Q_1 \ Q_2)=
B^{-1/2}(P_1 \ P_2)
\left(
\begin{matrix}
U & 0 \\
0 & V
\end{matrix}
\right)=
B^{-1/2} (P_1U \ P_2V), \\
&Q^{-1}=Q^TB.
\end{align*}  

Matrizearen dimentsioak laburtuz, $Q=(Q_1 \ Q_2) \in \mathbb{R}^{s \times s}$, $Q_1 \in \mathbb{R}^{s \times m}$ eta $Q_2 \in \mathbb{R}^{s \times (s-m)}$.

\end{enumerate}

\subsection*{(s+1) ekuazio-sistema.}

Aldagai berri bat kontsideratuko dugu $z \in \mathbb{R}^d$ eta honako ekuazio sistema kontsideratuko dugu,
\begin{align*}
&Y_{n,i}=y_n+\frac{z}{2}+ h\ \sum^s_{j=1}{\bar{a}_{ij}\ f(t_n+c_jh,Y_{n,j})}\ \ \ \ \ i=1 ,\dots, s,\\
&z=h \sum_{i=1}^{s} {b_i f(t_n+c_jh,Y_{n,i})}.
\end{align*} 
Ekuazio berri hau, sistemari gehituz, $(s+1) \times d$ dimentsioko ekuazio sistema baliokidea dugu,
\begin{align}
\label{eq:s1s}
(I_s \otimes I_d- h \ \bar{A} \otimes J) \ \Delta Y - \frac{1}{2}(e_s \otimes I_d) \ \Delta z =r,\\
(-h e_s^T B \otimes J) \ \Delta Y+  \Delta z=0,
\end{align}
non $e_s^T=(1,\dots,1) \in \mathbb{R}^{1 \times s}$ den. Argi dagoenez, $(\Delta Y, \Delta z)$ ekuazio-sistema honen (\ref{eq:s1s}) soluzioa bada, $\Delta Y$ gure jatorrizko ekuazio sistemaren (\ref{eq:eqlineala}) soluzioa da.

\paragraph*{} Ekuazio-sistemaren adierazpen matriziala lagungarria izan daiteke,
\begin{equation*}
\begin{bmatrix}
    &      &      &  & \ \ -I_d/2 \\
    &      &      &  & \ \ -I_d/2 \\
    &      &      &  & \ \      \\    
    &  & I_s \otimes I_d- h \ \bar{A} \otimes J & & \ \ \vdots \\
    &      &      &  & \ \      \\
    &      &      &  & \ \ \ \ -I_d/2    \\
-hb_1 J & -hb_2 J & \dots & -hb_s J &  I_d\\ 
\end{bmatrix}
\begin{bmatrix}
\Delta Y_1 \\
\Delta Y_2 \\
\vdots \\
\Delta Y_s \\
\Delta z\\
\end{bmatrix}=
\begin{bmatrix}
r_1 \\
r_2 \\
\vdots \\
r_s \\
0\\
\end{bmatrix}
\end{equation*} 

\subsubsection*{IRK metodo orokorren garapena.}

Lehengo, garapena metodo sinplektikoentzat egingo dugu eta ondoren, metodoa simetrikoentzat. Honako aldagai aldaketa aplikatuz,
\begin{equation*}
 \Delta Y = (Q \otimes I_d) \ W,
\end{equation*}

eta metodoa simetrikoa izateagatik (\ref{eq:syb}) , ekuazio sistema baliokidea lortuko dugu,
\begin{multline*}
\left(
\begin{matrix}
I_m \otimes I_d & \ \ -h D \otimes J \\
hD^T \otimes J &  \ \ I_{s-m} \otimes I_d 
\end{matrix}
\right)
 W- \frac{1}{2}(Q^{-1} e_s \otimes I_d) \ \Delta z = (Q^{-1} \otimes I_d) \ r,
\end{multline*}
\begin{equation*}
- h \ (e_s^T \ B \ Q \otimes J) \ W + \Delta z =0.
\end{equation*}

Eranskinean (\ref{serans:A31}), ekuazio baliokideak lortzeko eman diren urratsen zehaztapenak eman ditugu. 


\subsubsection*{IRK metodo simetriko eta sinplektikoen garapena.}

Honako aldagai aldaketarekin,
\begin{equation*}
 \Delta Y = (Q \otimes I_d) W= (Q_1 \otimes I_d) W'+ (Q_2 \otimes I_d) W'',
\end{equation*}
non,
\begin{align*}
W=\left(
\begin{matrix}
W^{'} \\
W^{''} 
\end{matrix}
\right),\ \ W^{'} \in \mathbb{R}^{m \times d}, \ \ W^{''} \in \mathbb{R}^{(s-m) \times d},
\end{align*}
\begin{align*}
Q=(Q_1 \ \ Q_2), \ \  Q_1 \in \mathbb{R}^{s \times m}, \ \ Q_2 \in \mathbb{R}^{s \times (s-m)},
\end{align*}

eta metodoa simetrikoa nahiz sinplektikoa izateagatik $e_s^TBP_2=0, \ e_s^TBQ_2=e_s^TBP_2V=0$~berdintasunak aplikatuz, honako ekuazio-sistema baliokidea lortuko dugu,
\begin{align*}
 W^{'}-h \ (D \otimes J) \ W^{''} -\frac{1}{2}\ (Q_1^T B \ e_s \otimes I_d) \ \Delta z &= (Q_1^T B \otimes I_d) \ r,\\
 h \ (D^T \otimes J) \ W^{'}+W^{''} &= (Q_2^T B \otimes I_d) \ r,\\
 - h \ (e_s^T B \ Q_1 \otimes J) \ W^{'} + \Delta z &=0. 
\end{align*}

Eranskinean (\ref{serans:A32}), ekuazio baliokideak lortzeko eman diren urratsen zehaztapenak eman ditugu.

\paragraph*{Matrizearen egitura.} Ekuazio sistemaren adierazpen matrizialarekin ($s=6$), matrizearen egitura berezia ikus daiteke. Aldagai aldaketarekin lortutako ekuazio sistema baliokideak, blokeka diagonala da eta hau aprobetxatuz, Newton iterazioaren inplementazio eraginkorra lortuko dugu.    
\begin{equation*}
\begin{bmatrix}
 I_d        &             &                     & \multicolumn{1}{|c}{-h\sigma_1 J} &            &                  &\multicolumn{1}{|c}{-\frac{\alpha_1}{2} I_d}\\
            & I_d         &                     & \multicolumn{1}{|c}{}           & -h\sigma_2 J &                
 &\multicolumn{1}{|c}{-\frac{\alpha_2}{2} I_d}\\
            &             & I_d                 & \multicolumn{1}{|c}{}           &            & -h\sigma_3 J  
 &\multicolumn{1}{|c}{-\frac{\alpha_3}{2} I_d}\\\cline{1-7}    
h\sigma_1 J &             &                     & \multicolumn{1}{|c}{I_d}        &            &            
 &\multicolumn{1}{|c}{0}\\
            & h\sigma_2 J &                     & \multicolumn{1}{|c}{}           &  I_d       &             
 &\multicolumn{1}{|c}{0}\\
            &             &  h\sigma_3 J        & \multicolumn{1}{|c}{}           &            &  I_d        
 &\multicolumn{1}{|c}{0}\\\cline{1-7}
 -h\alpha_1 J       & -h\alpha_2 J              &  -h\alpha_3 J                    & \multicolumn{1}{|c}{$0$}        &  $0$       &  $0$         
 &\multicolumn{1}{|c}{ I_d}\\
\end{bmatrix}
\begin{bmatrix}
         \\
 W^{'} \\
    \\
\cline{1-1} \\
    \\
 W^{''}   \\
    \\
    \cline{1-1} \\
 \Delta z  \\
\end{bmatrix}=
\begin{bmatrix}
         \\
 R^{'} \\
    \\
\cline{1-1} \\
    \\
 R^{''}   \\
    \\
    \cline{1-1} \\
 0  \\
\end{bmatrix}
\end{equation*} 
non 
\begin{equation*}
\begin{bmatrix}
 R^{'}=(Q_1^{T}B^{1/2} \otimes I_d) \ r \\
 R^{''}=(Q_2^{T}B^{1/2} \otimes I_d) \ r
\end{bmatrix}, \ \
\left(
\begin{matrix}
\alpha_1 \\
\vdots \\
\alpha_m
\end{matrix}
\right)=Q_1^TB \ e_s.
\end{equation*}

\paragraph*{}Jarraian, ekuazio sistemaren ezezagunak ($\Delta z,W^{'},W^{''}$) askatzeko aplikatuko ditugun espresioak laburtuko ditugu.
\paragraph*{$W^{''}$ kalkulatzeko ekuazioak.}

Bigarren ekuaziotik $W^{''}$ askatu,
\begin{equation}
W^{''}= -h \ (D^T \otimes J) \ W^{'}+(Q_2^T B \otimes I_d) \ r.
\end{equation}

\paragraph*{$W^{'}$ kalkulatzeko ekuazioak.}

Eta lehen ekuazioan ordezkatuz,
\begin{align*}
(I_m \otimes I_d+ h^2DD^T \otimes J^2) \ W^{'}- \frac{1}{2}(Q_1^T B \ e_s \otimes I_d)\ \Delta z&=R, \\
- h \ (e_s^T B \ Q_1 \otimes J) \ W^{'} + \Delta z &=0,
\end{align*}
non $R=(Q_1^T B \otimes I_d) \ r + h \  ( D Q_2^T B \otimes J)\,  r \in \mathbb{R}^{md}.$

\paragraph*{}Goiko ekuazio sistema honako notazioaren arabera,  
\begin{equation*}
R=\begin{bmatrix}
R_1 \\
\vdots \\
R_m
\end{bmatrix}, \ \ \
W^{'}=\begin{bmatrix}
W_1 \\
\vdots \\
W_m
\end{bmatrix}, 
\ \ R_i,W_i \in \mathbb{R}^d, \ \ i=1,\dots,m  
\end{equation*}

era honetan berridatziko dugu,
\begin{align*}
(I_d+h^2\sigma_i^2J^2) \ W_i- \frac{\alpha_i}{2}\ \Delta z &=R_i, \ i=1,\dots,m,\\
-h \ J \sum\limits_{i=1}^{m} \alpha_i W_i+\Delta z &=0,
\end{align*}
non,
\begin{align*}
\left(
\begin{matrix}
\alpha_1 \\
\vdots \\
\alpha_m
\end{matrix}
\right)=Q_1^TB \ e_s,
\end{align*}
eta $\sigma_1 \ge \cdots \sigma_{s/2}, \ K$ matrizearen balio singularrak diren; $s$ bakoitia denean $\sigma_m=0$ dela gogoratu (\ref{eq:Dmat}). 

\paragraph*{$\Delta z$ kalkulatzeko ekuazioak.}

Lehen ekuaziotik,$W_i$ askatuz, 
\begin{equation*}
W_i=(I_d+h^2\sigma_i^2J^2)^{-1} (R_i+\frac{\alpha_i}{2} \ \Delta z)
\end{equation*}

eta bigarren ekuazioan ordezkatuz, $\Delta z \in \mathbb{R}^d$ askatzeko ekuazioak lortuko ditugu,
\begin{equation}
M\ \Delta z=h \ J\sum\limits_{i=1}^{m}\alpha_i (I_d+h^2\sigma_i^2J^2)^{-1}R_i,
\end{equation}
non
\begin{equation}M=I_d+ J \ \frac{h}{2}\ \sum\limits_{i=1}^{m} \alpha_i^2 (I_d+h^2 \sigma_i^2 J^2)^{-1} \in \mathbb{R}^{d \times d}.
\end{equation}


\section{IRK-Newton estandarra (formulazio berria).}

\subsection*{Sarrera.}

IRK-Newton inplementazioarentzat, IRK puntu-finkoaren inplementazioan erabilitako birformulazio (\ref{chap:IRK-PF}atala) berdina aplikatuko dugu. Newton iterazio metodoan ordea, $L_i$ ($i=1,\dots,s$) aldagai ezezagunatzat eta $Y_i$ ($i=1,\dots,s$) aldagai laguntzailea kontsideratzea \cite{Olsson2000} izango da egokiena,
\begin{align}
\label{eq:62}
&\ L_{n,i}=hb_if(Y_{n,i}), \ \ Y_{n,i}=y_n+ \sum\limits_{j=1}^{s} \mu_{ij} \ L_{n,j},  \ \ i=1,\dots,s,\\
&y_{n+1}=y_n+\sum\limits_{i=1}^{s} L_{n,i},
\end{align}
non  $\mu_{ij}=a_{ij}/{b_j}$, \ \  $1 \le i,j \le s$.

\subsection*{Newton sinplifikatuaren iterazioa.}

Newton metodo sinplifikatua, $k=1,2,\dots$  $L_i^{[k]}$ hurbilpenak kalkulatzeko,
\begin{enumerate}
\item 
$g_i^{[k]}=-L_i^{[k-1]}+h b_i f(t+c_ih,\ y_n+ \sum\limits_{j=1}^{s} \mu_{ij} L_{j}^{[k-1]}), \ \ i=1,\dots,s.$

\item Askatu $\Delta L_i^{[k]}$,

$\Delta L_i^{[k]} - h b_i \ J_i^{[0]} \sum_{j=1}^{s} \mu_{ij}  \ \Delta L_j^{[k]} = g_i^{[k]}  , \ \ i=1,\dots,s.$

\item $L_i^{[k]} = L_i^{[k-1]}+ \Delta L_i^{[k]}, \ \  i=1,\dots,s.$

\end{enumerate}
non $J_i^{[0]}=\frac{\partial f}{\partial y} (t+c_ih, Y_i^{[0]}) \ \  i=1,\dots,s$ den.

\paragraph*{}Modu baliokidean, ekuazio lineala notazio matriziala erabiliz laburtu daiteke,
\begin{equation*}
\left (I_s \otimes I_d - h  
\begin{bmatrix}
b_1 \mu_{11} J_1^{[0]} & \dots & b_1 \mu_{1s} J_1^{[0]} \\
b_2 \mu_{21} J_2^{[0]} & \dots & b_2 \mu_{2s} J_2^{[0]} \\
\dots          & \ddots & \dots \\
b_s \mu_{s1} J_s^{[0]} & \dots & b_s \mu_{ss} J_s^{[0]} \\ 
\end{bmatrix} \right) \Delta L^{[k]} =g^{[k]}.
\end{equation*}

non,
\begin{equation*}
L^{[k]}=\begin{bmatrix}
L_1^{[k]} \\
\vdots \\
L_s^{[k]}
\end{bmatrix} \in \mathbb{R}^{sd}, \ \ \
g^{[k]}=\begin{bmatrix}
g_1^{[k]} \\
\vdots \\
g_s^{[k]}
\end{bmatrix} \in \mathbb{R}^{sd},  
\end{equation*}

\begin{equation*}
\label{eq:907}
J_{is}(y)=\left(\partial f^i/\partial y^j (y)\right)_{i,j}^d=
\begin{bmatrix}
    \frac{\partial f^1}{\partial y^1} & \cdots & \frac{\partial f^1}{\partial y^d}\\    
    \vdots & \ddots & \vdots \\    
    \frac{\partial f^d}{\partial y^1} & \cdots & \frac{\partial f^d}{\partial y^d}\\    
\end{bmatrix} \in \mathbb{R}^{d \times d}, \ is=1,\cdots,s.
\end{equation*}

\subsection*{Newton super-sinplifikatuaren iterazioa.}

Honako bigarren sinplifikazioarekin, $J_i^{[0]}=\partial f / \partial y \ (t+c_ih, Y_i^{[0]}), \ \  i=1,\dots,s$ matrizeak,  $J_i^{[0]} \approx J, \ i=0,\cdots,s$ hurbilpenarekin ordezkatuz, ekuazio sistema lineal hau lortuko dugu,
\begin{equation}
(I_s \otimes I_d - h \ BAB^{-1} \otimes J) \ \Delta L = g. 
\end{equation}
non
\begin{equation*}
I_s,\ I_d \ \ \text{identitateak eta }B=
\begin{bmatrix}
   b_{1} & 0      & \dots & 0 \\
   0     & b_{2}  & \dots & 0 \\
    \vdots & \vdots & \ddots  & \vdots \\
   0     & 0      & \dots & b_{s}\\
\end{bmatrix}.     
\end{equation*}

\subsection*{Algoritmoa.}

Formulazio berrian Newton super-sinplifikatua deitu dugun algoritmoa laburtu dugu (Algoritmoa \ref{alg:nssli}).

\begin{algorithm}[h!]
 \BlankLine
  $\tilde{y}_0=fl(y_0)$\;
  $e_0=fl(y_0-\tilde{y}_0)$\;
  \For{$n\leftarrow 0$ \KwTo ($endstep-1$)}
  {
   \BlankLine
   $k=0$\;
   \text{Hasieratu}  $L_{n,i}^{[0]} \ \ , \ \ i=1,\dots,s $\;
   $Y_{n,i}^{[0]}=y_{n} + \ \big(e_n+\sum\limits_{j=1}^{s} \mu_{ij} L_{n,j}^{[0]}\big)  $\;
   \BlankLine
   $J=\frac{\partial f}{\partial y}(y_n) $\; 
   $M=LU(I_s \otimes I_d - h \ BAB^{-1} \otimes J)$\;
   \BlankLine
   \While{ (\text{not konbergentzia})}
   {
    \BlankLine 
    $k=k+1$\;
    $g_i^{[k]}=-L_i^{[k-1]}+h b_i f(t+c_ih,\ y_n+ \sum\limits_{j=1}^{s} \mu_{ij} L_{j}^{[k-1]})$\;
    Solve $(M \Delta L^{[k]}=g^{[k]})$\;
    $L^{[k]}=L^{[k-1]}+\Delta L^{[k]}$\;
    $Y_{n,i}^{[k]}=y_{n} + \ \big(e_n+\sum\limits_{j=1}^{s} \mu_{ij} L_{n,j}^{[k]}\big)  $\;
    $\text{konbergentzia} \leftarrow \text{GeratzeErizpidea}(L^{[k]},L^{[k-1]},\Delta_{min}) $\;
   }
   \BlankLine
   \If{($\exists j \ \text{non} \ \Delta_j^{[K]}\neq 0$)}
   {
    \If{$(NormalizeDistance(Y^{[k]},Y^{[k-1]})>1$}
    {$\text{fail convergence}$\;}
   }
   $(\tilde y_{n+1}, e_{n+1})\leftarrow \text{baturakonpensatua}(\tilde y_{n},e_{n},L_{n}^{[k]})$\;
 }
 \caption{IRK (Newton super-sinplifikatua).}
 \label{alg:nssli}
\end{algorithm}

\paragraph*{Interpolazio koefizienteak.} $L_{n,i}^{[0]}$ atalen hasieraketarentzat dagokien koefizienteak era honetan definituko ditugu. IRK puntu finkoaren inplementazioan finkatu genituen interpolazio koefizienteetatik abiatuta (\ref{eq: interpLi}) modu errezean definituko ditugu formulazio honi dagozkion interpolazio koefizienteak.
\begin{align}
\left \{ \begin{array}{c}
  Y_{n,i}^{[0]}=y_n+\sum_{j=1}^{s} \mu_{ij} L_{n,j}^{[0]} \\[.25cm]
  Y_{n,i}^{[0]}=y_n+\sum_{j=1}^{s} \nu_{ij} L_{n-1,j} \\
          \end{array} \right. 
\Rightarrow \ \ L_n^{{[0]}}=(Mu^{-1} Nu) L_{n-1}, \ \ (Mu^{-1} Nu)_{i,j}^{s}=\lambda_{ij}/a_{ij}.
\end{align}

\paragraph*{Geratze erizpidea.} Puntu-finkoan definitutako geratze erizpide berdina aplikatuko dugu baina $L_{n,i}, \ i=1,\cdots,s$ aldagei.
\begin{equation*}
\Delta^{[k]}=(L_1^{[k]}-L_1^{[k-1]},\dots,L_s^{[k]}-L_s^{[k-1]}) \in \mathbb{R}^{sd},
\end{equation*}

Iterazioak jarraitu, honako baldintza betetzen den artean,
\begin{multline}
\label{eq:not_stopping Li}
\exists j \in \{1,\ldots,s d\} \quad \mbox{non} \quad \\
%|\Delta_j^{[1]}| >\cdots > |\Delta_j^{[k-1]}|>0 \mbox{ and } \Delta_j^{[k]} \neq 0. 
\min \left(\{|\Delta_j^{[1]}|,\cdots ,|\Delta_j^{[k-1]}|\} \ /\{0\} \right)>|\Delta_j^{[k]}| \mbox{ and } \Delta_j^{[k]} \neq 0. 
\end{multline}

\paragraph*{Batura konpensatua.}

Batura konpesatua egiteko modua zehaztuko dugu. Lehenik $\Delta L_i$ gaiak gehituko ditugu,
\begin{algorithm}[h]
$\beta_{n}={e}_{n} + \sum\limits_{j=1}^{s}\Delta L_{n,j}^{[k]}$\;
\end{algorithm}

Bigarrenik, $y_{n+1}=y_{n}+ \sum_{i=1}^{s} L_{n,i}^{[k-1]} + \beta_{n}$ batuketa egiteko

\begin{algorithm}[H]
  \SetAlgoLined\DontPrintSemicolon
  \SetKwFunction{algo}{algo}\SetKwFunction{BaturaKonpensatua}{BaturaKonpensatua}
  \SetKwProg{myalg}{Algorithm}{}{}
  \SetKwProg{myproc}{Function}{}{}
  \myproc{\BaturaKonpensatua {$y_n$,\ $\beta_n$,\ $L_n^{[k-1]}$}}{
     \BlankLine
     $s_0=y_n$\;
     $ee=\beta_n$\;
     \For{$i\leftarrow 1$ \KwTo ($s$)}
      {
        $s_1=s_0$\;
        $\delta= L_{n,i}^{[k-1]} +ee$\;
        $s_0=s_1+\delta$\;
        $ee=(s_1 - s_0)+ \delta$\;   
      }
     $y_{n+1}=s_0$\;
     $e_{n+1}=ee$\;    
    \KwRet ($y_{n+1}$,$e_{n+1}$) \;}
  \caption{BaturaKonpensatua}
\end{algorithm} 


\section{IRK-Newton eraginkorra (formulazio berria).}


Formulazio berrian honakoa da, modu eraginkorrean askatu behar dugun ekuazio-lineala,
\begin{equation}
(I_s \otimes I_d - h \ BAB^{-1} \otimes J) \ \Delta L = g, 
\end{equation}
$J \in \mathbb{R}^{d \times d}$  eta $g \in \mathbb{R}^{s \times d}$ emandako matrizeak izanik. 


\subsection*{Formulazio estandarretik formulazio berrirako urratsa.}

Formulazio berriaren inplementazio eraginkorra, formulazio estandarrean emandako algoritmoaren (\ref{sec:s74} atala) moldatuz zehaztuko dugu.

\paragraph*{} Hauek izango dira bi formulazioen aldagaien arteko erlazioak,

\begin{enumerate}

\item Aldagai aldaketa.
\begin{align*}
\Delta L &=(B \otimes I_d) \ \Delta Y, \\
\Delta L &=(B Q \otimes I_d) \ W, \\
\Delta L &=(B Q_1 \otimes I_d) \ W^{'}+(B Q_2 \otimes I_d) \ W^{''}.
\end{align*}

\item $R$ matrizea.

$g \in \mathbb{R}^{sd} \ \  \rightarrow \ \ r=(B^{-1} \otimes I_d) g$.
\begin{align*}
R=&(Q_1^TB \otimes I_d) \ r + h \ (D Q_2^T B \otimes J) \ r ,\\
R=&(Q_1^T \otimes I_d) \ g + h \ (D Q_2^T \otimes J) \ g,  \\
R=& (Q_1^T) \ g  + (h D Q_2^T) \ g \ J^T.
\end{align*}

\item $W^{''}$ matrizea.
\begin{align*}
W^{''}&= -h \ (D^T \otimes J) \ W^{'}+ (Q_2^T B \otimes I_d) \ r, \\
W^{''}&= -h \ (D^T \otimes J) \ W^{'}+ (Q_2^T \otimes I_d) \ g, \\
\end{align*}


\end{enumerate}

\paragraph*{}Formulazio berrian IRK-Newton sinplifikatuaren inplementazioaren urratsak hauek dira,

\begin{enumerate}

\item $\Delta z$  askatu.
\begin{equation*}
M \ \Delta z=h \ J\sum\limits_{i=1}^{m}\alpha_i (I_d+h^2\sigma_i^2J^2)^{-1}R_i,
\end{equation*}
non $M=I_d+J \ \frac{h}{2}\sum\limits_{i=1}^{m} \alpha_i^2 (I_d+h^2 \sigma_i^2 J^2)^{-1} \in \mathbb{R}^{d \times d}$.

\item  $W^{'}$ askatu.
\begin{align*}
W^{'}_i=(I_d+h^2\sigma_i^2J^2)^{-1} (R_i+\frac{\alpha_i}{2} \ \Delta z) \in \mathbb{R}^d, \ i=1,\dots,m
\end{align*}

\item $W^{''}$.
\begin{align*}
W^{''} &= -h \ (D^T \otimes J) \ W^{'}+(Q_2^T B \otimes I_d) \ r, \\
W^{''} &= -h (D^T \otimes J) W^{'}+ (Q_2^T \otimes I_d) \ g, \\
W^{``} &= (-hD^T) \ W^{`} \ J^T+Q_2^T \ g,
\end{align*}


\item $\Delta L$.
\begin{align*}
\Delta L &=(B Q_1 \otimes I_d) W^{'}+(B Q_2 \otimes I_d) W^{''}, \\
\end{align*}

\end{enumerate}

\subsection*{AlgoritmoaV1.}

IRK-Newton gure inplementazioaren algoritmoa laburtuko dugu (Algoritmoa \ref{alg:IRK-Newton-Li}), 

\begin{algorithm}[h!]
 \BlankLine
  $\tilde{y}_0=fl(y_0)$\;
  $e_0=fl(y_0-\tilde{y}_0)$\;
  \For{$n\leftarrow 0$ \KwTo ($endstep-1$)}
  {
   \BlankLine
   $k=0$\;
   \text{Hasieratu}  $L_{n,i}^{[0]} \ \ , \ \ i=1,\dots,s $\;
   $Y_{n,i}^{[0]}=y_{n} + \ \big(e_n+\sum\limits_{j=1}^{s} \mu_{ij} L_{n,j}^{[0]}\big)  $\; 
   \BlankLine
   $J=\frac{\partial f}{\partial y}(y_n) $\; 
   \BlankLine
   $ M=I_d- \ J \ \frac{h}{2}\ \sum\limits_{i=1}^{m} \alpha_i^2 (I_d+h^2 \sigma_i^2 J^2)^{-1} $\;
   $ \text{LUM}=LU(M)$\;
   \BlankLine  
   \While{ (\text{not konbergentzia})}
   {
    \BlankLine 
    $k=k+1$\;
    \BlankLine
    $g= (hB F(L)-L )$\;
    \BlankLine
    $R=Q_1^T \ g  + (h D Q_2^T) \ g \ J^T $\;
    $d=h \ J \sum\limits_{i=1}^{m}\alpha_i (I_d+h^2\sigma_i^2J^2)^{-1}R_i$\;
    $\text{Solve}(M \Delta z = d)$\;
    \BlankLine 
    $W^{'}_i=(I_d+h^2\sigma_i^2J^2)^{-1} (R_i+\frac{\alpha_i}{2} \ \Delta z), \ i=1,\dots,m$\;
    \BlankLine
    $W^{``}= (-hD^T) \ W^{`} \ J^T+Q_2^T \ g$\;
    \BlankLine
    $\Delta L =B Q_1 \ W^{`}+B Q_2 \ W^{``} $\;
    $L^{[k]}=L^{[k-1]}+\Delta L^{[k]}$\;
    $Y_{n,i}^{[k]}=y_{n} + \ \big(e_n+\sum\limits_{j=1}^{s} \mu_{ij} L_{n,j}^{[k]}\big)  $\;  
    $\text{konbergentzia} \leftarrow \text{GeratzeErizpidea}(L^{[k]},L^{[k-1]},\Delta_{min}) $\;
   }
 \BlankLine
   \If{($\exists j \ \text{non} \ \Delta_j^{[K]}\neq 0$)}
   {
    \If{$(NormalizeDistance(Y^{[k]},Y^{[k-1]})>1$}
    {$\text{fail convergence}$\;}
   }
   $(\tilde y_{n+1}, e_{n+1})\leftarrow \text{baturakonpensatua}(\tilde y_{n},e_{n},L_{n}^{[k]})$\;
 }
 \caption{IRK (NSS-EraginkorraV1).}
 \label{alg:IRK-Newton-Li}
\end{algorithm}


\clearpage


\section{IRK Sasi-Newton.}


\subsection*{Sarrera.}

Doitasun laukoitzerako abantaila asko izango dituen proposamen berria egingo dugu. Doitasun laukoitzeko exekuzioetan, funtzio ebaluaztapena oso garestia da eta funtzio ebaluaztapen kopurua gutxitzea bilatuko dugu.

Newton osoa aplikatzen dugunean pena mereziko du (Newton osoak ez du atalen hasieraketa ona behar eta Newton sinplifikatuak bai).

Ideia nagusia hauxe da; Newton osoaren iterazioaren ekuazio sistema, modu iteratiboan askatzea,
\begin{align*}
&\Delta L_i^{[k]} - h b_i \ J_i^{[k]} \sum_{j=1}^{s} \mu_{ij}  \ \Delta L_j^{[k]} = g_i^{[k]}  , \ \ i=1,\dots,s,\\
&J_i^{[k]}=\frac{\partial f}{\partial y} (t+c_ih, Y_i^{[k]}), \ \  i=1,\dots,s,\\
\end{align*}

Notazio berria finkatzen badugu, hau da askatu beharreko ekuazio sistema,
\begin{equation*}
M^{[k]} \Delta L^{[k]}=g^{[k]},
\end{equation*}

eta $G()$  funtzioa era honetan definituz,
\begin{align*}
G(\Delta L):&=g-M \ \Delta L,
\end{align*}
askatu nahi dugun ekuazioa,
\begin{align*}
G(\Delta L) \ &=0.
\end{align*}

Aurreko atalean,  $M^{[k]} \approx \tilde{M}$  hurbilketa, modu eraginkorrean askatzen ikusi dugu, 
\begin{align*}
\tilde{M} \ =(I_s \otimes I_d - h \ BAB^{-1} \otimes J).
\end{align*}

Eta beraz,
\begin{algorithm}[h!]
 \BlankLine
  $\Delta L^{[0]}=0$\;
  \For{$l=0,1,2,\cdots$}
  {
    $\Delta L^{[l+1]} = \Delta L^{[l]}+ \tilde{M}^{-1} \ G^{[l]}$\;
  }
 \caption{.}
\end{algorithm}

\subsection*{Sasi-Newton.}


\begin{algorithm}[h!]
  $ \Delta L^{[k,1]} = \tilde{M}^{-1} \ g^{[k]}$\;
  \BlankLine
  \For{ (l=2,3,\dots)}
  {
   \BlankLine
   $G^{[k,l]} = g^{[k]}- M^{[k]} \Delta L^{[k,l-1]}$\;
   \BlankLine
   $\Delta L^{[k,l]}=\Delta L^{[k,l-1]}+ \tilde{M}^{-1} \ G^{[k,l]}$\;
  }
 \caption{Sasi-Newton Benetakoa.}
 \label{alg:901}
\end{algorithm}

Non
\begin{align*}
 G_i^{[k,l]} &=g_i^{[k]}- (\Delta L_i^{[k,l-1]}-hb_i J_i^{[k]} \sum_{j=1}^{s} \mu_{ij} \Delta L_{j}^{[k,l-1]}), \ \ i=1,\dots,s,\\
 G_i^{[k,l]} &=J_i^{[k]} \cdot (\sum_{j=1}^{s} hb_i\mu_{ij} \Delta L_{j}^{[k,l-1]})+(g_i^{[k]}-\Delta L_i^{[k,l-1]}), \ \ i=1,\dots,s.
\end{align*}

\paragraph*{} Proposamen berri honetan,
\begin{itemize}
\item Iterazioa=$1$: \emph{Newton Super-Sinplifikatua} (NSS-eraginkorra).
\item Iterazioa=$\infty$: \emph{Newton Benetako} (NB-eraginkorra).
\item Iterazioa=$2$ edo $3$: \emph{Sasi-Newton Benetakoa} (SNB-eraginkorra).
\end{itemize}

\section{Laburpena.}