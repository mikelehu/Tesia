\chapter{Koma higikorreko aritmetika.}
\label{sec:4}

\section{Sarrera.}

Konputagailuetan, zenbaki errealak ($\mathbb{R}$) bit kopuru finituaren bidez adierazi behar dira eta honetarako, koma-higikorreko adierazpen sistema ($\mathbb{F}$) erabiltzen da. Zenbaki erreal batzuk, $\mathbb{F}$ sisteman adierazpen zehatza dute, baina beste batzuk hurbildu egin behar dira.  Era berean, eragiketa aritmetikoen ($+,-,*,/$) kalkulu gehienetan ere, emaitzaren hurbilpena egin beha da. $\mathbb{R}$ sistematik $\mathbb{F}$ sistemara bihurtzeko funtzioari biribiltzea esaten zaio. Oro har, konputazio zientzian, biribiltze errore honen eragina garrantzitsua da eta errorea gutxitzeko ahalegin berezia beharrezkoa da.

Egungo konputagailuen koma-higikorreko aritmetikaren inplementazioak, \emph{IEEE-$754$} estandarrean \cite{IEEE2008} oinarritzen dira. 
\emph{IEEE-$754$} estandarrak, koma-higikorreko aritmetikaren konputaziorako formatu eta metodoak definitzen ditu. Konputazioen fidagarritasuna eta aplikazioen portabilitatea bermatzen ditu.    
 
Atal honetan, koma-higikorreko aritmetika eta biribiltze errorearen oinarria azalduko ditugu. Ondoren, konputazioetan biribiltze erroreak gutxitzeko teknika ezagun batzuk azalduko ditugu. 

\section{\emph{IEEE-754} estandarra.}

Koma-higikorreko zenbaki multzoa finitua da eta ${\mathbb{F}}$ izendatuko dugu. Koma-higikorreko adierazpen zehatza duten zenbaki errealei koma-higikorreko zenbakiak deritzogu, 
\begin{equation*}
\mathbb{F}\subset \mathbb{R}.
\end{equation*}

$\mathbb{F}$ zenbaki multzoa, \ref{fig:FloatNumberLine}irudian laburtu dugu. Bai zenbaki positiboentzat, bai negatiboentzat, adieraz daitekeen zenbaki handienaren eta txikienaren arteko balio bakanez osatuta dago. Multzoaren kanpoaldean zenbaki hauek guztiak ditugu: batetik overflow tartean $(-\infty,\max_{x \in \mathbb{F_{-}}}|x|)$  eta $(\max_{x \in \mathbb{F_{+}}}|x|,\infty+)$ daudenak; bestetik underflow tartean  $(\min_{x \in \mathbb{F_{-}}}|x|,0)$ eta $(0,\min_{x \in \mathbb{F_{+}}}|x|)$ daudenak. 

\begin{figure}[h]
\centerline{\includegraphics[width=14cm, height=3cm] {ZenbakiErrealak}}
\caption[Koma-higikorreko zenbakien multzoa]{Koma-higikorreko zenbakien multzoa}
\label{fig:FloatNumberLine}
\end{figure} 

IEEE-754 estandarraren arabera, $n$-biteko koma-higikorreko adierazpenak bi zati ditu (ikus \ref{fig:32bitKomaHigikorra} irudiko adibidea),
\begin{enumerate}
\item $m$ bitez osatutako zatia, mantisa ($M$) izenekoa. Horietako bit batek ($S$) zeinua adierazten du. Bestalde $M$ mantisa modu normalizatu honetan emana da, $\pm 1.F$ eta zati erreala ($F$) bakarrik gorde behar da.   
\item Esponentea ($E$), $(n-m)$ bitez adierazitako zenbaki osoa. Zeinuarentzat ez da bit zehatzik, baizik \emph{bias} izeneko balio bat kenduz adierazten dira zenbaki positiboak eta negatiboak.  
\end{enumerate}

Beraz, oinarri bitarrean koma-higikorreko zenbaki hauek adierazten dira,
\begin{equation*}
M \times b^E, \ b=2,
\end{equation*}
eta biribiltze unitatea (\emph{unit roundoff}) era honetan definituko dugu,
\begin{equation*}
u=2^{-m}.
\end{equation*} 

\begin{figure}[h]
\centerline{\includegraphics[width=12cm, height=2cm] {ZenbakiErrealak2}}
\caption[32-biteko koma-higikorra]{\small $32$-biteko koma-higikorreko zenbakiaren adierazpena: esponentearentzat  8-bit eta mantisarentzat  $24$-bit (bit bat zeinuarentzat eta beste $23$ bit, $1.F$ eran normalizatutako mantisarentzat) banatuta.}
\label{fig:32bitKomaHigikorra}
\end{figure} 

IEEE-$754$ estandarrean, oinarri bitarreko koma-higikorreko hiru formatu definitzen dira: bata doitasun arrunta (\emph{single precision}), bestea doitasun bikoitza (\emph{double precision}) eta hirugarrena doitasun laukoitza (\emph{quadruple precision}) izenekoak (\ref{tab:koma-higikorreko-aritmetikak} Taula).

\begin{table} [h!]
\caption[IEEE-754 koma-higikorreko formatuak]{IEEE-754 koma-higikorreko formatuak}
\label{tab:koma-higikorreko-aritmetikak}       % Give a unique label
\centering
\begin{tabular}{ l c c c l c} 
 \hline
 Formatoa      &  Tamaina    & Mantisa   & Esponentea  & Tartea           &  $u=2^{-m}$          \\
               &    n        & m         & n-m         &                  &                      \\
   \hline
% Half     & 16 bit      & 11  & 5  & $2^{\pm 16}$     &  $5 \times 10^{-4}$   \\ 
 Arrunta   & 32 bit      & 24  & 8  & $10^{\pm 38}$    &  $6 \times 10^{-8}$   \\	    
 Bikoitza  & 64 bit      & 53  & 11 & $10^{\pm 308}$   &  $1 \times 10^{-16}$   \\
 Laukoitza & 128 bit     & 113 & 15 & $10^{\pm 11356}$ &  $1 \times 10^{-35}$   \\
\hline
\end{tabular}
\end{table}


Doitasun bikoitzeko oinarrizko eragiketak (batuketa, kenketa, biderketa, zatiketa, eta erro karratua) hardware bidez exekutatzen dira \cite{Muller2009} eta azkarrak dira. Makina ziklo bakoitzeko, $2$ eta $4$ batuketa, kenketa edo biderketa egin ohi dira; zatiketa eta erro karratua aldiz, eragiketa motelagoak dira. Bestalde, doitasun arruntaren  aritmetika, doitasun bikoitza baino azkarragoa da: garraiatu behar den bit kopuru erdia delako eta gainera, hardware bereziei esker (adibidez \emph{Intel} makinetan \emph{SSE} moduluak), eragiketa aritmetikoak azkarragoak direlako. 2008. urtean, IEEE-$754$ estandarrak, $128$-biteko koma-higikorreko aritmetika onartu zuen, baina  inplementazioa  softwarez bidezkoa da eta exekuzioa, gutxi gorabehera, doitasun bikoitzeko aritmetika baino 10-15 aldiz motelagoa da.

Problema batzuk, doitasun bikoitza baino doitasun handiagoa behar dute \cite{Joldes2016}. Doitasun laukoitza edo altuagoa, software liburutegien bidez emulatu ohi dira. Doitasun altuko zenbakiak adierazteko nagusiki bi modu bereizten dira:   

\begin{enumerate}
\item \emph{Digitu-anitzeko adierazpena}. Zenbakiak esponente bakarra eta mantisa bat baino gehiagorekin adierazten dira (adb. \emph{GNU MPFR liburutegia} \cite{Fousse2007}).
\item \emph{Termino-anitzeko adierazpena}. Zenbakiak  ebaluatu gabeko hainbat koma-higikorreko makina zenbaki estandarren batura gisa adierazten dira (adb. Bailey QD liburutegia) \cite{Hida2001} eta exekuzioaren ikuspegitik, hardware bidezko inplementazioaren abantaila dute.    
\end{enumerate}

Doitasun laukoitzeko gure esperimentuetarako, \emph{GCC libquadmath} liburutegia \cite{libquad} erabili dugu. Doitasun laukoitzean exekutatutako integrazioen zenbakizko soluzioak, soluzio zehatzak kontsideratu ditugu eta  doitasun bikoitzeko inplementazioaren errorea, soluzio zehatzarekiko diferentzia gisa kalkulatu dugu. 

Laskar-ek epe luzeko eguzki-sistemaren simulazioaren ($-250$ eta $+250$ milioitako integrazio tartea) konputaziorako kalkuluak \cite{Laskar2011}, kontu handiz eta doitasun handian egin behar ditu. Dena den, era honetako problemak salbuespenak dira eta ez da ohikoa izaten doitasun handian lan egin beharra. Egia da ere, neurri fisiko oso gutxi ezagutzen direla  hain doitasun handian (adibidez $50$-bitekin, Lurra eta Ilargiaren arteko distantzia, milimetroko errorearekin adieraz daiteke).  


\section{Biribiltze errorea.}

Zenbakizko integrazioen errorea, trunkatze eta biribiltze errorez osatuta dago. Urrats luzera nahi bezain txikia aukeratuz, trunkatze errorea biribiltze errorea baino txikiago izango da eta beraz, zenbakizko integrazio hauetan errorean biribiltze errorea nagusitzen da. Epe luzeko eta doitasun handiko integrazioetan, urrats luzera txikia erabiltzen denez, biribiltze errorea gutxitzea funtsezkoa izango da.     

Bi biribiltze errore mota bereiziko ditugu, bata adierazpenaren errorea eta bestea, aritmetikaren errorea.  

\subsection*{Adierazpenaren errorea.} 

Zenbaki erreal batzuk, $\mathbb{F}$ koma-higikorreko multzoan zehazki adieraz daitezke eta beste batzuk ordea, hurbilpen batez adierazi behar dira. $x \in \mathbb{R}$ izanik, $fl: \mathbb{R} \rightarrow \mathbb{F}$ koma-higikorreko zenbakia esleitzen dion funtzioari deituko diogu:  $x \in \mathbb{R}$ balioaren gertuen dagoen  $fl(x) \in \mathbb{F}$ itzultzen duen funtzioa bezala definitzen da. Hau da, $f_1,f_2 \in \mathbb{F}$ jarraian dauden koma-higikorreko zenbakiak  badira eta $x \in \mathbb{R}, \ f_1\leqslant x \leqslant f_2$ bada,
\begin{equation*}
fl(x)=
\left\{
        \begin{array}{lc}
        f_1 & \mathrm{if} \ |x-f_1| < |x-f_2| \\
        f_2 & \mathrm{if} \ |x-f_1| \geqslant |x-f_2| 
        \end{array}.
\right.
\end{equation*}  

Jarraian, koma-higikorreko adierazpenaren errore absolutua eta errore erlatiboa finkatuko ditugu.
\begin{itemize}
\item Errore absolutua,
\begin{equation*}
\triangle x= fl(x)-x= \tilde{x}-x. 
\end{equation*} 
\item Errore erlatiboa, 
\begin{equation*}
\delta x =\frac{\triangle x}{x} = \frac{\tilde{x}-x}{x}. 
\end{equation*}
\item Aurreko bi definizioen ondorioz honako formula erabilgarria dugu,
\begin{equation*}
\tilde{x}= x+\triangle x = x \ (1+\delta x).
\end{equation*}
\end{itemize}

Koma-higikorreko zenbaki sistema bitarrean ($m=$ mantisa adierazteko bit kopurua izanik) $|x|$ balioa, $\mathbb{F}$ multzoaren zenbaki txikienaren eta handienaren artean badago,
\begin{equation*}
 |\delta x|< u \ \ \text{non} \ \ u=2^{-m},
 \end{equation*}
bermatuta dagoela froga daiteke \cite{Corless2013}.

\subsection*{Aritmetikaren errorea.} 

Koma-higikorreko zenbakien arteko eragiketa baten emaitzak, ez du zertan $\mathbb{F}$ multzoan adierazpen zehatza izan  eta orduan, emaitza biribildu egingo  da. Adibidez, $m$ digituzko bi zenbakien biderketaren emaitza zehatza adierazteko, $2m$ digituzko mantisa behar dugu ($m$ digituzko galera) \cite{Fukushima2001}. Salbuespena, biderkagaietako bat $2$-ren berretura denean gertatzen da, orduan biderketa zehatza baita.

\paragraph*{Adibidea.} Demagun lau digitu hamartar errealeko aritmetikarekin ari garela lanean.

Emaitza zehatza, $1,343 \times 2,103 = 2,824229$. 

Hiru digitu hamartar errealeko aritmetika, $1,343 \times 2,103 \approx 2.824$.

\paragraph*{} Hauek zenbaki errealen arteko funtsezko eragiketak badira,  $\ast: \mathbb{R}^2\rightarrow \mathbb{R}$, 
\begin{equation*}
\ast\in \{+,-,\times,/ \},
\end{equation*}
koma-higikorreko zenbakien arteko funtsezko eragiketak era honetan izendatuko ditugu  $\circledast: \mathbb{F}^2\rightarrow \mathbb{F}$,
\begin{equation*}
\circledast\in \{\oplus,\ominus,\otimes,\oslash \}.
\end{equation*}

$\tilde x,\tilde y \in \mathbb{F}$ emanik eta $z= \tilde x \ast \tilde y$ emaitza zehatza bada, $\tilde z= \tilde x \circledast \tilde y$ (edo $\tilde z= fl(\tilde x \ast \tilde y$)) eragiketaren emaitzaren errore absolutua eta errore erlatiboa definituko ditugu,

\begin{itemize}
\item Errore absolutua,
\begin{equation*}
\triangle z=\tilde z-z =(\tilde x \circledast \tilde y) -(\tilde x \ast \tilde y).
\end{equation*} 
\item Errore erlatiboa,
\begin{equation*}
\delta z=\frac{\triangle z}{z}==\frac{(\tilde x \circledast \tilde y) -(\tilde x \ast \tilde y)}{(\tilde x \ast \tilde y)}.
\end{equation*} 
\item Honako erlazio hau ondorioztatu daiteke,
\begin{equation*}
\tilde z=(\tilde x \circledast \tilde y)=z+\triangle z=z \ (1+\delta z).  
\end{equation*}
\end{itemize}

Koma-higikorreko aritmetikan, \ $|\delta z|<u$ \ , non  $u=2^{-m}$, beteko dela froga daiteke \cite{Corless2013}.

\paragraph*{} Zenbakizko algoritmoen biribiltze errorearen eraginaren azterketa formalak, propietate hauetan oinarritzen dira. Bestalde, errore erlatiboak emaitzaren digitu zuzenak neurtzen du:
\begin{equation*}
\delta z \approx 10^{-k} \Rightarrow \ \approx \ k \ \mbox{digitu hamartar zuzen}.
\end{equation*}  


\subsection*{Biribiltze errorearen hedapena.}


Ohiko konputazioetan, eragiketa aritmetiko kopuru handia egin behar dugu emaitza lortzeko. Batzuetan, eragiketen biribiltze erroreak elkar ezereztatzen dira baina kasu txarrenean, biribiltze errorea metatu eta magnitude handikoa izan daiteke.   

\paragraph*{Adibidea.} 
Modu honetako batura batean , non $n>2$ eta $\tilde x_1,\dots,\tilde x_n \in \mathbb{F}$,  
\begin{equation*}
\bigoplus_{i=1}^{n}(\tilde x_i)=(\sum\limits_{i=1}^{n} \tilde x_i)(1+\delta),
\end{equation*}
$|\delta|<u \ \text{non} \  u=2^{-m}$ beteko denik, ezin daiteke bermatu. 

\paragraph*{}Analisi zehatza egiten badugu $n=3$ adibiderako, honako espresioa lortzen dugu,
\begin{equation*}
((\tilde x_1 \oplus \tilde x_2) \oplus \tilde x_3)  = 
  \big((\tilde x_1 + \tilde x_2)(1+\delta_1)
  +\tilde x_3 \big) (1+\delta_2), \ \ \delta_1,\delta_2<u.
\end{equation*}

\subsection*{Ezabapen arazoa.}

Algoritmoen kalkuluetan, doitasun galera azkarra gerta daiteke. Horren adibidea ezabapen arazoa dugu: oso antzekoak diren bi zenbakiren arteko kendura egiten dugunean gerta daitekeena. 

\paragraph*{Adibidea.} Mathematican kalkulatutako adibide honetan, ezabapen errorea nola gertatzen den erakutsi dugu. 
\begin{lstlisting} [language=Mathematica]
>>  InputForm[N[Pi]]
>> 3.141592653589793

>> y=N[Pi]*10^(-10);
>> InputForm[y]
>> 3.1415926535897934*10^(-10)

>> z=1.+y;
>> InputForm[z]
>> 1.0000000003141594           # 16-digitu hamartar zuzenak.

>> InputForm[z-1.]
>> 3.141593651889707*10^(-10)   # 6-digitu hamartar zuzenak.

\end{lstlisting}


\section{Biribiltze errorea gutxitzeko teknikak.}
\label{sec:4.4}

Batuketa eta biderketa eragiketen biribiltze errorea kalkulatzeko algoritmoak ezagunak dira \cite{Dekker1971,Higham2002}. Algoritmo hauek, \emph{termino-gaitzeko adierazpenetan} oinarritzen dira eta baturaren kasuan, batura konpentsatu izeneko algoritmoaren oinarria da. Ikusiko dugun bezala, algoritmo sinpleak dira eta konputazio kostu txikia dute.  

Teknika hauek, zenbakizko integrazioaren inplementazioaren kalkulu "kritikoetan" erabiliko ditugu, soluzioaren doitasuna handitzeko asmoarekin.

\subsection*{Batura: Fast2Sum.}

\emph{Fast2Sum} algorithmoa, 1971.ean Dekker-ek  asmatu zuen \cite{Dekker1971}. Koma-higikorreko $\tilde x,\tilde y \in \mathbb{F} \ \text{non} \ |\tilde x| \geq |\tilde y| \ \text{bi zenbakien}$ arteko $\tilde z= \tilde x \oplus \tilde y$ batuketari dagokion $e$ biribiltze errorea  era honetan kalkulatu daiteke,
%\ \text{non} \ \tilde z+ e=\tilde x+\tilde{y}$ den

\begin{algorithm}[H]
 \BlankLine
 {$\tilde{z}=\tilde{x} \oplus\tilde{y}$\;
  $e=\tilde{y} \ominus (\tilde{z}\ominus\tilde{x})$\;
 }
 \BlankLine
 \caption{Fast2Sum.}
 \label{alg:FastSum}
\end{algorithm}

\ref{fig:fast2sum}irudiaren laguntzarekin hobeto uler daiteke batuketaren biribiltze errorearen kalkulua \cite{Higham2002}.

\begin{figure}[h!]
\centerline{\includegraphics[width=14cm, height=8cm] {Fast2Sum}}
\caption[Batuketaren biribiltze errorea]{Batuketaren biribiltze errorea}
\label{fig:fast2sum}
\end{figure} 

\subsubsection*{Batura konpensatua.}

Era honetako batugai askoren arteko batuketan,
\begin{equation*}
z_{n+1}= z_0+\sum\limits_{i=0}^{n} x_i,
\end{equation*}
biribiltze errorea gutxitzeko teknika ezaguna da \cite{Higham2002,Muller2009,Hairer2006}.
Ideia da, bi zenbakien baturan egindako biribiltze errorea lortu, eta errore hau hurrengo baturan erabiltzea. Jarraian azaltzen den moduan, urrats bakoitzaren amaieran $e_{i}$ errore estimazioa  kalkulatuko dugu eta hurrengo urratsean, batugaiari gehituko diogu.

\begin{algorithm}[H]
 \BlankLine
  $\tilde z_0= z_0; \ e_0=0$\;
  \For{$i\leftarrow 0$ \KwTo $n$}
  {
   \BlankLine
    $x=\tilde z_i$\;
    $y= x_i+e_i$\;
    $\tilde z_{i+1}=x+y$\;
    $e_{i+1}=(x-z)+y$\;
   \BlankLine
  }
 \caption{Kahan-en batura konpentsatua.}
   \label{alg:KahanBK}
\end{algorithm}

Knuth-ek eta Kahan-ek \cite{Muller2009} frogatu zuten,  batura konpentsatuko algoritmoaren bidez kalkulatutako $z_{n+1}$ baturak honakoa betetzen duela:
\begin{equation*}
\left | z_{n+1} - (z_0+\sum_{i=0}^{n} x_i) \right | \leq (2u+ \mathcal{O}(nu^2)) \left(|z_0|+\sum_{i=1}^{n} |x_0|\right).
\end{equation*}

Jakina da, batugaiak bektoreak diren kasurako, hau da, $\tilde z_0, e_0, x_0, x_1, \dots, x_n \in \mathbb{F}^d$, ~algoritmoa orokor daitekeela. Beraz, \ref{alg:KahanBK} algoritmoa $n$ eta $d$ parametroak dituen funtzio familia gisa interpreta daiteke,
\begin{equation}
\label{eq:batsd}
S_{n,d} : \mathbb{F}^{(n+3)d} \rightarrow \mathbb{F}^{2d},
\end{equation}
zeinek $\tilde z_0, e_0, x_0, x_1, \dots, x_n \in \mathbb{F}^d$ argumentuak emanik, $\tilde z_{n+1}, e_{n+1} \in \mathbb{F}^d$ balioak itzultzen dituen, eta ($\tilde z_{n+1}+e_{n+1}) \approx \tilde (z_0+e_0+x_0+x_1+ \dots+x_n$) hurbilketa den.

\subsubsection*{Zenbakizko integrazioak.}
 
Zenbakizko integrazioetan, $n=1,2,\dots$ balioentzat era honetako baturak kalkulatu behar ditugu \cite{Hairer2006},
\begin{equation*}
y_{n+1}=y_n+\delta_n,
\end{equation*}  
non $|\delta_n|<|y_n|$ izan ohi den. Beraz, integrazioaren batura honen birbiltze errorea gutxitzeko, batura konpentsatua erabiliko dugu.  

$y_{n+1} \in \mathbb{R}^{d},\quad y_{n+1}=\tilde y_{n}+\tilde \delta_n$ batura zehatza izanik eta $\tilde y_{n+1} \in \mathbb{F}^{d}, \quad \tilde y_{n+1}=\tilde y_{n} \oplus \tilde \delta_n$ koma-higikorreko hurbilpena izanik, batura konpentsatuaren bidez lortutako errorearen estimazioa $e_{n+1}$, \ref{alg:batkp}~algoritmoa jarraituz lor daiteke eta baturan egindako biribiltze errore zehatza da, 
\begin{equation}
y_{n+1}=\tilde {y}_{n+1}+e_{n+1}. 
\end{equation}

\begin{algorithm}[H]
 \BlankLine
  $\tilde{y}_{0}=fl(y_{0}); \ e_0=fl(y_0-\tilde{y}_0)$\;
 \BlankLine
  \For{$n=0,1,2,\dots \quad$}
  {
   \BlankLine
    $inc=\tilde {\delta}_n \oplus e_n$\;
    $\tilde {y}_{n+1}=\tilde{y}_n \oplus inc$\;
    $e_{n+1}=(\tilde{y}_n \ominus \tilde {y}_{n+1}) \oplus inc$\;
   \BlankLine
  }
 \caption{Batura konpentsatua (zenbakizko integrazioa).}
 \label{alg:batkp}
\end{algorithm}


Goian aipatutako ideia,  beste ikuspegi batetik ere uler daiteke. Zenbakizko soluzioa, doitasun bikoitzeko bi balioen batura gisa $y_n=\tilde{y}_n+e_n$ (ia doitasun laukoitza), adierazten ari gara  eta beraz, interpretazio honen arabera, konputazio eragiketa batzuk ia doitasun laukoitzean egiten ariko ginateke. Zentzu honetan gure inplementazioan, hasierako balio zehatza $y_0=y(t_0)$, bi balioen batura gisa $y_0=\tilde{y}_0+e_0$ ulertu behar da eta era honetan hasieratuko dugu,
\begin{align*}
\tilde{y}_0 &=fl(y_0) ,\\
e_0 &=fl(y_0-\tilde{y}_0).
\end{align*}

\subsection*{Bidekerta: 2MultFMA.}

\emph{IEEE 754-2008} estandarrean, \emph{FMA} \cite{Muller2009} (\emph{fused multiply-add}) instrukzioa gehitu zen eta hurrengo urteetan, ordenagailu arruntetan zabaltzea espero da. Instrukzio honen garrantzia handia da: orokorrean konputazioak azkartzen ditu eta biderketa eskalarren, matrize biderkaduren eta polinomio ebaluazioen biribiltze errorea txikitzen du. \emph{FMA} instrukzioa, zatiketa eta erro karratuaren algoritmo azkarren diseinuan ere erabiltzen da.

\emph{FMA} instrukzioak, era honetako konputazioetan biribiltze errore bakarra bermatzen du,
\begin{equation*}
fl(\tilde x \times \tilde y \pm \tilde z)= (\tilde x \times \tilde y\pm \tilde z) (1+\delta), \ \delta<u \ \ \text{non} \ \ u=2^{-m}.
\end{equation*}
 

\emph{FMA}  instrukzioa erabilgarri dagoenean, biderketaren biribiltze errorea kalkulatzea erraza da; $\tilde x,\tilde y \in \mathbb{F}$ bi zenbakien arteko biderketari $\tilde z= fl(\tilde x \times \tilde y)$ dagokion biribiltze errorea $e, \ \text{non} \  \tilde{z}+ e=\tilde x \times \tilde y$ den, era honetan kalkulatu daiteke,

\begin{algorithm}[H]
 \BlankLine
 {$\tilde{z}=fl(\tilde{x}\times\tilde{y})$\;
  $e=fl(\tilde{x}\times\tilde{y}- \tilde{z})$\;
 }
 \BlankLine
 \caption{2MultFMA.}
 \label{alg:2MultFMA}
\end{algorithm}

\subsection*{Sterbenz Teorema.}
Sterbenz teoremaren arabera \cite{Sterbenz1973}, bi zenbaki elkarrekiko  gertu daudenean, honako baldintza betetzen bada, horien arteko kendura zehatza da.
\begin{equation}
\label{eq:4311}
x,y \in \mathbb{F}, \ \ \frac{y}{2}\leq x \leq 2y \ \ \ \Rightarrow \ \ \ x-y\in \mathbb{F}.
\end{equation}


\section{Laburpena.}

Atal honetan, koma-higikorreko aritmetikaren deskribapena egin ondoren, konputazioen doitasuna handitzeko tresnak azaldu ditugu. Tresna hauek, konputazio kostu txikia dute eta zenbakizko integrazioetan, biribiltze errorea txikitzeko aplikatuko ditugu.  

Koma-higikorreko aritmetikan sakontzeko honako bibliografia azpimarratuko dugu: \cite{Overton2001,Muller2009,Higham2002,Corless2013}.

