\chapter{Nire-Argibideak}.

Niretzako dokumentatutako argibide batzuk. Ez ditut tesian sartu behar.

\section{IRK-Newton.}

\subsection*{Newton sinplifikatuaren iterazioa.}
Newton sinplifikatuaren iterazioan formulazio berrian ekuazio honen jatorria:

\begin{align*}
\mbox{Askatu} \ \ & \triangle L_i^{[k]} \\
\ \ &\triangle L_i^{[k]} - h b_i \ J_i \sum_{j=1}^{s} \mu_{ij}  \ \triangle L_j^{[k]} = g_i^{[k]}  , \ \ i=1,\dots,s.
\end{align*}

\subsubsection*{Frogapena.}

Abiapuntua.
\begin{align*}
& \triangle L_i=L_i-L_i^{[k]},\\
& L_i-h b_i f(y_n+\sum_{j=1}^{s} \mu_{ij}L_j)=0
\end{align*}

Honako garapena egingo dugu.
\begin{align*}
L_i-h b_i f(y_n+\sum_{j=1}^{s} \mu_{ij}L_j)= L_i^{[k]}+\triangle L_i-h b_i f(y_n+\sum_{j=1}^{s} \mu_{ij} (L_j^{[k]}+\triangle L_j)).
\end{align*}
Eta $f(y_n+\sum_{j=1}^{s} \mu_{ij} (L_j^{[k]}+\triangle L_j))$ linealizatuz,
\begin{align*}
& \approxeq L_i^{[k]}+\triangle L_i-h b_i f(y_n+\sum_{j=1}^{s} \mu_{ij}L_j^{[k]})-h b_i f'(y_n+\sum_{j=1}^{s} \mu_{ij} L_j^{[k]}) (\sum_{j=1}^{s} \mu_{ij} \triangle L_j))= \\
& \triangle L_i- hb_i J_i \ (\sum_{j=1}^{s} \mu_{ij} \triangle L_j))= -L_i^{[k]}+h b_i f(y_n+\sum_{j=1}^{s} \mu_{ij}L_j)
\end{align*}
non $J_i=f'(y_n+\sum_{j=1}^{s} \mu_{ij} L_j^{[k]})=f'(Y_i)$.

\section{FMA.}

Nola jakin gure \emph{linux} konputagailu batek \emph{FMA} instrukzioak dituen ala ez ? Honako agindua exekutatu eta \emph{fma} flaga azaltzen den ala ez begiratu behar dugu.
\begin{lstlisting}
$ grep fma < /proc/cpuinfo
\end{lstlisting}
eta konpilatzeko \emph{-mfma} flag-a zehaztu behar da,
\begin{lstlisting}
$ gcc -O2 -Wall -std=c99 -fno-common -mfma adibidea.c
\end{lstlisting}

\section{Hitz-zerrenda.}

\begin{table}
\caption{Hitz-zerrenda.}
\label{tab:21}       % Give a unique label
\begin{tabular}{ l l c c c } 
 \hline
 Euskaraz                                &  Ingelesez                           & Laburdura    & Adibidea \\
 \hline
 Ekuazio diferentzial arrunta            &  Ordinary Differential equation      & \emph{ODE}   & $\dot{y}=f(t,y)$ \\
 Hasierako baliodun problema             &  Initial value problem               & \emph{IVP}   &                  \\
 Zurruna								 &  Stiff                               &              &                  \\
 Sinplektikoa                            &  Symplectic                          &              &                  \\
 Desplazamendu                           &  Drift                               &              &                  \\
                                         &  Splitting methods                   &              &                  \\
                                         &  Conposition methods                 &              &                  \\
 Runge-Kutta Esplizitua                  &  Explicit Runge-Kutta                &  \emph{ERK}  &                  \\
 Runge-Kutta Inplizitua                  &  Inplicit Runge-Kutta                &  \emph{IRK}  &                  \\
                                         &  A-stability, B-stability            &              &                  \\
                                         &  Implicit Midpoint method            &              &                  \\
                                         &  Adjoint                             &              &                  \\
                                         &  bias                                &              &                  \\
                                         &  compensated summation               &              &                  \\
 Biribiltzea                             &  roundoff                            &              &                  \\
 Portabilitate                           &  portable                            &              &                  \\
 Doitasun arrunta                        &  Single precision                    &              &                  \\
 Doitasun bikoitza                       &  Double precision                    &              &                  \\
 Doitasun laukoitza                      &  Quadruple precision                 &              &                  \\
 Multiple-digit representation           &  Digito-anitzeko adierazpena         &              &                  \\
 Multiple-term representation            &  Termino-anitzeko adierazpena        &              &                  \\  
 Haria                                   &  Thread                              &              &                  \\
                                         &  Graphical Processor Unit            &  \emph{GPU}  &                  \\
                                         &  Least Square                        &              &                  \\
                                         &  Least Eigenvalues problems          &              &                  \\
 Balio singularren deskonposaketa        &  Singular values descomposition      & \emph{SVD}   &                  \\
 Ezentrizidadea                          &  Eccentricity                        &              &                  \\
                                         &  Eccentric anomaly                   &              &                  \\
 Astronomical unit (AU)                  &  Unitate astronomikoa                &              &                  \\
 Julian date                             &  Data juliotar                       &              &                  \\
 LU Decomposition  (low, up)             &  LU-deskonposaketa                   &              &                  \\ 
 Flops                                   &  Koma-higikorreko eragiketa segunduko &             &                  \\   
 Peak                                    &  Exekuzio gaitasuna                   &             &                  \\
 Wall-time, elapsed-time                 &                                       &             &                  \\
 CPU-time                                &                                       &             &                  \\
 Cache memoria                           &                                       &             &                  \\ 
 spatial/data locality                   &                                       &             &                  \\
 Single Instruction Multiple Data        &                                       & SIMD        &                  \\
 Fork-join                               &                                       &             &                  \\
 Application programming interface       & Interfaze aplikazio programa          & API         &                  \\
 Eraginkortasun altuko konputazioa       & High performance computing            & HPC         &                  \\
 Problem solving enviroments             & Problemak ebazteko inguruneak         & PSE         &                  \\
 Portable                                &                                       &             &                  \\
 Linear least square problems            &                                       &             &                  \\
 Eigenvalues problems                    &                                       &             &                  \\
 Sparse matrices                         & Matrize bakanak                       &             &                  \\
                                                            
 
                                
 \hline
 \end{tabular}
\end{table}
