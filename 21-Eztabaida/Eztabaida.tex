\chapter{Eztabaida.}

\section{Eguzki-sistemaren integraziorako egungo metodoak.}


Tesiaren sarreran azaldu genuenez, gure helburua eguzki-sistemaren epe luzeko eta doitasun handiko inplementazio eraginkorra proposatzea da. Atal honetan, egungo eguzki-sistemaren simulazioetan erabiltzen diren metodo eta inplementazioen laburpena egingo dugu. Bi eratako simulazioak aztertuko ditugu. Batetik, eguzki-sistemaren  efemerideak ditugu; eguzki-sistemaren eredu konplexuak eta epe tarte txikietarako (ehunka urtekoak) integrazioak konputatzen dituzte. Bestetik, astronomi arloaren inguruko ikerketetarako eguzki-sistemaren integrazio luzeak ditugu: eguzki-sistemaren eredu sinpleagoak eta milioika urteko integrazioak konputatzen dituzte. 

\subsection*{Efemerideak.}

Konputagailuen aurreko garaian, efemerideak teoria analitikoetan oinarritutako serie funtzioen bidez kalkulatzen ziren. Soluzio hauetan, Fourier-en serie trigonometriko luzeen ebaluazioa egin behar zen. $1960$ hamarkadan, eguzki-sistemaren ezagutza hobetu zenean (espazio bidaiak eta behatoki astronomikoen aurrerapenak medio), serie oso luzeak kalkulatu behar zituzten, eta orduan zenbakizko integrazioen bidezko soluzioak eraginkorragoak bilakatu ziren \cite{Kaplan2015}.   
   
Eguzki-sistemaren gorputzen efemeride modernoak, mugimenduaren ekuazio diferentzialen (\ref{eq:nbody}) zenbakizko integrazioaren bidez kalkulatzen dira. Integrazioaren hasierako balioak eta ereduaren parametroak, sateliteen bidez jasotako datuekin kontrastatu egiten dira.

Efemerideak, \emph{Chebyshev} polinomio moduan adierazten dira. Integrazio tarteak, $2.000.$ urte inguruko ehunka urtekoak izaten dira. Zenbakizko integrazio hauetan, biribiltze errorea gai garrantzitsua da. $128$-biteko aritmetikaren aukera baztertzen da, konputazioa oso garestia delako eta $64$-biteko doitasuneko aritmetika hobetzen dituzten teknika konputazionalki merkeagoak, aplikatzen dira. 

Efemerideetarako, eguzki-sistemaren eredu konplexua aplikatzen da. Gorputz nagusien arteko indar grabitazionalez gain, erlatibitate efektua, asteroideek eragindako grabitazio indarrak, gorputzen formen eragina eta beste hainbat indar ez grabitazionalak kontutan hartzen dituzte. Mugimenduaren ekuazio diferentzialak, era honetakoak izaten dira \cite{Feinga2015},      
      \begin{equation*}
      \ddot{x}_{Planet}= \sum_{A \neq B} \mu_B \frac{r_{AB}}{\|r_{AB}\|^3}+\ddot{x}_{GR} (\beta,\gamma,c^{-4})+ \ddot{x}_{AST,300}+ \ddot{x}_{J_2}.
      \end{equation*}

Hauek dira, ekuazio hauen konplexutasuna erakusten duten ezaugarri batzuk:      
      \begin{itemize}
      \item Gorputz kopurua: $8$ planetak, ilargia, Pluton eta 300 asteroide.
      \item Erlatibitate efektua (GR): Einstein-Imfeld-Hoffmann, $c^{-4}$ PPN hurbilketa.
      \item $J_2$: eguzkia esferikoa ez izatearen eragina. 
      \item Urrats luzera, $h=0.055$ egunekoa da.
      \end{itemize}   

\subsubsection*{Erlatibitate efektua.}
Eguzkiaren erlatibitate efektua kontutan hartzen duten ekuazio diferentzialak azalduko ditugu \cite{Kopeikin2011}.
\begin{equation}
\dot{q_i}=v_i, \  i=0,1,\dots N
\end{equation}
\begin{multline} 
\dot{v_i}= \sum_{j=0,j \neq i}^{N} \frac{Gm_j}{\|q_j-q_i\|^3} (q_j-q_i)
           \bigg(1- \frac{2(\beta+\gamma)}{c^2} \sum\limits_{k=0, k \neq i}^{N} \frac{Gm_k}{\|q_k-q_i\|} 
                  - \frac{2\beta-1}{c^2}        \sum\limits_{k=0, k \neq j}^{N} \frac{Gm_k}{\|q_k-q_j\|} \\
                  + \gamma \big(\frac{v_i}{c}\big)^2 + (1+\gamma) \big(\frac{v_j}{c} \big)^2 
                  - \frac{2(1+\gamma)}{c^2} v_i \ v_j \\
                  - \frac{3}{2c^2} \big(\frac{(q_i-q_j) v_j}{\|q_j-q_i\|} \big)^2+                  
                  \frac{1}{2c^2}(q_j-q_i) \dot{v_i} \bigg) \\
           + \frac{1}{c^2} \sum_{j=0,j \neq i}^{N} \frac{Gm_j}{\|q_j-q_i\|^3} 
             ((q_i-q_l) ((2+2\gamma)v_i-(1+2\gamma)v_j)) (v_i-v_j) \\
           + \frac{3+4\gamma}{2c^2} \sum_{j=0,j \neq i}^{N} \frac{Gm_j \dot{v_j}}{\|q_j-q_i\|}                                      
\end{multline}

\begin{table}[h]
\caption{Konstanteak}
\label{tab:1}       % Give a unique label
\centering
\begin{tabular}{l l l }
\hline
  c             &  $299792.458$ km/s           & Argiaren abiadura  \\
%\hline
  au            &  $149597870.700$ km          & Unitate Astronomikoa  \\
%\hline 	       
$\beta$          & $1.0$                       & PPN parametroa     \\
%\hline 
$\gamma$         & $1.0$                       & PPN parametroa     \\
\hline
\end{tabular}
\end{table}

\subsubsection*{Asteriodeak.}
Asteroideek, bereziki Marte planetaren mugimenduarengan eragina dute (\ref{fig:asteroideak} irudia) eta kontutan hartzekoak, barne-planeten mugimenduaren doitasun handiko emaitzak behar ditugunean . Bost asteroide nagusiren masak (Ceres, Pallas, Vesta, Iris eta Bamberga) Merkurio eta Pluton planeten mailakoak direnez, integrazioetan gehitzen dira. Asteroide txikien talde handia, estimazioen bidez simulatzen dira.

\begin{figure} [h]
\centerline{\includegraphics [width=6cm, height=4cm] {Asteroideak}}
\caption{Asteroideak.}
\label{fig:asteroideak}
\end{figure} 

  
\subsubsection*{Hiru efemerideak.}

Gaur-egun, eguzki-sistemaren planeten hiru efemeride kalkulatzen dira.
\begin{enumerate}
\item Jet Propulsion Laboratory (\emph{EEBB}) \emph{NASA}-ko erakundeak DE (Development Ephemerides) izeneko efemerideak konputatzen dituzte.

      $1984$. urtean kalkulatu zen lehen efemeridea (DE-200) eta $2.014$. urteko \emph{DE-430} \cite{Folkner2014} efemeridea da, publikatutako azken efemeridea. Azken efemeride honen integrazio tartea, $1550-2650$ urtetakoa izan da.

      Zenbakizko integrazio metodoa, urrats luzera eta  ordena aldakorreko \emph{Multistep Adams} metodoa \cite{Krogh1997} (\emph{DIVA}/\emph{QIVA}) da. \emph{QIVA} doitasun laukoitzeko ($128$-bit) bertsioari deitzen zaio: mugimenduaren ekuazioen Newton zatia, doitasun laukoitzean kalkulatzen da eta ekuazioaren gainontzeko zatia, doitasun bikoitzean.

\item Institut de Méchanique Céleste et de Calcul des Ephémérides (IMCCE,Paris Observatory) INPOP (Intégrateur Númerique Planétaire de l'Observatoire de Paris) izeneko efemerideak.
      
      $2.000$. urte arte, efemerideak kalkulatzeko teori analitikoetan oinarritu ziren. $2.003$. urtean, zenbakizko integrazioaren bidezko lehen efemeridea kalkulatu zuten eta \emph{INPOP13c}  \cite[$2.014$]{Fienga2008} publikatutako azkena da.
           
	  Zenbakizko integrazio metodoa: $12$ ordeneko \emph{Adams-Cowell} metodoa da eta urrats finkoa aplikatzen dute.
	  
	  Doitasuna: C lengoaian inplementatuta dago eta \emph{Intel} makinetako $80$-biteko doitasuna erabiltzen du. Era berean, modu merkean  doitasun laukoitza simulatuz, urrats zuzentzaile (corrector step) bat aplikatzen dute \cite{Fienga2008}.  
	  
  
\item Institute of Applied Astronomy (\emph{IAA}, St. Petersburg), EPM (Ephemerides Planets-Moon) izeneko efemerideak.
      
      $1.980$. urtetik aurrera, zenbakizko integrazioen bidezko efemerideak kalkulatu dituzte eta  \emph{EPM2.013}  \cite[2.014]{Pitjeva2014} publikatutako azken efemeridea da.
      
      Zenbakizko integrazio metodoa, \emph{Everhart} izeneko \emph{IRK} metodoa (Gauss-Radau) da. $23$ ordeneko metodoa eta urrats luzera finkoa aplikatzen dute.
            
      Doitasuna. Inplementazioak (software package ERA), \emph{Intel} makinetako $80$-biteko doitasuna erabiltzen dute.
      
\end{enumerate}


%\ref{fig:668} taulan, planeten efemerideen doitasunaren eboluzioa ikus daiteke. Hiru efemerideak antzeko doitasuna azaltzen dutela gehitu beharra dago. 
%\begin{figure} [h]
%\centerline{\includegraphics [width=10cm, height=6cm] {Efemerideak}}
%\caption{Efemerideen doitasunaren eboluzioa.}
%\label{fig:668}
%\end{figure} 
%

\subsection*{Eguzki-sistemaren integrazio luzeak.} 


A.Morbidellik \cite{Morbidelli2002} eguzki-sistemaren zenbakizko integrazioen algoritmoen garapenaren azterketan, garai hauek bereizten ditu:
\begin{enumerate}

\item Garai klasikoa.

$90$. hamarkada hasiera arte, urrats luzera aldakorreko integratzaileak erabiltzen dira: Runge-Kutta (Dormand et al. $1987$), Bulirsch and Stoer ($1966$), Radau (Everharht, $1985$), eta Störmer ($1990$). Garai honetan integrazio tarteak, $10^4-10^6$ urte artekoak dira.  

\item Garai sinplektikoa.

Wisdom eta Holman-en \cite[1991]{Sussman1992} lanarekin, eguzki-sistemaren azterketarako integratzaile sinpletikoen erabilera zabaldu zen. Garai honetan, ($10^8-10^9$) urte arteko eguzki-sistemaren integrazioak egin ziren.  

\item Garai estatistikoa.

Planeten eta gorputz txikien (asteroide, meteoritoak) arteko kolisio gertuko egoerak kalkulatzen dituzten algoritmoak garatu ziren. Inplementazio berri hauetan, milaka gorputzen integrazio azkarra egin daiteke. Horrela, asteroide eta meteoritoen orbiten distribuzio azterketa estatistikoak egin ziren.

\item Planeten sorrera garaiko azterketak.

Eguzki-sistemaren sorrerari buruzko simulazioak nagusituko dira; masa handiko gorputzen arteko kolisio gertuko egoerak gertatzen diren integrazioak konputatzen dira. 
 
\end{enumerate}

Eguzki-sistemaren integrazioetarako nagusiki, bi integratzaile famili aplikatzen dira: 
\begin{enumerate}
\item Integratzaile simetrikoak.

Metodo simetrikoen artean nagusiena, 4 ordenako \emph{Hermite} \cite{Aarseth2008} integratzailea da.
Urrats luzera tamaina aldakorreko integratzaile da, modu errazean inplementatu daiteke.  Hermite integratzailea konputazionalki garestia da eta bereziki, gorputz kopuru handia dituzten eta kolisio gertuko egoerak maiz gertatzen diren problemetan (eguzki-sistemaren sorrera, \dots) aplikatzen da.  

\item Sinplektikoak.

Gaur-egun, eguzki-sistemaren epe luzeko integrazioetarako, integratzaile sinplektikoak nagusitu dira. 

\end{enumerate}


\subsubsection*{Eguzki-sistemari egokitutako integratzaile sinplektikoak.}

Wisdom-ek eta Holman-ek \cite[1991]{Sussman1992}, eguzki-sistemaren epe luzeko simulazioetarako integratzaile  sinplektikoak (\emph{WH}) arrakasta izan zuen. Eguzki-sistema, mugimendu perturbatua duen sistema dinamikoa da eta ezaugarri honi egokitutako integratzaile eraginkorra garatu zuten. Jacobi koordenatuak  aplikatuz, N-gorputzen problemaren Hamiltondarra, bi zatitan banatu zuten,
\begin{equation*}
H(q,p)=H_K(p)+H_I(q) \ \ \ , \ \ H_K\gg H_I,
\end{equation*}
non $H_K$, Hamiltondar Kepleriarra (planeten eguzkiarekiko mugimendu kleperiarra) eta $H_I$, interakzioen Hamiltondarra (planeten arteko grabitazio interakzioak) diren. Integrazioaren urrats bakoitzean , Hamiltondar bakoitzaren soluzioa tartekatuz, problema osoaren ebazpena kalkulatzen da.  

\emph{WH} integratzailea, ondorengo metodo askoren aurrekaria kontsideratu bada ere, bere aplikagarritasuna mugatua da. Batetik, izar anitzeko planeten sistemak edo planeta-ilargiak sistemak integratzeko ez da egokia. Bestetik, \emph{WH} metodo sinplektikoa denez, urrats luzera finkoa aplikatu behar da eta hau, gorputzen arteko kolisiotik gertuko egoerak dituzten problemak modu eraginkorrean integratzeko eragozpen bat da. Arazo hauek gainditzeko, urteetan zehar algoritmo honen aldaerak proposatu dira eta jarraian, nagusienak aipatuko ditugu. 

Levinson eta Duncan-ek \cite[1994]{Levison1994}, \emph{WH} inplementazioa, integratzaile ez sinplektiko batekin konbinatu zuten, kolisiotik gertuko egoeren kalkulua hobetzeko. \emph{SWIFT} paketean, \emph{RMVS3} izeneko integratzailea inplementatu zuten. Duncan, Levinson eta Lee-k \cite[1998]{Duncan1998}, koordenatu heliozentrikoak erabiliz, Hamiltondarra beste modu honetan banatu zuten,  
\begin{equation*}
H(q,p)=H_K(p)+(H_C(p)+H_I(q))
\end{equation*}
eta kolisiotik gertuko egoerei, urratsa luzera txikituz aurre egin zioten. Inplementazio honek, \emph{SYMBA} izena du. Chambers-ek \cite{Chambers1999} koordenatu heliozentrikoetan oinarritu zen eta kolisiotik gertuko egoerak gertatzen diren uneetan, beste integratzaile  (Bulirsch-Stoer metodoa) batekin konbinatuz integratzen du. Inplementazio honek, \emph{MERCURY} izena du. Levinson eta Duncan-ek ($2000$), aurreko inplementazioaren arazo batzuk konponduz, \emph{Modified SYMBA} izeneko garapen berria burutu zuten.
Kvaerno eta Leimkuhler \cite{Kvaerno2000} eta beste autore batzuk ere, antzeko ideiak landu dituzte.

Wisdom eta Holmanek proposatutko Hamiltondarraren banaketa, (\ref{eq:stverlet})~\emph{Leapfrog} metodoaren bidez integratzen da eta beraz, 2 ordeneko da. Orden altuagoko ($p>2$) metodoak definitzeko, koefiziente negatiboak erabili behar zela uste zen \cite{Yoshida1993,Laskar2001} eta orduan ez dira interesgarriak, \emph{Leapfrog} metodoak eraginkorragoak baitira. 

%\paragraph*{Adibidea.}
%Yoshidaren $p=4$ ordenako metodoa. $\phi_h$ oinarrizko metodoa \emph{Leapfrog} izanik, metodoaren konposaketari dagokion 4 ordeneko konposizio metodoa, era honetan definitzen da,
%\begin{equation*}
%\Psi_h=\phi_{\gamma_1 h} \circ \phi_{\gamma_0 h} \circ \phi_{\gamma_{1 h}},
%\end{equation*}  
%non  $\gamma_0=-2^{1/3}/(2-2^{1/3})$ eta $\gamma_1=1/(2-2^{1/3})$.

Beranduago, McLachlan-ek \cite[1995]{McLachlan1995} eta Laskar-ek  \cite[2001]{Laskar2001} koefiziente negatiboen arazoa gainditu zuten eta koefiziente positiboekin definitutako ordena altuko Splitting eskemak aurkitu zituzten. Berriki, Blanes-ek \cite[2012]{Blanes2013} ordena altuko Splitting eskema berriak eta eraginkorrak aurkitu ditu. 

Hernandez eta Bertschinger-ek \cite[2015]{Hernandez2015} N-gorputzen problema grabitazional eta kolisiodunetarako $2$ ordenako integratzaile sinplektiko berri bat proposatu dute. Hernandez eta Bertschinger-ek \cite{Hernandez2015} koordenatu kartesiarretan oinarrituz, N-gorputzen problema, 2-gorputzen azpiproblemetan banatzen dute.

% eta honako Hamiltondarraren banaketa proposatzen dute,
%\begin{align}
%\begin{split}
%&H=T+V, \\
%&H=T+ \sum_{i} \sum_{i>j} V_{ij}, \\
%&H=T+ \sum_{i} \sum_{i>j} (K_{ij}-T{ij})
%\end{split}
%\end{align}

\subsection*{Splitting metodoak.}

Splitting metodoak, era honetako perturbatutako Hamiltondar sistemak,
\begin{equation}
\label{eq:Hban}
H=H_A+\epsilon H_B,
\end{equation}
integratzeko aplika daitezke, non $H_A$ eta $H_B$ independenteki integragarriak diren. Pentsatzeko da, eguzki-sistemaren eredu errealistetan hainbat indar ez grabitazionalak modu honetan gehitzeko, zailtasunak izan ditzakegula.  

Laskar-ek \cite[2011]{Laskar2011}, epe luzeko zenbakizko integraziorako  $1/c^2$ ordenako eguzkiaren erlatibitate efektua   kontsideratu zuen. Horretarako, Saha eta Tremain-ek \cite{Saha1994} finkatutako teknikan oinarritzen da. Teknika honetan, (\ref{eq:Hban}) Hamiltondar banagarriei egokitzen zaien erlatibitate efektuaren espresioa lortu zuten eta gainera modu eraginkorrean kalkulatu daiteke. Kontutan hartuko bagenitu, bi planeta handienen (Jupiter eta Saturno) erlatibitate efektua, ekuazioak ez dira gehiegi konplikatzen eta errorea txikiagoa lortuko genuke. Splitting metodoetan ez da posible erlatibitate efektu hauek gehitzea eta horregatik, ez direla kontutan hartzen. 
 
Eguzki-sistemaren integrazioetarako, (\ref{eq:Hban}) Hamitondarraren banaketa, Jacobi koordenatuak edo heliozentrikoak aplikatuz lortzen da.
Koordenatu cartesiarrak, ezin daitezke erabili.

Muga hauek ez ditugu, IRK metodoak aplikatzeko. Ekuazio diferentzial orokorrei aplika daitekeenez, askatasun osoa dugu behar diren ekuazioak gehitzeko eta interesatzen zaigun koordenatu sistema aplikatzeko.  
 

\section{Eredu deskonposaketa.}


IRK metodoen inplementazio eraginkorren azterketaren hasierako fasean, problemaren deskonposaketan oinarritutako ideia ikertu dugu. Eguzki-sistemaren eredua, problema sinple (konputazionalki merkea) eta problema konplexu batean (konputazionalki garestia) bana daiteke.   
Problemaren alde konplexuaren balioztapen kopuru txikituz, inplementazio eraginkorra lortzeko asmoarekin, Newton sinplifikatuaren iterazio berezi bat aplikatu dugu. Newton sinplifikatuaren iterazio berezi honetan, ekuazio-sistema lineala LU deskonposaketaren bidez askatu beharrean, puntu-finkoaren bidez ebatzi dugu.

Atal honen bukaeran emango ditugun arrazoiengatik, beste bide batzuk ikertzea zuzenago dela ikusi dugu eta planteamendu hau alde batera utzi dugu. Dena den, planteamendu honen laburpen bat emango dugu eta inplementazio bide honetan sortutako hainbat ideia interesgarri jaso ditugu.

\subsection*{Inplementazioa.}
Eguzki-sistema, perturbatutako mugimendu Kepleriar gisa har daiteke,  
\begin{align}
\label{eq:fkg}
 f(y)=  k(y) + \epsilon \ g (y), \ \ g\ll k,   
\end{align}
 
$k(y)$ sistema dinamikoaren alde sinplifikatua da eta ebaluatzeko kostua txikia du. $g(y)$, ordea, sistemaren alde konplexua da eta ebaluatzeko kostua handia du. Ekuazio diferentzialaren (\ref{eq:fkg}) deskonposaketa badugu, IRK metodoaren atalen ekuazio sistema ebazteko,
\begin{equation*}
Y_i=y_n+h\ \ \sum^s_{j=1}{a_{ij} \ f({Y_j}) },  
\end{equation*} 
era honetako iterazio proposatzen dugu,
\begin{align}
\begin{split}
&k=1,2,\dots \\
&Y_i^{[k]}=y_n+h\ \ \sum^s_{j=1}{a_{ij}(\ k({Y_j}^{[k]})+g({Y_j}^{[k-1]})) }
\end{split}
\end{align}

Iterazio hau aplikatzea, Jacobiarraren hurbilpentzat, $J_i=\partial k/ \partial y (Y_i), \ i=1,\dots,s$ hartzen duen Newton sinplifikatua aplikatzearen baliokidea da. Hurrengo \ref{alg:metalg} meta-algoritmoan, iterazio aplikatzeko zehaztasun gehiago ematen ditugu,
 
\begin{algorithm}[H]
 \For{$n\leftarrow 1$ \KwTo $endstep$}{
  \BlankLine
  \text{Hasieraketa}\ $Y^{[0]}_i, W^{[0]}_i$\;
  k=1\;
  \text{Askatu} \ ${Y_i}^{[k]}=y_n+h\  \sum^s_{j=1}{a_{ij}\ k({Y_j}^{[k]})+W^{[k-1]}_i\ \ \ }$\;
  \BlankLine
  \While{(\text{konbergentzia ez lortu})}{
     \BlankLine
     k++\;
     $W^{[k-1]}_i= h\ \sum^s_{j=1}{a_{ij}\ g({Y_j}^{[k-1]})}$\; 
     \text{Askatu} \ ${Y_i}^{[k]}=y_n+h\  \sum^s_{j=1}{a_{ij}\ k({Y_j}^{[k]})+W^{[k-1]}_i\ \ \ }$\;
     \BlankLine
  }  
  $y_{n+1}=y_n+h\ \sum^s_{j=1}{b_{i}\ f({Y_j}^{[k]}) \ }$\;
  \BlankLine
 }
 \caption{Meta-algoritmoa}
 \label{alg:metalg}
\end{algorithm}

Meta-algoritmoaren barne ekuazio-sistema askatzeko, metodo egokiena aplikatzeko askatasuna dugu. Pentsa liteke,  problema zurruna ez bada puntu-finkoaren bidez askatzea eta problema zurruna bada, berriz,  Newton sinplifikatuaren iterazioaren bidez.

Barne ekuazio-sistema,
\begin{equation*}
 Y_i^{[k]}=y_n+h\  \sum^s_{j=1}{a_{ij}\ k({Y_j}^{[k]})+W^{[k-1]}_i},
\end{equation*}
puntu-finkoaren iterazioaren bidez askatzeko, \ref{alg:bpf} barne-iterazioa aplikatuko dugu. 

\begin{algorithm}[H]
 \BlankLine
  $l=0$\;
  $Y_{i}^{[k,0]}=Y_{i}^{[k-1]}$\;
  \While{ (\text{konbergentzia ez lortu})}
  {
   \BlankLine
   $l=l+1$\;  
   \BlankLine
   $K_{i}^{[k,l]}=k(Y_{i}^{[k,l-1]})$\;
   $Y_{i}^{[k,l]}=y_{n} + h \sum\limits_{j=1}^{s} \ a_{ij} \ K_{j}^{[k,l]}  +  W_{i}^{[k-1]} $\;
  }
 \caption{Barne-iterazioa: puntu-finkoaren iterazioa}
 \label{alg:bpf}
\end{algorithm}

\subsection*{Orokorpena.}

\subsubsection*{Eredu konplexuak.}
Aurreko atalean, maila bakarreko ereduen deskonposaketa aztertu dugu. Ideia orokortuz, eredu deskonposaketa maila ezberdinetan aplika daiteke.  $\dot{y} =f(y)$ problema emanik, 
\begin{align*}
&\mbox{1. maila} \ \
\left \{ \begin{array}{c}
  \mbox{Eredu osoa.   } f(y) \\[.25cm]
  \mbox{Eredu sinplea.    } \tilde{f}(y)  \\
\end{array} \right.
\ \Rightarrow \ \
f =\tilde{f}+(f-\tilde{f})  
\end{align*}

\begin{align*}
&\mbox{2. maila} \ \
\left \{ \begin{array}{c}
  \mbox{Eredu osoa.   }\tilde{f}(y) \\[.25cm]
  \mbox{Eredu sinplea.    }\tilde{\tilde{f}}(y)  \\
\end{array} \right.
\ \Rightarrow \ \
\tilde{f} =\tilde{\tilde{f}}+({\tilde{f}}-\tilde{\tilde{f}})\\
&\dots  
\end{align*}

\paragraph*{Adibidea.}
Eguzki-sistemaren eredu konplexuaren ekuazio-sisteman, alde Kepleriarra eta perturbazio maila ezberdinak bereiz daitezke. Meta-algoritmoa, deskonposaketaren maila bakoitzari modu errekurtsiboan aplika daiteke. Demagun, eguzki-sistemaren problemaren ekuazio diferentzial hauek ditugula,
\begin{equation*}
\dot{y}=f(y), \ f(y)=k(y)+g(y)+r(y),
\end{equation*}
non
\begin{align*}
&k(y): \ \text{kepleriarra.}\\
&g(y): \ \text{planeten arteko grabitazio interakzioak.}\\
&r(y): \ \text{planeten eguzkiarekiko erlatibitate efektua.}
\end{align*}

Meta-algoritmoa modu honetan aplika daiteke,
\begin{align*}
&\mbox{1. maila}\\ 
&Y_i=y_n+h \ \sum^s_{j=1}{a_{ij} \ f(Y_j)}.\\
&\mbox{2. maila}, \ k=1,2,\dots\\
&Y_i^{[k]}=y_n+h\  \sum^s_{j=1}{a_{ij} \ k({Y_j}^{[k]})}+ \delta_1^{[k-1]},\\
& \text{non} \ \ \delta_1^{[k-1]}= h\  \sum^s_{j=1}{a_{ij} (g({Y_j}^{[k-1]})+r({Y_j}^{[k-1]}))}. \\
&\mbox{3. maila}, , \ l=1,2,\dots\\
&Y_i^{[k,l]}=y_n+h\ \ \sum^s_{j=1}{a_{ij} \left(k({Y_j}^{[k,l]})+g({Y_j}^{[k-1,l]})\right)}+\delta_2^{[l-1]}, \\
& \text{non} \ \ \delta_2^{[l-1]}= h \ \sum^s_{j=1}{a_{ij} \ r({Y_j}^{[k-1,l-1]})}.
\end{align*}


\subsubsection*{Problema independenteak.}

Batzuetan, problemaren eredu sinplifikatua, azpiproblema independentetan bana daiteke,
\begin{align*}
f\left ( \begin{array}{c}
   y_1 \\
   y_2 \\
\end{array} \right)=
\left ( \begin{array}{c}
   k_1(y_1) \\
   k_2(y_2) \\
\end{array} \right)+
\left ( \begin{array}{c}
   g_1(y_1,y_2) \\
   g_2(y_1,y_2) \\
\end{array} \right).
\end{align*}
%
Hortaz, azpiproblema bakoitzari dagokion barne-iterazioak independenteak dira eta paraleloan kalkula daitezke. Eguzki-sistema eredu grabitazionala, azaldutakoaren adibide argia da; planeta bakoitzarentzat, $k(y)$ eguzkiarekiko interakzioa azpiproblema independentea da. Dena den, zenbakizko esperimentuetarako aukera eraginkorrena, eredu sinplifikatua bi azpiproblemetan banatzea dela konprobatu dugu: alde batetik, barne-planeten eredu sinplifikatuak osatutako azpiproblema eta beste aldetik, kanpo-planeten eredu sinplifikatua osatutakoa.    

\subsubsection*{Tolerantzia aldakorra.}

Kanpo eta barne-iterazioetarako geratze irizpide berdinak definitu ditugu. Kanpo-iterazioetarako, tolerantzia finkoa erabili dugu eta barne-iterazioetarako, ordea, tolerantzia aldakorra. Tolerantzia aldakorra, kanpo-iterazio (konputazionalki garestia) bakoitzarentzat, barne-iterazio (konputazionalki merkea) kopuru nahikoak eta beharrezkoak eman dadin, definitu dugu.     
 

\subsection*{Eragozpenak.}
Esan dugunez, planteamendu honetan Newton sinplifikatua aplikatu dugu eta ekuazio-sistema lineala LU deskonposaketaren bidez askatu beharrean, puntu-finkoaren iterazioaren bidez askatu dugu. Planteamendu honi IRK-Newton sinplifikatuaren inplementazio estandarrarekiko, hiru desabantaila aurkitu dizkiogu: 
\begin{enumerate}
\item Barne kalkuluen doitasuna. 

Planteamendu honen barne-iterazioak konputazioak, inplementazioaren oinarrizko doitasunean kalkulatu behar dira.
IRK-Newton sinplifikatuaren inplementazioaren eragiketa konplexuenak, doitasun txikiagoan kalkula daitezke \cite{Baboulin20092526}. 

\item Jacobiarraren balioztapena.

Aztertutako planteamendu honetan, Jacobiarra iterazio bakoitzean aldatzen denez, iterazio guztietan balioztatu behar da.
IRK-Newton sinplifikatu estandarrean, Jacobiarra urrats bakoitzean behin bakarrik kalkulatu behar da. 

\item Ereduen deskonposaketa.\\
Ereduen deskonposaketak, nolabaiteko konplexutasuna gehitzen du. Hiru eredu ezberdin bereizi behar ditugu: eredu osoa $f(y)$, eredu sinplifikatua $\tilde{f}(y)$ eta perturbazioa  $g=f(y)-\tilde{f}(y)$. 
\end{enumerate}

\section{IRK inplementazioaren oinarriak.}


\subsection*{Atalen hasieraketa.}

Iterazio metodoetan aplika daitezkeen atalen hasieraketa teknika ezberdinak ikertu ditugu. 
\begin{enumerate}
\item Metodo esplizituak.

Atalen hasieraketa metodo esplizitu bat aplikatuz lor daiteke. Hurbilpen merkea lortze aldera, ordena txikiko metodoak aplikatu ditugu: 
Euler  $\mathcal{O}(h)$ eta Euler hobetuaren $\mathcal{O}(h^2)$ metodoa. Atalen hasieraketa metodo hauek, ekuazio diferentzialaren balioztapena beharrezkoa da eta era honetan aplikatu ditugu:
\begin{align*}
&\text{Euler}:\\
& \quad Y_i=Y_{i-1}+h (c_i-c_{i-1}) f(Y_{i-1}).\\
&\text{Euler hobetua}: \\
& \quad Y_i=Y_{i-1}+h (c_i-c_{i-1}) f(k_i),\\
& \quad \text{non} \ k_i=Y_{i-1}+\frac{h}{2} (c_i-c_{i-1}) f(Y_{i-1}).
\end{align*} 


\item Kepler fluxua.

Eguzki-sistemaren, perturbatutako sistema Kepleriarra denez, atalak Kepler fluxuaren bidez hasieratu daitezke. Kepler fluxuaren funtzioak, $y(t_n)$ une bateko kokapen eta abiadurak emanik, $\Delta t_n$ denbora igarotakoan kokapen eta abiadura zehatzak itzultzen du,
\begin{equation*}
\text{Keplerflow}(\Delta t_n, y(t_n)) \rightarrow y(t_n+\Delta t_n).
\end{equation*}

Beraz, $y_n$ egoeratik abiatuta, atal bakoitzari dagokion hasieraketa,
\begin{equation*}
Y_{n,i}^{[0]}=y(t_n+hc_i), \ i=1,\dots,s,
\end{equation*}
Kepler fluxua $\Delta t_n=hc_i$ denbora aurrera eginez lortuko dugu.  

\item Interpolazio polinomioak.

Aurreko urratseko ataletako informazioa erabiliz, urrats berriaren atalen hasieraketa lortzen dugu. Problema zurruna ez denean interesgarria da. Era honetako hasieraketa merkea da, ez baita funtzio balioztapen berririk egiten. Era honetan aplikatzen da,
\begin{equation*}
Y_{n,i}^{[0]}=y_n+h \sum_{j=1}^{s} \lambda_{ij} \ f(Y_{n-1,j}),
\end{equation*}
non $\lambda_{ij}$ aurre-kalkulatutako interpolazio koefizienteak diren. 


Bide honetatik, teknika aurreratuagoak ere aztertu ditugu (maila txikiko polinomioen bidezko hasieraketak). Era berean, B-Serieak \cite{Chartier2010} teknikan oinarrituz, interpolazio estandarra \cite{Laburta1998} hobetzeko saiakera egin dugu baina ez dugu hau hobetzea lortu. Metodo estandarrarekin, urrutien dagoen ataletarako hasieraketa txarra lortzen da eta atal askotako metodoentzat, arazo hau, eragozpen handia izan daiteke.  

\end{enumerate}

Problema ez-zurrunetarako, interpolazio polinomioen bidezko hasieraketa ona eta merkea lortzen da. Problema zurrunetarako, ordea, atalak aurreko urratseko soluzioarekin hasieratuko ditugu,
\begin{align*}
Y_{n,i}^{[0]}=y_n, \ i=1,\dots,s.  
\end{align*}
    

\subsection*{Geratze irizpidea.}

Iterazio metodo bat aplikatzeko, geratze irizpide sendoa finkatzea funtsezkoa da.  Iterazioak, biribiltze errorearen eragina azaltzen denean geratu behar dira. Batetik, iterazioak beranduegi geratzen baditugu, alferrikako iterazioak emango ditugu, eraginkortasunaren kalterako. Bestetik, iterazioak goizegi geratzen baditugu, biribiltze errorea handitu eta trunkatze errorea eragingo ditu.  

Geratze irizpide estandarrak aplikatzerakoan, hainbat arazo aurkitu ditugu eta ondorioz, geratze irizpide berri bat definitzeko beharra ikusi dugu. Geratze irizpide berria definitzeko, aukera ezberdinak aztertu ditugu. Garapen honetako, geratze irizpide bertsio nagusienak azalduko ditugu.      

\subsubsection*{Norma.}
Iterazio bakoitzaren hobekuntza ($\Delta^{[k]}$) neurtzeko, norma ezberdinak aplikatzen dira. Honakoa da, Hairer-ek puntu-finkoaren iterazioan oinarritutako IRK inplementazioaren geratze irizpidean aplikatzen duen norma,
\begin{equation*}
\Delta^{[k]}= \max_{i=1,\dots,s} \|Y_i^{[k]}-Y_i^{[k-1]}\|_{\infty}.
\end{equation*}

Norma hau zalantzan jarri dugu eta iterazioaren hobekuntza hobeto neurtzen duen norma finkatzen saiatu gara. Honako aukerak aztertu ditugu,
\begin{enumerate}
\item Lehen bertsioa.

Diferentziak modu erlatiboan neurtzeko beharrak bultzatuta era honetan definitu dugu,
\begin{align*}
\Delta=\max_{1 \leqslant j \leqslant d} \frac{\max_{1 \leqslant i \leqslant s} |\Delta Y_i^j|}
                                                {(\max_{1 \leqslant i \leqslant s}|Y_i^j|)+\delta},
\end{align*}
non $\delta \approx 10^{-20}$, zatitzailea zero ez izateko finkatzen dugun balio txikia  eta $y=(y^1,\dots,y^d)$ den.

\item Bigarren bertsioa.

Kokapenen ($Q_i \in \mathbb{R}^d$) eta abiaduren ($V_i \in \mathbb{R}^d$) izaera ezberdina dela jabetuta, norman bi kontzeptu hauek banatu ditugu, 
\begin{align*}
 & \Delta =\\
 & max\left({\  {\max_{1\le j\le d} \frac{{\max_{1\le i\le s} {\left\|Q^{j}_i-{\tilde{Q}}^{j}_i\right\|}^2\ }}{{\max_{1\le i\le s} {\left\|Q^{j}_i\right\|}^2\ }}\ }\ },\ {\max_{1\le j\le d} \frac{{\max_{1\le i\le s} {\left\|V^{j}_i-{\tilde{V}}^{j}_i\right\|}^2\ }}{{\max_{1\le i\le s} {\left\|V^{j}_i\right\|}^2\ }}\ }\right),
\end{align*}
non,
\begin{align*}
Y=\left( \begin{array}{cccccc}
Q^{1}_1 & \dots  & Q^{d}_1 & V^{1}_1 & \dots  & V^{d}_1 \\ 
Q^{1}_2 & \dots  & Q^{d}_2 & V^{1}_2 & \dots  & V^{d}_2 \\ 
\vdots  & \ddots  & \vdots  & \vdots  & \ddots  & \vdots  \\ 
Q^{1}_s & \dots  & Q^{d}_s & V^{1}_s & \dots  & V^{d}_s \end{array}
\right)\ \  
\tilde Y=\left( \begin{array}{cccccc}
\tilde Q^{1}_1 & \dots  & \tilde Q^{d}_1 & \tilde V^{1}_1 & \dots  & \tilde V^{d}_1 \\ 
\tilde Q^{1}_2 & \dots  & \tilde Q^{d}_2 & \tilde V^{1}_2 & \dots  & \tilde V^{d}_2 \\ 
\vdots  & \ddots  & \vdots  & \vdots  & \ddots  & \vdots  \\ 
\tilde Q^{1}_s & \dots  & \tilde Q^{d}_s & \tilde V^{1}_s & \dots  & \tilde V^{d}_s \end{array}
\right). 
\end{align*}

\item Hirugarren bertsioa.

Iterazioaren hobekuntza, norma jakin bat aplikatuz zenbaki eskalar bakarrarekin neurtu beharrean,
\begin{equation}
\label{eq:DD3}
\Delta^{[k]}=|Y^{[k]}-Y^{[k-1]}| \in \mathbb{R}^{sd}
\end{equation}
matrizea erabiliko dugu. Modu honetan, geratze irizpidea normaren independentea izango da eta gainera, geratze irizpide segurua eraikitzeko aukera emango digu.

\end{enumerate} 


\subsubsection*{Geratze irizpidea.}

Oraingoan ere, geratze irizpidearen abiapuntua Hairer-en geratze irizpidea izan da,
\begin{equation*}
\Delta ^{[k]}=0 \ \text{or} \ \Delta^{[k]} \geqslant \Delta^{[k-1]}.
\end{equation*}

Geratze irizpide hau hobetzeko proposamen batzuk garatu ditugu,
\begin{enumerate}
\item Lehen bertsioa.

Abiapuntuko geratze irizpidea, batzuetan goizegi geratzen da eta ziurtasuna handitu beharra dago.  Horregatik, geratze irizpidean, $k-2.$ iterazioaren informazioa erabiltzea aztertu dugu, 
\begin{align*}
(\Delta^{[k]} = 0) \ \text{or} \ ( \Delta^{[k]}\geqslant \Delta^{[k-1]} \ \ \text{and} \ \ \Delta^{[k]}\geqslant 0.81*\Delta^{[k-2]}).
\end{align*}

\item Bigarren bertsioa.

Bertsio honetan, $k-3.$ iterazioaren informazioa erabiliz saiakera berri proposatu dugu,
\begin{align*}
\left(\Delta^{[k]} \leqslant tol \right) \ \text{or} \ \left( zat^{[k]} \geqslant koef*\max(zat^{[k-1]},zat^{[k-2]}) \right),
\end{align*}
non $zat^{[k]}={\Delta^{[k]}}/{\Delta^{[k-1]}}, \ tol\approx10^{-16}$ eta $koef=10$ den.

\item Hirugarren bertsioa.
\begin{align*}
\left(\Delta^{[k]} \leqslant tol\right) \ \text{or} \ \left(\frac{zat^2}{(1-zat)} \Delta^{[k-1]} \leqslant tol\right) \ \text{or} \ \left(\frac{zat}{(1-zat)}\Delta^{[k-2]} \leqslant tol\right),
\end{align*}
non $zat=\max(zat^{[k-1]},zat^{[k-2]})$, $tol\approx10^{-16}$ den.

Geratze irizpide honetan, iteraziotik irteten denean, $z\geqslant1$ betetzen bada,
\begin{align*}
&(1). \ \Delta {[k]} > c \ tol \rightarrow \text{konbergentzia arazoak (exekuzio geratu)},\\
&(2). \ \Delta {[k]} \leqslant c \ tol \rightarrow \text{birbiltze errorea (exekuzio jarraitu)},
\end{align*}   
non $c\approx 10^{4}$ konbergentzi koefizientea den. 

\item Laugarren bertsioa.

Azkenik, honakoa proposatu dugu: (\ref{eq:DD3}) $\Delta ^{[k]} \in \mathbb{R}^{sd}$ izanik, iterazioak  $k=1,2,\ldots$ jarraitzea , $ \Delta^{[k]} =0$ bete arte edo honako baldintza bi iterazio jarraietan betetzen den artean,
\begin{equation*}
\forall j \in \{1,\ldots,s d\},  \quad
\min \left(\{|\Delta_j^{[1]}|,\cdots ,|\Delta_j^{[k-1]}|\} \ /\{0\} \right) \leqslant |\Delta_j^{[k]}|.
\end{equation*}

Geratze irizpide honetan, iterazioa gertatu bada $\exists j,  \ \Delta_{j}^{[K]} \neq 0$ delarik, orduan urratsa onargarria den ala ez erabaki behar dugu. 


\end{enumerate}

 
\section{Doitasun altuko konputazioak.}


Zientzia-konputazioan, nagusiki doitasun bikoitzeko aritmetika (64-bit) aplikatzen da baina eguzki-sistemaren epe luzeko integraziorako, doitasun bikoitza ez da nahikoa. Jarraian, zenbakizko integrazio hauetan \cite{Laskar2015}, doitasuna hobetzeko bi bide azalduko ditugu. 

\subsection*{Double-extended doitasuna ($80$-bit).}

Egungo \emph{IEEE-754} estandarrak ($2008$), koma-higikorreko hiru formatu definitzen ditu: arrunta ($32$-bit), bikoitza ($64$-bit) eta laukoitza ($128$-bit).  \emph{IEEE-754} estandar zaharrak ($1985$), $80$-biteko formatuaren (double-extended) inplementazioa aholkatzen zuen \cite{Overton2001} baina egungo estandarretik kanpo utzi dute.  

C lengoaiaren \emph{long double} datu motak, $80$-biteko koma-higikorreko sustatzen du.  \emph{IA-32} koma-higikorreko eragiketa multzoak \cite{Muller2009}, $80$-biteko zenbaki formatua onartzen du; $64$-biteko mantisa eta $15$-biteko esponentea. \emph{IA-64} eragiketa multzoak, $82$-biteko formatu onartzen du; $64$-biteko mantisa eta $17$-biteko esponentea. $80$-biteko doitasuneko zenbakiak, $19$ zifra hamartarreko doitasuna dugu. 

Intel $x86$  prozesadoreek, esaterako Intel Xeon Phi \cite{IntelXeon2013}, doitasun arrunta eta bikoitza hardwarez inplementatzen dute; laukoitza, softwarez; eta $80$-biteko doitasuna ere hardwarez.

Laskar-ek \cite{Laskar2011,Laskar2015}, eguzki-sistemaren epe luzeko zenbakizko integratzeko, batura konpensatua eta $80$-biteko koma-higikorra erabili zituen. Integratzaile sinplektikoa C lengoaian berridatzi zuen, Intel-en $80$-biteko aritmetika erabili ahal izateko.
Era berean, efemerideen kalkuluetarako antzekoa gertatu da: EPM \cite{Pitjeva2014} (aplikazioa berridatzi zuten) eta INPOP \cite{Fienga2008} efemerideen kalkulutarako, $80$-biteko aritmetika erabiltzen dute.

Honek, astronomoentzat doitasuna hobetzeko duen garrantzia erakusten du. $128$-biteko aritmetikaren inplementazioa software bidezkoa da eta erabiltzeko garestiegia. $80$-biteko aritmetikarekin eta batura konpensatuarekin, zifra hamartar gutxi batzuk irabazten dute baina inplementazioen eraginkortasunarekiko eragin txikiarekin.
 
Gure IRK metodoen inplementazioak, doitasun bikoitzarekin garatu ditugu. Ez zaigu zuzena iruditu $80$-biteko aritmetika aplikatzea. \emph{IEEE-754} estandarretik kanpo dago eta hardwarearen menpekoa da. Gainera, Laskar-en \cite{Laskar2011} zenbakizko integrazioen arabera, konputazioak $x2$ garestitu ditu. 


\subsection*{Mixed-precision.}


Azken urteetan, konputazio kostu txikiarekin beharrezko doitasuna lortzeko, koma-higikorreko doitasunak konbinatzeko joera dago.

Doitasun arrunteko aritmetikak ($32$-bit), doitasun bikoitzarekiko $x2$ aldeko eraginkortasun erratio  du \cite{Dongarra2017}. Aplikazio batzuetarako, doitasun arrunteko doitasuna nahikoa da eta beste batzuetarako, konputazioaren funtsa doitasun arruntean egin daiteke eta galdutako digituak modu merkean berreskuratu.

Doitasun laukoitza ($128$-bit), software bidez modu errazean  inplementatu daiteke. Doitasun laukoitzeko zenbaki bat, doitasun bikoitzeko bi zenbakiren bidez adieraz daitezke eta eragiketek, doitasun bikoitzeko eragiketekiko $\mathcal{O}(10)$ ordenako kostua dute. Ideia da, konputazioa doitasun bikoitzean egitea eta doitasun laukoitzean integrazioaren kalkulu zehatz batzuk egitea. Zenbakizko integrazioetarako batura konpensatua era honetan interpretatu daiteke. 

Batura konpensatua merkea da. Eguzki-sistemaren gure inplementazioan batura konpensatua aplikatzeak abantaila berezia dakar: urrats bakoitzean gehitzen dugun balioa txikia denez, batura konpensatuarekin zifra hamartar asko mantentzen ditugu.            

\section{Paralelizazioa.}


Hasierako baliodun problemen,
\begin{equation}
 \label{eq:eztivp}
\dot{y}(t)=f(t,y(t)), \ f: \mathbb{R}^{d+1} \longrightarrow \mathbb{R}^d, \ \ y(t_0)=y_0,
\end{equation} 
integrazioetan, denboran zehar zenbakizko soluzioa sekuentzialki lortzen goaz eta beraz, ez dago zuzeneko paralelizazioa aplikatzerik. Hala ere, hainbat teknika proposatu dira \cite{Burrage1993}, eta etorkizunean, paralelizazio garapen berriak espero dira.   

Saha eta Tremainek \cite{Saha1996}, eguzki-sistemaren epe luzeko integratzeko, metodo paraleloa proposatu zuten. Konputazio paraleloa lortzeko, integrazio tarteen (1-astea, 2-astea,\dots) integrazioak prozesadoreen artean banatzen dituzte. Ondoren, teknika hau garatuz, Laskar-ek \cite{Jimenez-Perez2011} bere proposamena egin zuen. Hairer-ek \cite{Gander2014}, Hamiltondar sistemetan inplementazio hauen  aplikagarritasunaren analisia egin zuen eta hainbat eragozpen ondorioztatu zituen.

Ebatzi nahi den problemaren ekuazio diferentzialen balioztapena konputazionalki garestia bada eta dimentsio handikoa, paralelizazioa beste maila batean aplikatu daiteke. Paralelizazioa aplikatzeko bide bat, dimentsio handiko ekuazio-sistemaren balioztapena, osagai independentetan banatzea eta hauen konputazioa, prozesadore batzuen artean modu paraleloan egitea da. Problema zurruna denean, metodo inplizituak aplikatzen direnez, urrats bakoitzean, dimentsio handiko ekuazio-sistemak ebatzi behar dira eta horretarako, paralelizazioan oinarritutako konputazio eraginkorrak ditugu. Era berean, N-gorputz kopuru handiko problemaren gorputzen arteko interakzioak ($\mathcal{O}(N^2)$) kalkulatzeko         
paralelizazioan oinarritutako hainbat inplementazio \cite{Barnes1986,Carrier1988,Driscoll2013} aplika daiteke. 

IRK metodoei dagokienez, $f(Y_i), \ i=1,\dots,s$ ataletako funtzio balioztapena paraleloan exekutatu daiteke. Eguzki-sistemaren eredu errealistetan, $f$ funtzioa konplexua izango da eta beraz, paralelizazioa aplikatzeak abantaila izango du. Bestalde, doitasun bikoitzeko integrazioetarako, Gauss metodoaren inplementazio estandar eraginkorrena, $s=6$ atalekoa kontsideratzen da \cite{Hairer2006}. Gure eguzki-sistemaren inplementazio berrian, ordea, $s=?$ ataletako metodoa da eraginkorrena eta honek, paralelizio maila handiago suposatzen du.
      

\section{Beste sistema dinamikoak.}


\subsection*{Exoplanetak.}

Eguzki-sistemari egokitutako splitting metodo sinplektikoak \cite{Wisdom1991,Laskar2001}, eguzki-sistemaren problemaren bi ezaugarriei abantaila atereaz diseinatzen dira. Lehenik, planeten eguzkiarekiko orbitak ia Kepleriarrak dira. Bigarrenik, N-gorputzen problemaren formulazio Hamiltondarran oinarritzen dira eta Hamiltondarra zehazki mantentzen dute, sinplektikoak direlako. 

Exoplaneta sistemak integratzeko metodo berriak behar dira \cite{Fabrycky2010}. Exoplaneten planeta-sistemen mugimenduak ez dira eguzki-sistemarenak bezain egonkorrak. Exoplaneta sistemetan, eszentrikotasun handiko orbitak, periodo txikikoak, eta izar anitzeko planeta-sistemak aurkituko ditugu. Splitting metodo sinplektikoak, ez dira ondo egokitzen era honetako sistemak integratzeko. Gure eguzki-sistemaren inplementazioa, IRK Gauss kolokazio metodo orokorrean oinarritzen da eta ekuazio-sistemari Kepler fluxuaren aldagai aldaketa modu egokienean aplikatzen diogu. Problema hauetarako, gure inplementazioa baliagarria izan daitekeela pentsatzen dugu.         
 

\subsection*{Molekulen sistema dinamikoak.}

Molekulen dinamikaren problema Hamiltondarra da eta ekuazio diferentzialak era honetako idatzi daitezke,
\begin{align*}
&\dot{q}_i=v_i, \\
&\dot{v}_i=g(q_i), \ i=1,\dots,N,
\end{align*}
non $N$ partikulen kopurua eta $q_i,v_i \in \mathbb{R}^3$ diren.

Molekulen sistema dinamikoen integrazioak konputazionalki garestiak dira, $N$ balioa handia delako, $g$ funtzioaren konplexutasunagatik eta integrazioa epea luzea delako. Molekulen sistema dinamikoen integrazioetarako, \emph{Störmer-Verlet} (\ref{eq:stverlet}) metodo nagusienetakoa da. Sanz Sernak \cite{sserna1996}, Gauss metodoa molekulen sistema dinamikoen integraziorako eraginkortasuna aztertu zuen, eta \emph{Störmer-Verlet} metodoa eraginkorragoa zela ondorioztatu zuen.
Molekulen sistema dinamikoen integrazioetarako gure planteamenduak, ez duela interesik dirudi. 

