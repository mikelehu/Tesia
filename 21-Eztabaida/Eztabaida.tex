\chapter{Eztabaida.}

\section{Koordenatu sistemak.}

\section{Hamiltondar banagarriak.}

\section{IRK Newton.}

$(I_s \otimes I_d - h \ A \otimes J) \triangle Y = r$ ekuazio sistema modu eraginkorrean askatzeko, inplementazio estandarraren eta gure inplementazioen konplexutasunak konparatuko ditugu.

\subsection*{Inplementazio estandarra.}

Butcher edota Hairer-en inplementazio estandarraren konputazio bi modutan egin daiteke:

\begin{enumerate}
\item Zenbaki konplexuak.

$A$ matrizearen diagonalizazioak balio propio konplexuak ditu, 
\begin{equation*}
P^{-1}AP=\begin{bmatrix}
  \gamma_1 &                &          &                &           &                 \\
           & \bar{\gamma_1} &          &                &           &                 \\
           &                & \gamma_2 &                &           &                 \\
           &                &          & \bar{\gamma_2} &           &                 \\ 
           &                &          &                & \gamma_3   &                 \\
           &                &          &                &            & \bar{\gamma_3} \\  
\end{bmatrix},
\end{equation*}

eta ekuazio-sistema, zenbaki konplexuen aritmetika erabiliz ebatzi daiteke.
\begin{align*}
&(I-h \gamma_j J) \ X = b, \ \ j=1,\dots,3, \\
&(I-h \bar{\gamma}_j J) \ X = \bar{b}. 
\end{align*}

\item Zenbaki errealak.

Zenbaki konplexuekin ez bada lana egin nahi, zenbaki errealeko deskonposaketa baliokidea,
\begin{align*}
&\gamma_j=\alpha_j + i \ \beta_j,\\
&P^{-1}AP=\begin{bmatrix}
\alpha_{1} & -\beta_{1}   &  0            &  0            &  0           &    0       \\
\beta_{1}  & \alpha_{1}   & 0             &  0            &  0           &    0       \\
 0          & 0             & \alpha_{2}  & -\beta_{2}    &  0           &    0       \\
 0          & 0             & \beta_{2}   & \alpha_{2}    &  0           &    0       \\
 0          & 0             &  0          & 0             & \alpha_{3}  & -\beta_{3}  \\
 0          & 0             &  0          & 0             & \beta_{3}   & \alpha_{3}  \\
\end{bmatrix}
\end{align*}

\end{enumerate}

Hairer-en inplementazioan,
\begin{itemize}
\item $s$ bikoitia bada $\rightarrow$ $(2d \times 2d)$ tamainako $[s/2]$  LU deskonposaketa.
\item $s$ bakoitia bada $\rightarrow$ $(2d \times 2d)$ tamainako $(s+1)/2$  LU deskonposaketa.
\end{itemize}

\subsection*{Gure inplementazio berria.}

$\bar{A}$ matrizearen, $\bar{A}=P^{-1}DP$ diagonalizatzen dugu eta $D$ matrizeak, irudikari puruak ditu,
\begin{equation*}
\bar{A}=Q^{-1}RQ, \ \,
R=\begin{bmatrix}
0           & -\gamma_{1}   &            &               &             &           \\
 \gamma_{1} & 0             &            &               &             &           \\
            &               & 0           & -\gamma_{2}  &             &           \\
            &               & \gamma_{2}  & 0              &           &           \\
            &               &             &                & 0            & -\gamma_{3} \\
            &               &             &                & \gamma_{3}   & 0            \\
\end{bmatrix}
\end{equation*}

Gure inplementazioan,
\begin{itemize}
\item $s$ bikoitia bada $\rightarrow$ $(d \times d)$ tamainako $[s/2]+1$  LU deskonposaketa.
\item $s$ bakoitia bada $\rightarrow$ $(d \times d)$ tamainako $(s+1)/2$  LU deskonposaketa.
\end{itemize}

\subsection*{Konplexutasun konparaketa}

Lehenengo eragiketa aljebraikoen konplexutasunak gogoratuko ditugu,
\begin{align*}
&\text{LU deskonposaketa}:  \ \ 2s^3d^3/3+\mathcal{O}(d^2), \\
&\text{Back substitution}:  \ \ 2s^2d^2+\mathcal{O}(d), \\
&\text{inv}: \ \ 2s^3d^3
\end{align*}

Bi inplementazioen konplexutasunen laburpena,
\begin{table}[h!]
\caption[LU deskonposaketak] 
{\small{}}
\label{tab:Olu}       
\centering
{%
\begin{tabular}{ l l l l l } 
 \hline
\\
                 &  \multicolumn{2}{c}{LU}  & \multicolumn{2}{c}{Back Substitution}  \\
 s               & Estandarra  & Berria     &  Estandarra  &              Berria     \\
\\
 \hline
\\
 $2m$            &   $\frac{8m}{3} d^3$                              &  $\frac{2}{3} (2m+1) d^3$ 
                 &   $4m (2d^2)$      &     $(6m+4)d^2$                                             \\
 \\
 $2m+1$          &   $\left(\frac{8m}{3} + \frac{2}{3}\right) d^3$   &  $\left(\frac{4m}{3}+\frac{2}{3}\right) d^3$
                 &   $(8m+2)d^2$      &     $(6m+4)d^2$\\  
 \\  
   \hline
 \end{tabular}}
\end{table}