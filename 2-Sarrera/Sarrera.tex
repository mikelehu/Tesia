\chapter{Sarrera.}


\section{Ikerketaren testuingurua.}

Urte luzez, zientziaren arlo ezberdinek N-gorputzeko problema ikertu dute. Astronomoek eguzki-sistemaren planeten mugimendua ulertu nahian egindako lanak edo kimikariek erreakzio kimikoekin esperimentatzeko molekulen dinamikaren azterketak aipatu daitezke. Gainera,  N-gorputzen problemaren azterketak garrantzi berezia izan du matematikako eremu ezberdinen garapenean,  dinamika ez-lineal eta kaos teorian esaterako. 

Garai batean, N-gorputzen problemak teori analitikoen bidez aztertzen ziren baina konputagailuen sorrerarekin, zenbakizko integrazioak tresna nagusia bilakatu ziren. Azken hamarkadetan, bai konputazio teknologien aurrerapenari esker bai algoritmo berrien sorrerari esker, zenbakizko azterketek garapen handia izan dute. Zenbakizko simulazioen laguntzaz, eguzki-sistemaren dinamikaren funtsezko galdera batzuk ezagutu ditugu eta berriki, Karplus-en taldeak 2013. urteko kimika Nobel saria \cite{Karplus2014} jaso du kimika konputazionalean egindako lanarengatik.       

Guk lan honetan, N-gorputzen problema grabitazionala aztertuko dugu. Oro har eta gaia kokatzeko asmoarekin, N-gorputzen ohiko zenbakizko  integrazioak hiru taldetan sailka ditzakegu:
\begin{enumerate}
{
\item Epe motzeko eta doitasun handiko integrazioak. 
 Eguzki-sistemaren efemeride zehatzak \cite{Folkner2014} edo espazioko satelite artifizialen kokapenen \cite{Beylkin2014} kalkuluetarako erabili ohi dira.
\item Epe luzeko baina doitasun txikiko integrazioak.
 Denbora epe luzean, planeta-sistemen mugimendua ezagutzeko egindako ikerketak dira. Azterketa hauetan, helburua gorputzen mugimenduaren argazki orokorra (zehaztasun handirik gabe) ezagutzea da. Normalean, problema mota hauetan gorputzen arteko kolisioak edota kolisiotik gertuko egoerak ez dira izaten.     
\item N-gorputz kopurua edozein izanik, hauen arteko kolisioak gerta daitezkeen problemak.
 Integrazio hauetan, konplexutasun handiari aurre egin behar zaio. Gorputz kopurua miliotakoa \cite{Ishiyama2012} izan dateke eta kolisiotik gertuko egoeren ondorioz, kalkuluetan egindako zenbakizko errore txikiek soluzioan eragin handia izan dezakete.    
}
\end{enumerate}

Gure helburua, eguzki-sistemaren epe luzeko eta doitasun handiko integrazioetarako egoki izango den inplementazio eraginkorra garatzea da. Aurreko hamarkadetan, eguzki-sistemaren planeten epe luzeko zenbakizko integrazioa erronka garrantzitsua izan da. Adibidez, Sussman-ek eta Wisdom-ek \ycite[1993]{Sussman1992} eguzki-sistemaren 100 milioiko integrazioarekin, planeten mugimendua kaotikoa zela baieztatu zuten. Aldi berean, paleoklimatologi-zientzialariak orain milioika urte gertatutako klima zikloak (epel, hotz eta glaziazio aroak) azaltzeko, lurraren orbitan izandako aldaken eraginez gertatu zirela azaltzen duen teoria (Milankovitch 1941) \cite{Berger2012} baieztatzeko, planeten orbiten efemeride zehatzetan oinarritu dira.        

Epe luzeko integrazio hauetarako zenbakizko hainbat metodo erabiltzen dira, bereziki beren izaera Hamiltondarra mantentzen duten metodoak (metodo sinplektikoak).

Konputazio-teknologi aurrerapenak handiak izan arren, eguzki-sistemaren simulazio hauek konputazionalki oso garestiak dira eta exekuzio denbora luzeak behar dituzte; adibidez, Laskar-ek \ycite[2010]{Laskar2011} bere azken integrazioa burutzeko 18 hilabete behar izan zituen.
Azken urteotako konputagailu berrien arkitekturaren bilakaerak, algoritmo azkarren diseinua aldatu du: simulazioak azkartzeko algoritmoak, paralelizazioan oinarritu behar dira. Integrazio luze hauen erronka handienetako bat, biribiltze errorearen garapena zaintzea da. Biribiltze errore sistematikoaren hedapenak, errore globalean eragindako joerak ekidin behar dira \cite{Laskar2015}.
 
\section{Motibazioa.}
\label{intro}


Metodo sinplektikoen artean erabilienak, izaera esplizituko algoritmoak dira. Oro har problema zurruna ez bada, metodo esplizituak  metodo inplizituak baino eraginkorragoak dira. Metodo inplizituetan ekuazio sistema ez-lineala askatu behar da (eragiketa garestia) eta honek, metodo esplizituekiko CPU denbora gainkarga suposatzen du. Hala ere, ebatzi beharreko problema zurruna bada, metodo esplizituak urrats oso txikiak eman behar izaten ditu integrazio fidagarriak lortu ahal izateko. Horrek ere, integrazioa garestitzen du. Metodo inplizituetan ez da halakorik gertatzen, urrats luzeagoak eman ditzakete nahiz eta problema zurruna izan. 

Azken aldian, ordea, ezbaian jarri da problemaren zurruntasunaren araberako metodoen aukeraketarako joera hori. Lan honetan, zenbakizko integraziorako Gaussen metodo inplizitu sinplektikoaren azterketa egingo dugu. Hainbat autorek (Hairer \cite{Hairer2006,Hairer2008} eta Sanz Serna\cite{JMSanz-Serna1994}) metodo honen potentziala nabarmendu dute. Azken urtetan, espazioko satelite artifizialen arloan ere, Gaussen integrazio metodoarekiko interesa azaldu dute \cite{Bradley2014,Beylkin2014}. 

Gaussen integrazio metodo inplizituen abantaila nagusienetakoa malgutasuna da. Ekuazio inplizituak ebazteko, teknika ezberdinak konbinatu daitezke eta ondorioz, integratu nahi dugun problemari egokitzeko eta eraginkortasuna hobetzeko aukera asko eskaintzen dizkigu.

Sinplektikoak diren metodo esplizituak oso eraginkorrak direla ezin da ukatu, baina metodo hauen erabilera ez da beti posible: sistema Hamiltondar banagarrietan bakarrik erabil daitezke. Sistema Hamiltondar orokorrak edota lehen ordenako ekuazio diferentzialeko sistemak integratzeko metodo sinplektikoak, inplizituak izan behar dute. Bestalde, Gauss metodoak paralelizagarriak dira, hau da, ekuazio diferentzial konplexuak kalkulatu behar ditugunean, $s$-ataletako funtzio konputazioak paraleloan exekutatu daitezke. Azkenik ez dugu ahaztu behar, ordena altuko Gauss metodoak existitzen direla  eta hauek beharrezkoak ditugula doitasun handiko integrazioetarako.     

\paragraph*{}Atal hau bukatzeko, Sanz Sernaren  \cite[1992]{Sanz-Serna1992} hitzak berreskuratuko ditugu. 
\begin{displayquote}
On the other hand, little has been undertaken in the construccion of practical high-order methods and the design of serious symplectic software is still waiting consideration.
\end{displayquote}

\paragraph*{} V.A. Brumberg-ek \cite[2012]{Brumberg2013} lanean, eguzki-sistemaren epe luzeko simulazioak era honetan deskribatzen ditu.
\begin{displayquote}
Numerical integration of the equations of motion of celestial bodies over a long interval of time is also not a trivial problem. Analytical and numerical techniques of celestial mechanics have been permanently improved over the history of celestial mechanics. In its turn, it was a stimulatory for many branches of mathematics (the theory of special functions, linear algebra, differential equations, theory of approximation, etc.).
\end{displayquote}  

\section{Helburua eta esparrua.}

Gure helburua, eguzki-sistemaren epe luzeko integraziorako Gaussen metodo inplizituaren inplementazio eraginkorra proposatzea edota, bide horretan aurrerapausoak ematea da. Helburu hau lortzeko, honako aspektu hauek bereziki zainduko ditugu: eguzki-sistemaren problemaren ezaugarriak, biribiltze erroreen garapena eta egungo konputagailuen gaitasunari egokitutako algoritmo azkarren diseinua.  

N-gorputzeko problema grabitazionalari dagokionez, eguzki-sistemaren eredu sinplea integratuko dugu. Eguzki-sistemaren gorputzak masa puntualak kontsideratuko ditugu eta gure ekuazio diferentzialek, gorputz hauen arteko erakarpen grabitazionalak bakarrik kontutan hartuko dituzte. Beraz, eguzki-sistemaren eredu konplexuagoetako erlatibitate efektua, gorputzen formaren eragina, eta beste zenbait indar ez-grabitazionalak ez ditugu kontutan hartu.

Zeintzuk dira eguzki-sistemaren problemaren ezaugarri bereziak? Batetik, planeten mugimendu orbitala, perturbazio txikiak dituen mugimendu Kepleriarra da. Beraz, mugimendu Kepleriarra  zehazki kalkula daitekeenez, eguzki-sistemaren planeten orbiten konputazioaren oinarria da. Bestetik,  badugu gorputz nagusi bat (eguzkia) eta honen inguruan mugimenduan dauden planetak, bi multzotan bana ditzakegu: barne-planetak, masa txikikoak eta eguzkitik gertu daudenak eta kanpo-planetak, masa handikoak eta eguzkitik urrun daudenak. Kanpo-planeten eboluzioan, barne-planetak eragin oso txikia daukate, eragina, masaren eta distantziaren alderantzizkoaren proportzionala baita.  Eguzki-sistema egonkorra kontsideratzen da, hau da, hurrengo bilioi urteetan planeten arteko talkarik ez da espero gertatzea. Orbiten denbora eskalak anitzak dira; ilargiaren lurraren inguruko orbitaren periodoa $27.32$ egunetakoa, lurraren eguzkiaren ingurukoa $1$ urtekoa eta Neptunorena $163$ urtekoa.  Eguzki-sistemaren egitura aberats honi, abantaila gehien ateratzen dion planteamendua bilatuko dugu.
  
Konputagailuen koma-higikorreko aritmetika ondo ulertzea garrantzitsua da. Zenbaki errealen adierazpen finitua erabiltzen denez, bai zenbakiak memorian gordetzerakoan, bai hauen arteko kalkulu aritmetikoak egiterakoan, biribiltze errorea sortzen da. Integrazio luzeetan, biribiltze errorea hedatu egiten da eta une batetik aurrera, soluzioen zuzentasuna ezereztatzen da. Zentzu honetan, doitasuna hobetzeko biribiltze errorea gutxitzen duten teknika bereziak aplikatzea ezinbestekoa izaten da. Integrazio luzeetan, maiz doitasun handian lan egiteko aukera aipatzen da, baina doitasun altuko aritmetikaren ($128$-bit) inplementazioa software bidezkoa denez, oso motela da eta ez da erabilgarria. Exekuzio denbora onargarriak lortzeko tarteko irtenbideak landu behar dira, esate baterako, doitasun ezberdinak nahasten dituzten inplementazioak.       

Konputazioko teknologiaren garapenean, algoritmo azkarren diseinua baldintzatzen duten bi ezaugarri azpimarratu behar dira. Batetik, konputagailuak orokorrean paraleloak dira eta algoritmo azkarrak garatzeko, kodearen paralelizazio gaitasunari heldu behar zaio. Bestetik, konputazioaren alde garestiena, memoria eta prozesadorearen arteko datu mugimendua denez, prozesadorearen konputazio handiena komunikazio txikienarekin lortu behar da. 

Sarrera honetan paralelizazioari buruzko ohar batzuk ematea komeni da. Algoritmo baten kode unitateak paraleloan exekutatzeak badu gainkarga bat eta beraz,  algoritmoaren exekuzioa paralelizazioaz azkartzea lortzeko,  unitate bakoitzaren tamainak esanguratsua izan behar du. Gure eguzki-sistemaren eredua sinplea da eta logikoa da pentsatzea eredu konplexuagoetan, paralelizazioak abantaila handiagoa erakutsiko duela. Bestalde, gorputzen kopurua handia den problemetan, hauen arteko interakzio kopuru  handia kalkulatu behar da ($\mathcal{O}(N^2)$) eta indar hauen hurbilpena modu eraginkorrean kalkulatzeko metodo ezagunak daude: \textit {tree code}\cite{Barnes1986} eta \textit {fast multipole method}\cite{Carrier1988} izeneko metodoak. Baina gure probleman gorputz kopurua txikia denez, teknika hauek gure eremutik kanpo utzi ditugu. 


\section{Ekarpenak.}

Tesiaren lana hiru ataletan banatu dugu. Lehen urratsean, Gauss metodoaren urratsa emateko puntu-finkoaren iterazioaren  bidezko inplementazioa aztertu dugu eta gure inplementazioen oinarriak finkatu ditugu. Bigarren fasean, Gauss metodoaren urratsa emateko, Newton iterazioaren bidezko inplementazio eraginkorra lortzeko ahalegin berezia egin dugu. Problema zurruna denean, puntu-finkoaren iterazio ez da eraginkorra eta Newtonen iterazioa aplikatu behar da. Gainera problema ez-zurruna izanik ere, Newton iterazioak interesgarriak izan daitezke; bereziki doitasun altuko (doitasun laukoitza) konputazioetan, metodoaren konbergentzia ezaugarri onak direla-eta.  Hirugarren fasean ...

Jarraian, atal bakoitzean egindako ekarpen nagusienak laburtuko ditugu:

\begin{enumerate}
\item IRK puntu finkoa.

Gauss metodoaren puntu finkoaren inplementazioaren azterketa sakon bat egin dugu eta horretarako, Hairer-en inplementazioa \cite{Hairer2008} hartu dugu gure lanaren abiapuntua. Kalitatezkoa inplementazio hau hobetzeko aukerak ikusi ditugu eta inplementazio sendoago bat proposatu dugu. Gure ekarpenak hauek izan dira:  

\begin{enumerate}
\item Metodoaren birformulazioa.

Gauss inplizitua aplikatzen dugunean, metodoa definitzen duten biribildutako koefiziente errealak ($\tilde{a}_{ij}, \tilde{b}_i \in \mathbb{F}$) erabiltzen dira. Formulazio estandarra erabiliz, koefiziente hauek ez dute metodoa sinplektikoa izateko baldintza zehazki betetzen eta beraz, izaera sinplektikoaren propietate onak galtzen dira. Metodoaren birformulazio baliokide bat proposatu dugu, horrela sinplektizidade baldintza zehazki betetzen duten koefizienteak modu errazean finkatu daitezke. Hori dela-eta, Gauss metodoak integral koadratikoak kontserbatuko ditu.

\item Geratze irizpide berria.

Orokorrean, Hairer-en inplementazioaren puntu finkoaren iterazioaren geratze irizpidea zuzena dela ikusi dugu baina kasu batzuetan goizegi geratzen dela baieztatu dugu. Arrazoiak bi direla ikusirik, bere geratze irizpidea bi zentzutan garatu/zorroztu dugu. Lehenik, Hairer-en inplementazioan, hobekuntza neurketa ataletako diferentziaren norma batean oinarritzen da. Normarekiko independentea den geratze irizpidea aplikatzea zuzenagoa da, eta horregatik, ataletako edozein osagaiaren diferentzia txikitzen den bitartean iterazioak egiten jarraitzea finkatu dugu. Bigarrenik, iterazioetan osagai guztien hobekuntza ez du zertan beherakorra izan behar, eta okertzen diren tarteko iterazioak gerta daitezke. Arazo hau gainditzeko, iterazioren batean osagai guztien diferentzia handitzea gertatzen denean, seguritateko bi iterazio gehigarri emango ditugu, iteraziotik irten aurretik.   

\item Atalen espresioaren aldaketa.

Integrazioan batura konpensatu estandarra aplikatzen dugunean, zenbakizko soluzioa $\tilde{y}_n, e_n \in \mathbb{F}^d$ lortzen dugu non $\tilde{y}_n+e_n \approx y(t_n)$ den. Hori dela-eta, metodoaren atalen espresioan $\tilde{y}_n$-ren ordez, $\tilde{y}_n+e_n$ erabiltzea proposatu dugu. Aldaketa honekin, zenbakizko soluzioaren doitasuna zerbait hobetuko dela espero da.

\begin{equation*}
Y_{n,i}=y_n + \left(e_n+ \sum_{j=1}^{s}\mu_{ij} L_{n,j} \right).
\end{equation*}
  
\item Biribiltze errorearen estimazioa.

Zenbakizko soluzioaren $\tilde{y}_n+e_n \approx y(t_n), \ n=1,2,\dots$ biribiltze errorearen estimazioa, doitasun txikiagoko bigarren zenbakizko soluzioaren $\hat{y}_n+\hat{e}_n \approx y(t_n), \ n=1,2,\dots$  diferentzia gisa kalkulatuko dugu.
Erabiltzaileari zenbakizko soluzioaren estimazioa ezagutzeko, exekuzio bakarrean  eta \emph{CPU} gainkarga txikiarekin, bi integrazioak sekuentzialki kalkulatzeko aukera eskainiko zaio. 


\end{enumerate}


\item IRK Newton.

\begin{enumerate}
\item Ekarpen nagusia.

$S$-ataletako IRK metodoa,  Newton iterazioaren bidez $d$-dimentsioko ekuazio diferentzial  sistemari aplikatzeko, urrats bakoitzean $sd \times sd$ tamainako hainbat ekuazio sistema (iterazio bakoitzeko bat) askatu behar dira. Atal honetan, jatorrizko $sd$-dimentsioko ekuazio sistema, $(s+1)d$ dimentsioko ekuazio-sistema baliokide moduan berridatzi dugu. Ekuazio-sistema baliokidea,  $d \times d$ tamainako $[s/2]+1$ matrize errealen \emph{LU}-deskonposaketa bidez askatuko dugu. Tamaina txikiko matrizeen LU deskonposaketa azkarra denez, konputazionalki eraginkorra izatea espero dugu.   

\item Doitasun laukoitzeko inplementazioa.

Doitasun laukoitzeko exekuzioetan, funtzio balioztapena oso garestia da eta funtzio balioztapenen kopurua gutxitzea bilatuko dugu. Newton osoa aplikatzen dugunean pena mereziko du, Gaussen Newton iterazioko inplementazioaren ekuazio lineala askatzeko metodo iteratiboa aplikatzea. 

\item Newton mixtoa.


\end{enumerate}
  

\item IRK Eguzki-sistema.

Hirugarren urratsean, eguzki-sistemaren epe luzeko integrazioan arituko gara. Ekarpen handiena, atalen hasieraketa berri bat aplikatzea da alde Kepleriarraren fluxuan oinarrituz. IRK metodoak eskaintzen digun malgutasunari esker eta N gorputzetako problema grabitazionalaren ezaugarriez baliatuz inplementazio ezberdinak egin ditugu. Inplementazio hauen eraginkortasuna, egungo integratzaile sinplektiko esplizituekin konparatu ditugu.

\item Birparametrizazioa.

Azken urratsean, esperimentalki, eguzki-sistemaren integrazioan  denboraren birparametrizazio teknikaren aplikazio sinple bat erakutsiko dugu. Integratzaile sinplektikoak luzera finkoko urratsa eduki behar du eta zentzu honetan, birparametrizazioa eraginkortasuna hobetzeko beste bide bat da.      

\end{enumerate}        


\section{Tesiaren egitura.}

Tesiaren lehenengo $(1-5)$ kapituluetan, zenbakizko integrazio sinplektikoak eta zientzia konputazionalaren oinarriak azaldu ditugu. Bigarren kapituluaren lehen zatian, \emph{Zenbakizko integratzaile sinplektikoen} inguruko oinarrizko kontzeptuak azaldu ditugu. Kapitulu honen bigarren zatian, Gauss metodo estandarra deskribatu eta bere propietate nagusienak eman ditugu. Kapitulu honen azken zatian, eguzki-sistemaren integraziorako metodo sinplektiko eta esplizitu nagusienak laburtu ditugu. Hirugarren kapituluan, lan honetan zenbakizko esperimentuetan erabili ditugun hasierako baliodun problemen zehaztasunak eman ditugu. Laugarren kapituluan koma-higikorreko aritmetikan murgildu gara eta biribiltze errorearen inguruko gaiak argitu nahi izan ditugu. Bostgarren kapituluan, egungo konputazio zientziaren hardware eta software kontzeptu nagusienak ezagutarazi nahi izan ditugu.     

Tesiaren $(6-8)$ kapituluetan, gure inplementazio berriak garatu ditugu eta zenbakizko esperimentuen bidez, hauen eraginkortasuna erakutsi dugu. Lehenik, $6$.~kapituluan Gauss metodoaren puntu finkoaren iterazioaren inplementazio berria eman dugu. Ondoren, $7$.~kapituluan Gauss metodoaren Newtonen iterazioan oinarritutako inplementazio eraginkorrak azaldu ditugu. $8$.~kapituluan, eguzki-sistemaren integraziorako inplementazio deskribapena egin dugu.  

Tesiaren $(9-10)$ kapituluetan, gure hipotesiaren eztabaida eta lanaren konklusioak idatzi ditugu.

Tesiaren bukaeran, hiru eranskinetan lanaren informazio osagarria bildu dugu. A-eranskinean, tesian zehar erabilitako hainbat frogen zehaztapenak eman dira. B-eranskinean, garatutako kodeak bildu eta erabiltzaileari erabilgarri izan dakiokeen informazioa laburtu dugu. Azkenik C-eranskinean, erabilitako notazioaren inguruko argibideak eman ditugu.

      
      
\section{Laburpena}

