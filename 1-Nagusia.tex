\documentclass[12pt,a4paper]{basque-book}
\usepackage[utf8]{inputenc}


\usepackage{times}
\usepackage[basque]{babel}
\renewcommand{\=}{"-}
\selectlanguage{basque}
\input formatoa
%\usepackage[latin1]{inpuntenc}
\usepackage[T1]{fontenc}
\usepackage{graphicx}
% Mikel : Nik gehitutako paketeak
%\usepackage[english]{babel}
\graphicspath{ {0-title/figures/}
               {2-Sarrera/figures/}
               {3-MetodoSinplektikoak/figures/} 
               {4-Problemak/figures/}
               {5-KomaHigikorra/figures/} 
%               {6-ZientziaKonputazioa/figures/} 
               {7-IRKPuntuFinkoa/figures/}
               {9-IRKNewton/figures/}
               {10-IRKEguzkiSistema/figures/}
               {21-Eztabaida/figures/}
               {22-Konklusioak/figures/}            
%               /home/joseba/Mahaigaina/PIC/tesia/dokumentuak/nagusia/22-Konklusioak/figures             
               {A-Eranskina/figures/}
               {B-Eranskina/figures/}
               {C-Eranskina/figures/}}
\usepackage{listings}   % kodea azaltzeko
%\usepackage[ruled]{algorithm2e}
%\usepackage[linesnumbered,ruled,vlined]{algorithm2e}
\usepackage{algorithm2e}
\SetAlFnt{\footnotesize}
\SetAlCapFnt{\small}
\SetAlCapNameFnt{\small}


\renewcommand{\algorithmcfname}{Algoritmoa}
\usepackage {amsmath,amsfonts,amssymb}
\usepackage{subfig}
\usepackage{tikz}
\usepackage{epigraph}
\usepackage{csquotes}

%Higham (bibliostyle)
\usepackage{texnames}
\usepackage{url}
\usepackage{color}
\usepackage{graphicx}
\definecolor{hotpink}{rgb}{0.9,0,0.5}
\usepackage[colorlinks,urlcolor=blue,citecolor=hotpink,linkcolor=blue]{hyperref}
\def\ycite[#1#2#3#4#5]#6{\cite[$\mit{#1#2#3#4}$#5]{#6}}

\usepackage{makeidx}
\makeindex

% End-Mikel

% ShareLatex (listings)
\definecolor{codegreen}{rgb}{0,0.6,0}
\definecolor{codegray}{rgb}{0.5,0.5,0.5}
\definecolor{codepurple}{rgb}{0.58,0,0.82}
\definecolor{backcolour}{rgb}{0.95,0.95,0.92}
 
\lstdefinestyle{mystyle}{
    backgroundcolor=\color{backcolour},   
    commentstyle=\color{codegreen},
    keywordstyle=\color{magenta},
    numberstyle=\tiny\color{codegray},
    stringstyle=\color{codepurple},
    basicstyle=\footnotesize,
    breakatwhitespace=false,         
    breaklines=true,                 
    captionpos=b,                    
    keepspaces=true,                 
%    numbers=left,                    
    numbersep=5pt,                  
    showspaces=false,                
    showstringspaces=false,
    showtabs=false,                  
    tabsize=2
}
\lstset{style=mystyle}
% End ShareLatex (listings)

\pagenumbering{roman} %Required by SGPS

\newcommand{\M}{\mathcal{M}}
\newcommand{\N}{\mathcal{N}}
\renewcommand{\P}{\mathcal{P}}
\newcommand{\Q}{\mathcal{Q}}
\newcommand{\F}{\mathbb{F}}
\newcommand{\fl}{\mathrm{fl}}
\newcommand{\R}{\mathbb{R}}
\renewcommand{\algorithmcfname}{Algoritmoa}
\renewcommand{\leq}{\leqslant}
\renewcommand{\geq}{\geqslant}
\renewcommand{\dim}{d}
\newcommand{\eps}{\varepsilon}
\newcommand{\half}{\textstyle \frac12}

\begin{document}


%: ----------------------- generate cover page ------------------------

\frontmatter
%Title page
%%%%%%%%%%%%%%%%%%%%%%%%%%%%%%%%%%%%%   frontpage  %%%%%%%%%%%%%%%%%%%%%%%%%%%%%%%%%%%%%%%%%%%%%%%5
\begin{titlepage}
\newcommand{\HRule}{\rule{\linewidth}{0.5mm}}
\begin{center}
%\includegraphics[width=0.3\textwidth]{uwo.eps}\\[1cm]    
\HRule \\[0.4cm]
{ \Large \bfseries \sc N-Gorputzeko problema grabitazionalaren ebazpenerako zenbakizko metodoen azterketa.}\\[0.4cm]
% maximum of 60 characters for spine title
%{(Spine title: Backward Error Analysis as a Model of Computation)\\ (Thesis Format: Integrated-Article)}

\HRule \\[1cm]
%by
%\\[1cm]
{\Large Mikel Antonana Otano}
\\[.75cm]


%{\large Graduate Program in Applied Mathematics}
%\\[1.2cm]
%{\large A thesis submitted\\ in partial fulfillment of the requirements for\\ the degree of Master of Science}
%\\[.5cm]
{\large Konputazio zientzia eta Adimen Artifiziala saila \\[.1cm]
}
{\large Informatika fakultatea \\[2.cm]
}


\includegraphics[scale=1.5]{Ehu-Logoa}
\\[5cm]

{\large Doktorego tesia\\[.1cm]}
{\large Donostia 2017}




%\copyright Mikel Antonana 2017
\end{center}
\end{titlepage}\pagebreak %title page in a subfile. Optional, but looks nicer.
%\addtocounter{page}{1}

%\title{N-Gorputzeko problema grabitazionalaren ebazpenerako zenbakizko metodoen azterketa.}

%\author{ Mikel Antoñana Otaño}
%\date{Otsaila 2014}

%\begin{titlepage}
%\maketitle
%\end{titlepage}

\chapter*{}
%tesi zuzendariak

\begin{center}


%\\[3cm]

%\newcommand{\HRule}{\rule{\linewidth}{0.5mm}}
%\HRule \\[1cm]
%
%\begin{tabular}{c c}
%{\Large Zuzendariak:} & {\Large Ander Murua Uria}\\[.1cm]
%                      & {\quad \quad \quad \Large Joseba Makazaga Odria} \\[.1cm]\\                  
%\end{tabular}

{\Huge Zuzendariak}\\[.8cm]
%\HRule \\[1cm]
%
{\Large Ander Murua Uria}\\[.1cm] 
{\Large \ \ Joseba Makazaga Odria}\\[1.cm]
{\large Konputazio Zientzia eta Adimen Artifiazial Saila} \\[.1cm]
{\large Euskal-Herriko Unibertsitatea}
\\[.75cm]
%
\end{center}

\chapter*{}
\epigraph{Itxaropena ez da dena ondo aterako den konbentzimendua; nolanahi ateratzen dela ere, egiten dugunak zentzua duen ziurtasuna baizik.}
{\textit {Vaclav Havel}}


%\\[5cm]                   



\chapter*{}

\section*{Hitzaurrea}


Eskuartean duzun ikerketa hau, bide luze baten emaitza da. Hitzaurre honetan, ez naiz gehiegi luzatuko eta soilik, irakurlea animatu nahiko nuke. Tesiaren izenburua, ziurrenik ez da oso erakargarria baina irakurleak aurrera jarraitzea erabakitzen badu, ez da damutuko. 
Kontzeptu batzuk ulertzeko zailak izan daitezkeela badakigu, baina lan honetan batez ere irudimena aplikatu dugu eta oinarrian, ekarpenak sinpleak dira.

Ikerketaren nire esperientzia azaldu nahiko nuke eta horretarako, gustuko dudan bi ideietan lagunduko naiz. Batetik, Bernardo Atxagari entzundako elkarrizketa batean, zientziarekin zuen harremanari buruzko galderi erantzunez zera esan zuen: ''zientziarekin dudan harremana, zopa beroarekin dudan harreman bera da, kutxara zoparen erdi-erdian sartu beharrean, ertzetara jotzen dut eta hortxe sartzen dut nire kutxara txikia''. Bestetik, Elena Cattaneo ikerlari italiarrak Donostian emandako hitzaldi batean, ikerketa, desertuan egotearekin parekatu zuen: ''lur eremu zabal-zabal baten aurrean aurkitzen zara, nora jo  ez dakizun leku batean eta ikusten duzun gauza bakarra, denborak ia ezabatutako norbaiten arrasto  batzuk. Nire ustez, bi analogia hauek ezin hobeto deskribatzen dute zer den zientzia eta ikerketa.

Bukatzeko, Ander eta Josebari beraien ikerketan lan egiteko aukera ematea eskertu nahi diet. Ikerketa honetan, batzuetan gauzak ondo joan zaizkigu eta beste batzuetan ez dugu asmatu, baina garrantzitsuena da, lan zintzoa aurrera eraman dugula.
 
\vspace{20mm}

\begin{tabular}{l}
Mikel Antonana Otano\\
2017 
\end{tabular}


%\include{01-hitzaurrea/hitzaurrea}

%: ----------------------- contents ------------------------

\setcounter{secnumdepth}{1} % organisational level that receives a numbers
\setcounter{tocdepth}{0}    % print table of contents for level 3


{\expandafter\def\csname @starttoc\endcsname#1{\InputIfFileExists{\jobname.#1}{}{}}%
\tableofcontents}


\renewcommand*\contentsname{Gaien Aurkibidea}

\setcounter{tocdepth}{1}    % print table of contents for level 3
\tableofcontents



%: ----------------------- list of figures/tables ------------------------

%\newpage
%\listoffigures
%\newpage
%\listoftables
%\newpage



%\begin{abstract}  
 
%\keywords{First keyword \and Second keyword \and More}
% \PACS{PACS code1 \and PACS code2 \and more}
% \subclass{MSC code1 \and MSC code2 \and more}
%\end{abstract}


%: --------------------------------------------------------------
%:                  MAIN DOCUMENT SECTION
% --------------------------------------------------------------

\mainmatter





\part{Sarrera}
\chapter{Sarrera.}


\section{Ikerketaren testuingurua.}

Urte luzez, zientziaren arlo ezberdinek N-gorputzeko problema ikertu dute. Astronomoek eguzki-sistemaren planeten mugimendua ulertu nahian egindako lanak edo kimikariek erreakzio kimikoekin esperimentatzeko molekulen dinamikaren azterketak aipatu daitezke. Gainera,  N-gorputzen problemaren azterketak garrantzi berezia izan du matematikako eremu ezberdinen garapenean,  dinamika ez-lineal eta kaos teorian esaterako. 

Garai batean, N-gorputzen problemak teori analitikoen bidez aztertzen ziren baina konputagailuen sorrerarekin, zenbakizko integrazioak tresna nagusia bilakatu ziren. Azken hamarkadetan, bai konputazio teknologien aurrerapenari esker bai algoritmo berrien sorrerari esker, zenbakizko azterketek garapen handia izan dute. Zenbakizko simulazioen laguntzaz, eguzki-sistemaren dinamikaren funtsezko galdera batzuk ezagutu ditugu eta berriki, Karplus-en taldeak 2013. urteko kimika Nobel saria \cite{Karplus2014} jaso du kimika konputazionalean egindako lanarengatik.       

Guk lan honetan, N-gorputzen problema grabitazionala aztertuko dugu. Oro har eta gaia kokatzeko asmoarekin, N-gorputzen ohiko zenbakizko  integrazioak hiru taldetan sailka ditzakegu:
\begin{enumerate}
{
\item Epe motzeko eta doitasun handiko integrazioak. 
 Eguzki-sistemaren efemeride zehatzak \cite{Folkner2014} edo espazioko satelite artifizialen kokapenen \cite{Beylkin2014} kalkuluetarako erabili ohi dira.
\item Epe luzeko baina doitasun txikiko integrazioak.
 Denbora epe luzean, planeta-sistemen mugimendua ezagutzeko egindako ikerketak dira. Azterketa hauetan, helburua gorputzen mugimenduaren argazki orokorra (zehaztasun handirik gabe) ezagutzea da. Normalean, problema mota hauetan gorputzen arteko kolisioak edota kolisiotik gertuko egoerak ez dira izaten.     
\item N-gorputz kopurua edozein izanik, hauen arteko kolisioak gerta daitezkeen problemak.
 Integrazio hauetan, konplexutasun handiari aurre egin behar zaio. Gorputz kopurua miliotakoa \cite{Ishiyama2012} izan dateke eta kolisiotik gertuko egoeren ondorioz, kalkuluetan egindako zenbakizko errore txikiek soluzioan eragin handia izan dezakete.    
}
\end{enumerate}

Gure helburua, eguzki-sistemaren epe luzeko eta doitasun handiko integrazioetarako egoki izango den inplementazio eraginkorra garatzea da. Aurreko hamarkadetan, eguzki-sistemaren planeten epe luzeko zenbakizko integrazioa erronka garrantzitsua izan da. Adibidez, Sussman-ek eta Wisdom-ek \ycite[1993]{Sussman1992} eguzki-sistemaren 100 milioiko integrazioarekin, planeten mugimendua kaotikoa zela baieztatu zuten. Aldi berean, paleoklimatologi-zientzialariak orain milioika urte gertatutako klima zikloak (epel, hotz eta glaziazio aroak) azaltzeko, lurraren orbitan izandako aldaken eraginez gertatu zirela azaltzen duen teoria (Milankovitch 1941) \cite{Berger2012} baieztatzeko, planeten orbiten efemeride zehatzetan oinarritu dira.        

Epe luzeko integrazio hauetarako zenbakizko hainbat metodo erabiltzen dira, bereziki beren izaera Hamiltondarra mantentzen duten metodoak (metodo sinplektikoak).

Konputazio-teknologi aurrerapenak handiak izan arren, eguzki-sistemaren simulazio hauek konputazionalki oso garestiak dira eta exekuzio denbora luzeak behar dituzte; adibidez, Laskar-ek \ycite[2010]{Laskar2011} bere azken integrazioa burutzeko 18 hilabete behar izan zituen.
Azken urteotako konputagailu berrien arkitekturaren bilakaerak, algoritmo azkarren diseinua aldatu du: simulazioak azkartzeko algoritmoak, paralelizazioan oinarritu behar dira. Integrazio luze hauen erronka handienetako bat, biribiltze errorearen garapena zaintzea da. Biribiltze errore sistematikoaren hedapenak, errore globalean eragindako joerak ekidin behar dira \cite{Laskar2015}.
 
\section{Motibazioa.}
\label{intro}


Metodo sinplektikoen artean erabilienak, izaera esplizituko algoritmoak dira. Oro har problema zurruna ez bada, metodo esplizituak  metodo inplizituak baino eraginkorragoak dira. Metodo inplizituetan ekuazio sistema ez-lineala askatu behar da (eragiketa garestia) eta honek, metodo esplizituekiko CPU denbora gainkarga suposatzen du. Hala ere, ebatzi beharreko problema zurruna bada, metodo esplizituak urrats oso txikiak eman behar izaten ditu integrazio fidagarriak lortu ahal izateko. Horrek ere, integrazioa garestitzen du. Metodo inplizituetan ez da halakorik gertatzen, urrats luzeagoak eman ditzakete nahiz eta problema zurruna izan. 

Azken aldian, ordea, ezbaian jarri da problemaren zurruntasunaren araberako metodoen aukeraketarako joera hori. Lan honetan, zenbakizko integraziorako Gaussen metodo inplizitu sinplektikoaren azterketa egingo dugu. Hainbat autorek (Hairer \cite{Hairer2006,Hairer2008} eta Sanz Serna\cite{JMSanz-Serna1994}) metodo honen potentziala nabarmendu dute. Azken urtetan, espazioko satelite artifizialen arloan ere, Gaussen integrazio metodoarekiko interesa azaldu dute \cite{Bradley2014,Beylkin2014}. 

Gaussen integrazio metodo inplizituen abantaila nagusienetakoa malgutasuna da. Ekuazio inplizituak ebazteko, teknika ezberdinak konbinatu daitezke eta ondorioz, integratu nahi dugun problemari egokitzeko eta eraginkortasuna hobetzeko aukera asko eskaintzen dizkigu.

Sinplektikoak diren metodo esplizituak oso eraginkorrak direla ezin da ukatu, baina metodo hauen erabilera ez da beti posible: sistema Hamiltondar banagarrietan bakarrik erabil daitezke. Sistema Hamiltondar orokorrak edota lehen ordenako ekuazio diferentzialeko sistemak integratzeko metodo sinplektikoak, inplizituak izan behar dute. Bestalde, Gauss metodoak paralelizagarriak dira, hau da, ekuazio diferentzial konplexuak kalkulatu behar ditugunean, $s$-ataletako funtzio konputazioak paraleloan exekutatu daitezke. Azkenik ez dugu ahaztu behar, ordena altuko Gauss metodoak existitzen direla  eta hauek beharrezkoak ditugula doitasun handiko integrazioetarako.     

\paragraph*{}Atal hau bukatzeko, Sanz Sernaren  \cite[1992]{Sanz-Serna1992} hitzak berreskuratuko ditugu. 
\begin{displayquote}
On the other hand, little has been undertaken in the construccion of practical high-order methods and the design of serious symplectic software is still waiting consideration.
\end{displayquote}

\paragraph*{} V.A. Brumberg-ek \cite[2012]{Brumberg2013} lanean, eguzki-sistemaren epe luzeko simulazioak era honetan deskribatzen ditu.
\begin{displayquote}
Numerical integration of the equations of motion of celestial bodies over a long interval of time is also not a trivial problem. Analytical and numerical techniques of celestial mechanics have been permanently improved over the history of celestial mechanics. In its turn, it was a stimulatory for many branches of mathematics (the theory of special functions, linear algebra, differential equations, theory of approximation, etc.).
\end{displayquote}  

\section{Helburua eta esparrua.}

Gure helburua, eguzki-sistemaren epe luzeko integraziorako Gaussen metodo inplizituaren inplementazio eraginkorra proposatzea edota, bide horretan aurrerapausoak ematea da. Helburu hau lortzeko, honako aspektu hauek bereziki zainduko ditugu: eguzki-sistemaren problemaren ezaugarriak, biribiltze erroreen garapena eta egungo konputagailuen gaitasunari egokitutako algoritmo azkarren diseinua.  

N-gorputzeko problema grabitazionalari dagokionez, eguzki-sistemaren eredu sinplea integratuko dugu. Eguzki-sistemaren gorputzak masa puntualak kontsideratuko ditugu eta gure ekuazio diferentzialek, gorputz hauen arteko erakarpen grabitazionalak bakarrik kontutan hartuko dituzte. Beraz, eguzki-sistemaren eredu konplexuagoetako erlatibitate efektua, gorputzen formaren eragina, eta beste zenbait indar ez-grabitazionalak ez ditugu kontutan hartu.

Zeintzuk dira eguzki-sistemaren problemaren ezaugarri bereziak? Batetik, planeten mugimendu orbitala, perturbazio txikiak dituen mugimendu Kepleriarra da. Beraz, mugimendu Kepleriarra  zehazki kalkula daitekeenez, eguzki-sistemaren planeten orbiten konputazioaren oinarria da. Bestetik,  badugu gorputz nagusi bat (eguzkia) eta honen inguruan mugimenduan dauden planetak, bi multzotan bana ditzakegu: barne-planetak, masa txikikoak eta eguzkitik gertu daudenak eta kanpo-planetak, masa handikoak eta eguzkitik urrun daudenak. Kanpo-planeten eboluzioan, barne-planetak eragin oso txikia daukate, eragina, masaren eta distantziaren alderantzizkoaren proportzionala baita.  Eguzki-sistema egonkorra kontsideratzen da, hau da, hurrengo bilioi urteetan planeten arteko talkarik ez da espero gertatzea. Orbiten denbora eskalak anitzak dira; ilargiaren lurraren inguruko orbitaren periodoa $27.32$ egunetakoa, lurraren eguzkiaren ingurukoa $1$ urtekoa eta Neptunorena $163$ urtekoa.  Eguzki-sistemaren egitura aberats honi, abantaila gehien ateratzen dion planteamendua bilatuko dugu.
  
Konputagailuen koma-higikorreko aritmetika ondo ulertzea garrantzitsua da. Zenbaki errealen adierazpen finitua erabiltzen denez, bai zenbakiak memorian gordetzerakoan, bai hauen arteko kalkulu aritmetikoak egiterakoan, biribiltze errorea sortzen da. Integrazio luzeetan, biribiltze errorea hedatu egiten da eta une batetik aurrera, soluzioen zuzentasuna ezereztatzen da. Zentzu honetan, doitasuna hobetzeko biribiltze errorea gutxitzen duten teknika bereziak aplikatzea ezinbestekoa izaten da. Integrazio luzeetan, maiz doitasun handian lan egiteko aukera aipatzen da, baina doitasun altuko aritmetikaren ($128$-bit) inplementazioa software bidezkoa denez, oso motela da eta ez da erabilgarria. Exekuzio denbora onargarriak lortzeko tarteko irtenbideak landu behar dira, esate baterako, doitasun ezberdinak nahasten dituzten inplementazioak.       

Konputazioko teknologiaren garapenean, algoritmo azkarren diseinua baldintzatzen duten bi ezaugarri azpimarratu behar dira. Batetik, konputagailuak orokorrean paraleloak dira eta algoritmo azkarrak garatzeko, kodearen paralelizazio gaitasunari heldu behar zaio. Bestetik, konputazioaren alde garestiena, memoria eta prozesadorearen arteko datu mugimendua denez, prozesadorearen konputazio handiena komunikazio txikienarekin lortu behar da. 

Sarrera honetan paralelizazioari buruzko ohar batzuk ematea komeni da. Algoritmo baten kode unitateak paraleloan exekutatzeak badu gainkarga bat eta beraz,  algoritmoaren exekuzioa paralelizazioaz azkartzea lortzeko,  unitate bakoitzaren tamainak esanguratsua izan behar du. Gure eguzki-sistemaren eredua sinplea da eta logikoa da pentsatzea eredu konplexuagoetan, paralelizazioak abantaila handiagoa erakutsiko duela. Bestalde, gorputzen kopurua handia den problemetan, hauen arteko interakzio kopuru  handia kalkulatu behar da ($\mathcal{O}(N^2)$) eta indar hauen hurbilpena modu eraginkorrean kalkulatzeko metodo ezagunak daude: \textit {tree code}\cite{Barnes1986} eta \textit {fast multipole method}\cite{Carrier1988} izeneko metodoak. Baina gure probleman gorputz kopurua txikia denez, teknika hauek gure eremutik kanpo utzi ditugu. 


\section{Ekarpenak.}

Tesiaren lana hiru ataletan banatu dugu. Lehen urratsean, Gauss metodoaren urratsa emateko puntu-finkoaren iterazioaren  bidezko inplementazioa aztertu dugu eta gure inplementazioen oinarriak finkatu ditugu. Bigarren fasean, Gauss metodoaren urratsa emateko, Newton iterazioaren bidezko inplementazio eraginkorra lortzeko ahalegin berezia egin dugu. Problema zurruna denean, puntu-finkoaren iterazio ez da eraginkorra eta Newtonen iterazioa aplikatu behar da. Gainera problema ez-zurruna izanik ere, Newton iterazioak interesgarriak izan daitezke; bereziki doitasun altuko (doitasun laukoitza) konputazioetan, metodoaren konbergentzia ezaugarri onak direla-eta.  Hirugarren fasean ...

Jarraian, atal bakoitzean egindako ekarpen nagusienak laburtuko ditugu:

\begin{enumerate}
\item IRK puntu finkoa.

Gauss metodoaren puntu finkoaren inplementazioaren azterketa sakon bat egin dugu eta horretarako, Hairer-en inplementazioa \cite{Hairer2008} hartu dugu gure lanaren abiapuntua. Kalitatezkoa inplementazio hau hobetzeko aukerak ikusi ditugu eta inplementazio sendoago bat proposatu dugu. Gure ekarpenak hauek izan dira:  

\begin{enumerate}
\item Metodoaren birformulazioa.

Gauss inplizitua aplikatzen dugunean, metodoa definitzen duten biribildutako koefiziente errealak ($\tilde{a}_{ij}, \tilde{b}_i \in \mathbb{F}$) erabiltzen dira. Formulazio estandarra erabiliz, koefiziente hauek ez dute metodoa sinplektikoa izateko baldintza zehazki betetzen eta beraz, izaera sinplektikoaren propietate onak galtzen dira. Metodoaren birformulazio baliokide bat proposatu dugu, horrela sinplektizidade baldintza zehazki betetzen duten koefizienteak modu errazean finkatu daitezke. Hori dela-eta, Gauss metodoak integral koadratikoak kontserbatuko ditu.

\item Geratze irizpide berria.

Orokorrean, Hairer-en inplementazioaren puntu finkoaren iterazioaren geratze irizpidea zuzena dela ikusi dugu baina kasu batzuetan goizegi geratzen dela baieztatu dugu. Arrazoiak bi direla ikusirik, bere geratze irizpidea bi zentzutan garatu/zorroztu dugu. Lehenik, Hairer-en inplementazioan, hobekuntza neurketa ataletako diferentziaren norma batean oinarritzen da. Normarekiko independentea den geratze irizpidea aplikatzea zuzenagoa da, eta horregatik, ataletako edozein osagaiaren diferentzia txikitzen den bitartean iterazioak egiten jarraitzea finkatu dugu. Bigarrenik, iterazioetan osagai guztien hobekuntza ez du zertan beherakorra izan behar, eta okertzen diren tarteko iterazioak gerta daitezke. Arazo hau gainditzeko, iterazioren batean osagai guztien diferentzia handitzea gertatzen denean, seguritateko bi iterazio gehigarri emango ditugu, iteraziotik irten aurretik.   

\item Atalen espresioaren aldaketa.

Integrazioan batura konpensatu estandarra aplikatzen dugunean, zenbakizko soluzioa $\tilde{y}_n, e_n \in \mathbb{F}^d$ lortzen dugu non $\tilde{y}_n+e_n \approx y(t_n)$ den. Hori dela-eta, metodoaren atalen espresioan $\tilde{y}_n$-ren ordez, $\tilde{y}_n+e_n$ erabiltzea proposatu dugu. Aldaketa honekin, zenbakizko soluzioaren doitasuna zerbait hobetuko dela espero da.

\begin{equation*}
Y_{n,i}=y_n + \left(e_n+ \sum_{j=1}^{s}\mu_{ij} L_{n,j} \right).
\end{equation*}
  
\item Biribiltze errorearen estimazioa.

Zenbakizko soluzioaren $\tilde{y}_n+e_n \approx y(t_n), \ n=1,2,\dots$ biribiltze errorearen estimazioa, doitasun txikiagoko bigarren zenbakizko soluzioaren $\hat{y}_n+\hat{e}_n \approx y(t_n), \ n=1,2,\dots$  diferentzia gisa kalkulatuko dugu.
Erabiltzaileari zenbakizko soluzioaren estimazioa ezagutzeko, exekuzio bakarrean  eta \emph{CPU} gainkarga txikiarekin, bi integrazioak sekuentzialki kalkulatzeko aukera eskainiko zaio. 


\end{enumerate}


\item IRK Newton.

\begin{enumerate}
\item Ekarpen nagusia.

$S$-ataletako IRK metodoa,  Newton iterazioaren bidez $d$-dimentsioko ekuazio diferentzial  sistemari aplikatzeko, urrats bakoitzean $sd \times sd$ tamainako hainbat ekuazio sistema (iterazio bakoitzeko bat) askatu behar dira. Atal honetan, jatorrizko $sd$-dimentsioko ekuazio sistema, $(s+1)d$ dimentsioko ekuazio-sistema baliokide moduan berridatzi dugu. Ekuazio-sistema baliokidea,  $d \times d$ tamainako $[s/2]+1$ matrize errealen \emph{LU}-deskonposaketa bidez askatuko dugu. Tamaina txikiko matrizeen LU deskonposaketa azkarra denez, konputazionalki eraginkorra izatea espero dugu.   

\item Doitasun laukoitzeko inplementazioa.

Doitasun laukoitzeko exekuzioetan, funtzio balioztapena oso garestia da eta funtzio balioztapenen kopurua gutxitzea bilatuko dugu. Newton osoa aplikatzen dugunean pena mereziko du, Gaussen Newton iterazioko inplementazioaren ekuazio lineala askatzeko metodo iteratiboa aplikatzea. 

\item Newton mixtoa.


\end{enumerate}
  

\item IRK Eguzki-sistema.

Hirugarren urratsean, eguzki-sistemaren epe luzeko integrazioan arituko gara. Ekarpen handiena, atalen hasieraketa berri bat aplikatzea da alde Kepleriarraren fluxuan oinarrituz. IRK metodoak eskaintzen digun malgutasunari esker eta N gorputzetako problema grabitazionalaren ezaugarriez baliatuz inplementazio ezberdinak egin ditugu. Inplementazio hauen eraginkortasuna, egungo integratzaile sinplektiko esplizituekin konparatu ditugu.

\item Birparametrizazioa.

Azken urratsean, esperimentalki, eguzki-sistemaren integrazioan  denboraren birparametrizazio teknikaren aplikazio sinple bat erakutsiko dugu. Integratzaile sinplektikoak luzera finkoko urratsa eduki behar du eta zentzu honetan, birparametrizazioa eraginkortasuna hobetzeko beste bide bat da.      

\end{enumerate}        


\section{Tesiaren egitura.}

Tesiaren lehenengo $(1-5)$ kapituluetan, zenbakizko integrazio sinplektikoak eta zientzia konputazionalaren oinarriak azaldu ditugu. Bigarren kapituluaren lehen zatian, \emph{Zenbakizko integratzaile sinplektikoen} inguruko oinarrizko kontzeptuak azaldu ditugu. Kapitulu honen bigarren zatian, Gauss metodo estandarra deskribatu eta bere propietate nagusienak eman ditugu. Kapitulu honen azken zatian, eguzki-sistemaren integraziorako metodo sinplektiko eta esplizitu nagusienak laburtu ditugu. Hirugarren kapituluan, lan honetan zenbakizko esperimentuetan erabili ditugun hasierako baliodun problemen zehaztasunak eman ditugu. Laugarren kapituluan koma-higikorreko aritmetikan murgildu gara eta biribiltze errorearen inguruko gaiak argitu nahi izan ditugu. Bostgarren kapituluan, egungo konputazio zientziaren hardware eta software kontzeptu nagusienak ezagutarazi nahi izan ditugu.     

Tesiaren $(6-8)$ kapituluetan, gure inplementazio berriak garatu ditugu eta zenbakizko esperimentuen bidez, hauen eraginkortasuna erakutsi dugu. Lehenik, $6$.~kapituluan Gauss metodoaren puntu finkoaren iterazioaren inplementazio berria eman dugu. Ondoren, $7$.~kapituluan Gauss metodoaren Newtonen iterazioan oinarritutako inplementazio eraginkorrak azaldu ditugu. $8$.~kapituluan, eguzki-sistemaren integraziorako inplementazio deskribapena egin dugu.  

Tesiaren $(9-10)$ kapituluetan, gure hipotesiaren eztabaida eta lanaren konklusioak idatzi ditugu.

Tesiaren bukaeran, hiru eranskinetan lanaren informazio osagarria bildu dugu. A-eranskinean, tesian zehar erabilitako hainbat frogen zehaztapenak eman dira. B-eranskinean, garatutako kodeak bildu eta erabiltzaileari erabilgarri izan dakiokeen informazioa laburtu dugu. Azkenik C-eranskinean, erabilitako notazioaren inguruko argibideak eman ditugu.

      
      
\section{Laburpena}


\part{Oinarriak}
\chapter{Zenbakizko Integratzaile Sinplektikoak.}

\section{Sarrera.}

\subsection{Zenbakizko metodoak.}

Ekuazio diferentzial arruntetarako (\emph{ODE}) hasierako baliodun problemen (\emph{IVP}) formulazioa,
\begin{equation}
\label{eq:21}
\dot{\mathbf{y}}(t)=\mathbf{f}(t,\mathbf{y}(t)),\ \ \ \mathbf{y}(t_0)=\mathbf{y_0}, 
\end{equation}
non  $\bf{y}: \mathbb{R} \longrightarrow {\mathbb{R}}^d$ soluzioa, $\bf{y_0} \in \mathbb{R}^d$ hasierako balioa eta $\bf{f}: \ \mathbb{R} \times {\mathbb{R}}^d \ \longrightarrow {\mathbb{R}}^d$ bektore eremua deskribatzen funtzioa dugun ($\dot{\mathbf{y}}$ notazioa erabiliko dugu $d\mathbf{y}/dt$ adierazteko).

Goiko ekuazio-sistema bektoriala (\ref{eq:21}), ekuazio-sistema modu eskalarrean idatzi daiteke:

\[\dot{y_1}(t)=f_1(t,(y_1(t),y_2(t),\dots,y_d(t)), \ \ y_1(t_0)=y_{1,0}\]
\[\dot{y_2}(t)=f_2(t,(y_1(t),y_2(t),\dots,y_d(t)), \ \ y_2(t_0)=y_{2,0}\]
\[\dots\]
\[\dot{y_d}(t)=f_d(t,(y_1(t),y_2(t),\dots,y_d(t)),  \ \ y_d(t_0)=y_{d,0}\]


Metodo analitikoak (funtzio ezagunen araberako soluzio zehatza) eta erdi-analitikoak, ez dira problema askoren soluzioa bilatzeko teknika egokiak. Zenbakizko metodoak, aldiz, modu errazean aplika daitezke eta horregatik, soluzio metodo nagusiena kontsideratzen da. 

Zenbakizko metodo baten bidez, $\mathbf{y}(t)$ soluzioaren $\mathbf{y_n} \approx \mathbf{y}(t_n)$ hurbilpena lortuko dugu $t_n=t_{n-1}+h_n$  ($n=1,2,\dots$)  une diskretu ezberdinetarako. Zenbakizko soluzioa urratsez-urrats sekuentzialki eta zehaztutako tarte baterako ($t_0\le t \le t_f$) kalkulatuko dugu. Beraz, lortutako balio  multzoak $(t_o,\mathbf{y_0}),(t_1,\mathbf{y_1}),\dots,(t_f,\mathbf{y_f})$ zenbakizko soluzioa definitzen du.   

Nola jakin zenbakizko soluzioa matematika modeloarekiko zuzena dela? Zenbakizko soluzioaren errorea neurtzeko teknika ezberdinak ditugu.           

Azkenik argitu beharra dago ekuazio-sistema beti \emph{sistema autonomo} moduan, hau da, denborarekiko independentea idatz daitekeela. Hori horrela izanik, notazioa sinplifikatzeko era honetako sistemak kontsideratuko ditugu,   

\begin{equation}
%\label{eq:31}
\dot{\mathbf{y}}(t)=\mathbf{f}(\mathbf{y}(t)),\ \ \ \mathbf{y}(t_0)=\mathbf{y_0}.
\end{equation}


\paragraph*{}Jarraian zenbakizko metodoen oinarrizko kontzeptuak eta notazioa finkatuko dugu.

\begin{enumerate}

\item Fluxua.

Fase-espazioko edozein $\mathbf{y_0}$ puntuari, $\mathbf{y}(t_0)=\mathbf{y_0}$ hasierako balio duen $\mathbf{y}(t)$ soluzioa esleitzen dion mapping-ari deitzen diogu. Izendatzeko $\varphi_t$ notazioa erabiliko dugu,
\begin{equation*}
\varphi_t(\mathbf{y_0})=\mathbf{y}(t) \ \ \text{baldin} \  \mathbf{y}(t_0)=\mathbf{y_0}
\end{equation*}

\item Zenbakizko diskretizazioa.

$\mathbf{y_{n}},\mathbf{y_{n-1}},\dots ,\mathbf{y_0}$ balioak emanda, $\mathbf{y_{n+1}}\approx \mathbf{y}(t_{n+1})$ soluzioaren hurbilpena kalkulatzeko formulari \emph{zenbakizko fluxua} deritzogu. Honako notazioa erabiliko dugu,
\begin{equation*}
\mathbf{y_{n+1}}=\phi(\mathbf{y_{n+1}},\mathbf{y_{n}},\dots,\mathbf{y_0};h;f).
\end{equation*}

$\phi$ metodoa, $\mathbf{y_{n+1}}$ balioaren menpe ez dagoenean, $\mathbf{y_{n+1}}$ zuzenean kalkula daiteke eta metodoari \emph{esplizitua} dela esaten zaio. Aldiz, $\phi$ metodoa $\mathbf{y_{n+1}}$ menpe dagoenean, $\mathbf{y_{n+1}}$ askatzeko zeharkako bidea erabili behar da (adibidez Newton sinplifikatua edo puntu finkoaren metodoa) eta metodoari \emph{inplizitua} dela esaten zaio.  

\paragraph*{\textbf{Adibideak.}}

Ekuazio diferentzial arruntaren (\ref{eq:21}),  $t_0$ unetik $t_1=t_0+h$ une arteko integrazioak,
\begin{equation*}
\mathbf{y}(t_1)=\mathbf{y_o}+\int\limits_{t_0}^{t_1} \mathbf{f}(t,\mathbf{y}(t)) dt,
\end{equation*}

Integrala modu ezberdinean hurbilduz, zenbakizko bi metodo defini ditzakegu.
   \begin{enumerate}
    \item Eurler metodo esplizitua.
    \begin{equation*}
     \label{eq41}
     \mathbf{y_{1}}=\mathbf{y_0}+h  \ \mathbf{f}(\mathbf{t_0,y_0}).
    \end{equation*} 

    \item Metodo trapezoidal inplizitua.
    \begin{equation*}
    \label{eq41}
    \mathbf{y_{1}}=\mathbf{y_0}+\frac{h}{2}  \ (\mathbf{f}(t_0,\mathbf{y_{0}}+\mathbf{f}(t_1,\mathbf{y_{1}})). 
    \end{equation*} 
    \end{enumerate}

\item Metodoaren ordena.

\paragraph*{}Definizioa. \textbf{Errore globala}. Zenbakizko soluzioaren $t_0$ hasierako unetik $t_k$ une arteko errore globala $ge(t)$,
\begin{equation*}
ge(t_k)=y_k-y(t_k).
\end{equation*}
  
\paragraph*{} Definizioa. \textbf{$\mathbf{\phi}$ metodoaren ordena}. $h$ urrats luzera finkoko $\phi$ metodoak $p$ ordenekoa dela esaten da, errore globala $ge(t)$  $O(h^{p})$ ordenekoa bada  $h \rightarrow 0$,
\begin{equation*}
y_k-y(t_k)=O(h^{p}), \ \ h \rightarrow 0.
\end{equation*}   

\paragraph*{}Definizioa. \textbf{Errore lokala}. Zenbakizko soluzioaren urrats bakarreko $[t_k,t_{k+1}]$ errore lokala $le(t)$,
\begin{equation*}
le(t_{k+1})=y_{k+1}-y_k(t_{k+1})
\end{equation*}  
non $y_k(t)$, $y(t_k)=y_k$ hasierako balio lokaleko soluzio zehatza den. 

Metodoaren ordena $O(h^p)$ bada, errore lokala $O(h^{p+1})$ da.

\paragraph*{} Asymptotically, that is for small values of h, the local error is $Ch^{p+1}$, where $C$ depends on the particular problem as well as the method. The value of $p$ is thus a guide to how rapidly errors reduce as a consequence of a reduction in h (Butcher- Gauss method - encyclopedia).

\item  Metodo simetrikoak.

\paragraph*{\textbf{Adjoint method}.} Jatorrizko metodoaren alderantzizko eta kontrako denbora $-h$ esleipenari (map), $\phi_h$ metodoaren $\phi_h^{*}$ \emph{adjoint metodoa} esaten zaio.

\begin{equation*}
\phi_h^{*}=\phi_{-h}^{-1}
\end{equation*}

$\phi_h^{*}=\phi_h$ betetzen denean, metodoa \textbf{simetrikoa} dela esaten da. 

\end{enumerate}


\subsection{Problema motak.}

\begin{enumerate}

\item Problema kaotikoak. 

Hasierako balio edo parametroen perturbazioekiko, diskretizazio-erroreekiko (trunkatze) edo birbitze erroreekiko esponentzialki sentikorrak diren problemei esaten zaie.

\item Problema stiff.

Irakurri $12.1$ Solution of stiff problems ($555-559$).

Irakurri $13.2.3$ Convergence and consistency and $13.3$ Stiffness and Implicitness (600).

Stiff equations are problems for which explicit methods don't work (Hairer). 

\end{enumerate}


\subsection{Sistema-Hamiltondarrak.}


Ekuazio diferentzial arrunten (EDA) formulazio Hamiltondarra erabili ohi da errealitateko sistemak matematikoki adierazteko. 

$H(\mathbf{p},\mathbf{q})$ funtzio leuna izanik, non  $H: {\mathbb{R}}^{2d} \ \longrightarrow {\mathbb{R}}$  den eta  $\mathbf{p}=(p_1, \dots , p_d)$, $\mathbf{q}=(q_1, \dots , q_d)$ domeinuaren aldagaiak diren. $H$ funtzioari dagokion \emph{Hamiltondar sistema} osatzen duten $2d$ ekuazio diferentzialak era honetan definitzen dira,

\begin{align*}
\frac{d}{dt} \ {p}_i & =-\frac{\partial H }{\partial q_i} (\mathbf{p},\mathbf{q}), \\
\frac{d}{dt} \ {q}_i & =\ \ \frac{\partial H}{\partial p_i} (\mathbf{p},\mathbf{q}), \ \ \ \ i=1,\dots,d.
\end{align*}

$H(\mathbf{p},\mathbf{q})$ funtzioari \emph{Hamiltondarra} esaten zaio eta integrazioan zehar konstante mantentzen da. $\mathbf{p}$ eta $\mathbf{q}$ bektoreen $d$ dimentsioa sistemaren \emph{askatasun maila} deritzo. Beste notazio laburtu hau ere erabili ohi da,

\begin{equation*}
\dot{\mathbf{y}}=J^{-1}\triangledown H(\mathbf{y}),
\end{equation*}

non $\mathbf{y}=(\mathbf{p},\mathbf{q})$, $\triangledown H=(\partial H/\partial p_1,\dots,\partial H/\partial p_d; \partial H/\partial q_1,\dots,\partial H/\partial q_d)$ eta

\begin{equation*}
 J=\left(\begin{array}{cc}
   \ 0_{dxd} & \ I_{dxd} \\
    -I_{dxd} & \ 0_{dxd} \\
\end{array}\right).  
\end{equation*}

\subsubsection*{Adibidea.} Penduluaren problemaren (masa $m=1$, $l=1$ luzerako makila eta $g=1$ grabitazioa) $d=1$ askatasuneko sistema Hamiltondarra,

\begin{equation*}
H(p,q)=\frac{1}{2} p^2- cos q.
\end{equation*}

Ekuazio diferentzialak,
\begin{equation*}
\dot{p}= -sin q, \ \ \dot{q}=p.
\end{equation*}

\begin{figure}[h]
\centering
\subfloat[Pendulua.]{
\includegraphics[width=.250\textwidth]{SinglePendulum}
}
\caption{ \small Pendulua.}
\label{fig:pendulua}
\end{figure}

\textbf{Kang Feng} With the development of the modern mechanics,physics, chemistry, and biology, it is undisputed that almost all physical processes,whether they are classical, quantum, or relativistic, can be represented by an Hamiltonian system. Thus, it is important to solve the Hamiltonian system correctly.

\subsubsection*{Hamiltondar banagarriak.}

Hamiltondar banagarriak egitura bereziko sistema Hamiltondarrak ditugu. Maiz, sistema-mekanikoek era honetako Hamiltondarra dute $H(\mathbf{p},\mathbf{q})=T(\mathbf{p})+U(\mathbf{q})$.

Horien artean, \emph{bigarren ordeneko} ekuazio diferentzialak aipatu behar ditugu, zeintzuk Hamiltondar banagarri kasu partikularra bat diren,  

\begin{equation*}
H(\mathbf{p},\mathbf{q})=\frac{1}{2}\mathbf{p}^T\mathbf{p} +U(\mathbf{q}).
\end{equation*}

Beraz, dagokien ekuazio diferentzialak,
\begin{equation*}
\dot{\mathbf{p}}=-\frac{\partial U(\mathbf{q})}{\partial \mathbf{q}}, \ \ \dot{\mathbf{q}}=\mathbf{p}. 
\end{equation*}

\subsubsection*{Adibidea.}
\emph{Bi-gorputzen problema} edo \emph{Kepler problema}. Planoan elkar erakartzen diren bi gorputzen (adibidez eguzkia eta planeta bat) mugimendua kalkulatzeko, horietako gorputz baten kokapena koordenatu sistemaren jatorria kontsideratuko dugu eta beste gorputzaren kokapenaren koordenatuak $\mathbf{q}=(q_1,q_2)$ izendatuko ditugu. Normalizatutako Newton legearen araberako ekuazio diferentzialak,  

\begin{equation}
\dot{p}_1= -\frac{q_1}{(q_1^2+q_2^2)^{3/2}}, \ \, \dot{p}_2= -\frac{q_2}{(q_1^2+q_2^2)^{3/2}}.
\end{equation}
  
\begin{equation}
\dot{q}_i=p_i, \ \ i=1,2.
\end{equation}

Baliokide den sistema Hamiltondarra,
\begin{equation}
H(p_1,p_2,q_1,q_2)=\frac{1}{2}(p_1^2+p_2^2)-\frac{1}{\sqrt{q_1^2+q_2^2}}.
\end{equation}

\paragraph*{} Planetaren mugimendua orbita eliptiko bat da. Honako hasierako balioei dagokien soluzioa,
\begin{equation*}
q_1(0)=1-e, \ q_2(0)=0, \ p_1(0)=0, \ p_2(0)=\sqrt{ \frac{1+e}{1-e}}, 
\end{equation*}
$e$ ezentrizidade ($0\le e < 1$) duen elipsea da, eta $P=2\pi$ periododuna. 
 
\subsubsection*{Hamiltondar perturbatuak.}

Hamiltondar perturbatuak, honako egitura duten sistemak ditugu.
\begin{equation*}
H=H_A+\epsilon H_B \ \ (|H_B|\ll |H_A|).
\end{equation*}

\subsubsection*{Adibidea.} Eguzki-sistemaren probleman, Hamiltondarra modu honetan idatzi daiteke $H=H_k+H_I$, non alde nagusia $H_K$ planeta bakoitzaren eguzki inguruko mugimendu kepleriarra den eta $H_I$ aldiz, planeten arteko interakzioek eragiten duten perturbazio txikia.   

\subsection{Metodo sinplektikoak.}

Zenbakizko integratzaile tradizionalak ez dituzte ekuazio diferentzialen hainbat propietate zehazki mantentzen: energia, momentua, momentu angeluarra, simetriak,... 

Metodo sinpletikoak sistema Hamiltondarren soluzioen hurbilpena kalkulatzeko zenbakizko metodo egokiak dira eta eguzki-sistemaren epe luzeko integrazioetan oso erabiliak dira. Metodo sinplektikoen ezaugarri hauek azpimarratuko ditugu,

\begin{enumerate}

\item Metodo simetrikoak dira.

\item Urrats luzera finkoa. Metodo gehienak eraginkorrak izateko, errore estimazio baten arabera integrazioan zehar urrats luzera egokitzen dute. Integratzaile sinplektikoetan, urrats luzeera finkoa erabili behar da metodoaren propietateak ez galtzeko.  

\end{enumerate} 

\subsubsection*{Störmer-Verlet  metodoa.}

$p=2$ ordeneko metodo sinplektikoa garrantzitsua da. Izen ezberdinekin ezaguna da: \emph{Störmer metodoa} astronomian, \emph{Verlet metodoa} molekula dinamikoan edo \emph{Leap-Frog metodoa} ekuazio diferentzial partzialetan. Era honetan definitzen da, 

\[p_{{n+1}/{2}}=p_n-\frac{h}{2} \triangledown_q H(p_{{n+1}/{2}},q_n) \]
\begin{equation}
q_{n+1}=q_n+\frac{h}{2} \big(\triangledown_p H(p_{{n+1}/{2}},q_n)+ \triangledown_p H(p_{{n+1}/{2}},q_{n+1}) \big)
\end{equation}
\[p_{n+1}=p_{{n+1}/{2}}-\frac{h}{2} \triangledown_q H(p_{{n+1}/{2}},q_{n+1}) \]

edo

\[q_{{n+1}/{2}}=q_n+\frac{h}{2} \triangledown_p H(p_n,q_{{n+1}/{2}}) \]
\begin{equation}
p_{n+1}=p_n-\frac{h}{2} \big(\triangledown_q H(p_n,q_{{n+1}/{2}})+ \triangledown_q H(p_{n+1},q_{{n+1}/{2}}) \big)
\end{equation}
\[q_{n+1}=q_{{n+1}/{2}}+\frac{h}{2} \triangledown_p H(p_{n+1},q_{{n+1}/{2}}) \]


\paragraph*{}Bigarren ordeneko ekuazio diferentziala denean, metodoa esplizitua da eta modu honetan labur daiteke,

\[p_{{n+1}/{2}}=p_n+\frac{h}{2} f(q_n)\]
\begin{equation}
q_{n+1}=q_n+h \ p_{{n+1}/{2}}
\end{equation}
\[p_{n+1}=p_{{n+1}/{2}}+\frac{h}{2} \ f(q_{n+1})\]

edo

\[q_{{n+1}/{2}}=q_n+\frac{h}{2} p_n\]
\begin{equation}
p_{n+1}=p_n+h \ f(q_{{n+1}/{2}})
\end{equation}
\[q_{n+1}=q_{{n+1}/{2}}+\frac{h}{2} \ p_{n+1}\]


\section{Gauss metodoak.}

Runge-Kutta  metodo sinplektiko interesgarriena \emph{Gauss} metodoa da. Metodo inplizitua da eta s-ataleko orden altueneko metodoa ($p=2s$). ++

\subsection{Runge-Kutta metodoak.}

Runge-Kutta metodoak, urrats bakarreko ekuazio diferentzial arrunten zenbakizko integratzaileak dira.  $b_{i}$, $a_{ij}$ eta $c_i=\sum\limits_{i=0}^{s} a_{ij} \ (1 \leq i,j \leq s)$ koefiziente errealek s-ataleko Runge-Kutta metodoa definitzen dute . \emph{Butcher} izeneko taulan moduan laburtu ohi dira koefiziente hauek, 

\begin{equation}
\begin{array}{c|c}
  \bf{c} & \bf{A} \\
  \hline
         &  \bf{b}^T\\
\end{array}, \ \ \ \ \ \ \ \ \ \ \ \
\begin{array}{c|cccc}
  \ c_1 &  a_{11} & a_{12} & \dots & a_{1s}\\
  \ c_2 &  a_{21} & a_{22} & \dots &a_{2s}\\
  \ \vdots & \vdots & \ddots & & \vdots \\
  \ c_s & a_{s1} & a_{s2} & \dots & a_{ss}\\
  \hline
  \  & b_{1} & b_{2} & \dots & b_{s}\\
\end{array}
\end{equation}

\paragraph*{} Hasierako baliodun problema (\ref{eq:21}) baten $\mathbf{y}(t)$ soluzioaren $\mathbf{y}_n \approx \mathbf{y}(t_n)$ hurbilpena era honetan kalkulatzen da,

\begin{equation}  
\mathbf{y}_{n+1}=\mathbf{y}_n+h\sum^s_{i=1}{b_i \ \mathbf{f}(t_n+c_ih,\mathbf{Y}_{n,i})\ \ },\
\end{equation} 

non $\mathbf{Y}_{n,i}$ atalak era honetan definitzen diren,
\begin{equation}
\mathbf{Y}_{n,i}=\mathbf{y}_n+\ h\ \sum^s_{j=1}{a_{ij}\ \mathbf{f}(t_n+c_jh,\mathbf{Y}_{n,j})}\ \ \ \ \ i=1 ,\dots, s.\
\end{equation} 

\subsubsection*{Metodo esplizituak (ERK) eta inplizituak (IRK).}
Runge-Kutta bi mota nagusi bereizi ditzakegu: esplizituak (\emph {ERK}) non $\forall i\ge j, \ a_{ij}=0 $ eta inplizituak (\emph {IRK}) non $\exists i \ge j \ , a_{ij} \ne 0$ . 

\paragraph*{} \textbf{Adibidea}: ERK lau-ataletako metodo klasikoa. 
\begin{equation*}
\begin{array}{c|cccc}
  \ 0   &    &    &     &      \\
  \ 1/2 & 1/2 &   &     &      \\
  \ 1/2 & 0   & 1/2  &  &      \\
  \ 1   & 0   & 0    &  1   &   \\
  \hline
  \     & 1/6 & 2/6  &  2/6 & 1/6 \\
  \end{array} \\
\end{equation*}

Atalak $\mathbf{Y}_{n,i}$ esplizituki kalkula daitezke,
\begin{equation*}
\mathbf{Y}_{n,i}=\mathbf{y}_n+\ h\ \sum^{i-1}_{j=1}{a_{ij}\ \mathbf{f}(t_n+c_jh,\mathbf{Y}_{n,j})}\ \ \ \ \ i=1 ,\dots, s.
\end{equation*} 

\paragraph*{} \textbf{Adibidea}:  IRK metodoa (Implicit Midpoint method). 
\begin{equation*}
\begin{array}{c|c}
  \ 1/2 &  1/2\\
  \hline
  \     & 1 \\
\end{array} \\
\end{equation*}

$\mathbf{Y}_{n,1}$ atalaren balioa kalkulatzeko, honako ekuazio ez-linealaren ebazpena egin behar da,
\begin{equation*}
\mathbf{Y}_{n,1}=\mathbf{y}_n+\ \frac{h}{2} \ \mathbf{f}(t_n+\frac{h}{2},\mathbf{Y}_{n,1}).
\end{equation*} 


\emph{ERK} lau-ataletako  metodo klasikoa, $p=4$ ordeneko dugu. Orden altuko \emph{ERK} metodoak aurkitzea konplexua da,  koefizienteek bete behar dituzten baldintza kopurua esponentzialki hazten baitira. Orden altuko \emph (ERK) metodo hauek aurkitu dira: $p=8$ ordeneko metodoa $s=11$ atalekin, $p=10$ ordeneko metodoa $s=17$ atalekin eta  $p=12$ ordeneko metodoa $s=25$ atalekin. 

\emph{IRK} metodoak \emph{ERK} metodoak baino modu errazagoan eraiki daitezke.  \emph{Butcher sinplifikazio baldintzen} \cite{Butcher2008} arabera definitzen dira,
\begin{align*}
B(p) &: \ \ \ \sum\limits_{i=1}^{s} b_ic_i^{q-1} =\frac{1}{q}, \ \ & q=1,\dots,p. \\
C(\eta) &: \ \ \ \sum\limits_{j=1}^{s} a_{ij}c_j^{q-1}  =\frac{c_i^q}{q},& \ \ i=1,\dots,s, \ q=1,\dots,\eta.\\
D(\zeta) &:  \ \ \ \sum\limits_{i=1}^{s}  b_i c_i^{q-1}  a_{ij} = \frac{b_j}{q} (1-c_j^q),&  \ j=1,\dots,s, \  q=1,\dots,\zeta.
\end{align*}


\begin{table}
\caption{Runge-Kutta metodo inplizituak.}
\label{tab:21}       % Give a unique label
\begin{tabular}{ c c c c c } 
 \hline
 Metodoa          &  Baldintzak             &                        &                 & Ordena \\
 \hline
 Gauss            &  $B(2s)$                & $C(s)$                 & $D(s)$          & $2s$    \\
 \hline
 Radau IA         &  $B(2s-1)$              & $C(s-1)$               & $D(s)$          & $2s-1$  \\
 \hline 
 Radau IA         &  $B(2s-1)$              & $C(s)$                 & $D(s-1)$        & $2s-1$  \\
 \hline 
 Lobatto IIIA     &  $B(2s-2)$              & $C(s)$                 & $D(s-2)$        & $2s-2$  \\
 \hline
 Lobatto IIIB     &  $B(2s-2)$              & $C(s-2)$               & $D(s)$          & $2s-2$  \\
 \hline 
 Lobatto IIIC     &  $B(2s-2)$              & $C(s-1)$               & $D(s-1)$        & $2s-2$  \\
  \hline
 \end{tabular}
\end{table}

\subsubsection*{Gauss metodoa.}

Aurreko taulan (Taula \ref{tab:21}) \emph{IRK} metodo ezagunenak laburtu ditugu. Lehenik, Gauss metodoaren bi ezaugarri azpimarratu nahi ditugu:  Runge-Kutta metodo sinplektiko bakarra eta $s$-ataletako orden altueneko ($p=2s$) IRK da. 

\paragraph*{}Jarraian, Gauss metodoaren ezaugarri orokorrak azalduko ditugu:  
\begin{enumerate}
\item{Metodo sinplektikoa da.}

Sanz-Sernak  \cite{JMSanz-Serna1994}  Runge-Kutta metodoaren koefizienteek,  
\begin{equation} \label{eq:1}
b_{i}a_{ij}+b_{j}a_{ji}-b_{i}b_{j}=0, \ \ 1 \leqslant i,j \leqslant s,
\end{equation}

baldintza betetzen badute metodoa sinplektikoa dela frogatu zuen.  
 
\item{Metodo simetrikoa}.
 \begin{equation} \label {eq:2}
 b_{i} = b_{\sigma(i)} ,\  c_{\sigma(i)}= 1-c_{i}, \ \  i=1,2,\dots \lceil s/2\rceil
 \end{equation} 
  \begin{equation} \label{eq:3}
 b_{j}= a_{\sigma(i),\sigma(j)}a_{i,j}, \ \  i=1,2,\dots \lceil s/2\rceil
 \end{equation} 
non $\sigma(i)=s+1-i$.
 
 \begin{figure}[h]
 \centering
 \subfloat[Kolokazio metodoen simetria.]{
 \includegraphics[width=.850\textwidth]{SymmetryCollocation}
 }
 \caption{ \small Kolokazio metodoen simetria.}
 \label{fig:pendulua}
 \end{figure}
 
\item{Orden altuko metodoa.}
Gauss metodoa  edozein ordenekoa izan daiteke. Doitasun handiko konputazioetarako orden altuko metodoak behar dira: doitasun bikoitzeko aritmetikan ($u\approx10^{-16}$) $p\ge8$ ordeneko metodoak eta doitasun laukoitzeko aritmetikan ($u\approx10^{-35}$)  oraindik orden altuagoko metodoak gomendagarriak dira.  

\item{Metodo orokorra.}
Gauss metodoa edozein ekuazio diferentzialari aplika daiteke. Sistema Hamiltondarren problemetan, ez du zertan banagarria izan behar.

\item{Paralelizagarria.}
Ekuazio diferentzial garestiak ditugunean, $s$-ataletako funtzio konputazioak ($f(Y_i), \ i=1,\dots,s$) paraleloan kalkula daitezke.  

\item{Kolokazio metodoa.}
Zenbakizko soluzioa diskretizazio puntuetan ez ezik, integrazio tarte bakoitzean polinomio interpolatzaile batek modu jarraian emandako soluzioa adierazten du.

\item{A-stability and B-stability.}
A-stable methods have a central role in the numerical solution of stiff problems and Gauss methods are likely candidates.
  
\end{enumerate}


\paragraph*{} Gauss metodoaren desabantailak?


\paragraph{\textbf{Adibidea}.} $s=1$, $s=2$ eta $s=3$ ataletako Gauss metodoak.
\begin{equation*}
\begin{array}{c|c}
  \frac{1}{2} & \ \frac{1}{2} \\
  \hline
   & 1 \\
\end{array} \ \ \ ,  \ \ \ \ \ \ \ \ \
\begin{array}{c|c c}
  \frac{1}{2}-\frac{\sqrt{3}}{6} & \ \frac{1}{4} & \ \frac{1}{4}-\frac{\sqrt{3}}{6} \\
  \frac{1}{2}+\frac{\sqrt{3}}{6} & \ \frac{1}{4}+\frac{\sqrt{3}}{6} & \ \frac{1}{4} \\
  \hline
         &  \frac{1}{2} & \ \frac{1}{2} \\
\end{array}
\end{equation*}

\begin{equation*}
\begin{array}{c|c c c}
  \frac{1}{2}-\frac{\sqrt{15}}{10} & \ \frac{5}{36} & \ \frac{2}{9}-\frac{\sqrt{15}}{15} & \ \frac{5}{36}-\frac{\sqrt{15}}{30} \\
  \frac{1}{2}   & \ \frac{5}{36}+\frac{\sqrt{15}}{24} & \ \frac{2}{9} & \ \frac{5}{36}-\frac{\sqrt{15}}{24} \\
  \frac{1}{2}+\frac{\sqrt{15}}{10}   & \ \frac{5}{36}+\frac{\sqrt{15}}{30} & \ \frac{2}{9}+\frac{\sqrt{15}}{15} & \ \frac{5}{36} \\
  \hline
  & \frac{5}{18} & \ \frac{4}{9} & \ \frac{5}{18}
\end{array}
\end{equation*}
      

\subsubsection*{IRK algoritmo-I orokorra.}

\emph{IRK} metodoen erronka handiena ekuazio-sistema ez-linealaren zenbakizko soluzioaren inplementazio eraginkorra  da. Nonstiff problemetarako, atalen hasieraketa ($Y_i^{[0]}$) egoki bat duen puntu-finkoko iterazioa erabil daiteke. Stiff problemetarako, puntu-finkoa iterazioak urrats tamaina txikiegia izatea behartuko luke eta ondorioz, Newton sinplifikatua erabili ohi da.  

\begin{algorithm}[H]
 \BlankLine
  \For{$n\leftarrow 0$ \KwTo ($endstep-1$)}
  {
   \BlankLine
   Hasieratu  $Y_{n,i}^{[0]} \ \ , \ \ i=1,\dots,s $\;
    \BlankLine
   \While{ (konbergentzia lortu)}
   {
    \BlankLine 
    $F_{n,i}=f(t_n+c_ih,Y_{n,i}) \ \ , \ \  i=1,\dots,s$\;
    Askatu ($Y_{n,i}=y_{n}+ h \ \sum\limits_{j=1}^{s} a_{ij} F_{n,j}  \ \ , \ \  i=1,\dots,s$) \;  
   }
   \BlankLine
    $\delta_n=h \ \sum\limits_{i=1}^{s} b_i F_{n,i}$\;
    $y_{n+1}=y_{n}+ \delta_n $\;
   \BlankLine
 }
 \caption{IRK Algoritmo orokorra}\label{alg:IRK1}
\end{algorithm}


\paragraph*{}Algoritmo nagusiko agindu bakoitzari ohar moduko egingo diogu, IRK metodoaren hainbat zehaztapen emateko helburuarekin.
\begin{enumerate}
\item Hasieratu  $Y_{n,i}^{[0]}$.

Atalen hasieraketa egokia definitu behar da. Aukera sinpleena $Y_{n,i}^{[0]}=y_{n-1}$ hasieratzea da baina aurreko urratseko atalen informazioa erabiliz hurbilketa hobea lortu daiteke. Aurreko urratseko atalen polinomio interpolatzailearen bidezko hasieraketa era honetan adierazi dezakegu  $Y_{n,i}^{[0]}=g(Y_{n-1,i}) \ , \ i=1, \dots, s$.      

\item $F_{n,i}=f(t_n+c_ih,Y_{n,i})$.

Atal bakoitzaren ($i=1,\dots,s$) ekuazio diferentzialaren balioztapena independentea da eta paraleloan exekutatu daiteke.

\item   Askatu ($Y_{n,i}=y_{n}+ h \ \sum\limits_{i=1}^{s} a_{ij} F_{n,j}  \ \ , \ \  i=1,\dots,s$).

Ekuazio-sistema ez lineala metodo iteratibo bat erabiliz askatu beharra dago. Aplikatutako metodoa problemaren araberako izango da: puntu-finkoaren metodoa edo Newton  metodoa izan daiteke. Azpimarratu, Newton metodoaren iterazioak, puntu finkoko metodoaren iterazioak baino azkarrago konbergitzen duela baina konputazionalki garestiagoa da. 

\begin{enumerate}

\item Puntu-finkoko iterazioa.


\begin{algorithm}[H]
  \For{ (k=1,2,\dots)}
  {
   $F_{i}^{[k]}=f(t_n+c_ih,Y_i^{[k-1]})$\;
   $Y_{i}^{[k]}=y_{n}+ h \ \sum\limits_{j=1}^{s} a_{ij} F_{j}^{[k]} , \ \  i=1,\dots,s$\; 
  }
 \caption{Puntu-finkoko iterazioa.}
\end{algorithm}

Konbergentzia $\|Y^k-Y\|=O(h) \|Y^{k-1}-Y\|$.


\item Newton metodoaren iterazioa. 

Newton metodoa iterazio bakoitza konputazionalki garestia da. Batetik,  $\frac{\partial f}{\partial y}(t_n+c_ih,Y_i^{[k-1]}), \ i=1,\dots,s$ Jakobiarrak ebaluatu behar dira. Bestetik, s-ataletako IRK metodo baten bidez d-dimentsioko \emph{EDA} integratzeko, $sd \times sd$ matrizearen \emph{LU-deskonposaketa} kalkulatu behar da.    


\begin{algorithm}[H]
  \For{ (k=1,2,\dots)}
  {
   $r_{i}^{[k]}=-Y_i^{[k-1]}+y_n+h \ \sum\limits_{j=1}^{s} a_{ij} f(t_n+c_ih,Y_j^{[k-1]}) $\;
   Askatu $\triangle Y_i^{[k]}-h \ \sum\limits_{j=1}^{s} a_{ij} \frac{\partial f}{\partial y}(t_n+c_jh,Y_j^{[k-1]}) \ \triangle Y_i^{[k]}=r_i^{[k]}$\;
   $Y_i^{[k]}=Y_i^{[k-1]}+\triangle Y_i^{[k]}, \ \  i=1,\dots,s$\; 
  }
 \caption{Newton metodoaren iterazioa}
\end{algorithm}

Konbergentzia $\|Y^k-Y\|=O(h^2) \|Y^{k-1}-Y\|$.


\end{enumerate}


\item $y_{n+1}=y_{n}+ \delta_n $\;

Zenbakizko integrazio luzeetan urrats asko ematen direnez, urratsero batura errekurtsibo honen konputazioa beharrezkoa da. Normalean $|\delta_n|\ll |y_n| $ izango da eta batura honetan doitasun galera gertatuko da. Hau ekiditeko \emph{batura konpensatu} teknika erabili ohi da.

\end{enumerate} 

\subsubsection*{IRK algoritmoa-II (Hamiltondar banagarriak).}

Era honetako ekuazio diferentzialak garrantzitsuak dira,
\begin{equation*}
\dot{p}=f(q), \ \ \dot{q}=g(p).
\end{equation*}

Esaterako, Hamiltondar banagarriak $H(q,p)=T(p)+U(q)$ eta bigarren ordeneko ekuazio diferentzialak $\ddot{q}=f(q)$ era honetako ekuazio diferentzialen kasu partikularrak dira.

Hurbilpena $(p_{n+1},q_{n+1}) \approx (p(t_{n+1}),q(t_{n+1}))$ era honetan kalkulatuko dugu,
\begin{align*}
p_{n+1}=p_n+ h \sum\limits_{i=1}^{s} b_i \ f(t_n+c_ih,Q_{n,i}),\\
q_{n+1}=q_n+ h \sum\limits_{i=1}^{s} b_i \ g(t_n+c_ih,P_{n,i}),
\end{align*}

non $(P_{n,i},Q_{n,i}), \ i=1,\dots,s$ honako ekuazio sistema definitutako atalak diren, 
\begin{align*}
P_{n,i} &=p_n+ h \sum\limits_{j=1}^{s} a_{ij} \ f(t_n+c_jh,Q_{n,j}), \\
Q_{n,i} &=q_n+ h \sum\limits_{j=1}^{s} a_{ij} \ g(t_n+c_jh,P_{n,j}).
\end{align*}

Era honetako problemetan,  funtsean \emph{IRK} algoritmo orokorra (alg.\ref{alg:IRK1}) aplikatuko dugu. Baina Hamiltondarraren egituraren abantaila aprobetxatuz, puntu-finkoaren iterazioaren konbergentzia hobetuko dugu,   

\begin{algorithm}[H]
  \For{ (k=1,2,\dots)}
  {
   $P_{n,i}^{[k]}=p_{n}+ h \ \sum\limits_{j=1}^{s} a_{ij} \ f(t_n+c_jh,Q_{n,j}^{[k-1]})$\; 
   $Q_{n,i}^{[k]}=q_{n}+ h \ \sum\limits_{j=1}^{s} a_{ij} \ g(t_n+c_jh,P_{n,j}^{[k]}), \ \ \ \ i=1,\dots,s $\; 
  }
 \caption{Puntu-finkoaren iterazioa (Gauss-Seidel).}
\end{algorithm}
 

\subsubsection*{IRK algoritmoa-III (bigarren ordeneko EDA).}
 
 
Bigarren ordeneko ekuazio diferentzialen $\ddot{q}=f(q)$ (\emph{Runge-Kutta-Nyström}) azterketa egiteko, modu baliokide honetan idatziko dugu,
\begin{equation*}
\dot{p}=f(q), \ \ \dot{q}=p.
\end{equation*}

\paragraph*{}Hurbilpena $(p_{n+1},q_{n+1}) \approx (p(t_{n+1}),q(t_{n+1}))$ era honetan kalkulatuko dugu,
\begin{align*}
p_{n+1}=p_n+ h \sum\limits_{i=1}^{s} b_i \ f(t_n+c_ih,Q_{n,i}),\\
q_{n+1}=q_n+ h p_{n} + h^2 \sum\limits_{i=1}^{s} \beta_i \ f(t_n+c_ih,Q_{n,i}),
\end{align*}

non $Q_{n,i}, \ i=1,\dots,s$ honako ekuazio sistema definitutako atalak diren, 
\begin{align*}
Q_{n,i}=q_n+ h\gamma_i p_n+ h^2 \sum\limits_{j=1}^{s} \tilde{a}_{ij} \ f(t_n+c_jh,Q_{n,j}).
\end{align*}

\paragraph*{}{\textbf{IRK algoritmoa-III (bigarren ordeneko EDA)}.}


\begin{algorithm}[H]
 \BlankLine
  \For{$n\leftarrow 0$ \KwTo ($endstep-1$)}
  {
   \BlankLine
   Hasieratu  $Q_{n,i}^{[0]} \ \ , \ \ i=1,\dots,s $\;
    \BlankLine
   \While{ (konbergentzia lortu)}
   {
    \BlankLine 
    $F_{n,i}=f(t_n+c_ih,Q_{n,i}) \ \ , \ \  i=1,\dots,s$\;
    $Q_{n,i}=q_n+ h\gamma_i p_n+ h^2 \sum\limits_{j=1}^{s} \tilde{a}_{ij} \ f(t_n+c_jh,Q_{n,j}) \ \ , \ \  i=1,\dots,s$\;  
   }
   \BlankLine
    $\delta p_n=h \ \sum\limits_{i=1}^{s} b_i F_{n,i}$\;
    $\delta q_n=h^2 \sum\limits_{i=1}^{s} \beta_i F_{n,i}$\;    
    $p_{n+1}=p_{n}+ \delta p_n $\;
    $q_{n+1}=q_{n}+ h\gamma_i p_n+\delta q_n $\;
   \BlankLine
 }
 \caption{IRK algoritmoa-III (bigarren ordeneko EDA)}\label{alg:IRK2}
\end{algorithm}

\paragraph*{}Bigarren ordeneko ekuazio diferentzialak ditugunean, puntu-finkoko iterazioa,

\begin{algorithm}[H]
  \For{ (k=1,2,\dots)}
  {
   $Q_{n,i}^{[k]}=q_{n}+h \gamma_i p_{n}+ h^2 \ \sum\limits_{j=1}^{s} \tilde{a}_{ij} f(t_n+c_jh,Q_{n,j}^{[k-1]}) $\;  
  }
 \caption{Puntu-finkoko iterazioa (bigarren ordeneko EDA)}
\end{algorithm} 
 

\subsection{Kolokazio metodoak.}

Kolokazio metodoak ekuazio diferentzialen zenbakizko soluzioa azaltzeko beste modu bat dira. Gauss metodoak kolokazio metodoak ditugu eta hauen abantaila da, zenbakizko soluzioa diskretizazio puntuetan ez ezik, polinomio interpolatzaile batek modu jarraian emandako soluzioa adierazten duela. Honako definizioa emango dugu,

\begin{definizioa}
$c_1,c_2,\dots,c_s \ \ (0\leq c_i \leq 1)$ zenbaki errealak izanik, $s$-mailako $u(t)$  \emph{kolokazio polinomioak} honakoa betetzen du,
\begin{align*}
u(t_0) &=y_0, \\
\dot{u}(t_0+c_ih) &=f(t_0+c_i h, u(t_0+c_i h)), \ \ i=1,\dots,s.
\end{align*}

Orduan soluzioa $y_1=u(t_0+h)\approx y(t_0+h)$ da.
\end{definizioa}

\begin{figure}[h]
\centering
\subfloat[kolokazio metodoak.]{
\includegraphics[width=.450\textwidth]{KolokazioMetodoa}
}
\caption{ \small kolokazio metodoak.}
\label{fig:kolokazio metodoak}
\end{figure}

\begin{teorema}
\textbf{Theorem 1.4 (Guillou and Soule 1969, Wright 1970).}
Kolokazio metodoaren definizioa eta jarraian emandako moduan kalkulatutako s-ataleko Runge-Kutta metodoa baliokideak dira.

\begin{equation}
a_{ij}=\int_{0}^{c_i} l_j(\tau) \ d\tau, \ \ b_i=\int_{0}^{1} l_i(\tau) \ d\tau
\end{equation}

non $l_i(\tau)$ Lagrangiar polinomioa dugu $l_i(\tau)=\prod_{l\neq i} \frac{(\tau-c_l)}{(c_i-c_l)}$.
\end{teorema}

\paragraph{\textbf{Definizioa}.} Gauss metodoak $c_i$ ($1 \leq i \leq s)$ koefizienteak "sth shifted Legendre" polinomioaren zeroak aukeratuz,

\begin{equation*}
\frac{d^s}{dx^s} \big(x^s(x-1)^s \big),
\end{equation*} 

Nodo hauetan oinarritutako Runge-Kutta metodoa $p=2s$ ordena du.

\paragraph*{\textbf{Gauss metodoaren koefizienteak}} kalkulatzeko bi aukera ditugu. Lehen aukera,  \emph{Mathematicak} NDSolve paketean inplementatuta dagoen koefizienteak kalkulatzeko funtzioa: ImplicitRungeKuttaGaussCoefficients[]. Bigarren aukera, guk koefizienteak kalkulatzeko garatutako funtzioa erabiltzea.  


%\begin{lstlisting} [language=Mathematica]
%
%GaussCoefficients[s_Integer, doi_, 
%                  a_Symbol, b_Symbol, c_Symbol] :=
%                  
%    Module[{f, g, ff, glist, B, A},
% 
%         Do[c[i] = N[(Root[LegendreP[s, #] &, i] + 1)/2, doi]
%                   // Simplify, {i, s}
%           ];
%           
%         ff = Collect[InterpolatingPolynomial
%                     [Table[{c[i], f[i]}, {i, s}], x], f[_]];
%                     
%         glist = Table[g[i] = Collect[ff, f[_],
%                                      %Simplify[\!\(\*SubsuperscriptBox[\(\[Integral]\), \(0\), \(c[
%          i]\)]\(#1 \[DifferentialD]x\)\)] &], {i, 1, s}];
%          
%          
%               yy = Collect[ff, f[_], Simplify[\!\(
%\*SubsuperscriptBox[\(\[Integral]\), \(0\), \(1\)]\(#1 \
%\[DifferentialD]x\)\)] &];
%                B = Table[b[i] = \!\(
%\*SubscriptBox[\(\[PartialD]\), \(f[i]\)]yy\), {i, 1, s}];
%               
% A = Table[a[i, j] = D[g[i], f[j]], {i, 1, s}, {j, 1, s}];
%         {Array[c, s], B, A}
% ]
%
%\end{lstlisting}

\begin{figure}[h!]
\centerline{\includegraphics[width=14cm, height=18cm] {Code_Gauss}}
\caption{Code Gauss.}
\label{fig:bost}
\end{figure}


\section{Konposizio eta Splitting metodoak.}

\subsection{Sarrera.}

Konposizio eta Splitting ideietan oinarrituz, aplikazio eremu ezberdinetarako hainbat integratzaile sinplektiko garatu dira. Metodo hauek ez dira orokorrak, problema zehatzetan aplikagarriak baizik, eta metodo oso eraginkorrak dira.  

\subsection{Konposizio metodoak.}

Konposizio metodoak, oinarrizko metodo bat edo gehiago konposatuz eraikitako zenbakizko integrazio metodoak dira.  Oinarrizko metodoekin segidan exekutatutako azpi-urrats kopuru batek, konposizio metodoaren integrazioaren urrats bat osatzen du. Helburua, orden baxuko metodo batetik abiatuta, orden altuko metodoa eraikitzea da; konposizio metodoak automatikoki konposatutako metodoaren propietateak (simetri, sinplektikoa,\dots) jasotzen ditu. 

\subsubsection*{Definizioa orokorra.}
$\phi_h$ oinarrizko metodoa eta $\gamma_1,\dots,\gamma_s$ zenbaki errealak emanik, urrats luzera hauen $\gamma_1 h,\gamma_2 h,\dots,\gamma_s h$ konposaketari dagokion konposizio metodoa,
\begin{equation}
\Psi_h=\phi_{\gamma_s h} \circ \dots \circ \phi_{\gamma_{1 h}}.
\end{equation}

%\paragraph*{\textbf{Teorema}.}
%Demagun $\phi_h$ urrats bakarreko eta $p$ ordeneko metodoa. Konposizio metodoa gutxienez $p+1$ ordeneko izango da baldin,
%\[\gamma_1+\dots+\gamma_s=1,\] 
%\begin{equation}
%\gamma_1^{p+1}+\dots+\gamma_s^{p+1}=0.
%\end{equation}

\subsubsection*{Algoritmoa.}
s-ataletako konposizio metodoen algoritmo orokorra honakoa izango litzateke:

\begin{algorithm}[H]
 \BlankLine
  \For{$n\leftarrow 0$ \KwTo $(endstep-1)$}
  {
   \BlankLine
    $Y_{0,n}=y_{n} $\;
    \BlankLine
   \For{i=1,2,...,s}
   {
    \BlankLine 
    $Y_{i,n}=\phi_{\gamma_i h}(Y_{i-1,n})$\;
   }
   \BlankLine
    $y_{n+1}=Y_{s,n}$\;
   \BlankLine
 }
 \caption{Konposizio metodoak.}
\end{algorithm}
 
Konposizio metodoei buruzko hainbat ohar azpimarratuko ditugu:
\begin{enumerate}
\item{Esplizitua.}

Konposizio metodo hauek esplizituak dira. Metodo hauetan ez da ekuazio sistemarik askatu behar, eta beraz inplementazioa sinplea da. 

\item{Sekuentziala.}

Azpi-urrats bakoitzaren kalkulua modu sekuentzialean egin behar dugu.

\item{Memoria gutxi.}

Ez da tarteko baliorik eta datu-egitura berezirik memorian gorde behar.   

\item{Oinarrizko metodoa: Störmer-Verlet.}

Bigarren ordeneko ekuazio diferentzialak ditugunean, Störmer-Verlet integratzailean oinarritzen diren konposizio metodoekin urrats bakoitzean $s$ ekuazio diferentzialaren balioztapena egin behar ditugu.

\end{enumerate}


\subsubsection*{CO1035: $10$ ordeneko konposizio metodoa.}

Sofronio eta Spalettaren ($2004$), $s=35$ eta $p=10$ ordeneko metodoa \cite{Sofroniou2005}, orain arteko lortutako orden altueneko konposizio metodo eraginkorrena kontsideratu daiteke (taula \ref{tab:31}). \emph{IRK} metodoa konposizio metodo honekin alderatuko dugu.     


Konposizio metodo simetrikoa da. Oinarrizko metodoa simetrikoa eta $p=2$ dela baliatuz eraikitako metodoa da. 
Integrazio metodo hau aplika daitekeen problemak, Hamiltondarra $H(q,p)=T(p)+U(q)$ modukoa izan behar dira (????).  

\begin{table}
\caption[C01035 konposizio metodoa.] 
{\small{10 ordeneko metodoa konposizio metodoa (CO1035).}}
\label{tab:31}       % Give a unique label
\centering
\resizebox{\textwidth}{!}{%
\begin{tabular}{ c c c c } 
 \hline
 Koefiziente         &  Balioa  & Koefizientea & Balioa \\
 \hline
 $\gamma_1=\gamma_{35}$ & 0.07879572252168641926390768 &  $\gamma_{10}=\gamma_{26}$ & -0.39910563013603589787862981 \\
 $\gamma_2=\gamma_{34}$ & 0.31309610341510852776481247 & $\gamma_{11}=\gamma_{25}$ & 0.10308739852747107731580277 \\ 
 $\gamma_3=\gamma_{33}$ & 0.02791838323507806610952027 & $\gamma_{12}=\gamma_{24}$ & 0.41143087395589023782070412 \\
 $\gamma_4=\gamma_{32}$ &-0.22959284159390709415121340 & $\gamma_{13}=\gamma_{23}$ & -0.00486636058313526176219566 \\ 
 $\gamma_5=\gamma_{31}$ & 0.13096206107716486317465686 & $\gamma_{14}=\gamma_{22}$ & -0.39203335370863990644808194 \\  
 $\gamma_6=\gamma_{30}$ & -0.26973340565451071434460973 & $\gamma_{15}=\gamma_{21}$ & 0.05194250296244964703718290 \\  
 $\gamma_7=\gamma_{29}$ & 0.07497334315589143566613711 & $\gamma_{16}=\gamma_{20}$ & 0.05066509075992449633587434 \\ 
 $\gamma_8=\gamma_{28}$ & 0.11199342399981020488957508 & $\gamma_{17}=\gamma_{19}$ & 0.04967437063972987905456880 \\  
 $\gamma_9=\gamma_{27}$ & 0.36613344954622675119314812 &$\gamma_{18}$ & 0.04931773575959453791768001 \\ 
   \hline
 \end{tabular}}
\end{table}


\paragraph*{\textbf{Gure inplementazioa.}} Gure abiapuntua Haireren konposizio metodoaren Fortran kodea izan da. Konposizio metodoaren azalpenak liburuko \cite{Hairer2006}  \emph{II.4} eta \emph{V.3} ataletan ematen ditu. \emph{GNI-Comp} izeneko kodea eskuragarri dago \cite{HairerGniComp} helbidean. Kodearen C-lengoaiako bertsioa garatu dugu.

\paragraph*{}$\phi_h$ metodoa $p=2$ ordenekoa eta simetrikoa izanik, era honetako konposizioak aurkitu dira,
\begin{equation}
\Psi_h=\phi_{\gamma_s h} \circ \phi_{\gamma_s-1 h} \circ \dots \circ \phi_{\gamma_{2 h}} \circ \phi_{\gamma_{1 h}} 
\end{equation}
non $\gamma_s=\gamma_1, \gamma_{s-1}=\gamma_2,\dots$ 


\paragraph*{\textbf{Koefizienteak}.}

Konposizio metodoaren oinarrizko metodoa \emph{Stömer-Verlet}  $\phi_h=\varphi_{h/2}^{1} \circ \varphi_{h}^{2} \circ \varphi_{h/2}^{1}$  dugula kontutan hartuta,

\begin{equation*}
\Psi_h=(\varphi_{h \gamma_s/2}^{1} \circ \varphi_{h \gamma_s}^{2} \circ \varphi_{h \gamma_s/2}^{1}) \circ \dots 
       \circ
       (\varphi_{h \gamma_2/2}^{1} \circ \varphi_{h \gamma_2}^{2} \circ \varphi_{h \gamma_2/2}^{1}) 
       \circ
       (\varphi_{h \gamma_1/2}^{1} \circ \varphi_{h \gamma_1}^{2} \circ \varphi_{h \gamma_1/2}^{1})  
\end{equation*}

\paragraph*{}Beraz jarraian dauden $\varphi^1$ fluxuak elkartuz,

\begin{equation*}
\Psi_h=\varphi_{h a_{s+1}}^{1} \circ \varphi_{h b_s}^{2} \circ \dots 
       \circ
       \varphi_{h a_3} \circ \varphi_{h b_2}^{2} 
       \circ
       \varphi_{h a_2}^{1} \circ \varphi_{h b_1}^{2} \circ \varphi_{h a_1}^{1}  
\end{equation*}

non $a_1=a_{s+1}=\gamma_1/2$, $b_i=\gamma_i$, $a_k=(\gamma_k+\gamma_{k-1})/2$, $i=1,\dots,s$ eta $k=2,\dots,s$.

\paragraph*{}Tarteko urratsetan, lehen atala $\varphi_{h a_1}^{1}$ eta azkena $\varphi_{h a_1}^{1}$ bakar batean elkar daitezke, 

\begin{equation*}
\Psi_h=\varphi_{h 2 a_{s+1}}^{1} \circ \varphi_{h b_s}^{2} \circ \dots 
       \circ
       \varphi_{h a_3} \circ \varphi_{h b_2}^{2} 
       \circ
       \varphi_{h a_2}^{1} \circ \varphi_{h b_1}^{2}.
\end{equation*}

\subsection{Splitting metodoak.}

\emph{Splitting metodoak}, bektore eremua $f: \mathbb{R}^d \rightarrow \mathbb{R}^d$ sistema osoa integratzeko baino errazagoa diren azpiproblemetan, $f^{[i]}$ ($f=\sum\limits_{i=1}^{m} f^{[i]}$), deskonposatu daitezkeen ekuazio diferentzialetarako zenbakizko integrazio metodoak dira.  

\paragraph*{}Maiz, jatorrizko $\dot{\mathbf{y}}=\mathbf{f}(\mathbf{y})$ problema era honetan bana daiteke,
\begin{equation}
\dot{\mathbf{y}}=\mathbf{f^{[1]}}(\mathbf{y})+\mathbf{f^{[2]}}(\mathbf{y}),
\end{equation} 
non $\dot{\mathbf{y}}=\mathbf{f^{[1]}}(\mathbf{y})$ eta $\dot{\mathbf{y}}=\mathbf{f^{[2]}}(\mathbf{y})$ sistemen fluxu zehatzak, $\varphi_t^{[1]}$ eta $\varphi_t^{[2]}$ esplizituki kalkula daitezkeen. 

\paragraph*{\textbf{Lie-Trotter splitting}.}
$p=1$ ordeneko metodoak,
\begin{equation}
\phi_h = \varphi_h^{[1]} \circ \varphi_h^{[2]} \ \ \ edo \ \ \ \phi_h^{*} = \varphi_h^{[2]} \circ \varphi_h^{[1]} .
\end{equation}

\paragraph*{\textbf{Strang-Marchuk splitting}.}
$p=2$ ordeneko metodo simetrikoa,
\begin{equation}
\phi_h =  \varphi_{{h}/{2}}^{[1]} \circ \varphi_h^{[2]} \circ \varphi_{{h}/{2}}^{[1]}
\end{equation} 

\paragraph*{\textbf{Splittig metodo orokorrak}.}
Konposizio metodoen modu berean, oinarrizko Splitting metodoak konposatuz orden altuagoko metodoak lortzen dira, 

\begin{equation}
\Psi_h=\varphi^{[1]}_{a_{s+1} h} \circ \varphi^{[2]}_{b_s h} \circ \varphi^{[1]}_{a_s h} \circ \dots \circ \varphi^{[1]}_{a_2 h} \circ \varphi^{[2]}_{b_1 h} \circ \varphi^{[1]}_{a_1 h}.
\end{equation}

$a_i,b_i$ koefizienteek metodoaren ordena definitzen dute. Metodoa simetrikoa bada $\Psi_h=\Psi_h^{*}$,
\begin{equation*}
a_1=a_{s+1}, \ b_1=b_{s}, \ a_2=a_s, b_2=b_{s-1}, \dots
\end{equation*} 

\subsubsection*{Algoritmoa.}
\emph{Splitting metodoen} algoritmo orokorra honakoa izango litzateke:

\begin{algorithm}[H]
 \BlankLine
  \For{$n\leftarrow 0$ \KwTo ($endstep$-1)}
  {
   \BlankLine
    $Y_{0,n}=y_{n-1} $\;
    \BlankLine
   \For{i=1,2,...,s}
   {
    \BlankLine 
    $Y_{i,n}=(\varphi^{[2]}_{b_i h} \circ \varphi^{[1]}_{a_i h})(Y_{i-1,n})$\ ;
   }
   \BlankLine
    $y_{n+1}=Y_{s,n}$\;
   \BlankLine
 }
 \caption{Splitting metodoak.}
\end{algorithm}

Splitting metodoen algoritmoan, konposizio metodoen algoritmoei buruz aipatutako ezaugarri berdinak errepikatu beharko genituzke. 

\subsubsection*{Fluxu zehatza eta zenbakizko fluxua konbinatuz.}
Demagun sistema $\dot{\mathbf{y}}=\mathbf{f}(\mathbf{y})$ era honetan banatzen dugula,
\begin{equation}
\dot{\mathbf{y}}=\mathbf{f^{[1]}}(\mathbf{y})+\mathbf{f^{[2]}}(\mathbf{y}),
\end{equation} 

Suposatu bakarrik $\dot{\mathbf{y}}=\mathbf{f^{[1]}}(\mathbf{y})$ sistemaren fluxua zehatza $\varphi_t^{[1]}$ kalkulatu daitekeela eta $\phi_t^{[2]}$, $\dot{\mathbf{y}}=\mathbf{f^{[2]}}(\mathbf{y})$ sistemari aplikatutako zenbakizko integrazio dugula. Orduan konposizio metodoaren oinarrizko metodoa honakoa kontsideratu daiteke,

\begin{equation*}
\phi_h=\varphi_h^{[1]} \circ \phi_h^{[2]}, \ \ \ \ \ \  \phi_h^{*}=\phi_h^{[2]*} \circ \varphi_h^{[1]}.
\end{equation*}


\paragraph*{} \emph{} Splitting metodoak konposizio metodoen interpretazioa emanez,
\begin{equation}
\Psi_h=\varphi^{[1]}_{a_{s} h} \circ \phi^{[2]}_{a_s h} \circ \phi^{[2]*}_{b_s h} \circ \varphi^{[1]}_{(b_s+a_s-1) h} \circ \phi^{[2]}_{a_s h} \circ \dots  \circ \phi^{[2]*}_{b_1 h} \circ \varphi^{[1]}_{b_1 h}.
\end{equation}


\subsubsection*{Eguzki-sistemari egokitutako splitting metodoak.}

Honakoa dugu N-gorputzeko problema grabitazionalaren Hamiltondarra,
\begin{equation*}
H(p,q)=T(p)+U(q).
\end{equation*}

Koordenatu sistema egokia (\emph{Jacobi} edo koordenatu Heliozentrikoak) erabiliaz,  Hamiltondarra beste modu honetan berridatzi daiteke,
\begin{equation*}
H=H_K+H_I,  \ \ |H_I|\ll|H_K|,
\end{equation*}
non alde nagusia $H_K$ planeta bakoitzaren eguzki inguruko mugimendu kepleriarra den eta $H_I$ aldiz, planeten arteko interakzioek eragiten duten perturbazio txikia.    

Eguzki-sistemaren N-gorputzeko problema grabitazionalari egokitutako zenbakizko bi integratzaile sinplektiko azalduko ditugu. Lehena, Laskarrek eta Robutelek \cite{Laskar2001} definitutako \emph{$SABAC_4$} integratzailea eta bigarrena, Blanes-ek \cite{Blanes2013} \cite{Farres2013} definitutako \emph{$ABAH1064$} integratzailea. 

\begin{enumerate}
\item Laskarren ($2001$) $SABAC_4$ zenbakizko integratzailea \cite{Laskar2001}.
$SABA_4$ integratzailea definitzen duten koefizienteak (taula \ref{tab:32}).
 
\begin{table}
\caption[$SABA_4$ splitting metodoa.] 
{\small{$SABA_4$ splitting metodoa.}}
\label{tab:32}       % Give a unique label
\begin{tabular}{ c c | c c} 
 \hline
 Koefiziente         &  Balioa  & Koefiziente         &  Balioa  \\
 \hline
 $c_1$ & $\frac{1}{2}-\frac{\sqrt{525+70\sqrt{30}}}{70}$ 
       & $d_1$ & $\frac{1}{4}-\frac{\sqrt{30}}{72}$\\
 $c_2$ & $\frac{\big( \sqrt{525+70 \sqrt{30}}-\sqrt{525-70 \sqrt{30}} \big)}{70}$ 
       & $d_2$ & $\frac{1}{4}+\frac{\sqrt{30}}{72}$\\
 $c_3$ & $\frac{\sqrt{525-70\sqrt{30}}}{35}$ & & \\
  \hline
 \end{tabular}
\end{table}

Hamiltondarra, $H=H_A+\epsilon H_B$ izanik eta goiko notazioa erabiliz, era honetan definituko dugu metodoa,
\begin{equation*}
SABA_4=\varphi^{[A]}_{c_1 h} \circ \varphi^{[B]}_{d_1 h} \circ \varphi^{[A]}_{c_2 h} \circ \varphi^{[B]}_{d_2 h}
         \circ  \varphi^{[A]}_{c_3 h}   \circ
          \varphi^{[B]}_{d_2 h} \circ \varphi^{[A]}_{c_2 h} \circ   \varphi^{[B]}_{d_1 h}\circ  \varphi^{[A]}_{c_1 h}.
\end{equation*}

\paragraph*{Corrected integrator.} Urrats bat gehitutako integratzailea $SABAC_4$,
\begin{equation*}
SABAC_4=\varphi{[B]}_\frac{-c}{2} \circ SABA_4 \circ \varphi{[B]}_\frac{-c}{2},
\end{equation*}
non $c=0.003396775048208601331532157783492144$.\\

\item $ABAH1064$ (Blanes, 2013).

Eguzki sistemaren integraziorako koordenatu Heliozentrikoei dagokion Hamiltondarra era honetakoa dugu,
\begin{equation*}
H(p,q)=H_K(p,q)+H_I(p,q), \ \ H_I(p,q)=T_1(p)+U_1(q). 
\end{equation*}

$H_I(p,q)$ fluxua zehazki kalkulatu ordez honen hurbilpen bat erabiliko dugu,
\begin{equation*}
\varphi_t^I \approx \tilde{\varphi}_t^I= \varphi_{{tb_i}/{2}}^{[U_1]} \circ \varphi_{tb_i}^{[T_1]} \circ \varphi_{{tb_i}/{2}}^{[U_1]}.
\end{equation*}

$ABAH1064$, $p=10$ eta $s=9$ splitting metodoa aztertuko dugu,
\begin{equation*}
ABAH1064=\prod\limits_{i=1}^{s} \varphi_{a_ih}^K \circ \tilde{\varphi}_{b_ih}^I
\end{equation*}
non $a_i$,$b_i$ koefizienteak beheko taulan (taula \ref{tab:33}) definitzen diren.  

\begin{table}
\caption[$ABAH1064$ splitting metodoa.] 
{\small{$ABAH1064$ splitting metodoa.}}
\label{tab:33}       % Give a unique label
\centering
\resizebox{\textwidth}{!}{%
\begin{tabular}{ c c c c } 
 \hline
 Koefiziente         &  Balioa  & Koefiziente         &  Balioa \\
 \hline
 $a_1=a_9$ & $0.04731908697653382270404371796320813250988$ & $b_1=b_9$ & $0.1196884624585322035312864297489892143852$ \\
 $a_2=a_8$ & $0.2651105235748785159539480036185693201078$  & $b_2=b_8$ & $0.3752955855379374250420128537687503199451$ \\
 $a_3=a_7$ & $-0.009976522883811240843267468164812380613143$ & $b_3=b_7$ & $-0.4684593418325993783650820409805381740605$ \\
 $a_4=a_6$ & $-0.05992919973494155126395247987729676004016$ & $b_4=b_6$ & $0.3351397342755897010393098942949569049275$ \\
 $a_5$ & $0.2574761120673404534492282264603316880356$ &  $b_5$ &  $0.2766711191210800975049457263356834696055$ \\
  \hline
 \end{tabular}}
\end{table}

\end{enumerate} 


\subsection{Kepler fluxua.}

Bi gorputzen problema edo Kepler problemari dagokion Hamiltondarra,
\begin{equation}
H(\bf{p},\bf{q})=\frac{\mathbf{p}^2}{2m}-\frac{\mu}{\|\mathbf{q}\|}.
\end{equation}

Elkar erakartzen diren bi gorputzen mugimendua kalkulatzeko, gorputz baten kokapena koordenatu sistemaren jatorria kontsideratuko dugu. Notazio hau finkatuko dugu, 
\begin{equation*}
m=(1/m_1+1/m_2)^{-1},\ \ \mu=Gm_1m_2,
\end{equation*} 

Hamiltondarrari dagokion bigarren ordeneko ekuazio diferentzialak,
\begin{equation}
\ddot{\mathbf{q}}= - \frac{k\mathbf{q}}{\|\bf{q}\|^3} ,
\end{equation}
non $k= \mu / m$ eta  $\mathbf{q}\in \mathbb{R}^3$.


\subsubsection*{Inplementazioa}

\textbf{Ideia nagusia}. Koordenatu cartesiarretatik koordenatu eliptikoetara $(a,e,i,\Omega,E)$ itzulpena egingo dugu. Kontutan hartuta $E$ (izena??) aldagai ezik beste aldagaiek konstante mantentzen direla, $E_0$ abiapuntua harturik, $\triangle t$ denbora tartea aurrera egingo dugu $E_1$ balioa berria kalkulatzeko. Azkenik, koordenatu eliptikoetatik koordenatu cartesiarrak berreskuratuko ditugu kokapen eta abiadura berriekin. 

\begin{equation*}
(\bf{q_0},\bf{v_0}) \in \mathbb{R}^6 \ \ \ \longrightarrow \ \ \  (a,e,i,\Omega,E_0) \in \mathbb{R}^6 
\end{equation*}

\begin{equation*}
\downarrow \triangle t
\end{equation*}

\begin{equation*}
(\bf{q_1},\bf{v_1}) \in \mathbb{R}^6 \ \ \ \longleftarrow \ \ \  (a,e,i,\Omega,E_1) \in \mathbb{R}^6 
\end{equation*}

\paragraph*{\textbf{Newton metodoa}.} Kepler-en ekuazioan oinarrituz ($E-e\sin E=n (t-t_p)$),  $E_1=\triangle E+E_0$ balioa kalkulatuko Newtonen metodoa aplikatuz,

\begin{equation*}
f(\triangle E)=\triangle E - ce \sin(\triangle E)- se (\cos(\triangle E)-1)-n \triangle t=0
\end{equation*}
\begin{equation}
\triangle E^{[k+1]}=\triangle E^{[k]}- \frac{f(\triangle E^{[k]})}{f'(\triangle E^{[k]})}
\end{equation}

\paragraph*{\textbf{Ekuazioak}.} Gure inplementazioan erabilitako ekuazio guztien azalpenak eta definizoak eranskinean eman ditugu.

\section{Laburpena.}

Metodo sinplektikoei buruzko liburu monografiko hauek gomendatuko ditugu: Sanz-Serna and M.P. Calvo’s Numerical Hamiltonian Problems (1994) \cite{JMSanz-Serna1994}; E. Hairer, C. Lubich and G. Wanner’s Geometrical Numerical Integration (2001) \cite{Hairer2006}; B. Leimkuhler and S. Reich’s Simulating Hamiltonian Dynamics (2004) \cite{Leimkuhler2004}; Feng, Kang and Qin, Mengzhao  Symplectic geometric algorithms for hamiltonian systems (2010) \cite{Feng2010}.


\begin{table}{h}
\caption{Integrazio metodoen laburpena}
\label{tab:1}       % Give a unique label
\begin{tabular}{ c|c c c } 
           &  C1035             &  ABAH1064           & GAUSS-12           \\
 \hline
 	       & Konposizio met.    & Splitting met.     & IRK met.            \\
 	       & Sofronio (2004)    & Blanes et al. 2013 &                     \\
 \hline 
               &                    &                    &                 \\
 Hamiltoniarra & Orokorra           & Perturbatua        & Orokorra        \\ 	    
 Mota          & Esplizitua         & Esplizitua         & Inplizitua      \\ 
 Ordena        & 10                 & 10                 & 12              \\ 
 Atalak        & 35                 & 9                  & 6               \\ 
 Parall.       & Ez                 & Ez                 & Bai             \\  
\end{tabular}
\end{table}
 
 
\chapter{Problemak.}

\section{Sarrera.}


Tesiaren erdigunea N-gorputzeko problema grabitazionala da. Gure helburua, eguzki-sistemaren problemaren simulaziorako zenbakizko integratzaile eraginkorra inplementatzea da. Eguzki-sistemaren bi eredu sinple kontsideratu ditugu: lehena, \emph{kanpo-planeten} izeneko problema (eguzkia, kanpo-planetak eta Plutonek osatutakoa), eta bigarren eredu konplexuagoa, \emph{9-planeten problema} izeneko problema (eguzkia, 8 planetak eta Plutonek osatutakoa). Bi eredu hauetan, gorputzak masa puntualak dira eta gorputz hauen arteko erakarpen grabitazionalak bakarrik kontsideratu ditugu (erlatibitate efektua eta beste hainbat indar ez-grabitazionalak gure lan-eremutik kanpo utzi ditugu).

Zenbakizko metodo sinplektiko nagusienak esplizituak direla eta Hamiltondar banagarriak diren problemei aplika daitezkeela, gogoratu behar dugu. Hortaz gain, problema zurruna bada metodo esplizituak ez direla eraginkorrak eta metodo inplizituek abantaila azaltzen dutela esan beharra dago. Gauss metodoa orokorra  eta inplizitua izanik, gure inplementazioa problema zurrunetarako eta Hamiltondar banagarria ez den problemetarako  aplikagarria dela ere erakutsi nahi dugu. 
 
Aurretik esandakoari jarraituz, pendulu bikoitza, problema osagarri gisa aukeratu dugu zenbakizko esperimentuak modu aberatsago eta zabalago batean egiteko. Pendulu bikoitzaren bi bertsio kontsideratu ditugu: pendulu bikoitz arrunta eta pendulu bikoitz zurruna.

N-gorputzeko problema grabitazionalaren Hamiltondarra banagarria da baina pendulu bikoitzarena aldiz, ez da banagarria. Bestalde, pendulu bikoitzari malguki bat gehituz  problema zurruna bilakatuko dugu eta problema hauen zailtasunei nola aurre egin erakusteko. Gainera,  eguzki-sistema kaotiko \cite{Laskar1999}  kontsideratzen dela  jakinik, pendulu bikoitz arruntak izaera kaotikoa azaltzen duen hasierako balio zehatzak aukeratu ditugu. Problema kaotikoak,  hasierako balio edo parametroen perturbazioekiko, trunkatze edo birbitze erroreekiko esponentzialki oso sentikorrak dira.

Atal honetan, tesiaren zenbakizko esperimentuetan erabili ditugun problemak deskribatu ditugu, problema bakoitzari dagokion Hamiltondarra eta hasierako balioak zehaztuz. Lehenik, pendulu bikoitzaren problema eta N-gorputzen problema orokorra azaldu ditugu. Bigarrenik, eguzki-sistemaren problema grabitazionalean murgildu gara eta eredu ezberdin zehaztapenak eman ditugu: Kepler problema, Hiru-gorputzen problema, kanpo-planeten problema eta 9-planeten problema. 

\section{Pendulu bikoitza.}
\label{s:32}

Pendulu bikoitzaren bi bertsio deskribatuko dugu: lehena, pendulu bikoitz arrunta eta bigarren problema konplexuagoa, pendulu bikoitz zurruna. 

\subsection{Pendulu bikoitz arrunta.}
\label{ss:321}

Planoan pendulu bikoitzaren problema (ikus \ref{fig:dp} irudia) era honetan definituko dugu: $m_1$ eta $m_2$ masadun bi penduluz eta elkar lotuta dauden $l_1$ eta $l_2$ luzerako makilez (masa gabekoak kontsideratuko ditugunak) osatuta sistema mekanikoa. Sistemaren egoera aldagaiak, bi angelu $q=(\phi,\theta)$ eta dagozkion momentuak $p=(p_{\phi},p_{\theta})$ dira. $\phi$ lehen penduluaren ardatz bertikalarekiko angelua da eta bigarren penduluaren angelua, era honetan definituko dugu $\psi=\phi+\theta$.

\begin{figure} [h]
\centerline{\includegraphics [width=10cm, height=8cm] {MyDoublePendulum}}
\caption{Pendulu bikoitz arrunta.}
\label{fig:dp}
\end{figure} 

\paragraph*{Hamiltondar funtzioa} $H(q,p)$  honakoa da,

\begin{multline}
 \label{eq:2}
-\frac{ {l_1}^2 \ (m_1+m_2) \ {p_{\theta}}^2 +{l_2}^2 \ m_2 \ (p_{\theta} -p_{\phi})^2 + 2 \ l_1 \ l_2 \ m_2 \ p_{\theta} \ (p_{\theta} -p_{\phi}) \  \cos(\theta )} {{l_1}^2  \ {l_2}^2 \ m_2 \  (-2 \ m_1 - m_2 + m_2 \ \cos(2 \theta ))} \\
-g  \ \cos (\phi) \  (l_1 \ (m_1+m_2)+l_2 \ m_2 \ \cos(\theta))+g \ l_2 \ m_2 \ \sin(\theta) \sin(\phi),
\end{multline}

\paragraph*{Sistemaren parametroak.} 
Gure esperimentuetarako honako parametroak kontsideratuko ditugu,
\begin{equation*}
 \label{eq:17}
g=9.8 \ {m}/{s^2}\ ,\ \ l_1=1.0 \ m \ , \ l_2=1.0 \ m\ , \ m_1=1.0 \ kg\ , \ m_2=1.0 \ kg.
\end{equation*} 

\paragraph*{Hasierako balioak.}
Bi hasierako balio ezberdin kontsideratu ditugu \cite{Dumitru}: lehenak, izaera ez-kaotiko du eta bigarrenak, izaera kaotikoa duen mugimendua agertzen du.

\begin{enumerate}
   \item Hasierako balio ez-kaotikoak (NCDP): 
   \ $q(0)=(1.1, \ -1.1)$  eta $p(0)=( 2.7746,\ 2.7746)$. $T_{end}=2^{12}$ segundoko integrazioa egin dugu eta zenbakizko soluzioa, $m=2^{10}$ urratsero itzuli dugu.   
   
   \item Hasierako balio kaotikoak (CDP):      
    $q(0)=(0, \ 0)$ eta  $p(0)=(3.873,\ 3.873)$. $T_{end}=2^{8}$ segundoko integrazioa egin dugu eta zenbakizko soluzioa, $m=2^{8}$ urratsero itzuli dugu.  
\end{enumerate}

Urrats luzera, $h=2^{-7}$ aukeratu dugu, trunkatze errorea biribiltze errorea baino txikiagoa izan dadin.

\subsection{Pendulu bikoitz zurruna.}
\label{ss:322}

Pendulu bikoitz arruntari malguki bat gehitutako sistema da (ikus \ref{fig:dp_zurruna}~irudia) : $l_1$ eta $l_2$ luzerako makilez elkar lotuta dauden  $m_1$ eta $m_2$ masadun bi pendulu eta hauen artean $k$ malgutasun duen malgukia osatzen duten sistema mekanikoa. $k=0$ balioarentzat, problema ez da zurruna eta problemaren zurruntasuna, $k$ balioarekin batera handitzen da. 

\begin{figure} [h]
\centerline{\includegraphics [width=10cm, height=8cm] {MyDoublePendulumSTIFF}}
\caption{Pendulu bikoitza (zurruna).}
\label{fig:dp_zurruna}
\end{figure} 

Sistemaren egoera aldagaiak, bi angelu $q=(\phi,\theta)$ eta dagozkion momentuak $p=(p_{\phi},p_{\theta})$ dira.  $\phi$ lehen penduluaren ardatz bertikalarekiko angelua da eta bigarren penduluaren angelua, era honetan definituko dugu  $\psi=\phi+\theta$.

\paragraph*{Hamiltondar funtzioa.} 
Formulazio Lagrangiarrean ($L=T-V$), energia potentzialari $1/2 \ k \ \theta^2$ gaia gehituz, dagokion $H(q,p)$ funtzio Hamiltondarra lortuko dugu,
\begin{multline}
\label{eq:Hpb2}
-\frac{ {l_1}^2 \ (m_1+m_2) \ {p_{\theta}}^2 +{l_2}^2 \ m_2 \ (p_{\theta} -p_{\phi})^2 + 2 \ l_1 \ l_2 \ m_2 \ p_{\theta} \ (p_{\theta} -p_{\phi}) \  \cos(\theta )} {{l_1}^2  \ {l_2}^2 \ m_2 \  (-2 \ m_1 - m_2 + m_2 \ \cos(2 \theta ))} \\
-g  \ \cos (\phi) \  (l_1 \ (m_1+m_2)+l_2 \ m_2 \ \cos(\theta))+g \ l_2 \ m_2 \ \sin(\theta) \sin(\phi)+\frac{k}{2} \ \theta^2 ,
\end{multline}

\paragraph*{Sistemaren parametroak.} 
Honako parametroak kontsideratu ditugu,
\begin{equation*} \label{eq:17}
g=9.8 \ {m}/{s^2}\ ,\ \ l_1=1.0 \ m \ , \ l_2=1.0 \ m\ , \ m_1=1.0 \ kg\ , \ m_2=1.0 \ kg,
\end{equation*} 
eta $k$ malgutasun parametroaren balio batzuk finkatu ditugu, zurruntasun maila ezberdineko pendulu bikoitzaren dinamikak aztertzeko 
\begin{equation*}
k=2^{2i}, \ \ i=0,\dots,11.
\end{equation*}  

\paragraph*{Hasierako balioak.}
Hasierako balioak, era honetan aukeratu ditugu:
\begin{enumerate}
\item  $k=0$ problemarako, \cite{Dumitru} artikulutik izaera ez-kaotikoa duen hasierako balioak hartu ditugu: $q(0)=(1.1, -1.1)$ and $p(0)=(2.7746,2.7746)$.

\item  $k\neq 0$ problemetarako hasierako balioak,
\begin{equation*}
q(0)=\left(1.1, \frac{-1.1}{\sqrt{1+100k}}\right), \ \ 
p(0)=(2.7746,2.7746),
\end{equation*}
aukeratu ditugu, non sistemaren energia $k \rightarrow \infty$ doanean bornatua dagoen.

\end{enumerate}

$k=0$ problema ez-zurrunerako, $h=2^{-7}$ urrats luzera, trunkatze errorea biribiltze errorea baino txikiagoa izan dadin finkatu dugu eta gainontzeko integrazio guztietarako urrats luzera berdina erabili dugu. $k>0$ zurruntasun balio batetik aurrera, trunkatze errorea birbiltze errorea baino garrantzitsuago da. $T_{end}=2^{12}$  segundoko integrazioa egin dugu eta zenbakizko soluzioa, $m=2^{10}$ urrats oro itzuli dugu.   

\section{N-Gorputzen problema.}
\label{s:33}

Problemaren formulazio eman aurretik, K.Tanikawa-k eta T.Ito-k \cite{Ito2007} 3-gorputzen problemari buruzko deskribapena aipatuko nahi genuke,

\begin{displayquote}
Never be attracted to the three-body problem. It is too dangerous. The three-body problem has long been an attractive but dangerous subject for students. This is because it has quite a simple setting and it appears relatively easy. However, it has been investigated for so many years that it very difficult to obtain anything new.
\end{displayquote}

Newtonen lege grabitazionalen araberako N-gorputzen problemaren ekuazio diferentzialak era honetan definitzen dira,
\begin{equation}
m_i\ddot{q_i}= G \sum_{j=0,j \neq i}^{N} \frac{m_im_j}{\|q_j-q_i\|^3} (q_j-q_i) , \ \  i=0,1,\dots, N,
\end{equation}
non $(N+1)$ gorputz kopurua den, eta $q_i\in \mathbb{R}^3$, $m_i \in \mathbb{R}, \ \ i=0,\dots,N$ gorputz bakoitzaren kokapena eta masa den. 

\paragraph*{Hamiltondar sistema.}
Momentuen definizio hau ordezkatuz  $p_i=m_i*\dot{q}_i$, N-gorputzeko problemaren formulazio Hamiltondarra  lortzen da,  

\begin{equation}
H(q,p)=\frac{1}{2}\ \sum^N_{i=0}{\ \frac{{\|p_i\|}^2}{m_i}}-G\ \sum^N_{0\le i<j\le N}{\frac{m_im_j}{\|q_i-q_j\|}}. 
\end{equation}

\paragraph*{Ekuazio diferentzialak.} Abiaduraren eta kokapenaren araberako ekuazioak hauek dira,
\begin{align}
\begin{split}
\dot{q}_i &=v_i, \ \  i=0,1,\dots, N,\\
\dot{v}_i &= \sum_{j=0,j \neq i}^{N} \frac{Gm_j}{\|q_j-q_i\|^3} (q_j-q_i) , \ \  i=0,1,\dots, N
\end{split}
\end{align}


\paragraph*{Problemaren integralak.}
Integrazioan zehar konstante mantentzen diren kantitateei problemaren integralak edo inbarianteak deitzen zaie. N-gorputzen problemak $10$ integral ditu \cite{Klioner2016}:
\begin{enumerate}


\item Masa zentroaren sei integralak.

Era honetan definitzen dugun $P$, konstantea dela modu errazean froga daiteke, 
\begin{equation*}
P=\sum_{i=0}^{N} m_i \dot{q}_i=\sum_{i=0}^{N} p_i=kons. 
\end{equation*}
Eta ondorioz,
\begin{equation*}
O=\sum_{i=0}^{N} m_i {q}_i=Pt+B, \ B=kons. 
\end{equation*}

$P,B \in \mathbb{R}^3$ bektoreen osagaiei, masa zentroaren sei integralak esaten zaie. Masa zentroaren kokapen ($Q$) eta abiadura ($V$)  era honetan definitzen dira, 
\begin{equation*}
Q={\left(\sum\limits_{i=0}^{N} m_i \ q_i\right)}/{M}, \ V={\left(\sum\limits_{i=0}^{N} m_i \ \dot{q}_i\right)}/{M}
\end{equation*}
non $M=\sum\limits_{i=0}^{N}Gm_i$ den.


\item Momentu angeluarra.

Momentu angeluarra $L\in \mathbb{R}^3$, problemaren beste hiru integralak ditugu, 
\begin{equation*}
L=\sum_{i=0}^{N} p_i \times q_i=\sum_{i=0}^{N} \dot{q}_i \times m_i q_i=kons.
\end{equation*}

\item Energia.

Hamitondar sistema osoaren energia da eta problemaren beste integrala da,
\begin{equation*}
E=H(q,p)=kons.
\end{equation*}

\end{enumerate}

Problemaren $10$ integral hauek, sistemaren ordena gutxitzeko edo zenbakizko integrazioa doitasuna neurtzeko erabili daitezke. Guk koordenatu barizentrikoak (koordenatu sistemaren jatorria masa zentroaren kokapena) erabiliko ditugu eta koordenatu hauetan, $P=0$ eta $B=0$ dira. Beraz, gorputzen kokapen eta abiadurak $\hat{Q}=0$ eta $\hat{V}=0$ izan daitezen, integratzeko erabiliko ditugun hasierako balioak ($\hat{q}_i,\hat{v}_i$) modu honetan finkatuko ditugu,
\begin{align*}
&\hat{q}_i=q_i-R, \\
&\hat{v}_i=v_i-V, \ \ i=0,\dots,N.
\end{align*}

Momentu angeluarra eta energia, zenbakizko integrazioaren doitasuna neurtzeko erabili ohi dira. Energia zenbakizko integrazioen biribiltze errorea neurtzeko integral egokiena da.

\section{Eguzki-sistema.}
\label{ss:34}

\subsection{Sarrera.}

Eguzki-sistemaren planeten orbiten mugimenduaren eredu matematikoa, Hamiltondar sistema bati dagokion ekuazio diferentzial arrunten bidez formulatzen da. Ekuazio diferentzial arrunten sistema, $6N$ ordenakoa da ($N$ planeten kopurua izanik).

Eguzki sistemaren eredu sinplea integratuko dugu. Eguzki-sistemaren gorputzak masa puntualak kontsideratuko ditugu eta gure ekuazio diferentzialak definitzeko, soilik gorputz hauen arteko erakarpen grabitazionalak kontutan hartu ditugu. Eguzki-sistemaren eredu konplexuagoen erlatibitate efektua, gorputzen formaren eragina, eta beste zenbait indar ez-grabitazionalak ez ditugu kontsideratu.

Eguzki-sistemaren gorputz nagusien (eguzkia eta planetak) integrazioetara mugatuko gara. Ereduen gorputz kopurua txikia izango da, $N=6$ (eguzki-sistemaren eredu sinplea) eta $N=10$ (eguzki sistemaren eredu osoa) gorputz kopuruen artekoa.

Eguzkiak, planetak baino $1.000$ aldiz masa handiago du eta hauxe da, eguzki-sistemaren ezaugarri garrantzitsuenetakoa: eguzkiaren grabitazio indar nagusi batek eta planeten arteko perturbazio txikiek sortzen duten sistema dinamikoa da. Planetak eta bere sateliteen mugimenduan kolisio gertuko egoerarik edo eszentrikotasun handiko orbitarik ez dago. Beraz, ikuspuntu honetatik sistema dinamiko sinplea dela, kontsideratu daiteke. Eguzki-sistema egonkorra kontsideratzen da, hau da, hurrengo bilioi urteetan  planeten arteko talkarik edo planeten kanporatzerik gertatzea, ez da espero  \cite{Laskar1999} \cite{Hayes2007}.

Egonkortasunaren azterketa zehatzago batean, sistema erregularrak eta kaotikoak bereiz ditzakegu. Sistema erregularretan, hasierako balioen perturbazio txikien eragina, denboran zehar poliki hasiko da  eta beraz, epe luzeko integrazio zehatzak posible dira. Sistema kaotikoetan aldiz, eboluzioa hasierako balioekiko oso sentikorra da eta gertuko soluzioen diferentzia esponentzialki hasiko da. Hasierako uneko diferentzia $d(0)=d_0$ bada, orduan distantzia,
\begin{equation*}
d(t)\approx d(0)e^{t \lambda}
\end{equation*}  
espresioaren arabera hasiko da, non $\lambda$ \emph{Lyapunov esponentziala} esaten zaion. Eguzki-sistema kaotikoa kontsideratzen da eta Laskar-ek \cite{Laskar1999} \emph{Lyapunov denbora} $\lambda^{-1}\approx 5$ milioi urtetan finkatzen du eta honako espresioa, kalkulatzeko  proposatzen du,
\begin{equation*}
d(t)\approx d(0)10^{t / 10}.
\end{equation*}     
Espresio honen arabera,  $10^{-10}$ tamainako hasierako erroreak baditugu , orduan $d(10)\approx 10^{-9}$ eta $d(100)\approx 1$ dira. Ondorioz, eguzki sistemaren $10-20$ milioi urteko integrazioetarako, doitasun handiko soluzioak lortuko ditugu baina $100$ milioi urteko integrazioen soluzioak ez dira esanguratsuak izango.   

Eguzki-sistemaren jatorria, orain $5  \times 10^9$ urtetan finkatzen da eta beraz, eguzki-sistemaren eboluzioa integratzeko, urrats kopuru oso handia beharrezkoa da. Eguzki-sistemaren eredu osoaren (9-planeten problema) integrazioetako ohiko urratsa  $h=0.0025$ urteko bada (orbita txikieneko periodoaren $ \%1$ ), eman beharreko urrats kopurua $2 \times 10^{12}$ izango da. Era berean, eguzki-sistemaren eredu sinpleagoaren (kanpo-planeten problema) integrazioetako urrats tamaina handiago bada ere, urrats kopurua $5 \times 10^{10}$ ingurukoa da.   

Eguzki-sistemaren problemaren eskalak, anitzak dira. \ref{fig:lbes} irudian, planeta nagusien tamainak  eta eguzkiarekiko distantziak irudikatu ditugu. Ondorengo \ref{tab:eguz-sist} taulan, eguzki-sistemaren planeten ezaugarri nagusienak eman ditugu. Denbora eskalari dagokionez, ilargiak lurraren inguruko orbita $27.32$ egunetakoa da, lurrak eguzkiarekiko inguruko orbitaren periodoa $1$ urtekoa eta Neptunorena $163$ urtekoa.
\begin{figure}[h!]
\centering
\begin{tabular}{c c}
\subfloat[\small {Planeten tamainak.}]
{\includegraphics[width=.5 \textwidth]{PanetenMasak}}
&
\subfloat[\small {Planeten eguzkiarekiko distantziak.}]
{\includegraphics[width=.5\textwidth]{PlanetenDistantziak}}
\end{tabular}
\caption{ \small  Ezkerreko (a) irudian, planeten arteko tamainen proportzioak eta eskubiko irudian, planeten eguzkiarekiko distantziak irudikatu ditugu.}
\label{fig:lbes}
\end{figure} 


\begin{table} [h!]
\caption{Eguzki-sistemaren planeta nagusien masak, eguzkiarekiko distantziak, orbitaren periodoa eta eszentrikotasuna.}
\label{tab:eguz-sist}       % Give a unique label
\begin{tabular}{l l l l l} 
\hline
 Planeta   &  Masak                 & Distantzia   & Periodoa    & Eszentrikotasuna\\   
           &  kg                    & AU           &   urteak    &               \\ \hline
 Eguzkia   &  $1.99 \times 10^{30}$ &              &             &               \\         
 Merkurio  &  $3.30 \times 10^{23}$ & $0.39$       &  $0.24$     &  $0.205$      \\
 Artizarra &  $4.87 \times 10^{24}$ & $0.72$       &  $0.007$    &  $0.007$      \\
 Lurra     &  $5.97 \times 10^{24}$ & $1.00$       &  $1.007$    &  $0.017$      \\
 Marte     &  $6.42 \times 10^{23}$ & $1.52$       &  $1.88$     &  $0.094$      \\ \hline
 Jupiter   &  $1.90 \times 10^{27}$ & $5.20$       &  $11.86$    &  $0.049$      \\
 Saturno   &  $5.68 \times 10^{26}$ & $9.54$       &  $29.42$    &  $0.057$      \\
 Urano     &  $8.68 \times 10^{25}$ & $19.19$      &  $83.75$    &  $0.046$      \\
 Neptuno   &  $1.02 \times 10^{26}$ & $30.06$      &  $163.72$   &  $0.011$      \\
 Pluton    &  $1.31 \times 10^{22}$ & $39.53$      &  $248.02$   &  $0.244$      \\
\hline
\end{tabular}
\end{table}

Hiru dira erabiltzen diren koordenatu sistema nagusienak:

\begin{enumerate}
\item Koordenatu cartesiarrak.
\item Koordenatu heliozentrikoak.
\item Koordenatu jacobiarrak.
\end{enumerate}

Ohikoa da ekuazio diferentzialak koordenatu heliozentrikoen (eguzkiaren zentroarekiko) arabera definitzea. 
Ekuazioen garapen osoa eranskinean eman dugu.


\subsection{Problemak.}
\label{ss:342}

Eguzki-sistemaren simulaziorako test problemak deskribatuko ditugu. Bi gorputzen problematik abiatuta, gero eta problema konplexuagoak azalduko ditugu. PW, Sharp-ek \cite{Sharp2001} eguzki-sistemaren problemen bilduma interesgarria egin zuen , eta bertan problema guzti hauek problema ez-zurrunak kontsideratzen dituela nabarmentzekoa da.      

\subsubsection*{Kepler problema.}
\label{ss:keplerproblem}

Kepler problema, bi gorputzen problemaren kasu partikularra da eta  honako Hamiltondarra dagokio,
\begin{equation}
H(p,q)=\frac{p^2}{2m}-\frac{\mu}{\|q\|},
\end{equation}
non $m$ eta $\mu$ konstanteen balioak, formulazioaren araberakoak diren.

Koordenatu sistema $q=q_2-q_1$ duen formulazioa aukeratzen badugu, konstanteen balioak hauek dira,  
\begin{equation*}
m=(1/m_1+1/m_2)^{-1},\ \ \mu=Gm_1m_2,
\end{equation*} 

eta ekuazio diferentzialak era honetan definitzen dira,
\begin{equation}
\dot{q}=p, \ \ \dot{p}= - \frac{k \ q}{\|q\|^3} ,
\end{equation}
non $k= \mu / m$ eta  $q,p \in \mathbb{R}^3$.

Kepler hasierako baliodun problemaren soluzio zehatza kalkula daiteke: une bateko kokapen eta abiadurak emanik, denbora tarte bat ($\Delta t$) igarotakoan (positibo ala negatiboa), zehazki kokapen eta abiadura berriak ezagutu daitezke. Eguzki-sistemaren integrazio metodoentzat, Kepler problemaren doitasun handiko eta kalkulu eraginkorra konputatzea, funtsezkoa da. Erreferentziazko Kepler problemaren inplementazioak, Danby \cite{Danby1992} eta J.Wisdom  \cite{Wisdom2015} ditugu. 

Kepler problemaren inplementazioa garatu dugu eta ideia nagusia, hitzez azalduko dugu. Lehenik, koordenatu cartesiarretatik ($q,p\in \mathbb{R}^3$), koordenatu eliptikoetara $(a,e,i,\Omega,E)$, itzulpena egingo dugu. Koordenatu eliptikoetan, $E$ (\emph{eccentric anomaly}) aldagaia izan ezik, beste aldagaiak konstante mantentzen dira: beraz $E_0$ balioa emanda, $\Delta t$ denbora tartea aurrera egin eta $E_1$ balioa berria kalkulatuko dugu. Azkenik, koordenatu eliptikoetatik koordenatu cartesiarretara itzulpena eginez, kokapen eta abiadura berriak eskuratuko ditugu. 

\begin{equation*}
(q_0,v_0) \in \mathbb{R}^6 \ \ \ \longrightarrow \ \ \  (a,e,i,\Omega,E_0) \in \mathbb{R}^6 
\end{equation*}
\begin{equation*}
\quad \quad \quad \quad \quad \quad \quad \quad \downarrow \Delta t
\end{equation*}
\begin{equation*}
(q_1,v_1) \in \mathbb{R}^6 \ \ \ \longleftarrow \ \ \  (a,e,i,\Omega,E_1) \in \mathbb{R}^6 
\end{equation*}

Inplementazioaren garapenaren zehaztasun guztiak eranskinaren \ref{erans:A1} atalean eman ditugu.

\subsubsection*{Hiru-gorputzen problema.}

Hiru-Gorputzen Problema Murriztuan (\emph{R3BP}) \cite{Hairer1993,Corless2013},  
masa ezberdineko hiru gorputz elkarrekiko erakarpen eraginpean, espazioan libreki mugitzen dira. Honako hurbilpenak kontsideratuz, definitzen da Hiru Gorputzen Problema Murriztua:

\begin{itemize}
\item Ohikoa den moduan Newton eredu grabitazionalean, gorputzak masa puntualak kontsideratuko ditugu. 
\item  Horietako bi gorputz nagusiak ($m_1,m_2$), bere masa zentroaren inguruan orbita zirkularrean mugitzen dira. Adibidez, lurra eta ilargia kontsideratu daitezke. Ilargiak ezentrizidade txikiko ($e=0.05$) mugimendu eliptikoa du eta suposizio hau onargarria da.
\item Problema orokorrean, hiru gorputzen masak edozein izan daitezke. Murriztuan aldiz, bi gorputz nagusien masa edozein delarik, hirugarrenarena beste bi gorputzen masa baino askoz ere txikiagoa da. Adibidez satelite artifizial bat kontsideratzen badugu, bere masa $1.000$ tonatako ($10^6$ kg) ingurukoa izango da  eta ilargiaren masarekin konparatuko bagenu ($I_M=7.3477 \times 10^{22}$ kg), bere masa askoz ere txikiagoa da.
\item Hirugarren gorputzak ($m_3$) beste bi gorputz nagusien masarekin alderatuta oso masa txikia duenez, ez du $m_1$ eta $m_2$ gorputzen mugimenduarekiko eraginik. Hirugarren gorputza, bi gorputz nagusien eraginpean hauen orbitaren plano berdinean mugitzen dela suposatuko dugu.
\item Eguzki-sistemaren gainontzeko gorputzen eragina baztertu da. Lurra-Ilargi sisteman, eguzkiaren grabitazio indarra ez dagoela kontsideratu da. 
\end{itemize} 

$(x, y)\in \mathbb{R}^2$ masa txikiko gorputzaren kokapen koordenatuak izanik, hauek dira errotazio koordenatu sisteman dagokion mugimenduaren ekuazioak ($4$ dimentsioko sistema dinamikoa),
\begin{align*}
&\dot{x} =p_x+y,\\
&\dot{y} =p_y-x,\\
&\dot{p}_x =p_y-umu  \ \frac{x+\mu}{r_1^3}-\mu \ \frac{x-(umu)}{r_2^3},\\
&\dot{p}_y =-p_x-umu \ \frac{y}{r_1^3}-\mu \ \frac{y}{r_2^3},
\end{align*}
non $r_1=((x+\mu)^2+y^2))^{1/2}$, $r_2=((x-umu)^2+y^2)^{1/2}$, $-\mu=(-{m}/{M+m})$ eta $umu=1-\mu$ diren.

Problema ezberdinak deskribatzen dira $\mu$ parametroaren arabera: $\mu = 0.01277471$ balioarekin finkatuta, Lurra-Ilargia modeloari dagokio. Gorputz nagusi handiena (Lurra) ($-\mu$, $0$) eta gorputz
nagusi txikiena (Ilargia) ($1 - \mu$, $0$) posizioan kokatzen dira. Ilargiak orbita zirkularra du lurraren inguruan eta ekuazio diferentzialek satelitearen mugimendua deskribatzen dute.

Sistema dinamikoaren energia, soluzioan zehar konstante mantentzen da,
\begin{equation*}
E=\frac{1}{2} (p_x^2+p_y^2)+p_x y-p_y x - \left(\frac{1-\mu}{r_1}\right)-\frac{\mu}{r_2}-\frac{1}{2} \mu (1-\mu).
\end{equation*}

Ezaguna da \cite{Hairer1993}, hasierako balio hauetarako (\ref{tab:r3bp0} taula) soluzioa
\[t=17.0652165601579625588917206249\]
 periodikoa dela (\ref{fig:41a} irudia),

\begin{table}[h]
\caption[R3BP problemaren hasierako balioak.]{R3BP problemaren hasierako balioak.}
\label{tab:r3bp0}       % Give a unique label
\centering
\resizebox{\textwidth}{!}{%
\begin{tabular}{ c c c c}
\hline 
Gorputza         & Balioak \\
\hline
Satelitea        &  $x,y$         & $0.994$ & $0$  \\
                 &  $p_x,p_y$     & $0.$ & $-2.00158510637908252240537862224 + 0.994$ \\\hline
 
\end{tabular}}
\end{table}

\begin{figure}[h!]
\centering
\subfloat[Arenstorf orbita.]{
\includegraphics[width=.500\textwidth]{r3bp1}
%plot3a-2
}
\subfloat[Kolisiotik gertuko egoera.]{
\includegraphics[width=.400\textwidth]{r3bp2}
%plot3b
}
\caption[R3BP.]
        {\small R3BP. 
        Arenstorf orbita izeneko soluzio periodikoa.        
        $t=0$ unean kolisiotik gertuko egoera; Ilargia
        ($0.987723, 0$) eta satelitea ($0.994, 0$) posizioa.
        
        }
\label{fig:41a}
\end{figure}      

\subsubsection*{Kanpo-planeten problema.}


Kanpo-planeten problemaren ereduan, eguzkia, lau planeta nagusiak (Jupiter, Saturno, Urano, Neptuno) eta Pluton kontsideratuko ditugu. Eguzki-sistemaren kanpo-planeten  mugimenduaren azterketa interesgarria da. Lehenik, planetan nagusi hauen eboluzioa eguzki-sistema osoaren zati garrantzitsuena da eta barne-planeten mugimendua kontutan hartzeak ala ez, kanpo-planeten zenbakizko integrazioarengan oso eragin txikia du. Bigarrenik, urrats luzera handi erabili daiteke eta beraz, epe luzeko integrazioak errazten dira (konputazio denbora gutxiago behar delako). Hirugarrenik, Pluton orbitaren berezitasunak ikertzea,  $1960-1980$ urteetan interes handikoa izan zen.        


Hasierako balioak \cite{Hairer2006} liburutik hartu ditugu. Planetei dagokien masak \ref{tab:ossm0} taulan eta kokapenak/abiadurak \ref{tab:oss0} taulan laburtu ditugu. planeten masak eguzkiarekiko erlatiboak dira, hau da, eguzkiaren masa $1$ da eta grabitazio konstantea $G=2.95912208286 \ 10^{-4}$. Barne-planeten masak eguzkiaren masari gehitu zaio eta horregatik, eguzkiaren masak, $m_0=1.00000597682$ balioa hartzen du.

\begin{table}[h]
\caption{Kanpo-planeten masak.}
\label{tab:ossm0}       % Give a unique label
\centering
\begin{tabular}{ l r }
\hline 
  Gorputza         &  Masa        
\\\hline
  Eguzkia          &  $1.000005976823$ \\
  Jupiter          &  $0.000954786104043$ \\
  Saturno          &  $0.000285583733151$ \\
  Urano            &  $0.0000437273164546$ \\
  Neptuno          &  $0.0000517759138449$ \\
  Pluton           &  ${1}/{(1.3 \ 10^8)}$ \\
\hline  
\end{tabular}
\end{table}

\begin{table}[h]
\caption[Kanpoko planeten problema.]{Kanpo-planeten problemaren hasierako balioak, kokapenak ($x,y,z$) eta abiadurak ($v_x,v_y,v_z$).}
\label{tab:oss0}       % Give a unique label
\centering
\resizebox{\textwidth}{!}{%
\begin{tabular}{ l l r r r }
\hline 
  Gorputza       &  Balioa \\
  \hline
  Eguzkia        &  $x,y,z$         & $0.$ & $0$ &	$0.$    \\
                 &  $v_x,v_y,v_z$   & $0.$ & $0.$ & $0.$    \\
  Jupiter        &  $x,y,z$         & $-3.5023653$ &  $-3.8169847$ & $-1.5507963$ \\
                 &  $v_x,v_y,v_z$   & $0.00565429$ &  $-0.00412490$ & $-0.00190589$ \\                    
  Saturno        &  $x,y,z$         &  $9.0755314$	& $-3.0458353$ & $-1.6483708$	    \\
                 &  $v_x,v_y,v_z$   &  $0.00168318$ & $0.00483525$ & $0.00192462$     \\
  Urano          &  $x,y,z$         &  $8.3101420$ & $-16.2901086$ & $-7.2521278$  \\
                 &  $v_x,v_y,v_z$   &  $0.00354178$ & $0.00137102$ & $0.00055029$ \\
  Neptuno        &  $x,y,z$         &  $11.4707666$ &	$-25.7294829$	& $-10.8169456$    \\
                 &  $v_x,v_y,v_z$   &  $0.00288930$  &	$0.00114527$ & $0.00039677$     \\
  Pluton         &  $x,y,z$         &  $-15.5387357$ &  $-25.2225594$ & $-3.1902382$ \\
                 &  $v_x,v_y,v_z$   &  $0.00276725$ &	$-0.00170702$ & $-0.00136504$ \\
\hline       
\end{tabular}}
\end{table}


\subsubsection*{9-planeten problema.}


Eguzki-sistemaren 9-planeten zenbakizko integrazioak, kanpo-planeten problemak baino konplexutasun handiago du. Planeten eta eguzkiaren arteko interakzio kopurua $45$ (kanpo-planeten probleman $15$) da. Orbita periodoa txikiena  $50$ aldiz txikiagoa (Merkurioren $0.24$ urtetako periodoa, Jupiterren $11.86$ urtetako periodoarekin alderatuta) da. Merkurioren orbitaren eszentrikotasuna $e=0.206$ (Jupiterren orbitaren eszentrikotasuna $e=0.048$) da. 

Eredu honetan, lur-ilargi sistema (\emph{EMB}) masa puntual bakarra kontsideratzen da. Lur-ilargi sistemaren masa, bi gorputzen masen arteko batura da eta kokapena, lur-ilargi sistemaren barizentroan finkatzen da.
  
Hasierako balioak \emph{DE-430} ($2.014$) \cite{Folkner2014} azken efemeride artikulutik hartu ditugu. Eguzki eta planeten hasierako kokapenak (AU) eta abiadurak (AU/egun), Julian data (TDB) $2440400.5$ ($1969$. ekainaren $28$) eta ICRFR2 (International Celestial Reference Frame) koordenatu sisteman \ref{tab:9bodyhas} taulan laburtu dugu. Planeta bakoitzari dagokion, $GM$ balioa \ref{tab:9bodymas} taulan laburtu dugu. 

\begin{table}[h]
\caption{Planeten $GM$ balioak.}
\label{tab:9bodymas}       % Give a unique label
\centering
\begin{tabular}{l c }
\hline 
  Gorputza         &  GM ($au^3/day^3$)          \\
  \hline
  Eguzkia          &  $0.295912208285591100e-03$ \\
  Merkurio         &  $0.491248045036476000e-10$ \\
  Artizarra        &  $0.724345233264412000e-09$ \\
  Lurra            &  $0.888769244512563400e-09$ \\
  Marte            &  $0.954954869555077000e-10$ \\
  Jupiter          &  $0.282534584083387000e-06$ \\
  Saturno          &  $0.845970607324503000e-07$ \\
  Urano            &  $0.129202482578296000e-07$ \\
  Neptuno          &  $0152435734788511000e-07$ \\
  Pluton           &  $0.217844105197418000e-11$ \\
  Ilargia          &  $0.109318945074237400e-10$ \\
\hline
\end{tabular}
\end{table}

\begin{table}[h]
\caption[9-planeten problemaren hasierako balioak]{Eguzki eta 9 planeten hasierako balioak, kokapenak ($x,y,z$) eta abiadurak ($v_x,v_y,v_z$).}
\label{tab:9bodyhas}       % Give a unique label
\centering
\resizebox{\textwidth}{!}{%
\begin{tabular}{ l l r r r }
\hline 
  Gorputza       &  Balioa \\
  \hline
  Eguzkia        &  $x,y,z$         & $0.00450250878464055477$ & $0.00076707642709100705$ &	$0.00026605791776697764$    \\
                 &  $v_x,v_y,v_z$   & $-0.00000035174953607552$ & $0.00000517762640983341$ & $0.00000222910217891203$    \\
  Merkurio       &  $x,y,z$         &  $0.36176271656028195477$ & $-0.09078197215676599295$ &	$-0.08571497256275117236$ \\
                 &  $v_x,v_y,v_z$   &  $0.00336749397200575848$ & $0.02489452055768343341$ &	$0.01294630040970409203$ \\
  Artizarra      &  $x,y,z$         &  $0.61275194083507215477$ & $-0.34836536903362219295$	& $-0.19527828667594382236$ \\
                 &  $v_x,v_y,v_z$   &  $0.01095206842352823448$ & $0.01561768426786768341$ &	$0.00633110570297786403$\\
  EMB            &  $x,y,z$         &  $0.12051741410138465477$ & $-0.92583847476914859295$ &	$-0.40154022645315222236$\\
                 &  $v_x,v_y,v_z$   &  $0.01681126830978379448$ & $0.00174830923073434441$ &	$0.00075820289738312913$\\
  Marte          &  $x,y,z$         & $-0.11018607714879824523$ & $-1.32759945030298299295$ &	$-0.60588914048429142236$ \\
                 &  $v_x,v_y,v_z$   &  $0.01448165305704756448$ & $0.00024246307683646861$ & $-0.00028152072792433877$   \\
  Jupiter        &  $x,y,z$         &  $-5.37970676855393644523$ & $-0.83048132656339789295$ & $-0.22482887442656542236$ \\
                 &  $v_x,v_y,v_z$   & $0.00109201259423733748$ & $-0.00651811661280738459$ &	$-0.00282078276229867897$\\                  
  Saturno        &  $x,y,z$         &  $7.89439068290953155477$ & $4.59647805517127300705$ &	$1.55869584283189997764$	    \\
                 &  $v_x,v_y,v_z$   &  $-0.00321755651650091552$ & $0.00433581034174662541$ & $0.00192864631686015503$     \\
  Urano          &  $x,y,z$         &  $-18.26540225387235944523$ &	$-1.16195541867586999295$ &	 $-0.25010605772133802236$\\
                 &  $v_x,v_y,v_z$   &  $0.00022119039101561468$ & $-0.00376247500810884459$ &	$-0.00165101502742994997$ \\
  Neptuno        &  $x,y,z$         &  $-16.05503578023336944523$ &	$-23.94219155985470899295$ &	 $-9.40015796880239402236$    \\
                 &  $v_x,v_y,v_z$   & $0.00264276984798005548$ & $-0.00149831255054097759$ &	$-0.00067904196080291327$     \\
  Pluton         &  $x,y,z$         &  $-30.48331376718383944523$ & $-0.87240555684104999295$ &	 $8.91157617249954997764$ \\
                 &  $v_x,v_y,v_z$   &  $0.00032220737349778078$ & $-0.00314357639364532859$ &	$-0.00107794975959731297$\\
\hline       
\end{tabular}}
\end{table}
                 


\subsubsection*{Laskar-en eredua.}

Eguzki-sistemaren mugimenduaren azterketa zehatza egiteko, planeten eta ilargiaren orbiten mugimenduaren ekuazioak nahiz lur eta ilargiaren errotazio mugimenduaren ekuazioak integratu behar dira. 

Laskar-ek, $2.011.$ urteko epe luzeko zenbakizko integraziorako \cite{Laskar2011} eguzki-sistemaren eredua deskribatuko dugu. Hasierako integrazioetan, eguzkia, 8 planetak, Pluton eta ilargia bakarrik kontsideratu zituen. Eguzkiaren erlatibitate efektua (Saha eta Tremain-ek \cite{Saha1994} finkatutako teknikaren arabera) eta eredu errealistaren indar ez grabitazional garrantzitsuenak aplikatu zituen. Azken integrazioetan, Zeres, Palas, Vesta, Iris eta Bamberga asteroideak gehitu zituen zituen.  
 

\begin{table} [h!]
\caption{}
\label{tab:laskp}       % Give a unique label
\begin{tabular}{l r r r r } 
\hline
 Planeta   &  Distantzia   & Periodoa    & GM             & Ezentrizitatea \\   
           &   AU          &   urte      & ($au^3/egun^3$) & \\ \hline
 Eguzkia    &               &             & $0.2959e-03$   & \\          
 Merkurio   &   $0.39$      &  $0.24$     & $0.4912e-10$   & $0.205$ \\
 Artizarra  &   $0.72$      &  $0.007$    & $0.7243e-09$   & $0.007$\\
 Lurra      &   $1.00$      &  $1.007$    & $0.8887e-09$   & $0.017$\\
 Ilargia    &               &             & $0.1093e-10$   & $0.055$\\ 
 Marte      &   $1.52$      &  $1.88$     & $0.9549e-10$   & $0.094$\\ \hline
 Jupiter    &   $5.20$      &  $11.86$    & $0.2825e-06$   & $0.049$\\
 Saturno    &   $9.54$      &  $29.42$    & $0.8459e-07$   & $0.057$\\ 
 Urano      &   $19.19$     &  $83.75$    & $0.1292e-07$   & $0.046$\\
 Neptuno    &   $30.06$     &  $163.72$   & $0.1524e-07$   & $0.011$ \\ \hline
 Zeres      &   $2.77$      &  $4.6$      & $0.1400e-12$   & $0.07$ \\
 Palas      &   $2.77$      &  $4.61$     & $0.3104e-13$   & $0.23$ \\
 Vesta      &   $2.36$      &  $3.63$     & $0.3854e-13$   & $0.08$\\
 Iris       &   $2.38$      &  $3.68$     & $0.2136e-14$   & $0.21$ \\
 Bamberga   &   $2.68$      &  $4.39$     & $0.1388e-14$   & $0.34$ \\ \hline
 Pluton     &   $39.53$     &  $248.02$   & $0.2178e-11$   & $0.244$ \\
\hline
\end{tabular}
\end{table}

Ilargia gorputz independente gisa kontsideratu zuen. Ilargiaren lurrarekiko distantzia ($380.000$ km) , beste gorputzekiko distantziekin alderatzen badugu (eguzkiarekiko $150.000.000$ km eta Artizarrarekiko $45.000.000$ km) oso txikia da. Hori dela eta, ilargiaren kokapena, eguzki-sistemaren barizentroarekiko kontsideratu ordez, lurrarekiko kontsideratuz doitasun handiago lortuko da. Lurraren eguzkiarekiko kokapena ($q_e$) eta ilargiaren eguzkiarekiko kokapen ($q_m$), hurrenez-hurren, lur-ilargi sistemaren barizentroaren eguzkiarekiko kokapena ($q_B$) eta  ilargiaren lurrarekiko kokapena ($q_{em}$) aldagaiekin ordezkatzen dira,
\begin{align*}
& q_B =\frac{Gm_e \ q_e+Gm_m \ q_m}{Gm_e+Gm_m},\\
& q_{em} =q_m-q_e.
\end{align*}
Argitzea komeni da, ekuazio diferentzialaren eskubiko aldeko espresioa ebaluatzeko ($q_e,q_m$) aldagaiak erabiliko ditugula eta aldagai berriak. ($q_B,q_{em}$) integratzeko erabiliko ditugula.

\begin{algorithm}[H]
 \BlankLine
  $\mbox{Lurra, Ilargia}=\{q_B,q_{em}\}$\;
  \For{$i\leftarrow 1$ \KwTo $endstep$}
  {
   \BlankLine
     $\{q_e,q_m\} \leftarrow \{q_B,q_{em}\} $\;
     $\mbox{Ebaluatu} \ \dot{y}=f(y)$\;
     $ \{q_B,q_{em}\} \leftarrow \{q_e,q_m\} $\;
     $\mbox{Integrazioa}\ (q_B,q_{em})$\;
   \BlankLine
  }
 \caption{Ilargiaren kalkuluak.}
\end{algorithm}

\begin{table}[h]
\caption{Ilargiaren Lurrarekiko hasierako balioak.}
\label{tab:1}       % Give a unique label
\centering
\resizebox{\textwidth}{!}{%
\begin{tabular}{ c l c c c }
\hline 
  Gorputza       &  Balioa \\
\hline
  Ilargia        &  $x,y,z$         & $-0.00080817735147818490$ &	$-0.00199462998549701300$ &	$-0.00108726268307068900$    \\
                 &  $v_x,v_y,v_z$   & $0.00060108481561422370$ & $-0.00016744546915764980$ &	$-0.00008556214140094871$ \\
\hline
\end{tabular}}
\end{table}     
          

\section{Laburpena.}

Atal honetan, pendulu bikoitzaren problema eta eguzki-sistema grabitazionalaren eredu ezberdinen zehaztasunak eman ditugu. Batetik, pendulu bikoitzaren problemaren hasierako balioak \cite{Dumitru} artikulutik hartu ditugu. Bestetik, eguzki-sistema grabitazionalaren problemaren integraziorako hasierako balioak lan hauetatik hartu ditugu: kanpo-planeten problemarentzat, \cite{Hairer2006} liburutik eta 9-planeten problemarentzat $2014.$ urteko efemerideen \cite{Folkner2014} artikulutik hartu ditugu. Eguzki-sistemaren problemaren integraziorako hasierako balioak jasotzen dituzten beste lan hauek ere aipatu nahi genituzke: P.W. Sharp-ek eguzki-sistemaren problemen bilduma \cite{Sharp2001} eta Laskar-en \cite{Laskar2009} artikuluaren informazio osagarria.
\chapter{Koma higikorreko aritmetika.}
\label{sec:4}

\section{Sarrera.}

Konputagailuetan, zenbaki errealak ($\mathbb{R}$) bit kopuru finituaren bidez adierazi behar dira eta honetarako, koma-higikorreko adierazpen sistema ($\mathbb{F}$) erabiltzen da. Zenbaki erreal batzuk, $\mathbb{F}$ sisteman adierazpen zehatza dute, baina beste batzuk hurbildu egin behar dira.  Era berean, eragiketa aritmetikoen ($+,-,*,/$) kalkulu gehienetan ere, emaitzaren hurbilpena egin beha da. $\mathbb{R}$ sistematik $\mathbb{F}$ sistemara bihurtzeko funtzioari biribiltzea esaten zaio. Oro har, konputazio zientzian, biribiltze errore honen eragina garrantzitsua da eta errorea gutxitzeko ahalegin berezia beharrezkoa da.

Egungo konputagailuen koma-higikorreko aritmetikaren inplementazioak, \emph{IEEE-$754$} estandarrean \cite{IEEE2008} oinarritzen dira. 
\emph{IEEE-$754$} estandarrak, koma-higikorreko aritmetikaren konputaziorako formatu eta metodoak definitzen ditu. Konputazioen fidagarritasuna eta aplikazioen portabilitatea bermatzen ditu.    
 
Atal honetan, koma-higikorreko aritmetika eta biribiltze errorearen oinarria azalduko ditugu. Ondoren, konputazioetan biribiltze erroreak gutxitzeko teknika ezagun batzuk azalduko ditugu. 

\section{\emph{IEEE-754} estandarra.}

Koma-higikorreko zenbaki multzoa finitua da eta ${\mathbb{F}}$ izendatuko dugu. Koma-higikorreko adierazpen zehatza duten zenbaki errealei koma-higikorreko zenbakiak deritzogu, 
\begin{equation*}
\mathbb{F}\subset \mathbb{R}.
\end{equation*}

$\mathbb{F}$ zenbaki multzoa, \ref{fig:FloatNumberLine}irudian laburtu dugu. Bai zenbaki positiboentzat, bai negatiboentzat, adieraz daitekeen zenbaki handienaren eta txikienaren arteko balio bakanez osatuta dago. Multzoaren kanpoaldean zenbaki hauek guztiak ditugu: batetik overflow tartean $(-\infty,\max_{x \in \mathbb{F_{-}}}|x|)$  eta $(\max_{x \in \mathbb{F_{+}}}|x|,\infty+)$ daudenak; bestetik underflow tartean  $(\min_{x \in \mathbb{F_{-}}}|x|,0)$ eta $(0,\min_{x \in \mathbb{F_{+}}}|x|)$ daudenak. 

\begin{figure}[h]
\centerline{\includegraphics[width=14cm, height=3cm] {ZenbakiErrealak}}
\caption[Koma-higikorreko zenbakien multzoa]{Koma-higikorreko zenbakien multzoa}
\label{fig:FloatNumberLine}
\end{figure} 

IEEE-754 estandarraren arabera, $n$-biteko koma-higikorreko adierazpenak bi zati ditu (ikus \ref{fig:32bitKomaHigikorra} irudiko adibidea),
\begin{enumerate}
\item $m$ bitez osatutako zatia, mantisa ($M$) izenekoa. Horietako bit batek ($S$) zeinua adierazten du. Bestalde $M$ mantisa modu normalizatu honetan emana da, $\pm 1.F$ eta zati erreala ($F$) bakarrik gorde behar da.   
\item Esponentea ($E$), $(n-m)$ bitez adierazitako zenbaki osoa. Zeinuarentzat ez da bit zehatzik, baizik \emph{bias} izeneko balio bat kenduz adierazten dira zenbaki positiboak eta negatiboak.  
\end{enumerate}

Beraz, oinarri bitarrean koma-higikorreko zenbaki hauek adierazten dira,
\begin{equation*}
M \times b^E, \ b=2,
\end{equation*}
eta biribiltze unitatea (\emph{unit roundoff}) era honetan definituko dugu,
\begin{equation*}
u=2^{-m}.
\end{equation*} 

\begin{figure}[h]
\centerline{\includegraphics[width=12cm, height=2cm] {ZenbakiErrealak2}}
\caption[32-biteko koma-higikorra]{\small $32$-biteko koma-higikorreko zenbakiaren adierazpena: esponentearentzat  8-bit eta mantisarentzat  $24$-bit (bit bat zeinuarentzat eta beste $23$ bit, $1.F$ eran normalizatutako mantisarentzat) banatuta.}
\label{fig:32bitKomaHigikorra}
\end{figure} 

IEEE-$754$ estandarrean, oinarri bitarreko koma-higikorreko hiru formatu definitzen dira: bata doitasun arrunta (\emph{single precision}), bestea doitasun bikoitza (\emph{double precision}) eta hirugarrena doitasun laukoitza (\emph{quadruple precision}) izenekoak (\ref{tab:koma-higikorreko-aritmetikak} Taula).

\begin{table} [h!]
\caption[IEEE-754 koma-higikorreko formatuak]{IEEE-754 koma-higikorreko formatuak}
\label{tab:koma-higikorreko-aritmetikak}       % Give a unique label
\centering
\begin{tabular}{ l c c c l c} 
 \hline
 Formatoa      &  Tamaina    & Mantisa   & Esponentea  & Tartea           &  $u=2^{-m}$          \\
               &    n        & m         & n-m         &                  &                      \\
   \hline
% Half     & 16 bit      & 11  & 5  & $2^{\pm 16}$     &  $5 \times 10^{-4}$   \\ 
 Arrunta   & 32 bit      & 24  & 8  & $10^{\pm 38}$    &  $6 \times 10^{-8}$   \\	    
 Bikoitza  & 64 bit      & 53  & 11 & $10^{\pm 308}$   &  $1 \times 10^{-16}$   \\
 Laukoitza & 128 bit     & 113 & 15 & $10^{\pm 11356}$ &  $1 \times 10^{-35}$   \\
\hline
\end{tabular}
\end{table}


Doitasun bikoitzeko oinarrizko eragiketak (batuketa, kenketa, biderketa, zatiketa, eta erro karratua) hardware bidez exekutatzen dira \cite{Muller2009} eta azkarrak dira. Makina ziklo bakoitzeko, $2$ eta $4$ batuketa, kenketa edo biderketa egin ohi dira; zatiketa eta erro karratua aldiz, eragiketa motelagoak dira. Bestalde, doitasun arruntaren  aritmetika, doitasun bikoitza baino azkarragoa da: garraiatu behar den bit kopuru erdia delako eta gainera, hardware bereziei esker (adibidez \emph{Intel} makinetan \emph{SSE} moduluak), eragiketa aritmetikoak azkarragoak direlako. 2008. urtean, IEEE-$754$ estandarrak, $128$-biteko koma-higikorreko aritmetika onartu zuen, baina  inplementazioa  softwarez bidezkoa da eta exekuzioa, gutxi gorabehera, doitasun bikoitzeko aritmetika baino 10-15 aldiz motelagoa da.

Problema batzuk, doitasun bikoitza baino doitasun handiagoa behar dute \cite{Joldes2016}. Doitasun laukoitza edo altuagoa, software liburutegien bidez emulatu ohi dira. Doitasun altuko zenbakiak adierazteko nagusiki bi modu bereizten dira:   

\begin{enumerate}
\item \emph{Digitu-anitzeko adierazpena}. Zenbakiak esponente bakarra eta mantisa bat baino gehiagorekin adierazten dira (adb. \emph{GNU MPFR liburutegia} \cite{Fousse2007}).
\item \emph{Termino-anitzeko adierazpena}. Zenbakiak  ebaluatu gabeko hainbat koma-higikorreko makina zenbaki estandarren batura gisa adierazten dira (adb. Bailey QD liburutegia) \cite{Hida2001} eta exekuzioaren ikuspegitik, hardware bidezko inplementazioaren abantaila dute.    
\end{enumerate}

Doitasun laukoitzeko gure esperimentuetarako, \emph{GCC libquadmath} liburutegia \cite{libquad} erabili dugu. Doitasun laukoitzean exekutatutako integrazioen zenbakizko soluzioak, soluzio zehatzak kontsideratu ditugu eta  doitasun bikoitzeko inplementazioaren errorea, soluzio zehatzarekiko diferentzia gisa kalkulatu dugu. 

Laskar-ek epe luzeko eguzki-sistemaren simulazioaren ($-250$ eta $+250$ milioitako integrazio tartea) konputaziorako kalkuluak \cite{Laskar2011}, kontu handiz eta doitasun handian egin behar ditu. Dena den, era honetako problemak salbuespenak dira eta ez da ohikoa izaten doitasun handian lan egin beharra. Egia da ere, neurri fisiko oso gutxi ezagutzen direla  hain doitasun handian (adibidez $50$-bitekin, Lurra eta Ilargiaren arteko distantzia, milimetroko errorearekin adieraz daiteke).  


\section{Biribiltze errorea.}

Zenbakizko integrazioen errorea, trunkatze eta biribiltze errorez osatuta dago. Urrats luzera nahi bezain txikia aukeratuz, trunkatze errorea biribiltze errorea baino txikiago izango da eta beraz, zenbakizko integrazio hauetan errorean biribiltze errorea nagusitzen da. Epe luzeko eta doitasun handiko integrazioetan, urrats luzera txikia erabiltzen denez, biribiltze errorea gutxitzea funtsezkoa izango da.     

Bi biribiltze errore mota bereiziko ditugu, bata adierazpenaren errorea eta bestea, aritmetikaren errorea.  

\subsection*{Adierazpenaren errorea.} 

Zenbaki erreal batzuk, $\mathbb{F}$ koma-higikorreko multzoan zehazki adieraz daitezke eta beste batzuk ordea, hurbilpen batez adierazi behar dira. $x \in \mathbb{R}$ izanik, $fl: \mathbb{R} \rightarrow \mathbb{F}$ koma-higikorreko zenbakia esleitzen dion funtzioari deituko diogu:  $x \in \mathbb{R}$ balioaren gertuen dagoen  $fl(x) \in \mathbb{F}$ itzultzen duen funtzioa bezala definitzen da. Hau da, $f_1,f_2 \in \mathbb{F}$ jarraian dauden koma-higikorreko zenbakiak  badira eta $x \in \mathbb{R}, \ f_1\leqslant x \leqslant f_2$ bada,
\begin{equation*}
fl(x)=
\left\{
        \begin{array}{lc}
        f_1 & \mathrm{if} \ |x-f_1| < |x-f_2| \\
        f_2 & \mathrm{if} \ |x-f_1| \geqslant |x-f_2| 
        \end{array}.
\right.
\end{equation*}  

Jarraian, koma-higikorreko adierazpenaren errore absolutua eta errore erlatiboa finkatuko ditugu.
\begin{itemize}
\item Errore absolutua,
\begin{equation*}
\triangle x= fl(x)-x= \tilde{x}-x. 
\end{equation*} 
\item Errore erlatiboa, 
\begin{equation*}
\delta x =\frac{\triangle x}{x} = \frac{\tilde{x}-x}{x}. 
\end{equation*}
\item Aurreko bi definizioen ondorioz honako formula erabilgarria dugu,
\begin{equation*}
\tilde{x}= x+\triangle x = x \ (1+\delta x).
\end{equation*}
\end{itemize}

Koma-higikorreko zenbaki sistema bitarrean ($m=$ mantisa adierazteko bit kopurua izanik) $|x|$ balioa, $\mathbb{F}$ multzoaren zenbaki txikienaren eta handienaren artean badago,
\begin{equation*}
 |\delta x|< u \ \ \text{non} \ \ u=2^{-m},
 \end{equation*}
bermatuta dagoela froga daiteke \cite{Corless2013}.

\subsection*{Aritmetikaren errorea.} 

Koma-higikorreko zenbakien arteko eragiketa baten emaitzak, ez du zertan $\mathbb{F}$ multzoan adierazpen zehatza izan  eta orduan, emaitza biribildu egingo  da. Adibidez, $m$ digituzko bi zenbakien biderketaren emaitza zehatza adierazteko, $2m$ digituzko mantisa behar dugu ($m$ digituzko galera) \cite{Fukushima2001}. Salbuespena, biderkagaietako bat $2$-ren berretura denean gertatzen da, orduan biderketa zehatza baita.

\paragraph*{Adibidea.} Demagun lau digitu hamartar errealeko aritmetikarekin ari garela lanean.

Emaitza zehatza, $1,343 \times 2,103 = 2,824229$. 

Hiru digitu hamartar errealeko aritmetika, $1,343 \times 2,103 \approx 2.824$.

\paragraph*{} Hauek zenbaki errealen arteko funtsezko eragiketak badira,  $\ast: \mathbb{R}^2\rightarrow \mathbb{R}$, 
\begin{equation*}
\ast\in \{+,-,\times,/ \},
\end{equation*}
koma-higikorreko zenbakien arteko funtsezko eragiketak era honetan izendatuko ditugu  $\circledast: \mathbb{F}^2\rightarrow \mathbb{F}$,
\begin{equation*}
\circledast\in \{\oplus,\ominus,\otimes,\oslash \}.
\end{equation*}

$\tilde x,\tilde y \in \mathbb{F}$ emanik eta $z= \tilde x \ast \tilde y$ emaitza zehatza bada, $\tilde z= \tilde x \circledast \tilde y$ (edo $\tilde z= fl(\tilde x \ast \tilde y$)) eragiketaren emaitzaren errore absolutua eta errore erlatiboa definituko ditugu,

\begin{itemize}
\item Errore absolutua,
\begin{equation*}
\triangle z=\tilde z-z =(\tilde x \circledast \tilde y) -(\tilde x \ast \tilde y).
\end{equation*} 
\item Errore erlatiboa,
\begin{equation*}
\delta z=\frac{\triangle z}{z}==\frac{(\tilde x \circledast \tilde y) -(\tilde x \ast \tilde y)}{(\tilde x \ast \tilde y)}.
\end{equation*} 
\item Honako erlazio hau ondorioztatu daiteke,
\begin{equation*}
\tilde z=(\tilde x \circledast \tilde y)=z+\triangle z=z \ (1+\delta z).  
\end{equation*}
\end{itemize}

Koma-higikorreko aritmetikan, \ $|\delta z|<u$ \ , non  $u=2^{-m}$, beteko dela froga daiteke \cite{Corless2013}.

\paragraph*{} Zenbakizko algoritmoen biribiltze errorearen eraginaren azterketa formalak, propietate hauetan oinarritzen dira. Bestalde, errore erlatiboak emaitzaren digitu zuzenak neurtzen du:
\begin{equation*}
\delta z \approx 10^{-k} \Rightarrow \ \approx \ k \ \mbox{digitu hamartar zuzen}.
\end{equation*}  


\subsection*{Biribiltze errorearen hedapena.}


Ohiko konputazioetan, eragiketa aritmetiko kopuru handia egin behar dugu emaitza lortzeko. Batzuetan, eragiketen biribiltze erroreak elkar ezereztatzen dira baina kasu txarrenean, biribiltze errorea metatu eta magnitude handikoa izan daiteke.   

\paragraph*{Adibidea.} 
Modu honetako batura batean , non $n>2$ eta $\tilde x_1,\dots,\tilde x_n \in \mathbb{F}$,  
\begin{equation*}
\bigoplus_{i=1}^{n}(\tilde x_i)=(\sum\limits_{i=1}^{n} \tilde x_i)(1+\delta),
\end{equation*}
$|\delta|<u \ \text{non} \  u=2^{-m}$ beteko denik, ezin daiteke bermatu. 

\paragraph*{}Analisi zehatza egiten badugu $n=3$ adibiderako, honako espresioa lortzen dugu,
\begin{equation*}
((\tilde x_1 \oplus \tilde x_2) \oplus \tilde x_3)  = 
  \big((\tilde x_1 + \tilde x_2)(1+\delta_1)
  +\tilde x_3 \big) (1+\delta_2), \ \ \delta_1,\delta_2<u.
\end{equation*}

\subsection*{Ezabapen arazoa.}

Algoritmoen kalkuluetan, doitasun galera azkarra gerta daiteke. Horren adibidea ezabapen arazoa dugu: oso antzekoak diren bi zenbakiren arteko kendura egiten dugunean gerta daitekeena. 

\paragraph*{Adibidea.} Mathematican kalkulatutako adibide honetan, ezabapen errorea nola gertatzen den erakutsi dugu. 
\begin{lstlisting} [language=Mathematica]
>>  InputForm[N[Pi]]
>> 3.141592653589793

>> y=N[Pi]*10^(-10);
>> InputForm[y]
>> 3.1415926535897934*10^(-10)

>> z=1.+y;
>> InputForm[z]
>> 1.0000000003141594           # 16-digitu hamartar zuzenak.

>> InputForm[z-1.]
>> 3.141593651889707*10^(-10)   # 6-digitu hamartar zuzenak.

\end{lstlisting}


\section{Biribiltze errorea gutxitzeko teknikak.}
\label{sec:4.4}

Batuketa eta biderketa eragiketen biribiltze errorea kalkulatzeko algoritmoak ezagunak dira \cite{Dekker1971,Higham2002}. Algoritmo hauek, \emph{termino-gaitzeko adierazpenetan} oinarritzen dira eta baturaren kasuan, batura konpentsatu izeneko algoritmoaren oinarria da. Ikusiko dugun bezala, algoritmo sinpleak dira eta konputazio kostu txikia dute.  

Teknika hauek, zenbakizko integrazioaren inplementazioaren kalkulu "kritikoetan" erabiliko ditugu, soluzioaren doitasuna handitzeko asmoarekin.

\subsection*{Batura: Fast2Sum.}

\emph{Fast2Sum} algorithmoa, 1971.ean Dekker-ek  asmatu zuen \cite{Dekker1971}. Koma-higikorreko $\tilde x,\tilde y \in \mathbb{F} \ \text{non} \ |\tilde x| \geq |\tilde y| \ \text{bi zenbakien}$ arteko $\tilde z= \tilde x \oplus \tilde y$ batuketari dagokion $e$ biribiltze errorea  era honetan kalkulatu daiteke,
%\ \text{non} \ \tilde z+ e=\tilde x+\tilde{y}$ den

\begin{algorithm}[H]
 \BlankLine
 {$\tilde{z}=\tilde{x} \oplus\tilde{y}$\;
  $e=\tilde{y} \ominus (\tilde{z}\ominus\tilde{x})$\;
 }
 \BlankLine
 \caption{Fast2Sum.}
 \label{alg:FastSum}
\end{algorithm}

\ref{fig:fast2sum}irudiaren laguntzarekin hobeto uler daiteke batuketaren biribiltze errorearen kalkulua \cite{Higham2002}.

\begin{figure}[h!]
\centerline{\includegraphics[width=14cm, height=8cm] {Fast2Sum}}
\caption[Batuketaren biribiltze errorea]{Batuketaren biribiltze errorea}
\label{fig:fast2sum}
\end{figure} 

\subsubsection*{Batura konpensatua.}

Era honetako batugai askoren arteko batuketan,
\begin{equation*}
z_{n+1}= z_0+\sum\limits_{i=0}^{n} x_i,
\end{equation*}
biribiltze errorea gutxitzeko teknika ezaguna da \cite{Higham2002,Muller2009,Hairer2006}.
Ideia da, bi zenbakien baturan egindako biribiltze errorea lortu, eta errore hau hurrengo baturan erabiltzea. Jarraian azaltzen den moduan, urrats bakoitzaren amaieran $e_{i}$ errore estimazioa  kalkulatuko dugu eta hurrengo urratsean, batugaiari gehituko diogu.

\begin{algorithm}[H]
 \BlankLine
  $\tilde z_0= z_0; \ e_0=0$\;
  \For{$i\leftarrow 0$ \KwTo $n$}
  {
   \BlankLine
    $x=\tilde z_i$\;
    $y= x_i+e_i$\;
    $\tilde z_{i+1}=x+y$\;
    $e_{i+1}=(x-z)+y$\;
   \BlankLine
  }
 \caption{Kahan-en batura konpentsatua.}
   \label{alg:KahanBK}
\end{algorithm}

Knuth-ek eta Kahan-ek \cite{Muller2009} frogatu zuten,  batura konpentsatuko algoritmoaren bidez kalkulatutako $z_{n+1}$ baturak honakoa betetzen duela:
\begin{equation*}
\left | z_{n+1} - (z_0+\sum_{i=0}^{n} x_i) \right | \leq (2u+ \mathcal{O}(nu^2)) \left(|z_0|+\sum_{i=1}^{n} |x_0|\right).
\end{equation*}

Jakina da, batugaiak bektoreak diren kasurako, hau da, $\tilde z_0, e_0, x_0, x_1, \dots, x_n \in \mathbb{F}^d$, ~algoritmoa orokor daitekeela. Beraz, \ref{alg:KahanBK} algoritmoa $n$ eta $d$ parametroak dituen funtzio familia gisa interpreta daiteke,
\begin{equation}
\label{eq:batsd}
S_{n,d} : \mathbb{F}^{(n+3)d} \rightarrow \mathbb{F}^{2d},
\end{equation}
zeinek $\tilde z_0, e_0, x_0, x_1, \dots, x_n \in \mathbb{F}^d$ argumentuak emanik, $\tilde z_{n+1}, e_{n+1} \in \mathbb{F}^d$ balioak itzultzen dituen, eta ($\tilde z_{n+1}+e_{n+1}) \approx \tilde (z_0+e_0+x_0+x_1+ \dots+x_n$) hurbilketa den.

\subsubsection*{Zenbakizko integrazioak.}
 
Zenbakizko integrazioetan, $n=1,2,\dots$ balioentzat era honetako baturak kalkulatu behar ditugu \cite{Hairer2006},
\begin{equation*}
y_{n+1}=y_n+\delta_n,
\end{equation*}  
non $|\delta_n|<|y_n|$ izan ohi den. Beraz, integrazioaren batura honen birbiltze errorea gutxitzeko, batura konpentsatua erabiliko dugu.  

$y_{n+1} \in \mathbb{R}^{d},\quad y_{n+1}=\tilde y_{n}+\tilde \delta_n$ batura zehatza izanik eta $\tilde y_{n+1} \in \mathbb{F}^{d}, \quad \tilde y_{n+1}=\tilde y_{n} \oplus \tilde \delta_n$ koma-higikorreko hurbilpena izanik, batura konpentsatuaren bidez lortutako errorearen estimazioa $e_{n+1}$, \ref{alg:batkp}~algoritmoa jarraituz lor daiteke eta baturan egindako biribiltze errore zehatza da, 
\begin{equation}
y_{n+1}=\tilde {y}_{n+1}+e_{n+1}. 
\end{equation}

\begin{algorithm}[H]
 \BlankLine
  $\tilde{y}_{0}=fl(y_{0}); \ e_0=fl(y_0-\tilde{y}_0)$\;
 \BlankLine
  \For{$n=0,1,2,\dots \quad$}
  {
   \BlankLine
    $inc=\tilde {\delta}_n \oplus e_n$\;
    $\tilde {y}_{n+1}=\tilde{y}_n \oplus inc$\;
    $e_{n+1}=(\tilde{y}_n \ominus \tilde {y}_{n+1}) \oplus inc$\;
   \BlankLine
  }
 \caption{Batura konpentsatua (zenbakizko integrazioa).}
 \label{alg:batkp}
\end{algorithm}


Goian aipatutako ideia,  beste ikuspegi batetik ere uler daiteke. Zenbakizko soluzioa, doitasun bikoitzeko bi balioen batura gisa $y_n=\tilde{y}_n+e_n$ (ia doitasun laukoitza), adierazten ari gara  eta beraz, interpretazio honen arabera, konputazio eragiketa batzuk ia doitasun laukoitzean egiten ariko ginateke. Zentzu honetan gure inplementazioan, hasierako balio zehatza $y_0=y(t_0)$, bi balioen batura gisa $y_0=\tilde{y}_0+e_0$ ulertu behar da eta era honetan hasieratuko dugu,
\begin{align*}
\tilde{y}_0 &=fl(y_0) ,\\
e_0 &=fl(y_0-\tilde{y}_0).
\end{align*}

\subsection*{Bidekerta: 2MultFMA.}

\emph{IEEE 754-2008} estandarrean, \emph{FMA} \cite{Muller2009} (\emph{fused multiply-add}) instrukzioa gehitu zen eta hurrengo urteetan, ordenagailu arruntetan zabaltzea espero da. Instrukzio honen garrantzia handia da: orokorrean konputazioak azkartzen ditu eta biderketa eskalarren, matrize biderkaduren eta polinomio ebaluazioen biribiltze errorea txikitzen du. \emph{FMA} instrukzioa, zatiketa eta erro karratuaren algoritmo azkarren diseinuan ere erabiltzen da.

\emph{FMA} instrukzioak, era honetako konputazioetan biribiltze errore bakarra bermatzen du,
\begin{equation*}
fl(\tilde x \times \tilde y \pm \tilde z)= (\tilde x \times \tilde y\pm \tilde z) (1+\delta), \ \delta<u \ \ \text{non} \ \ u=2^{-m}.
\end{equation*}
 

\emph{FMA}  instrukzioa erabilgarri dagoenean, biderketaren biribiltze errorea kalkulatzea erraza da; $\tilde x,\tilde y \in \mathbb{F}$ bi zenbakien arteko biderketari $\tilde z= fl(\tilde x \times \tilde y)$ dagokion biribiltze errorea $e, \ \text{non} \  \tilde{z}+ e=\tilde x \times \tilde y$ den, era honetan kalkulatu daiteke,

\begin{algorithm}[H]
 \BlankLine
 {$\tilde{z}=fl(\tilde{x}\times\tilde{y})$\;
  $e=fl(\tilde{x}\times\tilde{y}- \tilde{z})$\;
 }
 \BlankLine
 \caption{2MultFMA.}
 \label{alg:2MultFMA}
\end{algorithm}

\subsection*{Sterbenz Teorema.}
Sterbenz teoremaren arabera \cite{Sterbenz1973}, bi zenbaki elkarrekiko  gertu daudenean, honako baldintza betetzen bada, horien arteko kendura zehatza da.
\begin{equation}
\label{eq:4311}
x,y \in \mathbb{F}, \ \ \frac{y}{2}\leq x \leq 2y \ \ \ \Rightarrow \ \ \ x-y\in \mathbb{F}.
\end{equation}


\section{Laburpena.}

Atal honetan, koma-higikorreko aritmetikaren deskribapena egin ondoren, konputazioen doitasuna handitzeko tresnak azaldu ditugu. Tresna hauek, konputazio kostu txikia dute eta zenbakizko integrazioetan, biribiltze errorea txikitzeko aplikatuko ditugu.  

Koma-higikorreko aritmetikan sakontzeko honako bibliografia azpimarratuko dugu: \cite{Overton2001,Muller2009,Higham2002,Corless2013}.


%\chapter{Zientzia konputazioa.}

\epigraph{Processor speed doubles every 18 months.}{\textit {Moore's Law (1965)}}
\epigraph{Number of cores per chip can double every two years.}{\textit Moore's Law Reinterpreter (2006)}

\section{Sarrera.}

Gaur-egungo konputagailuak (super-konputagailu, eramangarri,...) orokorrean paraleloak dira. 1986-2002 urteen artean, txip barruan transistore dentsitatea handitzen zen heinean, prozesadore bakarreko konputagailuen eraginkortasuna hobetuz joan zen. Baina teknologi-garapena muga fisikoetara iritsita, bide honetatik konputagailuen abiadura hobetzea ezinezkoa bilakatu zen (Irudia \ref{fig:51}). Horrela, 2005.urtetik aurrera fabrikatzaileek konputagailuen gaitasuna hobetzeko, txipan prozesadore bat baino gehiago erabiltzea erabaki zuten.      

\begin{figure}[h]
\centerline{\includegraphics[width=10cm, height=6cm] {ProcessorClock}}
\caption[Processor clock rate.]{\small Processor clock rate growth halted around 2005.}
\label{fig:51}
\end{figure} 

%\begin{figure}[h]
%\centerline{\includegraphics[width=12cm, height=8cm] {PerformanceDevelopment}}
%\caption[Konputagailuen eraginkortasuna.]{www.top500.org, Top: total computing power of top 500 computers. Middle: 1 %computer. Bottom: 500 computer.}
%\label{fig:61}
%\end{figure} 

Konputagailuen eredu aldaketa honen ondorioz, algoritmo azkarrak garatzeko kodearen paralelizazio gaitasunari heldu behar zaio. Programazio paralelo teknikak inplementatzeko, beharrezko da prozesadore berrien hardware arkitekturak nahiz software ingurune berriak ulertzea. Gaia nahiko konplexua izanik, ikuspegi orokorra eman ondoren, gure inplementazioan erabilitako hardware arkitektura eta software teknika zehatzak azalduko ditugu: memoria-konpartitutako sistemak eta OpenMP programazio eredua.

Bi dira, algoritmo azkarrak diseinatzeko erronkak: 
\begin{enumerate}
\item Paralizatzeko pisuko lana identifikatzea.
\item Memoria eta prozesadorearen arteko datu mugimendua gutxitzea. 
\end{enumerate}

Bestalde, inplementazio berrien garapenean optimizatutako liburutegiak erabiltzea komeni da. Horien artean, LAPACK eta BLAS algebra linealeko liburutegiak erabilgarriak izan zaizkigu. Liburutegi hauen gaineko azalpenak emango ditugu.

\section{Eraginkortasuna.}

\subsection*{\textbf{Zein azkarrak dira konputagailuak?}}

Gaur egungo prozesadoreen maiztasun-abiadura hertzetan neurtzen da, hau da,  \emph{makina ziklo segundoko} kopuruaren arabera. Une honetako prozesadoreak gigahertz mailakoak dira.
\begin{description}
\item {Kilo} = mila ($10^3$).
\item {Mega} = milioi ($10^6$).
\item {Giga} = bilioi ($10^9$).
\item {Tera} = trilioi ($10^{12}$).
\item {Peta} = $10^{15}$.
\item {Exa} = $10^{18}$. 
\end{description}

Koma-higikorreko oinarrizko eragiketa bat egiteko ($\oplus,\ominus,\otimes,\oslash$) ziklo gutxi batzuk behar dira. Honek esan nahi du, $1$ GHz-ko prozesadore batek,
$>100.000.000$ koma-higikorreko eragiketa segundoko egiten dituela ($>100$ megaflops).

\paragraph*{\textbf{Adibidea}.} 
Demagun $A,B$ eta $C \ (n \times n)$ dimentsioko matrizeak ditugula eta $C=AB$ matrize arteko biderketa egiteko behar dugun denbora jakin nahi dugula.
\begin{equation*}
c_{ij}=\sum\limits_{i,j=1}^{n} a_{ij}*b_{ji}.
\end{equation*}

\begin{itemize}
\item $c_{ij}$ gai bakoitza kalkulatzeko $n$ biderketa eta ($n-1$) batura egin behar ditugu.
\item $C$ matrizeak $n^2$ osagaia ditu $\Rightarrow$ $O(n^3)$ koma-higikorrezko ariketak exekutatu behar dira.
\end{itemize}

Adibidez, $n=100$ bada $\ \Rightarrow \ n^3=10^{6}$ eragiketa egin behar ditugu. $1$GHz prozesadorean exekutatzeko, $>10^(-2)$ segundo beharko genituzke. 

\paragraph*{} Zientzia konputazioaren eraginkortasuna neurtzeko, koma-higikorreko eragiketa kopurua (\emph{flops}) erabili ohi zen. Problema handia denean, datuen mugimendua koma-higikorreko eragiketak baino garestiagoa da, eta beraz eraginkortasuna eragiketa kopuruaren arabera neurtzea okerra izan daiteke. Kodearen exekuzioa azkartzeko derrigorrezkoa da konputagailuan datuen mugimendua minimizatzea.

\paragraph*{\textbf{Adibidea},} $n=400$ tamainako matrizeak hartzen baditugu, $15,6$ \emph{MB} memoria behar dugu (suposatuz konputagailuaren CACHE memoria baino handiagoa) eta datuen mugimenduaren eragina nabarituko da exekuzio denboran.

\subsubsection*{Timing code.}

Unix \emph{time} agindua erabili daiteke, konputazioen denborak ezagutzeko:

\begin{lstlisting} 
S time ./a.out
<kodearen irteera>

real 0m38.856s
user 0m38.789s
sys  0m0.004s
\end{lstlisting}

Agindu honekin, \emph{./a.out} C programa exekutatuko da eta ondoren, programa exekutatzeko behar izan duen denboraren informazioa pantailaratuko du: \emph{real} hasi eta bukatu arteko denbora (\emph{wall-time}); \emph{user} prozesadoreak gure programa exekutatzen erabili duen denbora (\emph{CPU-time}); \emph{sys} programa exekutatu ahal izateko, sistema eragile lanetan emandako denbora.   

\paragraph*{} C lengoaian badaude, denbora neurtzeko funtzioak. Jarraian, exekuzio denbora (\emph{Wall-time}) eta cpu denborak \emph{CPU-time} nola kalkulatu azaldu dugu.

\begin{lstlisting}[language=C]

#include <time.h>

    time_t  wtime0,wtime1;
    clock_t clock0, clock1; 

    wtime0= time(NULL);
    clock0= clock();

    <neurtu nahi den kodea>

    wtime1= time(NULL);
    clock1=clock();
    
    Wall_time=(wtime1 - wtime0);
    CPU_Time=(clock1 - clock0)/CLOCKS_PER_SEC);

\end{lstlisting}

\emph{Wall-time} deiturikoa izango da algoritmo baten denborak neurtzeko gure irizpidea. Programazio paraleloan, algoritmoen exekuzio denborak egokien neurtzen duen aldagaia da. Dena den, une berean programa bakarra exekutatzea behartuta gaude.       
 

\section{Hardwarea.}

\subsection{Memori hierarkia.}

Lehenik, konputagailuan dauden memoria mota ezberdinen hierarkia azalduko dugu. 
\begin{figure}[h]
\centerline{\includegraphics[width=10cm, height=4cm] {MemoryHierarchy}}
\caption{Memoria hierarkia.}
\label{fig:three}
\end{figure} 

\paragraph*{} CPU-k koma-higikorrezko eragiketak egiten ditu: erregistroetatik datuak irakurri, eragiketak egin eta emaitza erregistroetan idazten ditu. Memoria nagusia eta erregistroen artean, 2 edo 3 mailako Cache memoria dugu: lehen Cache memoria (L1) txikiena eta azkarrena da, eta beste mailak (L2,L3,...), handiagoak eta motelagoak. Memoria nagusian, exekutatzen diren programak eta datuak gordetzen dira ($1-4$ GB artekoa). Azkenik, disko gogorrean konputagailuko datu (argazki, bideo,...) eta erabilgarri ditugun programa guztiak gordetzen dira.  

Cache memoria lerroka egituratuta dago eta lerro bakoitza $64$ edo $128$ bytez ($8$ edo $16$ double zenbaki) osatuta dago. Programa batek datu bat behar duenean, memoria nagusitik lerro tamainako datu taldea irakurriko du eta Cachean idatziko ditu. Komunikazio hau minimizatzeko memorian datuak gordetzeko ordenak badu garrantzia. Beraz, datu-egiturak diseinatzen direnean, kontutan hartu behar da une berean beharko diren datuak memorian gertu gordetzea   

\paragraph*{\textbf{Adibidea}.}Badakigunez, C-lengoaian matrizeak lerroka gordetzen dira. Beheko adibidean,  matrizearen lehen osagaia $a(1,1)$ behar dugunean, memoria nagusitik Cachera osagai honetaz gain jarraiko 16 osagaiak ekarriko dira ($a(1,1),a(1,2),\dots,a(1,16)$). Honela, hurrengo $15$ batura egiteko behar ditugun datuak Cachean eskuara izango ditugu memoria irakurketa berririk egin gabe. 

\begin{algorithm}[h]
 \BlankLine
  $int \ n$\;
  $double \ a[n][m]$\;
  \BlankLine
  $sum=0$\;
  \For{$i\leftarrow 1$ \KwTo $n$}
  {
   \BlankLine
    \For{$j\leftarrow 1$ \KwTo $m$}
   {
    \BlankLine 
    $sum+=a(i,j)$\;
   }
 }
 \caption{Memoria atzipena.}
\end{algorithm} 

\begin{equation*}
a=\left(\begin{array}{ccccc}
  1    & 2    & 3    & \dots & 1000 \\
  1001 & 1002 & 1003 &\dots & 2000 \\
  2001 & 2002 & 2003 &\dots & 2000 \\
  \dots & \dots & \dots & \dots & \dots \\
  9001 & 9002 & 9003 &\dots & 10000 \\
  \end{array}\right).  
\end{equation*}

\paragraph*{}CPUk datu bat behar duenean, memoria hierarkian zehar bilatuko du: lehenik $L1$ cachean, ondoren $L2$ cachean,...eta hauetan ez badago, memoria nagusira joko du. Memoria nagusi eta cache memoria arteko irakurketa eta idazketa guzti hauetan,  informazio konsistentzia mantentzeko hainbat arau aurrera ematen dira.  

\subsection{Hardware motak.}

MIMD (Multiple instruction, multiple data) sistemak, guztiz independenteak diren prozesadore multzoak osatzen dituzte. Bi dira MIMD sistema nagusiak: memoria konpartitutako eta memoria banatutako sistemak. Memoria konpartitutako sistemetan, prozesadore guztiek memoria osoa konpartitzen dute eta inplizituki konpartitutako datuen atzipenaren bidez komunikatzen dira. Memoria banatutakotako sistemetan aldiz, prozesadore bakoitzak bere memoria pribatua du eta esplizituki bidalitako mezuen bidez komunikatzen dira.

Hirugarren hardware arkitektura ere aipatuko dugu, general purpose GPU computing (Graphical Processor Unit).
Jokoen eta animazio industrian, grafiko oso azkarrak beharrak bultzatuta  sortutako teknologia da. Oinarrian, imajinak pantailaratzeko prozesagailu asko paraleloan lan egiten dute. Azken hamarkadan, GPU unitate hauek zientzia konputaziora zabaldu dira.  

\begin{figure}[h]
\centerline{\includegraphics[width=12cm, height=4cm] {SharedMemorySystem}}
\caption{Memoria konpartitutako sistemak.}
\label{fig:61}
\end{figure}  

\begin{figure}[h]
\centerline{\includegraphics[width=12cm, height=4cm] {DistribuitedMemorySystem}}
\caption{Memoria banatutako sistemak.}
\label{fig:61}
\end{figure}  

\paragraph*{\textbf{Memoria konpartitutako sistemak}.}

Multicore bat edo gehiagoz osatutako sistema dugu. Multicore prozesadore bakoitzak txipean CPU bat baino gehiago ditu. Normalean CPU bakoitzak $L1$ bere cache memoria du. Aipatzeko da, era honetako sistemetan prozesadore kopurua ezin dela nahi adina handitu eta mugatua dela (normalean $\leq 32$ ).

 \begin{figure}[h]
 \centerline{\includegraphics[width=12cm, height=4cm] {SharedMemorySystemUMA}}
 \caption{Memoria konpartitutako sistemak (UMA).}
 \label{fig:61}
 \end{figure}  

\section{Softwarea.}


\subsection{Software liburutegiak.}

Matematika bi software errekurtso nagusienak aipatuko ditugu; BLAS (Basic Linear Algebra Subroutines) eta LAPACK (Linear Algebra Package). Kalitate handiko software orokorrak dira eta hauek erabiltzea abantaila asko ditu: 
\begin{enumerate}
\item Garapen berriak egiteko denbora aurrezten da. 
\item Problema askotan ondo probatutako softwareak dira.
\item Konplexutasun handikoak dira, modu seguruan eta azkarrean exekutatzeko diseinatu direlako. 
\end{enumerate}

Konputagailu hardware bakoitzerako optimizatutako bertsioak daude. Inplementazioa Fortranen egina dago eta datu-motei dagokionez:
\begin{enumerate}
\item S: float ($32$ bit).
\item D: double ($64$ bit).
\item C: complex.
\item Z: complex double.
\end{enumerate}   

\subsubsection*{\textbf{BLAS}.}

BLAS liburutegian, bektore eta matrizeen arteko funtzio estandarrak inplementatuta daude. Hiru mailetan banatuta dago: 

\begin{enumerate}
\item BLAS-1: bektore-bektore eragiketak.

 Adibidez: $y=\alpha*x+y$ , $2n$ flop eta $3n$ irakurketa/idazketa.
 
 Konputazio intentsitatea: $\frac{2n}{3n}=\frac{2}{3}$. 

\item BLAS-2: matrize-bektore eragiketak.

 Adibidez: $y=\alpha*A*x+\beta*x$, $O(n^2)$ flop eta $O(n^2)$ irakurketa/idazketa.
 
 Konputazio intentsitatea: $\approx \frac{2n^2}{n^2}=2$. 
 
\item BLAS-3: matrize-matrize eragiketak.

 Adibidez: $C=\alpha*A*B+\beta*C$, $O(n^3)$ flop eta $O(n^2)$ irakurketa/idazketa.
 
 Konputazio intentsitatea: $\approx \frac{2n^3}{4n^2}=\frac{n}{2}$. 

\end{enumerate}

Azpimarratu, BLAS-1 eta BLAS-2 funtzioen konputazio intentsitatea txikia dela eta beraz, datuen komunikazioa nagusia dela. BLAS-3 aldiz, konputazio intentsitatea handiagoa da eta ezaugarri honi esker, konputagailuaren konputazio gaitasuna ondo aprobetxatu ahal izango da.

\begin{figure}[h]
\centerline{\includegraphics[width=12cm, height=8cm] {BLASSpeed}}
\caption{BLAS speeds.}
\label{fig:61}
\end{figure}    

Fabrikatzaile bakoitzak optimizatutako BLAS liburutegiak (AMD ACML,Intel MKL) dituzte eta beraz, multi-threaded dira.
Beste aukera bat, optimizatutako BLAS instalazioa ATLAS (Automatically Tuned Linear Algebra Software) bidez egitea.    

\subsubsection*{\textbf{LAPACK}}.

Zenbakizko aljebra linealaren liburutegia da.

\begin{enumerate}
\item Sistema linealak: $AX=b$.
\item Least Square: choose $x$ to minimize $\|Ax-b\|$.
\item Eigenvalues.
\item Balio singularren deskonposaketa (SVD).
\end{enumerate}

Posible den guztietan, BLAS-3 funtzioetan oinarritzen da.

\subsection{Programazio paraleloa.}

C-lengoaia programazio paraleloan erabiltzeko, lengoaiaren bi extensio dira nagusienak: bata memori-banatutako sistemetarako diseinatuta  MPI (Message-Passing Inteface) eta bestea, memoria-konpartitutako sistemetarako diseinutakoa OpenMP (Open Specifications for MultiProcessing). MPI datu moten definizio, funtzio eta makroen liburutegia da. OpenMP liburutegia bat  eta C konpiladorearen aldaketa batzuk. OpenMP erabili dugu gure inplementaziorako eta jarraian honi buruzko idei nagusienak emango ditugu.

\paragraph*{\textbf{OpenMP}.}

Memoria konpartitutako programazio paraleloaren estandarra dugu. 
Programazioan paralelizazio kontrola, "fork-join" modeloa jarraituz egiten da.

\begin{enumerate}
\item OpenMP programen hasieran prozesu bakarra dago, hari (thread) nagusia. 
\item FORK: hari nagusiak hari talde paraleloa sortzen du.
\item JOIN: hariak kode paraleloa bukatzen dutenean, behin sinkronizatuta amaitzen dute eta hari nagusiak bakarrik jarraitzen du.
\end{enumerate}

% \begin{figure}[h]
% \centerline{\includegraphics[width=10cm, height=3cm] {ForkJoin}}
% \caption{Fork-Join.}
% \label{fig:61}
% \end{figure}  
 
 \begin{figure}[h]
 \centering
 \subfloat[Fork-Join.]{
 \includegraphics[width=.500\textwidth]{ForkJoin}
 }
 \subfloat[Fork-Join.]{
 \includegraphics[width=.200\textwidth]{OpenMP1}
 }
  \caption[OpenMp programazio modeloa.]{\small OpenMp programazio modeloa.}
 \label{fig:forkjoin}
 \end{figure}

Aldagai batean (threadcount) paralelizazioan zenbat hari erabili adierazten da eta ohikoa izaten da hari bat prozesadore bakoitzeko sortzea.  Konpilazio direktiben bidez,  paralelizazioa nola exekutatu behar den zehazten zaio.

\paragraph*{\textbf{Adibidea}}.

\begin{lstlisting}[language=C]
#    pragma omp parallel for num_threads(thread_count) 
     for (i = 0; i<n; i++)
     {
       ! Aginduak 
     }
\end{lstlisting}

OpenMP Version 4.5.	

gcc -v (gcc version 4.8.4 (Ubuntu 4.8.4-2ubuntu1~14.04.3))

A number of compilers from various vendors or open source communities implement the OpenMP API:

\begin{enumerate}
\item From GCC 4.7.0, OpenMP 3.1 is fully supported. 

\item From GCC 6.1, OpenMP 4.5 is fully supported in C and C++.

\end{enumerate}   


\subsection{Konpiladorea.}

\subsubsection*{Sarrera.}

Erabiliko dugun konpiladorea,

\begin{enumerate}

\item \emph{gcc} - \emph{GNU} open source compiler.

\begin{lstlisting}
$ gcc -v
$ gcc version 4.8.4 (Ubuntu 4.8.4-2ubuntu1~14.04.3) 
\end{lstlisting}

\item Several comercial compilers also are avalaible.

\end{enumerate}


C11 (formerly C1X) is an informal name for ISO/IEC 9899:2011,[1] the current standard for the C programming language.It replaces the previous C standard, informally known as C99. This new version mainly standardizes features that have already been supported by common contemporary compilers, and includes a detailed memory model to better support multiple threads of execution.

gcc requires you specify -std=c99 or -std=c11


\paragraph*{Optimizations} (-Olevel).
        
Optimizes the code for execution speed according to the level specified by level , which can be 1, 2, or 3. If no level is specified, as in –O , then 1 is the default. Larger numbers  indicate higher levels of optimization.

-O2 (default) Optimize for code speed. This is the generally recommended optimization level. -O3 Enable -O2 optimizations and in addition, enable more aggressive optimizations such as loop and memory access transformation, and prefetching. 

Every compiler offers a collection of standard optimization options (-O0,-O1,. . . ).  However, all compilers refrain from most optimizations at level -O0, which is hence the correct choice for analyzing the code with a debugger. At higher levels, optimizing compilers mix up source lines, detect and eliminate “redundant” variables, rearrange arithmetic expressions, etc.,      

\subsubsection*{Konpilazioa.}

\paragraph*{Oinarrizko erabilpena.}

\begin{enumerate}
\item Compiles and links and creates an executable adibidea.exe.
\begin{lstlisting}[language=C]
$ gcc adibidea.c -o adibidea.exe
\end{lstlisting}

\begin{lstlisting}[language=C]
$ ./adibidea.exe
\end{lstlisting}

\item Compile and link steps.

\begin{lstlisting}[language=C]
$ gcc adibidea.c  # creates adibidea.o
$ gcc adibidea.o -o adibidea.exe
\end{lstlisting}

\end{enumerate}

\begin{lstlisting}[language=C]
gcc -O2 -Wall -std=c99 -fno-common adibidea.c
\end{lstlisting}

\subsubsection*{Makefile.}

A common way of automating software builds and other complex tasks with dependencies.

A Makefile is itself a program in a special language.

\paragraph*{Adibidea.}
Demangun programa bat hiru fitxategieten banatuta dugula,

\begin{lstlisting}[language=C]
/*file: main.c*/
void main()
{
    printf("Main program");
    sub1();
    sub2();
}
\end{lstlisting}

\begin{lstlisting}[language=C]
/*file: sub1.c*/
void sub1()
{
    printf("sub1");
}
\end{lstlisting}

\begin{lstlisting}[language=C]
/*file: sub2.c*/
void sub2()
{
    printf("sub2");
}
\end{lstlisting}

Programa exekutagarria lortzeko makefile fitxategia,

\begin{lstlisting} [language=C]
main.exe: main.o sub1.o sub2.o
	      gcc main.o sub1.o sub2.o -o main.exe
main.o: main.c
        gcc -c main.c
sub1.o: sub1.c
        gcc -c sub1.c        
sub2.o: sub2.c
        gcc -c sub2.c        
\end{lstlisting}

\begin{lstlisting}
$ make main.exe
gcc -c main.c
gcc -c sub1.c
gcc -c sub2.c
gcc main.o sub1.o sub2.o -o main.exe
\end{lstlisting}

Typical element in the simple Makefile:

\begin{lstlisting}
target: dependencies
>TAB>  command(s) to make the target
\end{lstlisting}

Typing "make target" means:
\begin{itemize}
\item Make sure all dependencies are update (those that are also targets).
\item If target older than any dependency, recreate it using specified commands.
\item The rules are applied recursively.
\end{itemize}

\paragraph*{\textbf{MakefileV2}}

\begin{lstlisting} [language=C]

CC = /usr/bin/gcc
FLAGS=-O2 -Wall -std=c99 -fno-common 
OBJECTS=  main.o sub1.o sub2.o
.PHONY: clean help

main.exe: $(OBJECTS)
	      ${CC} $(OBJECTS) -o main.exe
	      
%.o: %.c
     ${CC} ${FLAGS} -c $<	      
	      
clean:
     rm -f $(OBJECTS) main.exe

help:
    @echo "Valid targets;"
    @echo " main.exe"
    @echo " main.o"
    @echo " sub1.o"
    @echo " sub2.o"
    @echo " clean"
             
\end{lstlisting}


\section{Kode Optimizazioak.}

Applications have two general challenges:
\begin{enumerate}
\item Numerical Method.

Performance required computation the shortest account of time.

\item. Computer Program.

Express the algorithm as fast computer problem, you have realized computer hardware eficiently.

\end{enumerate}

Optimization areas are:
\begin{enumerate}
\item Vectorization.
\item Paralelization.
\item Memory trafic control.
\end{enumerate}

Optimization areas:
\begin{enumerate}
\item scalar optimization (compiler friendly practices).

\item vectorization (must use 16 or 8 wide vectors).

\item multi-threading (must  scale to $100+$ threads).

\item memory access (streaming acces).

\item communication (offload, MPI traffic control).

\end{enumerate}


\subsection*{Scalar Tuning and General Optimization.}

Optimization of scalar arithmetics.

One of the most important scalar optimizazion techincs is  \textbf{strength reduction} : replace expensive operation for less expensive operations.

\begin{figure}[h]
 \centerline{\includegraphics[width=12cm, height=4cm] {Optimization_Strength_Reduction}}
 \caption{Optimization.}
 \label{fig:61}
\end{figure}  

\paragraph*{} Precision control.

\begin{enumerate}
\item Precision Control for transcendental functions.

\item Floating-point semantics.

\item Consistency of precision: constants and constants.

Using incorrect function names in the single precsion is a common mistake

\end{enumerate}

\subsection*{Optimization of vectorization.}

\begin{enumerate}

\item Preferably unid-stride access to data.
Very important step in the optimization.

Artikulua "Auto-Vectorization with the Intel Compilers".

Most CPU architectures today include Single Instruction Multiple Data (SIMD) parallelism in the form
of a vector instruction set. Serial codes (i.e., running with a single thread), as well as instruction-parallel cal-
culations (running with several threads) can take advantage of SIMD instructions and significantly increase
the performance of some computations. Each CPU core performs SIMD operations on several numbers (in-
tegers, single or double precision floating-point numbers) simultaneously, when these variables are loaded
into the processor’s vector registers, and a vector instruction is applied to them. SIMD instructions include
common arithmetic operations (addition, subtraction, multiplication and division), as well as comparisons,
reduction and bit-masked operations (see, e.g., the list of SSE 2 intrinsics). Libraries such as the Intel Math
Library provide SIMD implementations of common transcendental functions, and other libraries provide
vectorized higher-level operations for linear algebra, signal analysis, statistics, etc.

\item Data Alignment and Padding.

An important consideration for efficient vectorization is data alignment.


\end{enumerate}

\subsection*{Multi-threading.}

Do you have enough parallelism in your code? 

Expanding iteration space, if it is no enough iterations in parallel loop.

Three layers of parallelism: MPI processes, OpenMP threads, vectorization.


\subsection*{Memory access.}

Memory access and Cache utilization.

loop tiling technics

\begin{figure}[h]
 \centerline{\includegraphics[width=12cm, height=4cm] {Optimization_LoopTiling}}
 \caption{Optimization.}
 \label{fig:61}
\end{figure}  

\section{Laburpena.}

Best practices for code vectorization and parallelization, and additional tips and tricks.

Algoritmo bat inplementatzen dugunean kontutan hartu beharrekoa:

\begin{enumerate}

\item Lerro edo zutabe araberako iterazioak exekuzio denboran eragin handia du.

\item Kodea garbia eta ulergarria mantendu behar da.

\item Badaude kodearen exekuzio denboraren analisia egiteko tresnak (adibidez gprof). Algoritmoaren funtzio bakoitzaren exekuzio denborari buruzko informazio erabilgarria lortuko dugu. Zenbait gauza modu sinplean azkartu daitezke baina zenbait beste gauza azkartzeko esfuertzu handia eskatu dezake.

\item Optimizatutako beste hainbat kode erabiltzea komenigarria da. LAPACK aljebra lineal paketea Fortran eta C-lengoaitetatik deitu daiteke. Eraberean, LAPACKek BLAS subrutinak erabiltzen ditu.  Subrutinak hauek matrizen arteko biderketak, "inner product", ... BLAS konputagailu arkitektura ezberdinetarako optimizatutako bertsioak daude.
 

\end{enumerate}


%\include{8-Review/Review}

\part{Ekarpenak.}
\chapter{IRK: Puntu-Finkoa.}
\label{chap:IRK-PF}

\section{Sarrera.}


Puntu-finkoan oinarritutako sistema Hamiltondar ez-zurrunen doitasun altuko zenbakizko integraziotarako, IRK metodoaren inplementazioa proposatuko dugu. Konputazioetan koma-higikorreko aritmetika erabiltzen denez, doitasun altuko integrazioetan,  biribiltze errore nagusitzen da. Hortaz, epe luzeko doitasun altuko zenbakizko integrazioetarako inplementazioetan, biribiltze errorearen eragina txikia izatea eta biribiltze errorearen estimazioa ezagutzea interesgarria da. 

Integrazioaren exekuzio denborak onargarriak izan daitezen, honako suposizioa egingo dugu: ekuazio diferentzialaren eskuin aldeko funtzioaren sarrera eta irteera argumentuak makina zenbakiak (koma-higikorreko aritmetika hardware bidezko exekuzioa azkarra duen datu-mota) direla. Gaur-egun zientzia-konputazioa,  $64$-biteko koma-higikorreko aritmetikan (\emph{double} datu-mota) oinarritzen da eta beraz, erabiltzaileak ekuazio diferentziala datu-mota honetan zehaztuko duela suposatuko dugu. 
 
Lehenengo, Hairer-en IRK metodoaren inplementazioa  \cite{Hairer2008} aztertuko dugu eta inplementazio horretan aurkitu ditugun hainbat arazo azalduko ditugu. Ondoren, IRK inplementazioa hobetzeko gure proposamenak egingo ditugu eta azkenik, zenbakizko integrazioen bidez bere abantailak erakutsiko ditugu.

\section{Hairer-en inplementazioa.}

\subsubsection*{IRK inplementazio estandarra.}

Gure abiapuntua, Hairer-ek \cite{Hairer2008} proposatutako IRK metodoaren inplementazio da.  
Lan honetan, puntu-finkoan oinarritutako IRK metodo sinplektikoaren  inplementazio estandarrean, biribiltze erroroeak energian errore sistematiko bat eragiten zuela ohartu ziren. Metodo sinplektiko esplizituetan, ordea, ez da horrelakorik gertatzen. Bere azterketaren ondorioaren arabera, errore sistematiko honen jatorriak bi direla aipatzen da:

\begin{enumerate}
 \item Praktikan aplikatutako IRK metodoa ez da sinpletikoa, integrazioan $a_{ij}, b_i \in \mathbb{R}$ koefiziente zehatzak erabili ordez, biribildutako $\tilde a_{ij},\tilde b_i \in \mathbb{F}$ erabiltzen direlako. 
\item Puntu-finkoaren geratze irizpide estandarrak, 
\begin{equation}
\Delta ^{[k]} = \max_{i=1,\dots,s}\|Y_i^{[k]}-Y_i^{[k-1]}\|_{\infty} \le \delta
\end{equation}
(finkatutako $\delta$ tolerantzia baten araberakoa), urrats bakoitzean errore sistematikoa eragiten du.

\end{enumerate}    

\subsubsection*{Konponbideak.}

Biribiltze errorearen eragina gutxitzeko helburuarekin, inplementazio estandarrean honako aldaketak proposatu zituzten:
\begin{enumerate}
\item Doitasun handiagoko koefizienteak erabiltzea, hauetako bakoitza bi koma-higikorreko koefizienteen batura kontsideratuz,
\begin{equation}
\label{eq:hkoef}
a_{ij}= a^{\ast}_{ij}+\tilde a_{ij}, \ b_i= b^{\ast}_i+\tilde b_i
\end{equation} 
non $a^{\ast}_{ij}>\tilde a_{ij}$ eta  $b^{\ast}_i>\tilde b_i$ diren. 

\paragraph*{}Adibidez, koefizienteak era honetan zehaztu daitezke,
\begin{equation*}
a^{\ast}_{ij}=(a_{ij} \otimes 2^{10}) \oslash 2^{10},\ \ \tilde a_{ij}= a_{ij}\ominus a^{\ast}_{ij}.
\end{equation*}

\item Puntu-finkoaren iterazioen geratze irizpide berria; iterazioak geratu, definitutako norma txikitzeari uzten dionean edo konbergentzia lortu duenean,
\begin{equation}
\label{eq:hstop}
\Delta^{[k]} = 0 \ \ \text{edo} \  \Delta^{[k]} \geqslant \Delta^{[k-1]}.
\end{equation}
  	 	
\end{enumerate}

\subsubsection*{Hairer-en algoritmoa.}

Hairer-ek bere \emph{Fortran} inplementazioa eskuragarri du (\href{http://www.unige.ch/~hairer/preprints.html}{Fortran kodea}). Jarraian, bere inplementazioaren algoritmoa (Algoritmoa.\ref{alg:Hairer-IRK}) eta erabilitako batura konpensatuaren teknika (Algoritmoa.\ref{alg:Hairer-BK}) zehaztu ditugu.
 
\begin{algorithm}[h!]
 \BlankLine
  $y_0=y(t_0); \ e_0=0$\;
  \For{$n\leftarrow 0$ \KwTo ($endstep-1$)}
  {
   \BlankLine
   $k=0$\;
   $Y_{n,i}^{[0]}=y_n+h \ c_i \ f(y_n) $\; 
   \BlankLine
   \While{ ($\Delta^{[k]} \ != 0 \ \ and \  \Delta^{[k]} < \Delta^{[k-1]}) $}
   {
    \BlankLine 
    $k=k+1$\;
    $F_{n,i}^{[k]}=f(Y_{n,i}^{[k-1]}) $\;
    $Y_{n,i}^{[k]}=y_n+ h \ \big(\sum\limits_{j=1}^{s} a^{\ast}_{ij} F_{n,j}^{[k]} \big) 
                          + h \ \big(\sum\limits_{j=1}^{s} \tilde a_{ij} F_{n,j}^{[k]} \big)$\; 
    $\Delta ^{[k]} = \max_{i=1,\dots,s}\|Y_{n,i}^{[k]}-Y_{n,i}^{[k-1]}\|_{\infty}$\;
   }
   \BlankLine
   $(y_{n+1},e_{n+1})\leftarrow BaturaKonpensatua(y_n,e_n,F_n^{[k]})$\;      
   \BlankLine
 }
 \caption{Hairer (IRK)}
 \label{alg:Hairer-IRK}
\end{algorithm}


\begin{algorithm}[h!]
%  \SetAlgoLined\DontPrintSemicolon
  \SetKwFunction{algo}{algo}\SetKwFunction{BaturaKonpensatua}{BaturaKonpensatua}
  \SetKwProg{myalg}{Algorithm}{}{}
  \SetKwProg{myproc}{Function}{}{}
  \myproc{\BaturaKonpensatua {$y_n$, $e_n$, $F_n^{[k]}$}}{  
     \BlankLine
     $\delta^{\ast}_{n}=h \ \big(\sum\limits_{i=1}^{s} b^{\ast}_i F_{i}^{[k]} \big)$\;
     $\tilde{\delta}_{n}=h \big(\sum\limits_{i=1}^{s} \tilde b_i F_{i}^{[k]} \big)$\;
     $ee=\delta^{\ast}_{n}+e_{n}$\;
     $yy=y_n+ee$\;
     $ee=(y_n-yy)+ee$\;
     \BlankLine
     $ee=\tilde{\delta}_{n}+ee$\;
     $y_{n+1}=y_{n}+ee$\;
     $e_{n+1}=(yy-y_{n+1})+ee$\; 
     \BlankLine  
    \KwRet ($y_{n+1}$,$e_{n+1}$) \;
    \BlankLine }
  \caption{Hairer (Batura konpensatua)}
  \label{alg:Hairer-BK}
\end{algorithm} 


\subsubsection*{Hairer-en inplementazioaren arazoak.}

Hairer-ek bere inplementazio berria, \emph{Hénon-Helies} eta eguzki-sistemaren kanpo planeten problemetarako energian errore sistematikorik ez zegoela baieztatu zuen. Energia errorea, $k\sqrt{t_n}$ espresioaren arabera handitzen dela erakutsi zuen eta beraz, inplementazio berriak \emph{Brouwer legea} \cite{Grazier2005} betetzen duela ondorioztatu zuen. Era berean, integrazioen azterketa estatistikoa egin zuen biribiltze errorearen ausazkotasuna baieztatzeko. Perturbatutako $1.000$ hasierako balio ezberdinen integrazio egin zituen eta integrazio horien energia errorearen batezbestekokoa ($\mu$) zero eta desbiazio tipikoa ( $\sigma$) $\sqrt{t_n}$ espresioaren proportzionala zirela erakutsi zuen. Azkenik, integrazioen bukaerako energi erroreen histogramak, $N(\mu,\sigma)$ distribuzio normala betetzen duela erakutsi zuen.

Gure aldetik Hairer-en inplementazioarekin egindako zenbakizko integrazio berriak egin ditugu, eta zenbait kasuetan geratze erizpidea ez duela ondo funzionatzen konprobatu dugu. Gainera, bere inplementazioaren biribiltze errorearen propagazioa optimoa ez dela uste dugu. 


\section{Gure inplementazioa.}

IRK metodoaren puntu-finkoaren inplementazioan lau proposamen berri egingo ditugu. Lehen bi proposamenak  Hairer-ek bere lanean proposatutako konponbideen hobekuntzak dira. Batetik, IRK-ren birformulazio bat erabiliz, IRK metodoaren koma-higikorreko koefizienteak sinplektizidade baldintza zehazki betetzea lortuko dugu. Bestetik, puntu-finkoaren geratze irizpidean arazo batzuk topatu ditugu eta arazo hauek gainditzen dituen geratze irizpide sendoagoa garatu dugu. Azken bi proposamenetan, batura-konpensatuaren hobekuntza  eta biribiltze errorea monitorizatzeko teknika proposatzen dira.
  

\subsection{Metodoaren birformulazioa (1.proposamena).}

IRK metodoa definitzen duten $a_{ij},b_i  \in \mathbb{R}$ koefizienteen ordez (\ref{eq:btchtaula}),~aplikatzen ditugun biribildutako koefizienteen $\tilde a_{ij},\tilde b_i \in \mathbb{F}$ ~hurbilpenek ez dute sinplektizidade baldintza \cite{JMSanz-Serna1994} betetzen,
\begin{equation}
\label{eq:simplektik2}
b_{i}a_{ij}+b_{j}a_{ji}-b_{i}b_{j}=0, \ \ 1 \leqslant i,j \leqslant s.
\end{equation}  
  
Hairer-ek doitasun handiagoko koefizienteak (\ref{eq:hkoef}) erabiltzea proposatu zuen  baina era honetan, ez da sinplektizidade baldintza betetzen. Arazo hau gainditzeko, IRK metodoa era honetan birformulatuko dugu,
\begin{align}
\label{eq:irk}
Y_{n,i}&=y_n+ \sum\limits_{j=1}^{s} \mu_{ij} L_{n,j},  \ \ L_{n,i}=hb_if(Y_{n,i}), \ \ i=1,\dots,s,\\
y_{n+1}&=y_n+\sum\limits_{i=1}^{s} L_{n,i},
\end{align}
non 
\begin{equation*}
\mu_{ij}=a_{ij}/{b_j}, \ \ 1 \leqslant i,j \leqslant s.
\end{equation*}

Eta sinplekzidade baldintza (\ref{eq:simplektik2}) modu honetan berridatziko dugu,
\begin{equation}
\label{eq:sinplekmij}
\mu_{ij}+\mu_{ji}-1=0, \ \ \ 1 \leqslant i,j \leqslant s.
\end{equation}
 
Birformulazio berri honen sinplektizidade baldintzaren espresioan, ez da biderketarik agertzen eta ezaugarri honek, baldintza zehazki betetzen duten $\tilde \mu_{ij} \in \mathbb{F}$ koefizienteak aurkitzeko bidea errezten du. Jarraian, koefizienteak finkatzeko urratsak azalduko ditugu:

\begin{enumerate}
\item $\mu_{ij}$~koefizienteak.

$S$-ataleko Gauss metodoetan, diagonaleko koefizienteek  balio finkoa dute ($\tilde{\mu}_{ii}:=1/2, \ i=1,\dots,s$) eta balio honek, koma-higikorreko adierazpen zehatza du.

Gainontzeko koefizienteak, bi urratsetan kalkulatuko ditugu:
\begin{itemize}
\item Lehenengo, koefiziente matrizearen behe-diagonaleko balioak finkatuko ditugu,
\begin{equation*}
 \tilde{\mu}_{ij}:=fl(\mu_{ij}), \ 1 \leqslant j < i \leqslant s.
\end{equation*}

\item Bigarrenik, koefiziente matrizearen goi-diagonaleko balioak esleituko ditugu,
\begin{equation*}
\tilde{\mu}_{ji}:=1-\tilde{\mu}_{ij} , \ 1 \leqslant j < i \leqslant s.
\end{equation*}

\end{itemize}
Sterbenz-en teoremak (ikus. \ref{eq:4311}), $1/2 < |\mu_{ij}| <2$ denez, $1-\tilde{\mu}_{ij}$ balioak koma-higikorreko adierazpen zehatza izango duela ziurtatzen du. 

Laburtuz, hauek ditugu birformulatutako simplektizitate baldintza (\ref{eq:sinplekmij}) zehazki betetzen duten koma-higikorreko $\tilde{\mu}_{ij}\in \mathbb{F}$ koefizienteak,   
\begin{equation}
\tilde{\mu}=\left(\begin{array}{cccc}
    1/2       & 1-fl(\mu_{21}) & \dots & 1-fl(\mu_{s1})      \\
    fl(\mu_{21})      & 1/2    & \dots & 1-fl(\mu_{s2})      \\
    \vdots            & \ddots         &       & \vdots      \\
    fl(\mu_{s1})      & fl(\mu_{s2})   & \dots & 1/2          \\ 
     \end{array}\right).
\end{equation}

\item $hb_{i}$~koefizienteak.

Gure inplementazioan, $hb_i=h \times b_i$ koefizienteak aurre-kalkulatuko ditugu. Koefiziente hauek simetrikoak direla eta  $\sum\limits_{i=1}^{s} hb_i=h$ berdintza bete behar dela jakinda, modu honetan kalkulatuko ditugu
\begin{align*}
hb_i & = fl(h \times b_i), \ \ \ i=2,\dots,s-1, \\
hb_1 & =hb_s:= \left(h - \sum\limits_{i=2}^{s-1} hb_i \right)/2.
\end{align*}

\item $\nu_{ij}$ interpolazio koefizienteak.

Formulazio estandarraren $\lambda_{ij}$ koefizienteetatik abiatuta (ikus ~\ref{ss:2.2.3.2}atala), formulazio berriari dagozkion interpolazio $\nu_{ij}$ koefizienteak era honetan definituko ditugu,
\begin{align}
\label{eq: interpLi}
Y_{n,i}^{[0]} &= y_n+ h \sum\limits_{j=1}^{s} \nu_{ij} L_{n-1,j}, \ \ \nu_{ij}=\lambda_{ij}/b_j \ \ 1\leqslant i,j \leqslant s.
\end{align} 

\end{enumerate}

\subsection{Geratze irizpidea (2.proposamena).}

Ekuazio inplizituaren (\ref{eq:62}) soluzioaren hurbilpena lortzeko puntu-finkoko iterazioa era honetan definituko dugu: iterazioaren abiapuntua $Y_i^{[0]}$  finkatu eta $k=1,2,\dots$ iterazioetarako $Y_i^{[k]}$ hurbilpenak lortu dagokigun geratze irizpidea bete arte.
%\begin{equation}
%L_i^{[k]}=hb_if(Y_i^{[k-1]}), \ \ Y_i^{[k]}=y_n+\sum\limits_{j=1}^{s} \mu_{ij} L_j^{[k]}
%\end{equation}

\begin{algorithm}[H]
 \For{ (k=1,2,\dots konbergentzia lortu arte)}
  {
   $L_i^{[k]}=hb_if(Y_i^{[k-1]}) $\;
   $Y_i^{[k]}=y_n+\sum\limits_{j=1}^{s} \mu_{ij} L_j^{[k]} , \ \  i=1,\dots,s $\; 
   }
 \caption{Puntu-finkoko iterazioa.}
 \label{alg:pf}
\end{algorithm}
 
\paragraph*{}IRK metodoaren inplementazio estandarrean, geratze irizpidea honakoa da,
\begin{equation*}
\Delta^{[k]}=(Y_1^{[k]}-Y_1^{[k-1]},\dots,Y_s^{[k]}-Y_s^{[k-1]}) \in \mathbb{F}^{sd},
\end{equation*} 
\begin{equation}
\|\Delta^{[k]}\| \le tol
\end{equation}
non $\|.\|$ aurre-finkatutako bektore norma eta \emph{tol} tolerantzia errorea den . Tolerantzia txikiegia aukeratzen bada, tolerantzia hori ez lortzea eta infinituki iterazioak exekutatzea gerta daiteke. Baina tolerantzia ez bada behar bezain txikia  aukeratzen, iterazioa puntu-finkora iritsi aurretik geratuko da eta lortutako $Y_i^{[k]}$ hurbilpenak biribiltze errorea baino errore handiagoa izango du. Gogorarazi ere, Hairer-ek \cite{Hairer2008} iterazio errorea modu sistematikoan metatzen dela konprobatu zuela.   

Hairer-ek proposatu zuen geratze irizpidea gogoratuko dugu; $\Delta^{[k]} = 0$ (puntu-finkora iritsi delako) ;  edo   $\Delta^{[k]} \geqslant \Delta^{[k-1]}$ (biribiltze errorea nagusi delako). Orokorrean, geratze irizpide honek ondo funtzionatzen du baina esperimentalki zenbait kasutan, iterazioak goizegi geratzen direla ikusi dugu. Hairer-ek, eguzki-sistemaren kanpo planeten problemaren $h=500/3$ eguneko urrats luzeerarekin egindako integrazioan ondo funzionatzen du baina $h=1000/3$ urrats luzeera aukeratuz, energia errorearen tamaina oso handia da. Energia errore erlatiboaren garapena (\ref{fig:OSSh2}Irudia) irudikatu dugu. Integrazioaren lehen urratsaren iterazioak aztertuz,

\begin{equation*}
\|\Delta^{[1]}\|>\|\Delta^{[2]}\| \dots > \|\Delta^{[12]}\|=3.91\times 10^{-14} \leqslant \|\Delta^{[13]}\|=4.35 \times 10^{-14} 
\end{equation*} 
iterazioa $13.$iterazioan geratzen dela konprobatu dugu. Puntu-finkoko iterazioa goizegi geratu da, hurrengo iterazioetan , $\|\Delta^{[13]}\|>\|\Delta^{[14]}\|>\|\Delta^{[15]}\|>\|\Delta^{[16]}\|=0$ gertatzen baita. 

Hairer-en geratze erizpidea aztertu ondoren bi ondorio atera daitezke. Batetik, ezin daiteke suposatu $\Delta^{[k]}$ segida beherakorra denik. Bigarrenik, $\Delta^{[k]} \geqslant \Delta^{[k-1]}$ baldintzak, biribiltze errorea nagusia dela adierazten duen arren, badago $j \in \{1,\dots,sd\}$ osagairik,   $|\Delta_j^{[k]}| < |\Delta_j^{[k-1]}|$ hobetzeko tartea duena.  

\begin{figure}[h!]
\centering
\begin{tabular}{c c}
\subfloat[OSS: Hairer's stopping criterion]
{\includegraphics[width=.4\textwidth]{Fig1}}
&
\subfloat[OSS: New stopping criterion]
{\includegraphics[width=.4\textwidth]{Fig2}}
\end{tabular}
\caption{\small Evolution of relative error in energy for the outer solar system problem (OSS) with the original unperturbed initial values in~\cite{Hairer2008} and doubled step-size ($h=1000/3$ days).  (a) Hairer's stopping criterion, (b) New stopping criterion}
\label{fig:OSSh2}
\end{figure}

Arazoari soluzioa emateko, geratze erizpidea berri bat proposatuko dugu. Honako notazioa finkatuko dugu,
\begin{equation*}
\Delta_j^{[k]}, \ \text{non} \ \Delta^{[k]} \in \mathbb{F}^{sd}  \ (1\leqslant j \leqslant sd).
\end{equation*}
 Iterazioak  $k=1,2,\ldots$ jarraitzea , $ \Delta^{[k]} =0$ bete arte edo honako baldintza bi iterazio jarraietan betetzen den artean,
\begin{equation}
\label{eq:not_stopping}
\forall j \in \{1,\ldots,s D\},  \quad
\min \left(\{|\Delta_j^{[1]}|,\cdots ,|\Delta_j^{[k-1]}|\} \ /\{0\} \right) \leqslant |\Delta_j^{[k]}|.
\end{equation}

Hairer-ek, eguzki-sistemaren kanpo planeten problemaren $h=1000/3$ eguneko urrats luzeerarekin egindako integrazioa errepikatu dugu baina oraingoan geratze erzipide berria aplikatuz. Energia errorearen eboluzioa (\ref{fig:OSSh2}Irudia) irudikatu dugu. 


\subsubsection*{Tolerantzi testa.}

Iterazio gehienetan, $j$ guztietarako $\Delta_{j}^{[k]}=0$ delako geratuko da. Dena den $j$ batetarako $\Delta_{j}^{[k]} \neq 0$ izanik iterazio geratzen bada, orduan puntu-finkoaren iterazioa onargarria den ala ez erabaki behar dugu. Iterazioa geratu daiteke biri


 orduan iterazio gehigarri bat emango dugu. Iterazio gehigarri bat emanez, esperimetanlki  puntu-finkoa lortzen den urratsen portzentaia nabarmen handitzen dela konprobatu dugu.


Puntu-finkoaren iterazioak amaitzerakoan, urratsa eman aurretik erabiltzaileak definitutako tolerantzia lortu den ala ez aztertuko dugu. Horretarako iterazioaren azken bi hurbilpenak konparatuko ditugu eta honako notazioa finkatuko dugu,

\begin{equation*}
Y_i=Y_i^{[k]}, \ \ \tilde{Y}_i=Y_i^{[k-1]}, \ \ i=1,\dots,s.
\end{equation*}  

Erabiltzaileak finkatutako tolerantzia erlatiboa eta tolerantzia absolutuaren parametroen arabera ($rto_i,atol_i, i=1,\dots,d$) , \emph{distantzia normalizatua} definituko dugu,
\begin{equation*}
\max_{i=1,\dots,d} \frac{\max_{j=1,\dots,s} |Y_j^i-\tilde{Y}_j^i|}
                        {\bigg(\big((\max_{j=1,\dots,s} |Y_j^i|+\max_{j=1,\dots,s} |\tilde{Y}_j^i|)/2 \big) \ rtol_i+ atol_i \bigg)}.
\end{equation*}

Distantzi normalizatua $>1$ bada orduan ez da lortu tolerantzia eta integrazio amaituko dugu.

Tolerantzia ez dugu erabiliko puntu-finkoko iterazio geratzeko, behin iterazioa geratu denean, emaitza onargarria den ala ez erabakitzeko baizik. Puntu-finkoko iterazioa konbergentzia lortu duelako edo konbergentzia arazoak izan direlako gera daiteke. Doitasun zehatzean, biribiltze errorearen eragina handia izan daiteke eta beraz, ez dugu tolerantzia estua erabiliko. Diskriminatu nahi dugu, aritmetika zehatzean konbergitu duen ala ez. Biribiltze errorea ez balitz konbergitu duen erabaki nahi dugu.

\subsection{Biribiltze errorea gutxitzeko teknikak (3.proposamena).}

Koma higikorreko aritmetika atalean (ikus \ref{sec:4.4}atala), baturan nahiz biderketan egindako biribiltze errore zehatza modu errazean kalkulatu daitekeela  ikusi genuen. Eragiketa hauen biribiltze erroreak, ondorengo konputazioetan erabiliko ditugu soluzioaren doitasuna hobetzeko.

Batura errekurtsiboetan, konputazioaren doitasuna hobetzeko teknikari \emph{batura konpensatua} esaten zaio eta zenbakizko integrazioetan erabili ohi da. Atal honetan, batetik IRK metodoetan batura konpensatuaren aplikazio estandarra hobetzeko proposamena azalduko dugu. Beste aldetik, IRK metodoaren gure  inplementazioan biribiltze errorearen beste jatorri nagusiak (biderketa eta batuketa bat) modu finagoan kalkulatzea proposatuko dugu.    

Goian aipatutako ideia,  beste ikuspegi batetik ere azaldu daiteke. Ikuspegi honen arabera, gure zenbakizko soluzioa, bi \emph{double} balioren bidez ($\tilde{y}_n,e_n$)  adierazten ari gara (ia doitasun laukoitza) eta beraz, interpretazio honen arabera, konputazio eragiketa batzuk ia doitasun laukoitzean egiten ariko ginateke. Zentzu honetan gure inplementazioan, hasierako balio zehatza $y_0=y(t_0)$, doitasun bikoitzeko bi zenbakiren bidez ($\tilde{y}_0, e_0$) hasieratuko dugu,
\begin{align*}
\tilde{y}_0 &=fl(y_0) ,\\
e_0 &=fl(y_0-\tilde{y}_0).
\end{align*}


\subsubsection*{Batura konpensatua (aportazioa).}

$y_{n+1} \in \mathbb{R}^{d},\quad y_{n+1}=\tilde y_{n}+\tilde \delta_n$ batura zehatza izanik eta $\tilde y_{n+1} \in \mathbb{F}^{d}, \quad \tilde y_{n+1}=\tilde y_{n} \oplus \tilde \delta_n$ koma-higikorreko hurbilpena izanik, batura konpensatuaren bidez lortutako errorearen estimazioa $e_{n+1}$ ,

\begin{algorithm}[H]
 \BlankLine
  $\tilde{y}_{0}=fl(y_{0}); \ e_0=fl(y_0-\tilde{y}_0)$\;
 \BlankLine
  \For{$n=1,2,\cdots \quad$}
  {
   \BlankLine
    $inc=\tilde {\delta}_n \oplus e_n$\;
    $\tilde {y}_{n+1}=\tilde{y}_n \oplus inc$\;
    $e_{n+1}=(\tilde{y}_n \ominus \tilde {y}_{n+1}) \oplus inc$\;
   \BlankLine
  }
 \caption{Batura konpensatua.}
\end{algorithm}

\paragraph*{}baturan egindako  biribiltze errore zehatza da,
\begin{equation}
y_{n+1}=\tilde {y}_{n+1}+e_{n+1}. 
\end{equation}

Horregatik, IRK metodoaren inplementazioan, inplizituki $Y_{n,i}$ atalak askatzeko ekuazioetan, $\tilde {y}_n$ ordez ($\tilde{y}_n \oplus \tilde{e}_{n}$) erabiltzea proposatzen dugu, 
\begin{equation}
\label{eq:eqbk}
L_{n,i}^{[k]}=hb_if(Y_{n,i}^{[k-1]}), \ \ \ Y_{n,i}^{[k]}=\tilde{y}_n \oplus \big(\tilde{e}_{n} \oplus \sum\limits_{j=1}^{s} \mu_{ij} L_{n,j}^{[k]}\big).
\end{equation}

Aldaketa honekin, lortutako zenbakizko soluzioaren doitasuna batura konpensatu estandarrarekin baino pixka bat hobea izango dela espero dugu. 

\subsubsection*{Biderketaren biribiltze errorea.}

Urratsa emateko unean, $L_{n,i}=hb_i \ f(Y_{n,i}), \ i=1,\dots,s$ biderketen biribiltze errorea kalkulatu eta $e_{n}$ gaiari gehituko diogu. Biderketaren biribiltze errorea jasotzeko konputagailuaren \emph{FMA} eragiketan oinarrituko gara eta (\ref{sec:4.4})ataleko (\ref{alg:FastSum}.algoritmoa) aplikatuko dugu. 

\begin{algorithm}[h]
$L_{n,i}^{[k]}=hb_i \ f(Y_{n,i}^{[k-1]})$\; 
$E_{n,i}=hb_i \ f(Y^{[k-1]}_{n,i}) - L^{[k]}_{n,i} \ \ , \ \ i=1,\dots,s$\;
$\beta_{n}={e}_{n} + \sum\limits_{j=1}^{s}E_{n,j}$\;
% \caption{Biderketaren biribiltze errorea eta batura konpensatua.}
\end{algorithm}

\subsubsection*{Baturaren biribiltze errorea.}

$\delta_n=\left(\sum\limits_{i=1}^{s}L_{n,i}^{[k]}\right)+\beta_{n}$ baturaren biribiltze errorea.


\begin{algorithm}[H]
  \SetAlgoLined\DontPrintSemicolon
  \SetKwFunction{algo}{algo}\SetKwFunction{BaturaKonpensatua}{BaturaKonpensatua}
  \SetKwProg{myalg}{Algorithm}{}{}
  \SetKwProg{myproc}{Function}{}{}
  \myproc{\BaturaKonpensatua {$y_n$,\ $\beta_n$,\ $L_n^{[k]}$}}{
     \BlankLine
     $s_0=y_n$\;
     $ee=\beta_n$\;
     \For{$i\leftarrow 1$ \KwTo ($s$)}
      {
        $s_1=s_0$\;
        $\delta= L_{n,i}^{[k]} +ee$\;
        $s_0=s_1+\delta$\;
        $ee=(s_1 - s_0)+ \delta$\;   
      }
     $y_{n+1}=s_0$\;
     $e_{n+1}=ee$\;    
    \KwRet ($y_{n+1}$,$e_{n+1}$) \;}
  \caption{BaturaKonpensatua}
\end{algorithm} 

\subsection{Biribiltze errorearen estimazioa (4.proposamena).}

Zenbakizko integrazioaren biribiltze errorearen estimazioa, bigarren zenbakizko integrazio baten soluzioaren diferentzia gisa kalkulatuko dugu. Bigarren integrazio honetan, $\tilde{\tilde{L}}_{n,i}$ terminoak mantisa txikiagoko zenbakira biribilduko ditugu,  doitasun gutxiagoko soluzioa lortzeko. 

$r\ge0$ zenbaki osoa, eta $x \in \mathbb{F}$ ($m$-biteko doitasuneko koma-higikorreko zenbakia) izanik, honako funtzioa definituko dugu,

\begin{algorithm}[H]
  \SetAlgoLined\DontPrintSemicolon
  \SetKwFunction{algo}{algo}\SetKwFunction{floatR}{floatR}
  \SetKwProg{myalg}{Algorithm}{}{}
  \SetKwProg{myproc}{Function}{}{}
  \myproc{\floatR {x,r}}{
    $res=(2^r x + x)- 2^r x$\;
    \KwRet res \;}
  \caption{floatR}
\end{algorithm} 

\paragraph*{}Funtzio honek itzultzen duen balioa, $(m-r)$-biteko doitasuneko koma-higikorreko zenbakia da. Beste modu batera esanda, $m$ biteko koma-higikorreko $x$ zenbakiaren azken $r$ bitak zeroan jartzen dituen funtzioa da.

Biribiltze errorearen estimazioa, zenbakizko soluzio nagusiaren $(y_n+e_{n})$ eta $r$ ($r<m$) balio txiki baterako (adibidez $r=3$) kalkulatutako bigarren zenbakizko soluzioaren $(\tilde{\tilde{y}}_n+\tilde{\tilde{e}}_{n})$ arteko diferentziaren norma bezala kalkulatuko dugu. 
\begin{equation}
estimazioa_n^i=\|(y_n^i+e_n^i)-(\tilde{\tilde{y}}_n^i+\tilde{\tilde{e}}_{n}^i)\|_2, \ \ i=1,\dots,d.
\end{equation}

Gure algoritmoan estimazioa zuzenean lortzeko, bi integrazioak sekuentzialki modu eraginkorrean kalkulatuko ditugu. Urrats bakoitzean, bi integrazioen $Y_i,\tilde{\tilde{Y}}_i$ ($i=1,\dots,s$) ataletako balioak, biribiltze errorea estimazio handiegia ez den artean,  antzekoak mantentzen dira. Beraz, bigarren integrazioan iterazio kopuru txikia beharko dugu, lehen integrazioaren bukaerako $Y_i^{[k]}$ ($i=1,\dots,s$) atalen balioak, bigarren integrazioaren $\tilde{\tilde{Y}}_i^{[0]}$ (i = 1, . . . , s) atalen hasieraketarako erabiltzen baditugu (ikus \ref{alg:errore-estimazioa}.algoritmoa).  

\begin{algorithm}[h!]
  \BlankLine
  \For{$n\leftarrow 0$ \KwTo ($endstep-1$)}
  {
    \BlankLine
    $Y_n^{[0]}=G(Y_{n-1},h)$\;
    \BlankLine
    $\text{Lehen Integrazioa}$\;
	\BlankLine
    $(y_{n+1},e_{n+1})\leftarrow BaturaKonpensatua(y_n,\beta_n,L_n^{[k]})$\;      
    \BlankLine
    \BlankLine
    \eIf{$(initwithfirst)$}
    {$\tilde{\tilde{Y}}_{n}^{[0]}=Y_{n}^{[k]}+(\tilde{\tilde{y}}_n-y_n)$\;}
    {$\tilde{\tilde{Y}}_{n}^{[0]}=G(\tilde{\tilde{Y}}_{n-1},h)$\;}
    \BlankLine
    $\text{Bigarren Integrazioa}$\;
	\BlankLine
    $(\tilde{\tilde{y}}_{n+1},\tilde{\tilde{e}}_{n+1})\leftarrow BaturaKonpensatua(\tilde{\tilde{y_n}},\tilde{\tilde{\beta}}_n,floatR(\tilde{\tilde{L}}_n^{[k]},r))$\;  
    \BlankLine
    \BlankLine
    $estimation_{n+1}=\|(y_{n+1}+e_{n+1})-(\tilde{\tilde{y}}_{n+1}-\tilde{\tilde{e}}_{n+1})\|_2$\;
    \BlankLine
   }
 \caption{RKG2: errore estimazioa}
 \label{alg:errore-estimazioa}
\end{algorithm}


\subsection{Algoritmoa.}

Formulazio berriari dagokion algoritmo orokorra laburtuko dugu \ref{alg:IRK-Berria}.algoritmoa.

\begin{algorithm}[h!]
 \BlankLine
  $\tilde{y}_0=fl(y_0)$\;
  $e_0=fl(y_0-\tilde{y}_0)$\;
  \For{$n\leftarrow 0$ \KwTo ($endstep-1$)}
  {
   \BlankLine
   $k=0$\;
   \text{Hasieratu}  $Y_{n,i}^{[0]} \ \ , \ \ i=1,\dots,s $\;
   \BlankLine
   \While{ (\text{not konbergentzia})}
   {
    \BlankLine 
    $k=k+1$\;
    $F_{n,i}^{[k]}=f(Y_{n,i}^{[k-1]}) $\;
    $L_{n,i}^{[k]}=hb_i \ F_{n,i}^{[k]} $\;
    $Y_{n.i}^{[k]}=\tilde{y}_{n} + \ \big(e_n+\sum\limits_{j=1}^{s} \mu_{ij} L_{n,j}^{[k]}\big)  $\;  
    $\text{konbergentzia} \leftarrow \text{GeratzeErizpidea}(Y^{[k]},Y^{[k-1]},\Delta_{min}) $\;
   }
   \BlankLine
   \If{($\exists j \ \text{non} \ \Delta_j^{[K]}\neq 0$)}
   {
   \If{$(NormalizeDistance(Y^{[k]},Y^{[k-1]})>1$}
   {$\text{fail convergence}$\;}
   }
   $E_{n,i} = h\,   b_i\,f_{n,i}^{[k]}-L_{n,i}^{[k]}$\;
   $\beta_{n}=e_{n}+\sum_{i=1}^{s} E_{n,i}$\;
   $(\tilde y_{n+1}, e_{n+1})\leftarrow \text{baturakonpensatua}(\tilde y_{n},\beta_{n},L_{n}^{[k]})$\;
   \BlankLine
 }
 \caption{IRK (puntu-finkoa).}
 \label{alg:IRK-Berria}
\end{algorithm}

\subsubsection*{Zenbakizko soluzioa.}  

Integrazio tartea $[t_0,t_{end}]$ eta urrats tamaina $h$ bada, emandako urrats kopurua $N=(t_{end}-t_0)/h$ izango da. Zenbakizko soluzioa fitxategi bitar batean idatziko da eta erabiltzaileak definitutako $m$ urratsero itzuliko da. 

$k$. integrazioan $N$ urrats eman baditugu eta $m$ finkatutako soluzioen frekuentzia bada, $t_i=t_0+i*(m \ h), \ i=1,\dots,N/m$ uneetarako zenbakizko soluzioa lortuko dugu. Bi integrazio mota exekutatu daitezke:

\begin{enumerate}
\item Integrazio arrunta.

Integrazioan itzultzen den fitxategiaren egitura honakoa da:
\begin{align*}
& (t_i,y_i,e_i) \ \text{non} \ \ y_i,e_i \in \mathbb{R}^d.\\
& y_i=(q_i,p_i) \ \text{eta} \ e_i=(eq_i,ep_i).
\end{align*}

\item Integrazioa errore estimazioarekin.

Errorearen estimazioa itzultzen duen integrazioaren emaitza honakoa da,
\begin{align*}
& (t_i,y_i,e_i,est_i) \ \text{non} \ \ y_i,e_i,est_i \in \mathbb{R}^d.\\
&  y_i=(q_i,p_i), \ e_i=(eq_i,ep_i) \ \text{eta} \ est_i=(estq_i,estp_i).
\end{align*}

\end{enumerate}

 Zenbakizko soluzioa doitasun bikoitzeko bi zenbakien batura gisa adieraziko dugu (aurretik azaldutakoaren erreferentzia),
\begin{equation*}
(y_i,e_i)=(q_i,p_i,eq_i,ep_i), \ \ y_i,e_i \in \mathbb{R}^d.
\end{equation*}
\begin{equation*}
(q_i^{[k]}+eq_i^{[k]},p_i^{[k]}+ep_i^{[k]})\approx(q(t_i)^{[k]},p(t_i)^{[k]}), \ \ \ i=1,\dots,N/m.
\end{equation*}


\clearpage

\section{Inplementazio estadarraren xehetasunak.}

\subsection*{Hamiltondar banagarriak.}

Era honetako ekuazio diferentzialak garrantzitsuak dira,
\begin{equation*}
\dot{p}=f(q), \ \ \dot{q}=g(p).
\end{equation*}

Esaterako, Hamiltondar banagarriak $H(q,p)=T(p)+U(q)$ eta bigarren ordeneko ekuazio diferentzialak $\ddot{q}=f(q)$ era honetako ekuazio diferentzialen kasu partikularrak dira.

Hurbilpena $(p_{n+1},q_{n+1}) \approx (p(t_{n+1}),q(t_{n+1}))$ era honetan kalkulatuko dugu,
\begin{align*}
p_{n+1}=p_n+ h \sum\limits_{i=1}^{s} b_i \ f(t_n+c_ih,Q_{n,i}),\\
q_{n+1}=q_n+ h \sum\limits_{i=1}^{s} b_i \ g(t_n+c_ih,P_{n,i}),
\end{align*}

non $(P_{n,i},Q_{n,i}), \ i=1,\dots,s$ honako ekuazio sistema definitutako atalak diren, 
\begin{align*}
P_{n,i} &=p_n+ h \sum\limits_{j=1}^{s} a_{ij} \ f(t_n+c_jh,Q_{n,j}), \\
Q_{n,i} &=q_n+ h \sum\limits_{j=1}^{s} a_{ij} \ g(t_n+c_jh,P_{n,j}).
\end{align*}

Era honetako problemetan,  funtsean \emph{IRK} algoritmo orokorra (alg.\ref{alg:IRK1}) aplikatuko dugu. Baina Hamiltondarraren egituraren abantaila aprobetxatuz, puntu-finkoaren iterazioaren konbergentzia hobetuko dugu,   

\begin{algorithm}[H]
  \For{ (k=1,2,\dots)}
  {
   $P_{n,i}^{[k]}=p_{n}+ h \ \sum\limits_{j=1}^{s} a_{ij} \ f(t_n+c_jh,Q_{n,j}^{[k-1]})$\; 
   $Q_{n,i}^{[k]}=q_{n}+ h \ \sum\limits_{j=1}^{s} a_{ij} \ g(t_n+c_jh,P_{n,j}^{[k]}), \ \ \ \ i=1,\dots,s $\; 
  }
 \caption{Puntu-finkoaren iterazioa (Gauss-Seidel).}
\end{algorithm}
 

\subsection*{Bigarren ordeneko EDA.}
 
 
Bigarren ordeneko ekuazio diferentzialen $\ddot{q}=f(q)$ (\emph{Runge-Kutta-Nyström}) azterketa egiteko, modu baliokide honetan idatziko dugu,
\begin{equation*}
\dot{p}=f(q), \ \ \dot{q}=p.
\end{equation*}

\paragraph*{}Hurbilpena $(p_{n+1},q_{n+1}) \approx (p(t_{n+1}),q(t_{n+1}))$ era honetan kalkulatuko dugu,
\begin{align*}
p_{n+1}=p_n+ h \sum\limits_{i=1}^{s} b_i \ f(t_n+c_ih,Q_{n,i}),\\
q_{n+1}=q_n+ h p_{n} + h^2 \sum\limits_{i=1}^{s} \beta_i \ f(t_n+c_ih,Q_{n,i}),
\end{align*}

non $Q_{n,i}, \ i=1,\dots,s$ honako ekuazio sistema definitutako atalak diren, 
\begin{align*}
Q_{n,i}=q_n+ h\gamma_i p_n+ h^2 \sum\limits_{j=1}^{s} \tilde{a}_{ij} \ f(t_n+c_jh,Q_{n,j}).
\end{align*}

\paragraph*{}{\textbf{IRK algoritmoa-III (bigarren ordeneko EDA)}.}


\begin{algorithm}[H]
 \BlankLine
  \For{$n\leftarrow 0$ \KwTo ($endstep-1$)}
  {
   \BlankLine
   Hasieratu  $Q_{n,i}^{[0]} \ \ , \ \ i=1,\dots,s $\;
    \BlankLine
   \While{ (konbergentzia lortu)}
   {
    \BlankLine 
    $F_{n,i}=f(t_n+c_ih,Q_{n,i}) \ \ , \ \  i=1,\dots,s$\;
    $Q_{n,i}=q_n+ h\gamma_i p_n+ h^2 \sum\limits_{j=1}^{s} \tilde{a}_{ij} \ f(t_n+c_jh,Q_{n,j}) \ \ , \ \  i=1,\dots,s$\;  
   }
   \BlankLine
    $\delta p_n=h \ \sum\limits_{i=1}^{s} b_i F_{n,i}$\;
    $\delta q_n=h^2 \sum\limits_{i=1}^{s} \beta_i F_{n,i}$\;    
    $p_{n+1}=p_{n}+ \delta p_n $\;
    $q_{n+1}=q_{n}+ h\gamma_i p_n+\delta q_n $\;
   \BlankLine
 }
 \caption{IRK algoritmoa-III (bigarren ordeneko EDA)}\label{alg:IRK2}
\end{algorithm}

\paragraph*{}Bigarren ordeneko ekuazio diferentzialak ditugunean, puntu-finkoko iterazioa,

\begin{algorithm}[H]
  \For{ (k=1,2,\dots)}
  {
   $Q_{n,i}^{[k]}=q_{n}+h \gamma_i p_{n}+ h^2 \ \sum\limits_{j=1}^{s} \tilde{a}_{ij} f(t_n+c_jh,Q_{n,j}^{[k-1]}) $\;  
  }
 \caption{Puntu-finkoko iterazioa (bigarren ordeneko EDA)}
\end{algorithm} 


\section{Esperimentuak.}

\subsection{Sarrera.}

Lau problemetarako esperimentuetan, integrazio parametroak bateratu ditugu. 

\begin{enumerate}

\item $s=6$ ataletako Gauss IRK metodoa aplikatu dugu.

\item Epe luzeko integrazioak aztertu ditugu.

\item Urratsa, trunkatze errorea biribiltze errorea baino txikiagoa izan dadin aukeratu dugu. 

\item Integrazio mota bakoitza $P=1.000$ perturbatutako hasierako balio ezberdinekin integratu dugu. Perturbazioen kalkulurako $k=2^{-20}$ balioa finkatu dugu, hau da, $\mathcal{O}(10^{-16})$   mailako perturbazioak aplikatu ditugu problema guztietarako.  

\item Kokapen errorearen estimazioaren kalkuluan, integrazio nagusian $rdigits1=0$ eta bigarren integrazioan $rdigits2=3$ balioak erabili ditugu.

\end{enumerate}

Lau probema ezberdinetarako egin ditugu zenbakizko esperimentuak eta bakoitzean zehazki erabilitako integrazio tartea eta urratsa tamainak hauek izan dira:

\begin{enumerate}
\item Pendulu bikoitza ez-kaotikoa (NCDP).
\begin{align*}
& t_0=0, \ \ t_{end}=2^{12}, h=2.^{-7}, \\
& m=2^{10}, L=512.
\end{align*} 

\item Pendulu bikoitza kaotikoa (CDP).
\begin{align*}
& t_0=0, \ \ t_{end}=2^{8}, h=2.^{-7}, \\
& m=2^6, L=512.
\end{align*} 

\item Outer solar system (6-Body).
\begin{align*}
& t_0=0, \ \ t_{end}=10^{7}, h=500/3, \\
& m=120, L=500.
\end{align*} 

\item Solar system (10-Body).
\begin{align*}
& t_0=0, \ \ t_{end}=10^{6}, h=2, \\
& m=2000, L=250.
\end{align*} 

\end{enumerate}

\subsection{Energia errore jatorriak.}

Gure doitasun bikoitzeko inplementazioaren zenbakizko soluzioaren $\tilde{y}_n+e_n \approxeq y(t_n) \ (n=1,2,\cdots)$ errorea, jatorri ezberdineko errore mota ezberdinen konbinazioa da:

\begin{enumerate}
\item Trunkatze errorea. Hasierako baliodun problemaren soluzio zehatza $y(t_n)$, ($n=1,2,3,\cdots$), (\ref{eq:irk}) metodoa aplikatuz  ($b_i,\mu_{ij}$ koefiziente zehatzekin) lortutako zenbakizko soluzioa $y_n$ ordezkatzerakoan eragindako errorea. 
\item Iterazio errorea. Praktikan, puntu-finkoko iterazioa (\ref{alg:pf}) $K$ kopuru finitua aplikatzen da, eta $L_{n.i}, Y_{n,i} \ (i=1,\cdot,s)$ soluzioa, $L_{n.i}^{[K]}, Y_{n,i}^{[K]}$ hurbilpenarekin ordezkatzen da. Hurbilpen honen arabera dagokion zenbakizko soluzioa kalkulatzen da,
\begin{equation*}
y_{n+1}=y_{n}+\sum_{i=1}^{s} L_{n,i}^{[K]}.
\end{equation*}   
\item Funtzioa zehatza $f:\mathbb{R}^D \rightarrow \mathbb{R}^D$, bere doitasun bikoitzeko bertsioaz $\tilde{f}:\mathbb{F}^D \rightarrow \mathbb{F}^D$ ordezkatzerakoan eragindako bi mailetako errorea. Batetik, $K$ iterazio finituetan puntu-finkora iristeagatik eragindako ekidin ezineko iterazio errorea. Bestetik, $\tilde{f}$ konputazioan gertatutako biribiltze erroreak. 
\item IRK metodoaren koefiziente zehatzak $b_i,\mu_{ij} \in \mathbb{R}$, dagozkien doitasun bikoitzeko koefiziente $\tilde{b}_i,\tilde{\mu}_{ij} \in \mathbb{F}$ erabiltzeagatik eragindako errorea.
\item Algoritmoaren inplementazioaren eragiketa aritmetikoak, doitasun bikoitzean kalkulatzeagatik eragindako errorea.  
\end{enumerate} 

Errore jatorri hauek energian duten eragina simulatzeko, honako algoritmoak inplementatu ditugu:

\begin{enumerate}

\item Inplementazio zehatza.

Konputazio guztia (ekuazio diferentzialaren funtzioaren ebaluazioa barne) doitasun laukoitzeko ($128$-bit) koma higikorreko aritmetikan  egindako integrazioa. Zenbakizko integrazio hau, trunkatze errorea estimatzeko eta errore globala kalkulatzeko erreferentzi gisa (soluzio zehatza) kontsideratuko dugu. 

\item Inplementazio superideala.

Zenbakizko integrazio hau iterazio errorea estimatzeko erabiliko dugu. Konputazio guztia doitasun laukoitzean egindako integrazioa baina geratze erizpidea, doitasun bikoitzean neurtuko dugu,
\begin{equation*}
\Delta^{[k]}=|\text{double}(Y^{[k]})-\text{double}(Y^{[k-1]})|.
\end{equation*}

\item Inplementazio ideala.

Ekuazio diferentzialaren eskuin aldeko funtzioaren ebaluazioa izan ezik, beste eragiketa guztiak doitasun laukoitzean egiten dituen inplementazioa da. Ekuazio diferentziala doitasun bikoitzean kalkulatzeak, eragiten duen errorea neurtzeko erabiliko dugu eta integrazio hau hobetu ezin daitekeen integrazioa kontsideratuko dugu.  

\item Inplementazio sasi-ideala.

Doitasun bikoitzeko koefizienteak ($\tilde{\mu}_ij,\tilde{b}_i \in \mathbb{F}$) erabiltzeak eragiten duen errorea neurtzeko integrazioa da. Konputazio guztia doitasun laukoitzean kalkulatzen da baina doitasun bikoitzeko koefizienteen balioak erabiliz (doitasun bikoitzeko koefiziente koadrifikatuak esaten diogu). 

\end{enumerate}

We next plot (Fig. \ref{fig:SourceError}), for each of the three considered initial value problems, the evolutions of the energy errors corresponding to the items A--D in previous list. We confirm that for the time-step $h$ considered, truncation error is below round-error. The use of double precision coefficients ($\tilde{b}_i, \tilde{\mu}_{ij}$) is do not effect in the propagation of round-off. The iteration error is very similar to the round-off error and it could be cause an energy drift. 

\begin{figure}[h]
\centering
\begin{tabular}{c c}
\subfloat[NCDP ($h=2^{-7}$)]
{\includegraphics[width=.400\textwidth]{NCDP1A}}
&
\subfloat[NCDP ($h=2^{-7}/2$)]
{\includegraphics[width=.400\textwidth]{NCDP1B}} \\
% Zerbitzaria/1-Artikulua-Esperimentuak-4B/12-Esperimentua-V11/Experiments
\subfloat[CDP ($h=2^{-7}$)]
{\includegraphics[width=.400\textwidth]{CDP1A}}
&
\subfloat[CDP ($h=2^{-7}/2$)]
{\includegraphics[width=.400\textwidth]{CDP1B}} \\
% Zerbitzaria/1-Artikulua-Esperimentuak-4B/13-Esperimentua-V12/Experiments
\subfloat[6-body ($h=500/3$)]
{\includegraphics[width=.400\textwidth]{NBODY1A}}
&
\subfloat[6-body ($h=500/6$)]
{\includegraphics[width=.400\textwidth]{NBODY1B}} \\
%Zerbitzaria/1-Artikulua-Esperimentuak-4B/11-Esperimentua-V102/Esperiments
\subfloat[10-body ($h=2$)]
{\includegraphics[width=.400\textwidth]{10NBODY1A}}
&
\subfloat[10-body ($h=1$)]
{\includegraphics[width=.400\textwidth]{10NBODY1B}}
%Zerbitzaria/1-Artikulua-Esperimentuak-4B/11-Esperimentua-V102/Esperiments
\end{tabular}
\caption{\small We plot the evolution of energy relative error in logarithmic scale of the next algorithms implementations: A-algorithm  as estimation of truncation error (red), B-algorithm  as estimation of iteration error (green), C-algorithm  as estimation of the effect of the computation of double precision $\tilde{f}$ function (black) and D-algorithm  as estimation of the effect of using double precision $\tilde{b}_i, \tilde{\mu}_{ij} \in \mathbb{F}$ coefficients (blue). We show an figure for each initial value problem: Double Pendulum Non-Chaotic case (a), Double Pendulum Chaotic case (b) and outer solar system case (e).}    
\label{fig:SourceError}
\end{figure}

\subsection{Errore azterketa estadistikoa.}

Errorearen konparaketa fidagarriagoa egiteko helburuarekin,  azterketa estadistikoa egin dugu. Ausaz perturbatutako $P=1000$ hasierako balio ezberdinetarako integrazioak exekutatu ditugu eta emaitza hauen guztien batezbestekoan oinarritu gara, biribiltze errorearen azterketa egokia egiteko.    

\paragraph*{Ausazko perturbazioak.}
Perturbaziorik gabeko hasierako balioa $(u0,e0)$ eta $\mathcal{O}(10^{-6})$ ordeneko $k$ perturbazio tamaina finkatuta, era honetan kalkulatuko dugu $(u1D,e1)$ perturbatutako hasierako balioa,
\begin{lstlisting} [language=Mathematica]
k = 2^(-20);
u1 = u0 + u0*(k*RandomReal[{-1,1}]);
u1D = N[u1];
u1DD = SetPrecision[u1D, prec];
e1 = N[u1 - u1DD];
\end{lstlisting}

\paragraph*{}Puntu-finko interazio bidezko $6$-ataletako Gauss kolokazio metodoaren hiru inplementazio konparatu ditugu:

\begin{enumerate}
\item Inplementazio ideala. Ekuazio diferentzialaren eskuin aldeko funtzioaren ebaluazioa izan ezik, beste eragiketa guztiak doitasun laukoitzean egiten dituen inplementazioa da.

\item Doitasun bikoitzeko gure inplementazioa berria (DP).

\item Hairer-en inplementazioa ~\cite{Hairer2008}, jatorrizko bere IRK metodoaren Fortran kodea exekutatu dugu (\href{http://www.unige.ch/~hairer/preprints.html}{Hairer's preprints page}).     

\end{enumerate}

\begin{table}
\caption[Fixed-point percentage of steps and mean iterations.] 
{\small{Percentage of steps that reach a computational fixed-point and the number of fixed-point iterations per step for the computations of non-chaotic double pendulum (NCDP), chaotic double pendulum (CDP), and the outer solar system (OSS) problems. In columns, we compare three different implementations: FPIEA (ideal), DP (double precision) and Hairer's Fortran code.}}
\label{tab:fp}       % Give a unique label
\centering
%\resizebox{\textwidth}{!}
{%
\begin{tabular}{ l c c c c c c } 
 \hline
                 &  \multicolumn{2}{c}{FPIEA}  & \multicolumn{2}{c}{DP} & \multicolumn{2}{c}{Hairer} \\
 \hline
 NCDP            & $98.7\%$    & $9.5$   & $98.8\%$     & $8.6$   &  $98.5\%$ & $8.6$  \\ 
 CDP             & $98.9\%$    & $9.5$   & $98.9\%$     & $8.6$   &  $98.4\%$ & $8.6$  \\ 
 OSS             & $97.4\%$    & $15.2$  & $97.4\%$     & $14.2$  &  $87.5\%$ & $14.1$ \\ 
   \hline
 \end{tabular}}
 \end{table}


\subsubsection*{Distribution of energy jumps.}

Integratzailearen inplementazio egoki batean, biribiltze erroreak eragindako energiaren errore lokala $H(y_n)-H(y_{n-1}$), ausazkoa izatea espero da. Zenbakizko soluzioa $m$ urratsero jasotzen dugula jakinik, energi diferentzia $H(y_{km})-H(y_{km-m})$ ausazkoa izango da, $\mu$ batazbestekoa ($\mu=0$ ideala) eta $\sigma$ duen distribuzio Gausiarrarekin. Beraz, metatutako energia diferentzia
\begin{equation*}
H(y_{km})-H(y_0),
\end{equation*} 
$t_{mk}=t_0+kmh$ uneetarako, $k^{1/2} \sigma=(t_{mk}/(mh))^{1/2} \sigma$ desbiazio estandarra duen ausazko ibilbide Gausiar bat (\emph{random walk}) jarraituko du. Brouwer-ek Kepler problemarentzat zenbakizko integrazioaren birbiltze errorearen azterketa egin zuen ~\cite{Brouwer1937} eta horregatik, aurretik aipatutakoa Brouwer legea bezala ezaguna da konputazio zientzian ~\cite{Grazier2005}.

Integrazio tartea $[t_0, t_{end}]$ bada eta $P$ perturbatutako hasierako balio kopurua bada, $L \ P$ energia diferentzia balioak ditugu,
\begin{equation*}
\frac{H(y_{km})-H(y_{km-m})}{H(y_0)},
\end{equation*} 
non $L=(t_{end}-t_0)/(mh)$ den. (\ref{fig:hist}) irudian, $L \ P$ balioen histogramak eta $N(\mu,\sigma)$ distribuzio normalak irudikatu ditugu (non $\mu$ eta $\sigma$, balioei dagokien batazbestekoa eta desbiazio tipikoak diren). Gure helburua , gure DP inplementazioari dagozkion histogramak distribuzio Gausiarra ondo egokitzen zaion ikustea da.    

\begin{figure}[h!]
\centering
\begin{tabular}{c c}
\subfloat[\small {NCDP : FPIEA.}]
{\includegraphics[width=.4 \textwidth]{NCDP2A}}
&
\subfloat[\small {NCDP : DP.}]
{\includegraphics[width=.4\textwidth]{NCDP2B}}
\\
$(\mu=1.8\mathbf{e}{-18}$, \ $\sigma=1.5\mathbf{e}{-17}$) & ($\mu=5.3\mathbf{e}{-19}$, \ $\sigma=1.5\mathbf{e}{-17}$)\\
% Zerbitzaria/1-Artikulua-Esperimentuak-4B/13-Esperimentua-V122/Experiments/NCDP-ZAHAR
%
\subfloat[CDP : FPIEA.]
{\includegraphics[width=.4\textwidth]{CDP2A}}
&
\subfloat[CDP : DP.]
{\includegraphics[width=.4\textwidth]{CDP2B}}
%
\\
$(\mu=1.6\mathbf{e}{-19}$, \ $\sigma=2.6\mathbf{e}{-17}$) & ($\mu=1.4\mathbf{e}{-19}$, \ $\sigma=2.6\mathbf{e}{-17}$)\\
% Zerbitzaria/1-Artikulua-Esperimentuak-4B/13-Esperimentua-V12/Experiments
%
\subfloat[6-Body : FPIEA.]
{\includegraphics[width=.4\textwidth]{NBODY2A}}
&
\subfloat[6-Body : DP.]
{\includegraphics[width=.4\textwidth]{NBODY2B}}
\\
$(\mu={-3.2}\mathbf{e}{-19}$, \ $\sigma=3.3\mathbf{e}{-18}$) & ($\mu=-1.9\mathbf{e}{-19}$, \ $\sigma=3.5\mathbf{e}{-18}$)\\
% Zerbitzaria/1-Artikulua-Esperimentuak-4B/13-Esperimentua-V122/Experiments/6-Body (approx=1)
%
\subfloat[10-Body : FPIEA.]
{\includegraphics[width=.4\textwidth]{NBODY2A}}
&
\subfloat[10-Body : DP.]
{\includegraphics[width=.4\textwidth]{NBODY2B}}
\\
$(\mu={-3.2}\mathbf{e}{-19}$, \ $\sigma=3.3\mathbf{e}{-18}$) & ($\mu=-1.9\mathbf{e}{-19}$, \ $\sigma=3.5\mathbf{e}{-18}$)
% Zerbitzaria/1-Artikulua-Esperimentuak-4B/13-Esperimentua-V122/Experiments/6-Body (approx=1)
%
\end{tabular}
\caption{ \small Histograms of $K P$ samples of energy jumps against the normal distribution $N(\mu, \delta)$. The horizontal axis is multiplied by $10^{15}$ and vertical axis ???. for the FPIEA and  DP implementations for the three considered problems (NCDP, CDP, and OSS).}
\label{fig:hist}
\end{figure}

\subsubsection*{Evolution of mean and standard deviation of errors}


(\ref{fig:Htt}) irudian, energiaren errorearen batazbestekoa eta desbiazio tipikoa irudikatu ditugu.
\begin{align*}
&\Delta E_i^{[k]} =\frac{(E^{[k]}_i-E^{[k]}_0)}{E^{[k]}_0}, \ \ \ i=1,\dots,N/m \ \text{eta} \ k=1,\dots,P.\\
&\bar{\Delta E_i} =\frac{1}{P} \sum_{k=1}^{P} \Delta E_i^{[k]}, \ \  \sigma_i =\sqrt{\frac{1}{P} \sum_{k=1}^{P} (\Delta E_i^{[k]}-\bar{\Delta E_i})^2}.
\end{align*} 

eta kokapen erroreak: doitasun laukoitzean lortutako soluzioa, soluzio zehatza kontsideratuko dugu,           
\begin{equation*}
yexact^{[k]}_i=\hat{y}^{[k]}_i=(\hat{q}^{[k]}_i+\hat{eq}^{[k]}_i,\hat{p}^{[k]}_i+\hat{ep}^{[k]}_i),\ \ i=1,\dots,N/m .
\end{equation*}

eta $k.$ soluzioari ${y}^{[k]}_i=({q}^{[k]}_i+{eq}^{[k]}_i,{p}^{[k]}_i+{ep}^{[k]}_i)$ dagokion kokapen errorea,          
\begin{align*}
&Ge^{[k]}_i =\|(\hat{q}^{[k]}_i+\hat{eq}^{[k]}_i)-(q^{[k]}_i+eq^{[k]}_i)\|_2.
\end{align*}  

\begin{figure}[h!]
\centering
\begin{tabular}{c c}
\subfloat[NCDP: energy error]
{\includegraphics[width=.4\textwidth]{NCDP3A}}
&
\subfloat[NCDP: mean global error.]
{\includegraphics[width=.4\textwidth]{NCDP4A}}
\\
% Zerbitzaria/1-Artikulua-Esperimentuak-4B/13-Esperimentua-V122/Experiments/NCDP-ZAHAR
%
\subfloat[CDP: energy error.]
{\includegraphics[width=.4\textwidth]{CDP3A}}
&
\subfloat[CDP: mean global error.]
{\includegraphics[width=.4\textwidth]{CDP4A}}
% Zerbitzaria/1-Artikulua-Esperimentuak-4B/13-Esperimentua-V122/Experiments
%
\\
\subfloat[6-body: energy error.]
{\includegraphics[width=.4\textwidth]{NBODY3A}}
&
\subfloat[6-body: mean global error.]
{\includegraphics[width=.4\textwidth]{NBODY4A}}
% Zerbitzaria/1-Artikulua-Esperimentuak-4B/13-Esperimentua-V122/Experiments/6-Body (approx=1)
%
\\
\subfloat[10-body: energy error.]
{\includegraphics[width=.4\textwidth]{NBODY3A}}
&
\subfloat[10-body: mean global error.]
{\includegraphics[width=.4\textwidth]{NBODY4A}}
\end{tabular}
\caption{\small Evolution of mean ($\mu$) and standard deviation ($\sigma$) of errors in energy (left) and evolution of Mean of global errors in positions (right), for DP implementation (blue),  FPIEA implementation (orange), and  Hairer's implementation (yellow).  Non-Chaotic case (a,b), Chaotic case (c,d) and outer solar system case (e, f). }
\label{fig:Htt}
\end{figure}

\subsection{Round-off estimation}

\begin{figure}[h]
\centering
\begin{tabular}{c c}
\subfloat[NCDP: original initial values]
{\includegraphics[width=.4\textwidth]{NCDP5A}}
&
\subfloat[NCDP: $P=1000$ perturbed initial values]
{\includegraphics[width=.4\textwidth]{NCDP5B}}
% Zerbitzaria/1-Artikulua-Esperimentuak-4B/13-Esperimentua-V122/Experiments/NCDP-ZAHAR
%
\\
\subfloat[CDP: original initial values]
{\includegraphics[width=.4\textwidth]{CDP5A}}
&
\subfloat[CDP: $P=1000$ perturbed initial values]
{\includegraphics[width=.4\textwidth]{CDP5B}}
% Zerbitzaria/1-Artikulua-Esperimentuak-4B/13-Esperimentua-V122/Experiments
%
\\
\subfloat[6-Body: original initial values]
{\includegraphics[width=.4\textwidth]{NBODY5A}}
&
\subfloat[6-Body: $P=1000$ perturbed initial values]
{\includegraphics[width=.4\textwidth]{NBODY5B}}
% Zerbitzaria/1-Artikulua-Esperimentuak-4B/13-Esperimentua-V122/Experiments/6-Body (approx=1)
%
\\
\subfloat[10-Body: original initial values]
{\includegraphics[width=.4\textwidth]{NBODY5A}}
&
\subfloat[10-Body: $P=1000$ perturbed initial values]
{\includegraphics[width=.4\textwidth]{NBODY5B}}
\end{tabular}
\caption{\small Left plots: estimation of the round-off error with the original unperturbed initial values. We compare the evolution of our error estimation (orange) with the evolution of the global error (blue). Right plots:evolution of the mean error in positions (blue) of the application of our DP algorithm to $P=1000$ perturbed initial value problems, together with the evolution of the mean of the estimated errors in positions (orange).}
\label{fig:plot5}
\end{figure}


\clearpage
\section{zaharra}

\subsection{Integrazio motak.}

Erabiltzaileak ekuazio diferentzialaren eskuin aldeko funtzioa, doitasun bikoitzean definituko duela suposatu dugu eta gure inplementazioaren biribiltze errorearen propagazioa idealetik gertu dagoela erakutsi nahi dugu. Gure inplementazioaren zenbakizko soluzioaren $\tilde{y}_n+e_n \approxeq y(t_n) \ (n=1,2,\cdots)$ errorea, jatorri ezberdineko erroreen konbinazioa da. 

\begin{enumerate}

\item Integrazio zehatza.

Konputazio guztia (ekuazio diferentzialaren funtzioaren ebaluazioa barne) doitasun laukoitzeko ($128$-bit) koma higikorreko aritmetikan  egindako integrazioa. Zenbakizko integrazio hau, trunkatze errorea estimatzeko eta errore globala kalkulatzeko erreferentzi gisa (soluzio zehatza) kontsideratuko dugu. 

\item Integrazio superideala.

Zenbakizko integrazio hau iterazio errorea estimatzeko erabiliko dugu. Konputazio guztia doitasun laukoitzean egindako integrazioa baina geratze erizpidea, doitasun bikoitzean neurtuko dugu,
\begin{equation*}
\Delta^{[k]}=|\text{double}(Y^{[k]})-\text{double}(Y^{[k-1]})|.
\end{equation*}

\item Integrazio ideala.

Ekuazio diferentzialaren eskuin aldeko funtzioaren ebaluazioa izan ezik, beste eragiketa guztiak doitasun laukoitzean egiten dituen inplementazioa da. Ekuazio diferentziala doitasun bikoitzean kalkulatzeak, eragiten duen errorea neurtzeko erabiliko dugu eta integrazio hau hobetu ezin daitekeen integrazioa kontsideratuko dugu.   

\item Integrazio sasi-ideala.

Doitasun bikoitzeko koefizienteak ($\tilde{\mu}_ij,\tilde{b}_i$) erabiltzeak eragiten duen errorea neurtzeko integrazioa da. Konputazio guztia doitasun laukoitzean kalkulatzen da baina doitasun bikoitzeko koefizienteen balioak erabiliz (doitasun bikoitzeko koefiziente koadrifikatuak esaten diogu).  
   
\item Double-berria (batura konpensatu hobetua).

Guk proposatutako inplementazioa da. Doitasun bikoitzeko  koma higikorreko aritmetika ($64$-bit) eta gure proposamenak jarraituz inplementatutako \emph{batura konpensatua hobetua} erabiltzen duen inplementazioa, hau da, atalen eguneraketan, ($\tilde{y}_n \oplus \tilde{e}_n$) (ekuazioa \ref{eq:eqbk}) espresioa erabiltzen duen integrazioa.

\item Double-estandarra (batura konpensatu estandarra).

Doitasun bikoitzeko koma higikorreko aritmetika ($64$-bit) eta \emph{batura konpensatu estandarra} erabiltzen duen inplementazioa (atalen eguneraketan $\tilde{e}_n$ gaia kontsideratzen ez duen integrazioa). 

The implementation given in~\cite{Hairer2008}, we will refer to it as Hairer's implementation. We have run their original implicit Runge-Kutta Fortran code  (\href{http://www.unige.ch/~hairer/preprints.html}{Hairer's preprints page}).

\end{enumerate}

Integrazio bakarra egin ordez, ausaz perturbatutako $P=1000$ hasierako balio ezberdinetarako integrazioak exekutatu ditugu eta emaitza hauen guztien batezbestekoan oinarritu gara, biribiltze errorearen azterketa egokia egiteko.    

\subsubsection*{Ausazko perturbazioak.}

Perturbaziorik gabeko hasierako balioa $(u0,e0)$ eta $\mathcal{O}(10^{-6})$ ordeneko $k$ perturbazio tamaina finkatuta, era honetan kalkulatuko dugu $(u1D,e1)$ perturbatutako hasierako balioa,
\begin{lstlisting} [language=Mathematica]
k = 2^(-20);
u1 = u0 + u0*(k*RandomReal[{-1,1}]);
u1D = N[u1];
u1DD = SetPrecision[u1D, prec];
e1 = u1 - u1DD;
\end{lstlisting}

\subsubsection*{Neurtzeko faktoreak.}

Sistema Hamiltondarretan energia kontserbatzen da,
\begin{equation*}
E_i^{[k]}=H(q_i^{[k]}+eq_i^{[k]},p_i^{[k]}+ep_i^{[k]})=konst, \ \ \ i=1,\dots,N/m.
\end{equation*}

\paragraph*{}Energia eta kokapen erroreak honako faktoreen bidez neurtuko ditugu.

\begin{enumerate}

\item Energia errorea.

\begin{align*}
&\Delta E_i^{[k]} =\frac{(E^{[k]}_i-E^{[k]}_0)}{E^{[k]}_0}, \ \ \ i=1,\dots,N/m \ \text{eta} \ k=1,\dots,P.\\
&\bar{\Delta E_i} =\frac{1}{P} \sum_{k=1}^{P} \Delta E_i^{[k]}, \ \  \sigma_i =\sqrt{\frac{1}{P} \sum_{k=1}^{P} (\Delta E_i^{[k]}-\bar{\Delta E_i})^2},\\
&\bar{MaxE } =\max_{i=1,\dots,N} |\bar{\Delta E_i}|.
\end{align*}  

\item Energia diferentzia.

$P$ integrazio baditugu eta $L=N/m$, integrazio bakoitzean lortutako emaitza kopurua bada, $L\cdot P$ balio guztien energiaren diferentzien batezbestekoa ($\bar{\mu}$) eta desbideratze estandarra ($\bar{\sigma}$) kalkulatuko ditugu.            
\begin{align*}
& Dif_i^{[k]} =\frac{(E^{[k]}_i-E^{[k]}_{i-1})}{E^{[k]}_0},\ \ \ i=1,\dots,N/m \ \text{eta} \ k=1,\dots,P, \\
& \bar{\mu} = \frac{1}{L\cdot P} \bigg(\sum_{k=1}^{P} \sum_{i=1}^{L} {Dif_i^{[k]}\bigg)}, \ \
 \bar{\sigma} = \sqrt{\frac{1}{L\cdot P} \bigg(\sum_{k=1}^{P} \sum_{i=1}^{L} {(Dif_i^{[k]}-\bar{\mu)}^2}\bigg)}          
\end{align*}
           
\item Kokapen errore globala.

Doitasun laukoitzean lortutako soluzioa, soluzio zehatza kontsideratuko dugu,           
\begin{equation*}
yexact^{[k]}_i=\hat{y}^{[k]}_i=(\hat{q}^{[k]}_i+\hat{eq}^{[k]}_i,\hat{p}^{[k]}_i+\hat{ep}^{[k]}_i),\ \ i=1,\dots,N/m .
\end{equation*}

eta $k.$ soluzioari ${y}^{[k]}_i=({q}^{[k]}_i+{eq}^{[k]}_i,{p}^{[k]}_i+{ep}^{[k]}_i)$ dagokion kokapen errorea,          
\begin{align*}
&Ge^{[k]}_i =\|(\hat{q}^{[k]}_i+\hat{eq}^{[k]}_i)-(q^{[k]}_i+eq^{[k]}_i)\|_2\\
&\bar{Ge_i} = \left(\frac{1}{P}\sum_{k=1}^{P} Ge^{[k]}_i \right) \ , \
                          \bar{MaxGe}=\max_{i=1,\dots,N/m} (\bar{Ge_i})
\end{align*}     
           
\item Puntu-finkorainoko konbergentzia portzentaien batezbesteko  ($\bar{\Lambda}$).
           
$\Lambda^{[k]}$  $k.$ integrazioan, puntu-finkorainoko konbergentzia lortutako urratsen ehuneko portzentaia izanik,           
\begin{equation*}
\bar{\Lambda}= \frac{1}{p}\sum_{k=1}^{P}\Lambda^{[k]}.
\end{equation*}
 
\item Kokapen errore estimazioa ($\bar{\mu Q_i}$ , $\bar{\sigma Q_i}$). 
            
Biribiltze errorearen estimazioaren gogoratuz, 
\begin{equation}
Est_i^{[k]}=\|(q_i^{[k]}+eq_{i}^{[k]})-(\tilde{\tilde{q}}_i^{[k]}+\tilde{\tilde{eq}}_{i}^{[k]})\|_2
\end{equation}
            
Errore estimazioaren kalitatea neurtzeko,
\begin{align} \label{eq:eq_Qi}
Q_i^{[k]} &=\log_{10} \bigg(\frac{Est^{[k]}_i}{Ge^{[k]}_i}\bigg),\\
\bar{\mu Q_i} &={\frac{1}{P}\sum_{k=1}^{p} Q_i^{[k]}} \ , \ \ 
 \sigma Q_i={\sqrt{\frac{1}{P}\sum_{k=1}^{P} (Q_i^{[k]}-\bar{\mu Q_i})^2}}
\end{align}

\end{enumerate} 


\subsection*{Emaitza nagusiak.}

Esperimentuen emaitzak (\ref{tab:tabDPA}) taula orokorrean eta (\ref{fig:plotDPA}) irudian adierazi ditugu. 

\begin{enumerate}
\item (\ref{tab:tabDPA}) taula orokorra.\\
Taula honetan esperimentuen ikuspegi orokorrena laburtu dugu. Emaitza hauetan, gure inplementazioaren (\emph{Double-berria}) eta \emph{integrazio idealaren} artean diferentzia oso txikia dela baieztatu daiteke. Batetik,  bi integrazioetan energia errorearen eta kokapen errorearen maximoek ia balio berdina hartzen dute. Bestalde, puntu-finkorainoko konbergentzia batezbesteko portzentaia altuak,  urratsa gehienetan puntu-finkoaren iterazioaren bidez lortu daitekeen emaitza hoberena lortzen ari garela adierazten digu.    

\item (\ref{fig:plotDPA}) irudia.
\begin{enumerate}
\item \textbf{(a) irudia: energiaren eboluzioa.}
Irudi honetan \emph{integrazio idealaren}, \emph{Double-berria} eta \emph{Double-estandarra} integrazioak konparatu ditugu. Batetik, espero genuen moduan \emph{Double-berria} integrazioak, \emph{Double-estandarra} integrazioak baino emaitza hobea du. Bigarrenik, \emph{Double-berria} integrazioren eta \emph{integrazio idealaren} energiaren eboluzioak antzekoak dira.

Integrazio mota guztien energiaren batezbestekoaren eboluzioan, drift-a nabarmentzen da. Hurrengo atalean (\ref{Drift}) drift honen jatorria aztertuko dugu.  

\item \textbf{(b) irudia: kokapen errorearen eboluzioa.}
Integrazio tartean zehar, \emph{Double-berri} integrazioaren eta \emph{integrazio idealaren} kokapen errorea parekoak dira (irudian ia ez dira bereizten). Energia errorearekin gertatzen zaigun moduan, \emph{Double-berri} integrazioan kokapen errorea pixka bat handiagoa da. 

\item \textbf{(c) eta (d) irudiak: energia histogramak.}
Energia errorearen ausazkotasuna aztertu dugu. Esperimentu hauetan, $P=1.000$ perturbatutako hasierako balio ezberdinen kopurua bada eta $Nout=512$ integrazio bakoitzean itzulitako diskretizazio kopurua bada, integrazio hauen guztien $521.000$ energia diferentzien laginen histogramak irudikatu ditugu, lagin hauen batezbesteko eta desbideratze estandarreko banaketa normalarekin konparatuz. Bai (\textbf{(c) irudia}) \emph{integrazio idealaren}, bai (\textbf{(d) irudia}) \emph{Double-berri} integrazioaren energia diferentzien laginak bat datoz banaketa normalarekin. 

\item \textbf{(e) eta (d) irudiak: errorearen estimazioa.}     
\textbf{(e) irudian} errorearen estimazioa kokapen errorearekin konparatu dugu: gure estimazioa kokapen errorearen gainetik mantentzen da integrazioan zehar. Estimazioaren kalitatea neurtu dugu \textbf{(d) irudian}:  $0$ eta $2.1$ tartean mantentzen da, eta beraz gehienez estimazioa, kokapen errorea baino bi aldiz handiago da. 

\end{enumerate}

\end{enumerate}
 



\subsubsection*{Energia Drift-a.}
\label{Drift}

(\ref{fig:plotDPA} a) irudian energia batezbestekoak zero izan behar luke integrazio osoan baina drift-a azaldu zaigu. Lehenik, gure inplementazioaren arazoren batek drift hori ez duela sortzen baieztatu behar dugu. Horretarako bi esperimentu berri egin ditugu:

\begin{enumerate}
\item \emph{Integrazio superideala.}
Horretarako, integrazio mota berri bat exekutatu dugu \emph{superideala} deitu duguna: exekuzio osoa doitasun laukoitzean baina geratze irizpidea \emph{double moduan} aplikatua,
\begin{equation*}
\Delta =\|Double(Y^{[k]})-Double(Y^{[k-1]})\|.
\end{equation*}
\emph{Integrazio superidealaren} emaitzak (\ref{fig:plotDrift}(a) irudia) ikus daitezke eta bertan, \emph{superideala integrazioak} ere energia drift-a azaltzen duela ikusi daiteke.

\item Hairer-en inplementazioaren azterketa. Hairer-ek bere lanean \cite{Hairer2008} bi problema (Hénon-Helies eta kanpo eguzki-sistema) aztertu zituen eta energia dirft-aren arazoa konpondutzat eman zuen. Guk Hairer-en kodearekin Pendulu bikoitzaren problema integratu dugu eta bere inplementazioan ere, energiaren drift-a badagoela baieztatu dugu. Hala ere,  \ref{fig:plotDrift}(b) irudian energia drift-a gure inplementazioa baino txikiago duela adierazi behar dugu.

\end{enumerate}


\begin{figure}[!h]
\centering
\subfloat[Energi drift: gure inplementazioa]{
\includegraphics[width=.500\textwidth]{plotSuperIdeala}
}
\subfloat[Energi drift: Hairer-en inplementazioa]{
\includegraphics[width=.500\textwidth]{plotHairer}
}
\caption[Energi drift-a (I).]
        {\small Energia Drift-a (I) (Pendulu-bikoitza ez-kaotikoa).
        
         \textbf{(a) irudian}, gure inplementazioaren emaitzak ikusi daitezke: \emph{superidealaren} integrazio bakarraren energia errore erlatiboa  eta \emph{Double-Berri} bertsioaren $P=1.000$ integrazioen energia batezbestekoa eta desbideratze estandarra. 
%
%        05-09-2016: Esperimentua errepasatu.
%          SuperIdeala berriz exekutatu ?         
%          (/C-inplementazioa/Code-bertsioak/41-Bertsioa/Experiments/DoublePendulum(NonChaotic)V3).
%          (/C-inplementazioa/Code-bertsioak/41-Bertsioa/Experiments/DoublePendulum(NonChaotic)V9..Tesirako irudiak).
                 
         \textbf{(b) irudian}, Hairer-en inplementazioaren $p=1000$ integrazioen energia batezbestekoa eta desbideratze estandarra , gure inplementazioaren \emph{superidealaren} integrazio bakarraren energia errore erlatiboarekin konparatu ditugu.
         
%        /4-Hairer IRK (Fortran)/1-Hairer-IRK(Double Pendulum)/Experiment/1-Double PendulumV9 ... Tesi irudiak
%        
}
\label{fig:plotDrift}
\end{figure}        


Bigarrenik, energiaren drift-aren arrazoia bilatu nahi dugu. Gure ustez, energiaren drift-aren jatorria, puntu-finkoaren iterazioen erroreak eragindakoa da: puntu-finkoaren iterazioen konbergentzia $O(h)$ da. Zentzu honetan bi esperimentu berri egin ditugu: 

\begin{enumerate}
\item Newton sinplifikatua. 
Newton sinplifikatuaren iterazioen konbergentzia $O(h^2)$ da. Newton sinplifikatuan oinarritutako inplementazio berri batekin Pendulu bikoitzaren integrazioa errepikatu dugu eta energiaren drift-a ez dagoela baieztatu dugu (ikus \ref{fig:plotDrift2} (a) irudia). Beraz, puntu-finkoaren metodoak berezkoa duen errorea kontsidera daiteke.

\item Sasi-Newton.
 Hala ere, azken urratsa bezala, LU-deskonposaketa eragiketa eta jakobiarra  behar ez duen inplementazio eskaini nahi diogu erabiltzaileari. Esperimentalki frogatu dugu modu honetan ere, energiaren drift-a desagertu dela (ikus \ref{fig:plotDrift2} (b) irudia).
 
Jarraian, IRK puntu-finkoaren eta Newton Sinplifikatuaren arteko algoritmoa azalduko dugu. Newton metodo sinplifikatua, $k=1,2,\dots$  $L_i^{[k]}$ hurbilpenak kalkulatzeko algoritmoa hau bada,
\begin{enumerate}
\item 
$g_i^{[k]}=-L_i^{[k-1]}+h b_i f(t+c_ih,y_n+ \sum\limits_{j=1}^{s} \mu_{ij} L_{j}^{[k-1]}), \ \ i=1,\dots,s.$

\item Solve $\Delta L_i^{[k]}$,

$\Delta L_i^{[k]} - h b_i \sum_{j=1}^{s} \mu_{ij} \ J_j \ \Delta L_j^{[k]} = g_i^{[k]}  , \ \ i=1,\dots,s.$

\item $L_i^{[k]} = L_i^{[k-1]}+ \Delta L_i^{[k]}, \ \  i=1,\dots,s.$

\end{enumerate}

Modu merkean, $\Delta L_i^{[k]}$ era honetan hurbildu daiteke,  
\begin{align*}
\Delta L_i^{[k]}= \frac{h b_i}{\lambda} \bigg(f(Y_{i}^{[k]}+ \sum_{j=1}^{s} \lambda \mu_{ij} g_j^{[k]})-f(Y_{i}^{[k]}) \bigg)
\end{align*}
non $\lambda=2^{10}$.

\end{enumerate}

%\begin{align*}
%R_{n,i}^{[k]}&=hb_i \ f(Y_{n,i}^{[k]})-L_{n,i}^{[k]} \\
%\Delta L_{n,i}^{[k]}&=R_{n,i}^{[k]}+hb_i \ f'(Y_{n,i}^{[k]}) \ \sum\limits_{j=1}^{s} \mu_{ij} \Delta L_{n,i}^{[k]} \\
%L_{n,i}^{[k+1]}&=L_{n,i}+\Delta L_{n,i}^{[k]}
%\end{align*}

%\begin{align*}
%R_{n,i}^{[k]}&=hb_i \ f(Y_{n,i}^{[k]})-L_{n,i}^{[k]} \\
%\Delta L_{n,i}^{[k]}&=R_{n,i}^{[k]}+\frac{hb_i}{\lambda} \big(f(Y_{n,i}^{[k]}+\sum\limits_{j=1}^{s} \lambda \mu_{ij} \ %R_{n,i}^{[k]})-f(Y_{n,i}^{[k]}) \big) \\
%L_{n,i}^{[k+1]}&=L_{n,i}+\Delta L_{n,i}^{[k]}
%\end{align*}
%non $\lambda=2^{10}$.

\begin{figure}[!h]
\centering
\subfloat[Newton sinplfikatua]{
\includegraphics[width=.500\textwidth]{plotNewtonDP}
}
\subfloat[SasiNewton]{
\includegraphics[width=.500\textwidth]{SasiNewton}
}
\caption[Energi drift-a (II).]
        {\small Energia Drift-a (II) (Pendulu-bikoitza ez-kaotikoa).
        
        \textbf{(a) irudian } puntu-finkoaren iterazioaren ordez Newton sinplifikatua aplikatuz integrazioen energia batezbestekoa eta desbideratze estandarra.      
%        Kodea: /61-Newton Sinplifikatua/C-inplementazioa/Code-Bertsioak/2-Bertsioa/Experiments/NonChaoticDouble-3/
%                Exekuzioa: Zerbitzaria/1-Artikulu-Esp-12(Newton)/Experiments/NonChaoticDouble-2

        \textbf{(b) irudian }, puntu-finkoaren iterazioaren ordez SasiNewton aplikatuz integrazioen energia energiaren errorearen batezbestekoa eta desbideratze estandarra. Esperimentu honetan integrazio tarte txikiagoa exekutatu dugu ($t=2^9$).       
%        SasiNewton.Nire portatilean exekuzioa. /1-IRK Kodeak/61-Newton  Sinplifikatua/C-inplementazioa/Code-Bertsioak/3-Bertsioa (Beta)/Experiments/NonChaoticDouble-2   
}
\label{fig:plotDrift2}
\end{figure}   

    
\clearpage


\subsubsection*{Emaitza nagusiak.}

Esperimentuen emaitzak (\ref{tab:tabDPB}) taulan eta (\ref{fig:plotDPB}) irudian adierazi ditugu. Problema kaotikoa denez, kokapen errorea esponentzialki handitzen da eta horregatik integrazio tarte txikiagoa aukeratu da. Orokorrean, Pendulu bikoitza ez-kaotikoaren esperimentuen ezaugarriak mantentzen dira. Esan, esperimentu hauetan ez dela energia drift-arik azaltzen (integrazio tartea txikia delako) eta energia diferentzien histogramak ez datozela erabat dagokion banaketa normalarekin. 

\begin{table} [h]
\caption{Pendulu Bikoitza kaotikoaren laburpena.}
\label{tab:tabDPB}       % Give a unique label
\begin{tabular}{c|c c c c c} 
 Arithmetic   &  $\bar{\Lambda}$  &  $\bar{MaxE}$ & $\bar{\mu}$  & $\bar{\sigma}$   & $\bar{MaxGe}$  \\
                           &   \%            &       &          &            &         \\
 \hline
                         &                 &         &       &             \\
 Zehatza                 &   $94$          &  $2e10^{-19}$  & $4e10^{-22}$  & $9e10^{-21}$   &          \\	    
 Ideala                  &   $98$          &  $8e10^{-17}$  & $1e10^{-19}$  & $3e10^{-17}$   & $0.33$   \\
 Double-berria           &   $95$          &  $8e10^{-17}$  & $1e10^{-19}$  & $3e10^{-17}$   & $0.34$   \\
 Double-estandarra       &   $95$          &  $9e10^{-17}$  & $1e10^{-19}$  & $3e10^{-17}$   & $0.35$   \\
\end{tabular}
\end{table}

\begin{figure}[!h]
\centering
\subfloat[Energi errorea.]{
\includegraphics[width=.500\textwidth]{DPBplot1}
%plot3c-2
}
\subfloat[Kokapen errorea.]{
\includegraphics[width=.500\textwidth]{DPBplot3}
%plot3d
}
\vskip\baselineskip
\subfloat[Histograma ideala.]{
\includegraphics[width=.450\textwidth]{DPBplot4}
%brouwer4c
}
\subfloat[Histograma Double-berri.]{
\includegraphics[width=.450\textwidth]{DPBplot5}
%brouwer4d
}
\vskip\baselineskip
\subfloat[Errore estimazioa]{
\includegraphics[width=.500\textwidth]{DPBplot6}
%plot5c
}
\subfloat[Errore estimazioaren kalitatea]{
\includegraphics[width=.500\textwidth]{DPBplot7}
%plot5d 
}
\caption[Pendulu-bikoitza: kaotikoa.]
        {\small Pendulu-bikoitza kaotikoa.
        
         \textbf{(a) irudian},  energi errorearen batezbestekoaren ($\bar{\Delta E_i}$) eta  desbideratze estandarraren ($\sigma_i$) eboluzioa. \textbf{(b) irudian}, kokapen errorearen eboluzioa $\bar{Ge_i}$. Honako laburdurak erabili ditugu: B=Integrazio ideala, C=Double-Berria, D=Double-estandarra.
                
                 \textbf{(c) eta (d) iruditan} $P=1.000$ integrazio ezberdinen $521.000$ energi diferentzien laginen histogramak irudikatu ditugu, batezbesteko eta desbideratze estandar bereko banaketa normalarekin alderatuz.Ardatza horizontalak $10^{-16}$ unitatea adierazten du.
                
                 \textbf{(e) irudian} errorearen estimazio kokapen errorearekin konparatu dugu. \textbf{(f) irudian} estimazioaren kalitatea aztertu dugu: (\ref{eq:eq_Qi}) definizioaren arabera, estimazio-kalitatearen batezbestekoa $\bar{\mu Q_i}$  eta desbideratze estandarra $\bar{\sigma Q_i}$ irudikatu ditugu.        
        }
\label{fig:plotDPB}
\end{figure}  

\clearpage
\subsection{N-Body problema.}

Problema honetan erabilitako integrazio tartea eta urratsa luzera hauek izan dira,
\begin{align*}
& t_0=0, \ \ t_{end}=10^{6}, \\
& h=2, \\
& sampling=1000, step=1.\\
& P=1.000, \ N=500. 
\end{align*} 

(integrazioen iraupena=$xx$ egun).

Esperimentuen emaitzak (\ref{tab:tabNBODY}) taulan eta (\ref{fig:plotNBODY}) irudian adierazi ditugu. Azpimarratu nahi dugu N-Body probleman ez zaigula energia Drift-arik agertu. N-Body problema bigarren ordeneko ekuazio diferentzial arrunta da eta \emph{Gauss-Seidel} moduko iterazioa aplikatu dugu (konbergentzia azkarragoa).

\begin{table} [h]
\caption{N-Body problemaren laburpena.}
\label{tab:tabNBODY}       % Give a unique label
\begin{tabular}{c|c c c c c} 
 Arithmetic   &  $\bar{\Lambda}$  &  $\bar{MaxE}$ & $\bar{\mu}$  & $\bar{\sigma}$   & $\bar{MaxGe}$  \\
                           &   \%            &       &          &            &         \\
 \hline
                           &                 &         &       &           &          \\
 Zehatza                   &   $98.$        &  $2e10^{-19}$  & $5e10^{-24}$  & $2e10^{-20}$  &      \\	    
 Ideala                    &   $99.$        &  $6e10^{-18}$  & $6e10^{-21}$  & $3e10^{-19}$ &  $7e10^{-12}$\\
 Double-berria             &   $98.$       &  $5e10^{-18}$  & $4e10^{-21}$  & $3e10^{-19}$ &  $1e10^{-11}$\\
 Double-estandarra         &   $98.$       &  $5e10^{-18}$  & $4e10^{-21}$  & $3e10^{-19}$ &  $1e10^{-11}$\\
\end{tabular}
\end{table}


\begin{figure}[!h]
\centering
\subfloat[Energi errorea.]{
\includegraphics[width=.500\textwidth]
{NBODYplot1}
%{plot6a}
}
\subfloat[Kokapen errorea.]{
\includegraphics[width=.500\textwidth]
{NBODYplot3}
%{plot6b}
}
\vskip\baselineskip
\subfloat[Histograma ideala.]{
\includegraphics[width=.450\textwidth]
{NBODYplot4}
%{brouwer4e}
}
\subfloat[Histograma Double-berri.]{
\includegraphics[width=.450\textwidth]
{NBODYplot5}
%{brouwer4f}
}
\vskip\baselineskip
\subfloat[Errore estimazioa]{
\includegraphics[width=.500\textwidth]
{NBODYplot6}
%{plot7a}
}
\subfloat[Errore estimazioaren kalitatea]{
\includegraphics[width=.500\textwidth]
{NBODYplot7} 
%{plot7b}
}
\caption[N-Body.]
        {\small N-Body.
        
         \textbf{(a) irudian},  energi errorearen batezbestekoaren ($\bar{\Delta E_i}$) eta  desbideratze estandarraren ($\sigma_i$) eboluzioa. \textbf{(b) irudian}, kokapen errorearen eboluzioa $\bar{Ge_i}$. Honako laburdurak erabili ditugu: B=Integrazio ideala, C=Double-Berria, D=Double-estandarra.
                
                 \textbf{(c) eta (d) iruditan} $P=1.000$ integrazio ezberdinen $521.000$ energi diferentzien laginen histogramak irudikatu ditugu, batezbesteko eta desbideratze estandar bereko banaketa normalarekin alderatuz.Ardatza horizontalak $10^{-16}$ unitatea adierazten du.
                
                 \textbf{(e) irudian} errorearen estimazio kokapen errorearekin konparatu dugu. \textbf{(f) irudian} estimazioaren kalitatea aztertu dugu: (\ref{eq:eq_Qi}) definizioaren arabera, estimazio-kalitatearen batezbestekoa $\bar{\mu Q_i}$  eta desbideratze estandarra $\bar{\sigma Q_i}$ irudikatu ditugu.        
        }
\label{fig:plotNBODY}
\end{figure}      

\subsubsection*{Momentu angeluarra.}

N-Body probleman momentu angeluarra,
\begin{equation*}
O=\sum_{i=0}^{N} \mathbf{p_i} \times \mathbf{q_i}.
\end{equation*}
metodo sinplektikoek mantentzen duten inbariantea da. Hurrengo grafikoetan (irudia \ref{fig:plotNBODYMOM}) horrela baieztatu daiteke.

\begin{figure}[!h]
\centering
\subfloat[Momentu angeluarra(Lx).]{
\includegraphics[width=.500\textwidth]
{NBODYplot21}
}
\subfloat[Momentu angeluarra(Ly).]{
\includegraphics[width=.500\textwidth]
{NBODYplot22}
}
\vskip\baselineskip
\subfloat[Momentu angeluarra(Lz).]{
\includegraphics[width=.450\textwidth]
{NBODYplot23}
}
\caption[N-Body (Momentu angeluarra).]
        {\small N-Body (Momentu angeluarra).
        
         \textbf{(a) irudian},\textbf{(b) irudian} eta \textbf{(c) irudian}  momentu angeluarraren errorearen batezbestekoaren ($\bar{\Delta L_i}$) eta  desbideratze estandarraren ($\sigma_i$) eboluzioa.       
        }
\label{fig:plotNBODYMOM}
\end{figure}      

\subsection{Ondorioak.}

Lehen atal honetan, IRK metodoaren puntu-finkoko iterazioan oinarritutako inplementazioa hiru problema ez-stiff-etarako aztertu dugu. Gogoan eduki behar dugu, erabiltzaileak ekuazio diferentzialaren eskuin aldeko funtzioa doitasun bikoitzean definituko duela suposatu dugula. Ondorio hauek azpimarratuko ditugu:   

\begin{enumerate}

\item Hairer-en inplementazioan antzeman ditugun hainbat arazo konpondu ditugu, inplementazioaren hobekuntzak proposatuz eta inplementazio sendoagoa lortu dugula uste dugu. 

\item Doitasun laukoitzeko aritmetika ($128-bit$) erabiltzen duen inplementazioa (\emph{ideala}) ez duela merezi ikusi dugu. Doitasun bikoitzeko aritmetika ($64$ bit) eta biribiltze errorea gutxitzeko teknika zehatzak (batura konpensatua,\dots) erabiltzen dituen inplementazioarekin, \emph{integrazio idealaren} errore mailaren oso antzekoa lortu dugu. Ondorioz, doitasun laukoitzeko aritmetikarekin (\emph{ideala}) lortzen diren emaitzak, eskatzen duen CPU kostuak ez duela merezi esan dezakegu. 

\item Problema ez-stiff-tarako, puntu-finkoko iterazioa Newton sinplifikatua baino merkeagoa da eta beraz, iterazio metodo hau aplikatzea gomendatzen da \cite{Hairer2006}. Pendulu bikoitzaren (hasierako balio ez-kaotikoak) epe luzeko integrazioan, energia drift-aren arazoaz  ohartu gara. Puntu-finkoko iterazioaren errorea energi drift honen jatorria da. Puntu-finkoko iterazioaren konbergentzia $O(h)$ izanik, Newton sinplifikatua iterazioak $O(h^2)$ konbergentziarekin ez du energia drift-aren arazorik.   

\item Puntu-finkoak doitasun laukoitzean lan egiterakoan muga batzuk ditu. Esperimentua definitu ?? Atal askotako metodoak??

\item Aurreko bi arrazoi hauen ondorioz, problema ez-stiff-tarako ere, Newton sinplifikatua aplikatzea egokia izan daitekeela pentsa liteke. Newton sinplifikatua garestia da, iterazio bakoitzean \emph{LU-deskonposaketa} eta jakobiarra kalkulatu behar direlako.
Guzti hau kontutan hartuta, Newton iterazio metodoen inplementazio eraginkorrak, abantaila asko izan ditzakeela iruditzen zaigu.        

\end{enumerate} 


\section{Laburpena.}

\chapter{IRK: Newton.}

\epigraph{The top 10 algorithms in Applied Mathematics: 1.Newton and quasi-Newton methods.}
{\textit {Nick Higham (2016)}}

\section{Sarrera.}

Runge-Kutta metodo inplizituen inplementazio eraginkorren arazo handiena, ekuazio sistema ez-lineala metodo iteratibo baten bidez askatzea da. Problema zurruna denean, puntu finkoaren iterazio ez da eraginkorra eta Newton interazioa aplikatu behar da. Gainera problema ez-zurruna izanik ere, Newton iterazioak interesgarriak izan daitezke; bereziki doitasun altuko (doitasun laukoitza) konputazioetan iterazio metodoaren konbergentzi ezaugarri onak direla eta. 
 
Newton metodo iteratiboa, ekuazio sistema ez-linealen zenbakizko soluzioa aurkitzeko metodoa da. $u\in \mathbb{R}^{n}$ eta $F: \mathbb{R}^n \ \longrightarrow {\mathbb{R}}^n$ emanik, $F(u)=0$ betetzen duen $u^{[*]}$ soluzioa aurkitu nahi dugu. Hasierako soluzioaren $u^{[0]}$ estimazioa  emanik,  Newton metodoa era honetan definituko dugu (Algoritmoa \ref{alg:801}).

\begin{algorithm}[H]
  $ \text{Hasieratu} \ u^{[0]}$\;
  \For{ (k=1,2,\dots \text{konbergentzia lortu arte})}
  {
   \BlankLine
   $F^{[k]}=F(u^{[k-1]})$\;
   $\text{Askatu} \ \ J(u^{[k-1]}) \ \Delta u^{[k]}=- F^{[k]}$\;
   \BlankLine
   $u^{[k]}=u^{[k-1]}+\Delta u^{[k]}$\;
  }
 \caption{Newton metodoa.}
 \label{alg:801}
\end{algorithm}

non $J(u^{[k-1]})$, \ $n \times n$ tamainako matrize jacobiarra den ($J_{ij}(u)=\partial f_i/\partial u_j (u), \ \ 1 \leq i,j \leq n$).  Newton metodoak, $u^{[*]}$ soluzioarekiko konbergentzia koadratikoa du,
\begin{equation*}
\label{eq:801}
\|u^{[k+1]}-u^{[*]}\| \le C \ \|u^{[k]}-u^{[*]}\|^2.
\end{equation*}

Newton metodo interesgarria da, baina konputazionalki garestia; iterazio bakoitzean $J(u^{[k-1]})$ jacobiarraren ebaluazioa eta $n \times n$ tamainako matrizearen \emph{LU} deskonposaketa kalkulatu behar da. Horregatik, konputazionalki merkeagoak diren Newton metodoaren aldaerak erabili ohi dira. Newton metodoaren aldaera horien artean, Newton sinplifikatua (Algoritmoa \ref{alg:802}) dugu aukera nagusienetako bat.

\begin{algorithm}[H]
  $ \text{Hasieratu} \ \ u^{[0]}   \quad \quad \quad \quad \quad \quad \quad \quad\quad \quad \quad \quad \quad \quad \quad \quad \quad    (64-bit)$\;
  $J^{[0]}=\frac{\partial F}{\partial u}(u^{[0]})  \ \ \quad \quad \quad \quad \quad \quad \quad \quad\quad \quad \quad \quad \quad \quad \quad \quad \quad    (32-bit)$\;
  $M=LU(J^{[0]}) \ \ \quad \quad \quad \quad \quad \quad \quad \quad\quad \quad \quad \quad \quad \quad \quad \quad \quad    (32-bit)$\;
  \For{ (k=1,2,\dots \text{konbergentzia lortu arte})}
  {
   \BlankLine
   $F^{[k]}=F(u^{[k-1]}) \ \ \quad \ \quad \quad \quad \quad \quad \quad \quad \quad \quad \quad \quad \quad \ \   (64-bit)$\;
   $\text{Askatu} \ \ M \ \Delta u^{[k]}=- F^{[k]} \ \quad \ \quad \quad \quad \quad \quad \quad \quad \quad \ \ \  (32-bit)$\;
   \BlankLine
   $u^{[k]}=u^{[k-1]}+\Delta u^{[k]}  \ \ \ \quad \quad \quad \quad\quad \quad \quad \quad \quad \quad \quad \ \     (64-bit)$\;
  }
 \caption{Newton sinplifikatua.}
 \label{alg:802}
\end{algorithm}

Argia da Newton sinplifikatua konputazionalki merkeagoa dela: urrats bakoitzean, jacobiarraren ebaluazio $J^{[0]}=\frac{\partial F}{\partial u}(u^{[0]})$ eta dagokion \emph{LU} deskonposaketa behin bakarrik kalkulatu behar dira. 

Azpimarratzekoa da ere, Newton metodoaren eragiketa konplexuenak doitasun txikiagoan kalkula daitezkeela \cite{Baboulin20092526} eta honek, konputazionalki abantaila interesgarria suposatzen duela. Algoritmoaren (Algoritmoa \ref{alg:802}) eskubi aldean, doitasun bikoitzeko ($64$-bit) inplementazioa balitz, eragiketa bakoitzaren doitasuna zehaztu dugu: jacobiarraren balioztapena eta algebra linealeko eragiketak, doitasun arruntean ($32$-bit) kalkulatu daitezke.

$S$-ataletako IRK metodoa, Newton iterazioaren bidez $d$-dimentsioko ODE sistemari aplikatzeko, urrats bakoitzean $sd \times sd$ tamaineko hainbat ekuazio sistema (iterazio bakoitzeko bat) askatu behar dira. Atal honetan, jatorrizko $sd$-dimentsioko ekuazio sistema, $(s+1)d$ sistema baliokide moduan berridatziko dugu eta sistema baliokidea,  $d \times d$ tamaineko $[s/2]+1$ matrize errealen \emph{LU}-deskonposaketa bidez askatuko dugu. Tamaina txikiko matrizeen LU-deskonposaketa azkarra denez, konputazionalki eraginkorragoa izatea espero dugu.   

\section{IRK-Newton estandarra.}

\paragraph*{}Demagun honako hasierako baliodun problema,
\begin{equation}
\label{eq:802}
\dot{y}=f(t,y),\ \ \ y(t_0)=y_0, 
\end{equation}
non  $y=[q^1,\dots,q^n,p^1,\dots,p^n]^T \in \mathbb{R}^{d=2n}$  eta $f: \  {\mathbb{R}}^{d+1} \ \longrightarrow {\mathbb{R}}^d$ diren. 

\paragraph*{}S-ataletako IRK metodoa gogoratuz, hasierako baliodun problemaren $y(t)$ soluzioaren $y_{n}\approx y(t_{n})$ hurbilpena, era honetan kalkulatzen da,  
\begin{equation}  
\label{eq:803}
y_{n+1}=y_n+h\sum^s_{i=1}{b_i \ f(t_n+c_ih,Y_{n,i})\ \ },\
\end{equation} 

non $Y_{n,i}$ atalak, era honetan inplizituki  definitzen diren,
\begin{equation}
\label{eq:804}
Y_{n,i}=y_n+\ h\ \sum^s_{j=1}{a_{ij}\ f(t_n+c_jh,Y_{n,j})}\ \ \ \ \ i=1 ,\dots, s.\
\end{equation} 

$Y_{n,i} \in \mathbb{R}^d,\ i=1,\ldots,s$ ezezagunen eta $sd$ tamainako ekuazio sistema ez-lineala askatzeko (\ref{eq:803}), iterazio metodo bat aplikatu behar dugu. Aurreko atalean puntu-finkoaren iterazioa aztertu genuen eta atal honetan,  Newton metodoa modu eraginkorrean aplikatzeko bidea ikertuko dugu.

\subsection*{Newton iterazioa.}

Newton iterazioa, $k=1,2,\dots$  $Y_i^{[k]}$ atalen hurbilpenak kalkulatzeko algoritmoa, modu honetan definituko dugu,

\begin{align}
\label{eq:(1)Newton_iteration}
1) & \quad r_i^{[k]} := -Y_{i}^{[k-1]} + y + h \sum_{j=1}^{s}\, a_{ij}\, f(t + c_j h,Y_{j}^{[k-1]}), \quad  i=1 ,\ldots, s, \\
\label{eq:(2)Newton_iteration}
\begin{split}
2) & \quad \mathrm{Askatu \ } \Delta Y_{i}^{[k]},\\
& \quad \Delta Y_{i}^{[k]}  - h \sum_{j=1}^{s}\, a_{ij}\, J_j^{[k]} \Delta Y_{j}^{[k]} = r_i^{[k]} \quad  i=1 ,\ldots, s, \\
& \mbox{non} \quad  J_i^{[k]}=\frac{\partial f}{\partial y}(t + c_i h,Y_{i}^{[k]}) \quad \quad  i=1,\ldots, s, 
\end{split} \\
\label{eq:(3)Newton_iteration}
3)& \quad Y_i^{[k]} := Y_i^{[k-1]} + \Delta Y_i^{[k]}, \quad  i=1 ,\ldots, s,
\end{align}

\paragraph*{}Azpimarratu behar da iterazio bakoitzeko,  $J_i^{[k]}$ jakobiarraren $s$-ebaluazio eta $sd \times sd$ tamainako matrizearen LU-deskonposaketa kalkulatu behar ditugula. Eragiketa hauek konplexuak dira eta beraz, Newton osoaren inplementazioa konputazionalki garestia da. 

\subsection*{Newton sinplifikatuaren iterazioa.}

Newton sinplifikatuaren iterazioa aplikatzerakoan, $J_i^{[k]}$ jacobiarra $J_i^{[0]}=\partial f / \partial y \ (t+c_ih, Y_i^{[0]}) \ \ i=1,\cdots,s$ jacobiarraz ordezkatzen da eta askatu beharreko ekuazio sistema honakoa da,

\begin{equation*}
\Delta Y_{i}^{[k]}  - h \sum_{j=1}^{s}\, a_{ij}\, J_j^{[0]} \Delta Y_{j}^{[k]} = r_i^{[k]}, \quad  i=1 ,\ldots, s,
\end{equation*}
$\mbox{non} \quad  J_i^{[0]}=\frac{\partial f}{\partial y}(t + c_i h,Y_{i}^{[0]}) \quad \quad  i=1,\ldots, s$ den.

Lehen sinplifikazio honetan, integrazioaren urrats bakoitzeko,  $J_i^{[0]}$ jakobiarraren s-ebaluazio eta $sd \times sd$ tamainako matrizearen LU-deskonposaketa behin bakarrik kalkulatu behar ditugu. Modu baliokidean, ekuazio lineala notazio matriziala erabiliz laburtu daiteke,
\begin{equation*}
\label{eq:805}
\left (I_s \otimes I_d - h  
\begin{bmatrix}
a_{11}  J_1^{[0]} & \dots & a_{1s}  J_s^{[0]} \\
a_{21}  J_1^{[0]} & \dots & a_{2s}  J_s^{[0]} \\
\dots            & \ddots & \dots \\
a_{s1}  J_1^{[0]} & \dots & a_{ss}  J_s^{[0]} \\ 
\end{bmatrix} \right) \Delta Y^{[k]} =r^{[k]}.
\end{equation*}

non,
\begin{equation*}
\label{eq:806}
Y^{[k]}=\begin{bmatrix}
Y_1^{[k]} \\
\vdots \\
Y_s^{[k]}
\end{bmatrix} \in \mathbb{R}^{sd}, \ \ \
r^{[k]}=\begin{bmatrix}
r_1^{[k]} \\
\vdots \\
r_s^{[k]}
\end{bmatrix} \in \mathbb{R}^{sd},
\end{equation*}

\begin{equation*}
\label{eq:807}
J_{is}(y)=\left(\partial f^i/\partial y^j (y)\right)_{i,j}^d=
\begin{bmatrix}
    \frac{\partial f^1}{\partial y^1} & \cdots & \frac{\partial f^1}{\partial y^d}\\    
    \vdots & \ddots & \vdots \\    
    \frac{\partial f^d}{\partial y^1} & \cdots & \frac{\partial f^d}{\partial y^d}\\    
\end{bmatrix} \in \mathbb{R}^{d \times d},\ is=1,\cdots,s.
\end{equation*}

\subsection*{Newton super-sinplifikatuaren iterazioa.}

Bigarren sinplifikazioa bat aplika daiteke, $J_i^{[0]}=\partial f / \partial y \ (t+c_ih, Y_i^{[0]}), \ \  i=1,\dots,s$ matrizeak,  $J_i^{[0]} \approx J, \ i=1,\cdots,s$ hurbilpen bakarrekin ordezkatuz. Era honetako ekuazio sistema lortuko dugu,   

\begin{equation}
\label{eq:808}
(I_s \otimes I_d - h \ A \otimes J) \Delta Y^{[k]} = r^{[k]}.
\end{equation}
non $I_s,I_d$ identitate eta $A=(a_{ij})_{i,j}^s$ koefiziente matrizeak diren.

Ohikoa da $J=\partial f / \partial y \ (t+h/2, y_n)$ hurbilpena erabiltzea. Problema zurruna denean, atalen hasieraketa $Y_i^{[0]}=y_n$, ($i=1,\dots,s$)  erabili ohi da, eta jakobiarraren hurbilpen honekin ekuazio lineala askatzeak zentzua izango du.

Newton iterazio bertsio honi super-sinplifikatua deitu diogu. Iterazio bakoitzean $f$ funtzioaren $s$ ebaluazio eta $sd$ tamainako ekuazio-sistema lineala askatu behar da. $(I_s \otimes I_d - h \ A \otimes J)$ matrizea iterazio guztietarako berdina da,  bere LU-deskonposaketa behin bakarrik egin behar da eta konputazionalki garestia da \cite{Butcher1976} \cite{Hairer1996},
\begin{align*}
&\text{LU-deskonposaketa}, \ \ 2s^3d^3/3+\mathcal{O}(d^3), \\
&\text{back substitution}, \ \ 2s^2d^2+\mathcal{O}(d).
\end{align*}

Jarraian, Newton super-sinplifikatuaren inplementazioaren algoritmo orokorra laburtu dugu (Algoritmoa \ref{alg:nss}).

\subsection*{Algoritmoa.}

\begin{algorithm}[H]
 \BlankLine
  $\tilde{y}_0=fl(y_0)$\;
  $e_0=fl(y_0-\tilde{y}_0)$\;
  \For{$n\leftarrow 0$ \KwTo ($endstep-1$)}
  {
   \BlankLine
   $k=0$\;
   \text{Hasieratu} $Y_{n,i}^{[0]} \ \ , \ \ i=1,\dots,s $\;
   \BlankLine
   $J=\frac{\partial f}{\partial y}(t+h/2,y_n) $\; 
   $M=LU(I_s \otimes I_d - h \ A \otimes J)$\;
   \BlankLine
   \While{ (\text{not konbergentzia})}
   {
    \BlankLine 
    $k=k+1$\;
    $r_i^{[k]}=-Y_i^{[k-1]}+y+h \sum\limits_{j=1}^{s} a_{ij} f(t+c_jh,Y_j^{[k-1]}) $\;
    Solve $(M \Delta Y^{[k]}=r^{[k]})$\;
    $Y^{[k]}=Y^{[k-1]}+\Delta Y^{[k]}$\;
    $\text{konbergentzia} \leftarrow \text{GeratzeErizpidea}(Y^{[k]},Y^{[k-1]},\Delta_{min}) $\;
   }
   \BlankLine
   \If{($\exists j \ \text{non} \ \Delta_j^{[K]}\neq 0$)}
   {
    \If{$(NormalizeDistance(Y^{[k]},Y^{[k-1]})>1$}
     {$\text{fail convergence}$\;}
   }
   {$(\tilde y_{n+1},e_{n+1})\leftarrow \text{baturakonpensatua}(y_n,e_n,Y_n^{[k]})$\;}    
 }
 \caption{IRK (Newton super-sinplifikatua).}
 \label{alg:nss}
\end{algorithm}



\section{IRK-Newton eraginkorra.}
\label{sec:s74}

\subsection*{Sarrera.}

Atal honetan, honako ekuazio lineala modu eraginkorrean askatzeko inplementazioa proposatuko,
\begin{equation*}
(I_s \otimes I_d - h \ A \otimes J) \ \Delta Y = r,
\end{equation*}
$J \in \mathbb{R}^{d \times d}$  eta $r \in \mathbb{R}^{s \times d}$ emandako matrizeak izanik.

\paragraph*{}$S$-ataletako IRK metodoa, Newton iterazioaren bidez $d$-dimentsioko ODE sistemari aplikatzeko, urrats bakoitzean $sd \times sd$ tamaineko hainbat ekuazio sistema (iterazio bakoitzeko bat) askatu behar dira. Atal honetan, jatorrizko $sd$-dimentsioko ekuazio sistema, $(s+1)d$ sistema baliokide moduan berridatziko dugu eta sistema baliokidea,  $d \times d$ tamaineko $[s/2]+1$ matrize errealen \emph{LU}-deskonposaketa bidez askatuko dugu. Tamaina txikiko matrizeen LU-deskonposaketa azkarra denez, konputazionalki eraginkorragoa izatea espero dugu.      

\paragraph*{}Gauss nodoetan oinarritutako \emph{IRK} metodoak sinplektikoak eta simetrikoak dira. 
\begin{enumerate}
\item Sinplektikoa.
\begin{equation} 
\label{eq:sympl}
b_{i}a_{ij}+b_{j}a_{ji}-b_{i}b_{j}=0, \ \ 1 \leqslant i,j \leqslant s.
\end{equation} 
\item Simetrikoa.
\begin{align}
\label{eq:simm}
b_{s+1-i}=b_i, \ \ {c}_{s+1-i}=1-{c}_i,& \quad \ 1\leqslant i,j \leqslant s, \\
b_j={a}_{s+1-i,s+1-j}+a_{i,j},& \quad 1\leqslant i,j \leqslant s. 
\end{align} 
\end{enumerate}

Proposamen berria, bi propietate hauetan oinarrituz garatuko dugu.

\subsection*{Notazio alternatiboa.}

Koefiziente notazioa berri bat finkatuta,
\begin{equation*}
\bar{c}_i=c_i-\frac{1}{2}, \ \ \bar{a}_{ij}=a_{ij}-\frac{b_j}{2}, \ \ 1\leqslant i,j \leqslant s,
\end{equation*}
sinplektikotasun (\ref{eq:sympl}) eta simetrikotasun (\ref{eq:simm}) propietateak modu baliokidean era honetan adieraziko ditugu,
\begin{enumerate}
\item {Sinplektikoa.}
\begin{equation}
\label{eq:eqlineala}
(B \bar{A}) \ \ \mbox{antisimetrikoa da},
\end{equation}
non $\bar{A}=(\bar{a}_{ij})_{i,j=1}^s$ eta $B$, ($b_1,b_2,\dots,b_s$) balioen  matrize diagonala diren.

\item {Simetrikoa.}
\begin{align}
\label{eq:simm2}
b_{s+1-i}=b_i, \ \ \bar{c}_{s+1-i}=-\bar{c}_i,& \quad 1\leq i \leq s, \\
\bar{a}_{s+1-i,s+1-j}=-\bar{a}_{ij},& \quad 1\leq i,j \leq s. \\
\end{align} 

\end{enumerate}

Inplementazio berri honen matrizeen dimentsioak zehazteko, balio berri hauetan oinarrituko gara,
$m=[(s+1)/2]$, eta $s-m =[s/2]$. Beraz metodoaren $s$-atal kopurua bikoiti ala bakoiti den arabera,
\begin{itemize}
\item $s$ bikoitia.

Adibidea $s=6 \rightarrow m=3,s-m=3$.

\item $s$ bakoitia.

Adibidea $s=7 \rightarrow m=4, s-m=3$.
\end{itemize}


\subsubsection*{Sinplektikoa.}

$(B \bar{A})$ antisimetrikoa bada, $B^{1/2}\bar{A}B^{-1/2}$ antisimetrikoa da. Era berean, honek $\bar{A}$ diagonalizagarria dela eta irudikari balio propio puruak dituela suposatzen du. Eta beraz,  $Q$ ($s \times s$) tamaineko matrize ortogonala existitzen da,
\begin{align}
\label{eq:syb}
Q^{-1}\bar{A}Q=
\left(
\begin{matrix}
0 & D \\
-D^T & 0
\end{matrix}
\right)
\end{align}
non $D$, ($m \times (s-m)$) tamainako eta balio erreal positiboen matrize diagonala den. 

\subsubsection*{Simetrikoa.}

Sinplektikotasun eta simetri propietate hauetan oinarrituz jarraiko matrizeak definituko ditugu.

\begin{enumerate}
\item P matrizea.

Kontsideratu ($s \times s$) tamainako $P=(P_1 \ P_2)$ matrize ortogonala. $x=(x_1,\dots,x_s)^T \in \mathbb{R}^s$, $P_1^Tx=(y_1,\dots,y_m)^T$, eta $P_2^Tx=(y_{m+1},\dots,y_s)^T$ non,
\begin{align*}
&y_i = \frac{\sqrt{2}}{2} (x_{s+1-i}+x_i), \ \ i=1,\dots,[s/2], \\
&y_i =\frac{\sqrt{2}}{2} (x_{s+1-i}-x_{i}), \ \ i=m+1,\dots,s, \\
&y_{m} = x_{m}, \ \ s \ \ \mbox{bakoitia bada}.
\end{align*}  

Matrizearen dimentsioak laburtuz, $P=(P_1 \ P_2) \in \mathbb{R}^{s \times s}$, $P_1 \in \mathbb{R}^{s \times m}$ eta $P_2 \in \mathbb{R}^{s \times (s-m)}$.

\item K matrizea.

Batetik, simetri propietateak (\ref{eq:simm2}),  $P_i^TB^{\frac{1}{2}}\bar{A}B^{-\frac{1}{2}}P_i=0, \ i=1,2$  eta bestetik, propietate sinplektikoak $B^{1/2}\bar{A}B^{-1/2}$ antisimetrikoa dela ziurtatzen duenez, $\bar{A}$ matrizea honako matrizearen antzekoa dela ondorioztatu daiteke,
\begin{align}
P^TB^{\frac{1}{2}}\bar{A}B^{-\frac{1}{2}}P=
\left(
\begin{matrix}
0 & K \\
-K^T & 0
\end{matrix}
\right)
\end{align}
non $K=P_1^TB^{\frac{1}{2}}\bar{A}B^{-\frac{1}{2}}P_2 \ \in \mathbb{R}^{m \times (s-m)}$ den.

\item D matrizea.

$K=UDV^T$ balio singularraren deskonposaketa izanik, non $U \in \mathbb{R}^{m \times m}$, $V \in \mathbb{R}^{(s-m) \times (s-m)}$ matrize ortonormalak diren eta $D \in \mathbb{R}^{m \times (s-m)}$, $K$ matrizearen balio singularren ($\sigma_1, \dots, \sigma_{s-m}$) matrize diagonala da,
\begin{align}
\label{eq:Dmat}
D=
\left(
\begin{matrix}
\sigma_1 & 0 & \dots & 0 \\
0 & \sigma_2 & \dots & 0 \\
0 & 0 & \dots & 0 \\
0 & 0 & \dots & \sigma_{s-m}
\end{matrix}
\right), \ \ \
D=
\left(
\begin{matrix}
\sigma_1 & 0 & \dots & 0 \\
0 & \sigma_2 & \dots & 0 \\
0 & 0 & \dots & 0 \\
0 & 0 & \dots & \sigma_{s-m} \\
0 & 0 & \dots & 0
\end{matrix}
\right).
\end{align}
s bikoitia bada, $D$ matrizea ezkerrean eta s bakoitia bada, $D$ matrizea eskubian ($\sigma_m=0$) irudikatu dugu. 

\item Q matrizea.

Simetrikoa den (\ref{eq:syb}) propietateari esker,
\begin{align*}
&Q=(Q_1 \ Q_2)=
B^{-1/2}(P_1 \ P_2)
\left(
\begin{matrix}
U & 0 \\
0 & V
\end{matrix}
\right)=
B^{-1/2} (P_1U \ P_2V), \\
&Q^{-1}=Q^TB.
\end{align*}  

Matrizearen dimentsioak laburtuz, $Q=(Q_1 \ Q_2) \in \mathbb{R}^{s \times s}$, $Q_1 \in \mathbb{R}^{s \times m}$ eta $Q_2 \in \mathbb{R}^{s \times (s-m)}$.

\end{enumerate}

\subsection*{(s+1) ekuazio-sistema.}

Aldagai berri bat kontsideratuko dugu $z \in \mathbb{R}^d$ eta honako ekuazio sistema kontsideratuko dugu,
\begin{align*}
&Y_{n,i}=y_n+\frac{z}{2}+ h\ \sum^s_{j=1}{\bar{a}_{ij}\ f(t_n+c_jh,Y_{n,j})}\ \ \ \ \ i=1 ,\dots, s,\\
&z=h \sum_{i=1}^{s} {b_i f(t_n+c_jh,Y_{n,i})}.
\end{align*} 
Ekuazio berri hau, sistemari gehituz, $(s+1) \times d$ dimentsioko ekuazio sistema baliokidea dugu,
\begin{align}
\label{eq:s1s}
(I_s \otimes I_d- h \ \bar{A} \otimes J) \ \Delta Y - \frac{1}{2}(e_s \otimes I_d) \ \Delta z =r,\\
(-h e_s^T B \otimes J) \ \Delta Y+  \Delta z=0,
\end{align}
non $e_s^T=(1,\dots,1) \in \mathbb{R}^{1 \times s}$ den. Argi dagoenez, $(\Delta Y, \Delta z)$ ekuazio-sistema honen (\ref{eq:s1s}) soluzioa bada, $\Delta Y$ gure jatorrizko ekuazio sistemaren (\ref{eq:eqlineala}) soluzioa da.

\paragraph*{} Ekuazio-sistemaren adierazpen matriziala lagungarria izan daiteke,
\begin{equation*}
\begin{bmatrix}
    &      &      &  & \ \ -I_d/2 \\
    &      &      &  & \ \ -I_d/2 \\
    &      &      &  & \ \      \\    
    &  & I_s \otimes I_d- h \ \bar{A} \otimes J & & \ \ \vdots \\
    &      &      &  & \ \      \\
    &      &      &  & \ \ \ \ -I_d/2    \\
-hb_1 J & -hb_2 J & \dots & -hb_s J &  I_d\\ 
\end{bmatrix}
\begin{bmatrix}
\Delta Y_1 \\
\Delta Y_2 \\
\vdots \\
\Delta Y_s \\
\Delta z\\
\end{bmatrix}=
\begin{bmatrix}
r_1 \\
r_2 \\
\vdots \\
r_s \\
0\\
\end{bmatrix}
\end{equation*} 

\subsubsection*{IRK metodo orokorren garapena.}

Lehengo, garapena metodo sinplektikoentzat egingo dugu eta ondoren, metodoa simetrikoentzat. Honako aldagai aldaketa aplikatuz,
\begin{equation*}
 \Delta Y = (Q \otimes I_d) \ W,
\end{equation*}

eta metodoa simetrikoa izateagatik (\ref{eq:syb}) , ekuazio sistema baliokidea lortuko dugu,
\begin{multline*}
\left(
\begin{matrix}
I_m \otimes I_d & \ \ -h D \otimes J \\
hD^T \otimes J &  \ \ I_{s-m} \otimes I_d 
\end{matrix}
\right)
 W- \frac{1}{2}(Q^{-1} e_s \otimes I_d) \ \Delta z = (Q^{-1} \otimes I_d) \ r,
\end{multline*}
\begin{equation*}
- h \ (e_s^T \ B \ Q \otimes J) \ W + \Delta z =0.
\end{equation*}

Eranskinean (\ref{serans:A31}), ekuazio baliokideak lortzeko eman diren urratsen zehaztapenak eman ditugu. 


\subsubsection*{IRK metodo simetriko eta sinplektikoen garapena.}

Honako aldagai aldaketarekin,
\begin{equation*}
 \Delta Y = (Q \otimes I_d) W= (Q_1 \otimes I_d) W'+ (Q_2 \otimes I_d) W'',
\end{equation*}
non,
\begin{align*}
W=\left(
\begin{matrix}
W^{'} \\
W^{''} 
\end{matrix}
\right),\ \ W^{'} \in \mathbb{R}^{m \times d}, \ \ W^{''} \in \mathbb{R}^{(s-m) \times d},
\end{align*}
\begin{align*}
Q=(Q_1 \ \ Q_2), \ \  Q_1 \in \mathbb{R}^{s \times m}, \ \ Q_2 \in \mathbb{R}^{s \times (s-m)},
\end{align*}

eta metodoa simetrikoa nahiz sinplektikoa izateagatik $e_s^TBP_2=0, \ e_s^TBQ_2=e_s^TBP_2V=0$~berdintasunak aplikatuz, honako ekuazio-sistema baliokidea lortuko dugu,
\begin{align*}
 W^{'}-h \ (D \otimes J) \ W^{''} -\frac{1}{2}\ (Q_1^T B \ e_s \otimes I_d) \ \Delta z &= (Q_1^T B \otimes I_d) \ r,\\
 h \ (D^T \otimes J) \ W^{'}+W^{''} &= (Q_2^T B \otimes I_d) \ r,\\
 - h \ (e_s^T B \ Q_1 \otimes J) \ W^{'} + \Delta z &=0. 
\end{align*}

Eranskinean (\ref{serans:A32}), ekuazio baliokideak lortzeko eman diren urratsen zehaztapenak eman ditugu.

\paragraph*{Matrizearen egitura.} Ekuazio sistemaren adierazpen matrizialarekin ($s=6$), matrizearen egitura berezia ikus daiteke. Aldagai aldaketarekin lortutako ekuazio sistema baliokideak, blokeka diagonala da eta hau aprobetxatuz, Newton iterazioaren inplementazio eraginkorra lortuko dugu.    
\begin{equation*}
\begin{bmatrix}
 I_d        &             &                     & \multicolumn{1}{|c}{-h\sigma_1 J} &            &                  &\multicolumn{1}{|c}{-\frac{\alpha_1}{2} I_d}\\
            & I_d         &                     & \multicolumn{1}{|c}{}           & -h\sigma_2 J &                
 &\multicolumn{1}{|c}{-\frac{\alpha_2}{2} I_d}\\
            &             & I_d                 & \multicolumn{1}{|c}{}           &            & -h\sigma_3 J  
 &\multicolumn{1}{|c}{-\frac{\alpha_3}{2} I_d}\\\cline{1-7}    
h\sigma_1 J &             &                     & \multicolumn{1}{|c}{I_d}        &            &            
 &\multicolumn{1}{|c}{0}\\
            & h\sigma_2 J &                     & \multicolumn{1}{|c}{}           &  I_d       &             
 &\multicolumn{1}{|c}{0}\\
            &             &  h\sigma_3 J        & \multicolumn{1}{|c}{}           &            &  I_d        
 &\multicolumn{1}{|c}{0}\\\cline{1-7}
 -h\alpha_1 J       & -h\alpha_2 J              &  -h\alpha_3 J                    & \multicolumn{1}{|c}{$0$}        &  $0$       &  $0$         
 &\multicolumn{1}{|c}{ I_d}\\
\end{bmatrix}
\begin{bmatrix}
         \\
 W^{'} \\
    \\
\cline{1-1} \\
    \\
 W^{''}   \\
    \\
    \cline{1-1} \\
 \Delta z  \\
\end{bmatrix}=
\begin{bmatrix}
         \\
 R^{'} \\
    \\
\cline{1-1} \\
    \\
 R^{''}   \\
    \\
    \cline{1-1} \\
 0  \\
\end{bmatrix}
\end{equation*} 
non 
\begin{equation*}
\begin{bmatrix}
 R^{'}=(Q_1^{T}B^{1/2} \otimes I_d) \ r \\
 R^{''}=(Q_2^{T}B^{1/2} \otimes I_d) \ r
\end{bmatrix}, \ \
\left(
\begin{matrix}
\alpha_1 \\
\vdots \\
\alpha_m
\end{matrix}
\right)=Q_1^TB \ e_s.
\end{equation*}

\paragraph*{}Jarraian, ekuazio sistemaren ezezagunak ($\Delta z,W^{'},W^{''}$) askatzeko aplikatuko ditugun espresioak laburtuko ditugu.
\paragraph*{$W^{''}$ kalkulatzeko ekuazioak.}

Bigarren ekuaziotik $W^{''}$ askatu,
\begin{equation}
W^{''}= -h \ (D^T \otimes J) \ W^{'}+(Q_2^T B \otimes I_d) \ r.
\end{equation}

\paragraph*{$W^{'}$ kalkulatzeko ekuazioak.}

Eta lehen ekuazioan ordezkatuz,
\begin{align*}
(I_m \otimes I_d+ h^2DD^T \otimes J^2) \ W^{'}- \frac{1}{2}(Q_1^T B \ e_s \otimes I_d)\ \Delta z&=R, \\
- h \ (e_s^T B \ Q_1 \otimes J) \ W^{'} + \Delta z &=0,
\end{align*}
non $R=(Q_1^T B \otimes I_d) \ r + h \  ( D Q_2^T B \otimes J)\,  r \in \mathbb{R}^{md}.$

\paragraph*{}Goiko ekuazio sistema honako notazioaren arabera,  
\begin{equation*}
R=\begin{bmatrix}
R_1 \\
\vdots \\
R_m
\end{bmatrix}, \ \ \
W^{'}=\begin{bmatrix}
W_1 \\
\vdots \\
W_m
\end{bmatrix}, 
\ \ R_i,W_i \in \mathbb{R}^d, \ \ i=1,\dots,m  
\end{equation*}

era honetan berridatziko dugu,
\begin{align*}
(I_d+h^2\sigma_i^2J^2) \ W_i- \frac{\alpha_i}{2}\ \Delta z &=R_i, \ i=1,\dots,m,\\
-h \ J \sum\limits_{i=1}^{m} \alpha_i W_i+\Delta z &=0,
\end{align*}
non,
\begin{align*}
\left(
\begin{matrix}
\alpha_1 \\
\vdots \\
\alpha_m
\end{matrix}
\right)=Q_1^TB \ e_s,
\end{align*}
eta $\sigma_1 \ge \cdots \sigma_{s/2}, \ K$ matrizearen balio singularrak diren; $s$ bakoitia denean $\sigma_m=0$ dela gogoratu (\ref{eq:Dmat}). 

\paragraph*{$\Delta z$ kalkulatzeko ekuazioak.}

Lehen ekuaziotik,$W_i$ askatuz, 
\begin{equation*}
W_i=(I_d+h^2\sigma_i^2J^2)^{-1} (R_i+\frac{\alpha_i}{2} \ \Delta z)
\end{equation*}

eta bigarren ekuazioan ordezkatuz, $\Delta z \in \mathbb{R}^d$ askatzeko ekuazioak lortuko ditugu,
\begin{equation}
M\ \Delta z=h \ J\sum\limits_{i=1}^{m}\alpha_i (I_d+h^2\sigma_i^2J^2)^{-1}R_i,
\end{equation}
non
\begin{equation}M=I_d+ J \ \frac{h}{2}\ \sum\limits_{i=1}^{m} \alpha_i^2 (I_d+h^2 \sigma_i^2 J^2)^{-1} \in \mathbb{R}^{d \times d}.
\end{equation}


\section{IRK-Newton estandarra (formulazio berria).}

\subsection*{Sarrera.}

IRK-Newton inplementazioarentzat, IRK puntu-finkoaren inplementazioan erabilitako birformulazio (\ref{chap:IRK-PF}atala) berdina aplikatuko dugu. Newton iterazio metodoan ordea, $L_i$ ($i=1,\dots,s$) aldagai ezezagunatzat eta $Y_i$ ($i=1,\dots,s$) aldagai laguntzailea kontsideratzea \cite{Olsson2000} izango da egokiena,
\begin{align}
\label{eq:62}
&\ L_{n,i}=hb_if(Y_{n,i}), \ \ Y_{n,i}=y_n+ \sum\limits_{j=1}^{s} \mu_{ij} \ L_{n,j},  \ \ i=1,\dots,s,\\
&y_{n+1}=y_n+\sum\limits_{i=1}^{s} L_{n,i},
\end{align}
non  $\mu_{ij}=a_{ij}/{b_j}$, \ \  $1 \le i,j \le s$.

\subsection*{Newton sinplifikatuaren iterazioa.}

Newton metodo sinplifikatua, $k=1,2,\dots$  $L_i^{[k]}$ hurbilpenak kalkulatzeko,
\begin{enumerate}
\item 
$g_i^{[k]}=-L_i^{[k-1]}+h b_i f(t+c_ih,\ y_n+ \sum\limits_{j=1}^{s} \mu_{ij} L_{j}^{[k-1]}), \ \ i=1,\dots,s.$

\item Askatu $\Delta L_i^{[k]}$,

$\Delta L_i^{[k]} - h b_i \ J_i^{[0]} \sum_{j=1}^{s} \mu_{ij}  \ \Delta L_j^{[k]} = g_i^{[k]}  , \ \ i=1,\dots,s.$

\item $L_i^{[k]} = L_i^{[k-1]}+ \Delta L_i^{[k]}, \ \  i=1,\dots,s.$

\end{enumerate}
non $J_i^{[0]}=\frac{\partial f}{\partial y} (t+c_ih, Y_i^{[0]}) \ \  i=1,\dots,s$ den.

\paragraph*{}Modu baliokidean, ekuazio lineala notazio matriziala erabiliz laburtu daiteke,
\begin{equation*}
\left (I_s \otimes I_d - h  
\begin{bmatrix}
b_1 \mu_{11} J_1^{[0]} & \dots & b_1 \mu_{1s} J_1^{[0]} \\
b_2 \mu_{21} J_2^{[0]} & \dots & b_2 \mu_{2s} J_2^{[0]} \\
\dots          & \ddots & \dots \\
b_s \mu_{s1} J_s^{[0]} & \dots & b_s \mu_{ss} J_s^{[0]} \\ 
\end{bmatrix} \right) \Delta L^{[k]} =g^{[k]}.
\end{equation*}

non,
\begin{equation*}
L^{[k]}=\begin{bmatrix}
L_1^{[k]} \\
\vdots \\
L_s^{[k]}
\end{bmatrix} \in \mathbb{R}^{sd}, \ \ \
g^{[k]}=\begin{bmatrix}
g_1^{[k]} \\
\vdots \\
g_s^{[k]}
\end{bmatrix} \in \mathbb{R}^{sd},  
\end{equation*}

\begin{equation*}
\label{eq:907}
J_{is}(y)=\left(\partial f^i/\partial y^j (y)\right)_{i,j}^d=
\begin{bmatrix}
    \frac{\partial f^1}{\partial y^1} & \cdots & \frac{\partial f^1}{\partial y^d}\\    
    \vdots & \ddots & \vdots \\    
    \frac{\partial f^d}{\partial y^1} & \cdots & \frac{\partial f^d}{\partial y^d}\\    
\end{bmatrix} \in \mathbb{R}^{d \times d}, \ is=1,\cdots,s.
\end{equation*}

\subsection*{Newton super-sinplifikatuaren iterazioa.}

Honako bigarren sinplifikazioarekin, $J_i^{[0]}=\partial f / \partial y \ (t+c_ih, Y_i^{[0]}), \ \  i=1,\dots,s$ matrizeak,  $J_i^{[0]} \approx J, \ i=0,\cdots,s$ hurbilpenarekin ordezkatuz, ekuazio sistema lineal hau lortuko dugu,
\begin{equation}
(I_s \otimes I_d - h \ BAB^{-1} \otimes J) \ \Delta L = g. 
\end{equation}
non
\begin{equation*}
I_s,\ I_d \ \ \text{identitateak eta }B=
\begin{bmatrix}
   b_{1} & 0      & \dots & 0 \\
   0     & b_{2}  & \dots & 0 \\
    \vdots & \vdots & \ddots  & \vdots \\
   0     & 0      & \dots & b_{s}\\
\end{bmatrix}.     
\end{equation*}

\subsection*{Algoritmoa.}

Formulazio berrian Newton super-sinplifikatua deitu dugun algoritmoa laburtu dugu (Algoritmoa \ref{alg:nssli}).

\begin{algorithm}[h!]
 \BlankLine
  $\tilde{y}_0=fl(y_0)$\;
  $e_0=fl(y_0-\tilde{y}_0)$\;
  \For{$n\leftarrow 0$ \KwTo ($endstep-1$)}
  {
   \BlankLine
   $k=0$\;
   \text{Hasieratu}  $L_{n,i}^{[0]} \ \ , \ \ i=1,\dots,s $\;
   $Y_{n,i}^{[0]}=y_{n} + \ \big(e_n+\sum\limits_{j=1}^{s} \mu_{ij} L_{n,j}^{[0]}\big)  $\;
   \BlankLine
   $J=\frac{\partial f}{\partial y}(y_n) $\; 
   $M=LU(I_s \otimes I_d - h \ BAB^{-1} \otimes J)$\;
   \BlankLine
   \While{ (\text{not konbergentzia})}
   {
    \BlankLine 
    $k=k+1$\;
    $g_i^{[k]}=-L_i^{[k-1]}+h b_i f(t+c_ih,\ y_n+ \sum\limits_{j=1}^{s} \mu_{ij} L_{j}^{[k-1]})$\;
    Solve $(M \Delta L^{[k]}=g^{[k]})$\;
    $L^{[k]}=L^{[k-1]}+\Delta L^{[k]}$\;
    $Y_{n,i}^{[k]}=y_{n} + \ \big(e_n+\sum\limits_{j=1}^{s} \mu_{ij} L_{n,j}^{[k]}\big)  $\;
    $\text{konbergentzia} \leftarrow \text{GeratzeErizpidea}(L^{[k]},L^{[k-1]},\Delta_{min}) $\;
   }
   \BlankLine
   \If{($\exists j \ \text{non} \ \Delta_j^{[K]}\neq 0$)}
   {
    \If{$(NormalizeDistance(Y^{[k]},Y^{[k-1]})>1$}
    {$\text{fail convergence}$\;}
   }
   $(\tilde y_{n+1}, e_{n+1})\leftarrow \text{baturakonpensatua}(\tilde y_{n},e_{n},L_{n}^{[k]})$\;
 }
 \caption{IRK (Newton super-sinplifikatua).}
 \label{alg:nssli}
\end{algorithm}

\paragraph*{Interpolazio koefizienteak.} $L_{n,i}^{[0]}$ atalen hasieraketarentzat dagokien koefizienteak era honetan definituko ditugu. IRK puntu finkoaren inplementazioan finkatu genituen interpolazio koefizienteetatik abiatuta (\ref{eq: interpLi}) modu errezean definituko ditugu formulazio honi dagozkion interpolazio koefizienteak.
\begin{align}
\left \{ \begin{array}{c}
  Y_{n,i}^{[0]}=y_n+\sum_{j=1}^{s} \mu_{ij} L_{n,j}^{[0]} \\[.25cm]
  Y_{n,i}^{[0]}=y_n+\sum_{j=1}^{s} \nu_{ij} L_{n-1,j} \\
          \end{array} \right. 
\Rightarrow \ \ L_n^{{[0]}}=(Mu^{-1} Nu) L_{n-1}, \ \ (Mu^{-1} Nu)_{i,j}^{s}=\lambda_{ij}/a_{ij}.
\end{align}

\paragraph*{Geratze erizpidea.} Puntu-finkoan definitutako geratze erizpide berdina aplikatuko dugu baina $L_{n,i}, \ i=1,\cdots,s$ aldagei.
\begin{equation*}
\Delta^{[k]}=(L_1^{[k]}-L_1^{[k-1]},\dots,L_s^{[k]}-L_s^{[k-1]}) \in \mathbb{R}^{sd},
\end{equation*}

Iterazioak jarraitu, honako baldintza betetzen den artean,
\begin{multline}
\label{eq:not_stopping Li}
\exists j \in \{1,\ldots,s d\} \quad \mbox{non} \quad \\
%|\Delta_j^{[1]}| >\cdots > |\Delta_j^{[k-1]}|>0 \mbox{ and } \Delta_j^{[k]} \neq 0. 
\min \left(\{|\Delta_j^{[1]}|,\cdots ,|\Delta_j^{[k-1]}|\} \ /\{0\} \right)>|\Delta_j^{[k]}| \mbox{ and } \Delta_j^{[k]} \neq 0. 
\end{multline}

\paragraph*{Batura konpensatua.}

Batura konpesatua egiteko modua zehaztuko dugu. Lehenik $\Delta L_i$ gaiak gehituko ditugu,
\begin{algorithm}[h]
$\beta_{n}={e}_{n} + \sum\limits_{j=1}^{s}\Delta L_{n,j}^{[k]}$\;
\end{algorithm}

Bigarrenik, $y_{n+1}=y_{n}+ \sum_{i=1}^{s} L_{n,i}^{[k-1]} + \beta_{n}$ batuketa egiteko

\begin{algorithm}[H]
  \SetAlgoLined\DontPrintSemicolon
  \SetKwFunction{algo}{algo}\SetKwFunction{BaturaKonpensatua}{BaturaKonpensatua}
  \SetKwProg{myalg}{Algorithm}{}{}
  \SetKwProg{myproc}{Function}{}{}
  \myproc{\BaturaKonpensatua {$y_n$,\ $\beta_n$,\ $L_n^{[k-1]}$}}{
     \BlankLine
     $s_0=y_n$\;
     $ee=\beta_n$\;
     \For{$i\leftarrow 1$ \KwTo ($s$)}
      {
        $s_1=s_0$\;
        $\delta= L_{n,i}^{[k-1]} +ee$\;
        $s_0=s_1+\delta$\;
        $ee=(s_1 - s_0)+ \delta$\;   
      }
     $y_{n+1}=s_0$\;
     $e_{n+1}=ee$\;    
    \KwRet ($y_{n+1}$,$e_{n+1}$) \;}
  \caption{BaturaKonpensatua}
\end{algorithm} 


\section{IRK-Newton eraginkorra (formulazio berria).}


Formulazio berrian honakoa da, modu eraginkorrean askatu behar dugun ekuazio-lineala,
\begin{equation}
(I_s \otimes I_d - h \ BAB^{-1} \otimes J) \ \Delta L = g, 
\end{equation}
$J \in \mathbb{R}^{d \times d}$  eta $g \in \mathbb{R}^{s \times d}$ emandako matrizeak izanik. 


\subsection*{Formulazio estandarretik formulazio berrirako urratsa.}

Formulazio berriaren inplementazio eraginkorra, formulazio estandarrean emandako algoritmoaren (\ref{sec:s74} atala) moldatuz zehaztuko dugu.

\paragraph*{} Hauek izango dira bi formulazioen aldagaien arteko erlazioak,

\begin{enumerate}

\item Aldagai aldaketa.
\begin{align*}
\Delta L &=(B \otimes I_d) \ \Delta Y, \\
\Delta L &=(B Q \otimes I_d) \ W, \\
\Delta L &=(B Q_1 \otimes I_d) \ W^{'}+(B Q_2 \otimes I_d) \ W^{''}.
\end{align*}

\item $R$ matrizea.

$g \in \mathbb{R}^{sd} \ \  \rightarrow \ \ r=(B^{-1} \otimes I_d) g$.
\begin{align*}
R=&(Q_1^TB \otimes I_d) \ r + h \ (D Q_2^T B \otimes J) \ r ,\\
R=&(Q_1^T \otimes I_d) \ g + h \ (D Q_2^T \otimes J) \ g,  \\
R=& (Q_1^T) \ g  + (h D Q_2^T) \ g \ J^T.
\end{align*}

\item $W^{''}$ matrizea.
\begin{align*}
W^{''}&= -h \ (D^T \otimes J) \ W^{'}+ (Q_2^T B \otimes I_d) \ r, \\
W^{''}&= -h \ (D^T \otimes J) \ W^{'}+ (Q_2^T \otimes I_d) \ g, \\
\end{align*}


\end{enumerate}

\paragraph*{}Formulazio berrian IRK-Newton sinplifikatuaren inplementazioaren urratsak hauek dira,

\begin{enumerate}

\item $\Delta z$  askatu.
\begin{equation*}
M \ \Delta z=h \ J\sum\limits_{i=1}^{m}\alpha_i (I_d+h^2\sigma_i^2J^2)^{-1}R_i,
\end{equation*}
non $M=I_d+J \ \frac{h}{2}\sum\limits_{i=1}^{m} \alpha_i^2 (I_d+h^2 \sigma_i^2 J^2)^{-1} \in \mathbb{R}^{d \times d}$.

\item  $W^{'}$ askatu.
\begin{align*}
W^{'}_i=(I_d+h^2\sigma_i^2J^2)^{-1} (R_i+\frac{\alpha_i}{2} \ \Delta z) \in \mathbb{R}^d, \ i=1,\dots,m
\end{align*}

\item $W^{''}$.
\begin{align*}
W^{''} &= -h \ (D^T \otimes J) \ W^{'}+(Q_2^T B \otimes I_d) \ r, \\
W^{''} &= -h (D^T \otimes J) W^{'}+ (Q_2^T \otimes I_d) \ g, \\
W^{``} &= (-hD^T) \ W^{`} \ J^T+Q_2^T \ g,
\end{align*}


\item $\Delta L$.
\begin{align*}
\Delta L &=(B Q_1 \otimes I_d) W^{'}+(B Q_2 \otimes I_d) W^{''}, \\
\end{align*}

\end{enumerate}

\subsection*{AlgoritmoaV1.}

IRK-Newton gure inplementazioaren algoritmoa laburtuko dugu (Algoritmoa \ref{alg:IRK-Newton-Li}), 

\begin{algorithm}[h!]
 \BlankLine
  $\tilde{y}_0=fl(y_0)$\;
  $e_0=fl(y_0-\tilde{y}_0)$\;
  \For{$n\leftarrow 0$ \KwTo ($endstep-1$)}
  {
   \BlankLine
   $k=0$\;
   \text{Hasieratu}  $L_{n,i}^{[0]} \ \ , \ \ i=1,\dots,s $\;
   $Y_{n,i}^{[0]}=y_{n} + \ \big(e_n+\sum\limits_{j=1}^{s} \mu_{ij} L_{n,j}^{[0]}\big)  $\; 
   \BlankLine
   $J=\frac{\partial f}{\partial y}(y_n) $\; 
   \BlankLine
   $ M=I_d- \ J \ \frac{h}{2}\ \sum\limits_{i=1}^{m} \alpha_i^2 (I_d+h^2 \sigma_i^2 J^2)^{-1} $\;
   $ \text{LUM}=LU(M)$\;
   \BlankLine  
   \While{ (\text{not konbergentzia})}
   {
    \BlankLine 
    $k=k+1$\;
    \BlankLine
    $g= (hB F(L)-L )$\;
    \BlankLine
    $R=Q_1^T \ g  + (h D Q_2^T) \ g \ J^T $\;
    $d=h \ J \sum\limits_{i=1}^{m}\alpha_i (I_d+h^2\sigma_i^2J^2)^{-1}R_i$\;
    $\text{Solve}(M \Delta z = d)$\;
    \BlankLine 
    $W^{'}_i=(I_d+h^2\sigma_i^2J^2)^{-1} (R_i+\frac{\alpha_i}{2} \ \Delta z), \ i=1,\dots,m$\;
    \BlankLine
    $W^{``}= (-hD^T) \ W^{`} \ J^T+Q_2^T \ g$\;
    \BlankLine
    $\Delta L =B Q_1 \ W^{`}+B Q_2 \ W^{``} $\;
    $L^{[k]}=L^{[k-1]}+\Delta L^{[k]}$\;
    $Y_{n,i}^{[k]}=y_{n} + \ \big(e_n+\sum\limits_{j=1}^{s} \mu_{ij} L_{n,j}^{[k]}\big)  $\;  
    $\text{konbergentzia} \leftarrow \text{GeratzeErizpidea}(L^{[k]},L^{[k-1]},\Delta_{min}) $\;
   }
 \BlankLine
   \If{($\exists j \ \text{non} \ \Delta_j^{[K]}\neq 0$)}
   {
    \If{$(NormalizeDistance(Y^{[k]},Y^{[k-1]})>1$}
    {$\text{fail convergence}$\;}
   }
   $(\tilde y_{n+1}, e_{n+1})\leftarrow \text{baturakonpensatua}(\tilde y_{n},e_{n},L_{n}^{[k]})$\;
 }
 \caption{IRK (NSS-EraginkorraV1).}
 \label{alg:IRK-Newton-Li}
\end{algorithm}


\clearpage


\section{IRK Sasi-Newton.}


\subsection*{Sarrera.}

Doitasun laukoitzerako abantaila asko izango dituen proposamen berria egingo dugu. Doitasun laukoitzeko exekuzioetan, funtzio ebaluaztapena oso garestia da eta funtzio ebaluaztapen kopurua gutxitzea bilatuko dugu.

Newton osoa aplikatzen dugunean pena mereziko du (Newton osoak ez du atalen hasieraketa ona behar eta Newton sinplifikatuak bai).

Ideia nagusia hauxe da; Newton osoaren iterazioaren ekuazio sistema, modu iteratiboan askatzea,
\begin{align*}
&\Delta L_i^{[k]} - h b_i \ J_i^{[k]} \sum_{j=1}^{s} \mu_{ij}  \ \Delta L_j^{[k]} = g_i^{[k]}  , \ \ i=1,\dots,s,\\
&J_i^{[k]}=\frac{\partial f}{\partial y} (t+c_ih, Y_i^{[k]}), \ \  i=1,\dots,s,\\
\end{align*}

Notazio berria finkatzen badugu, hau da askatu beharreko ekuazio sistema,
\begin{equation*}
M^{[k]} \Delta L^{[k]}=g^{[k]},
\end{equation*}

eta $G()$  funtzioa era honetan definituz,
\begin{align*}
G(\Delta L):&=g-M \ \Delta L,
\end{align*}
askatu nahi dugun ekuazioa,
\begin{align*}
G(\Delta L) \ &=0.
\end{align*}

Aurreko atalean,  $M^{[k]} \approx \tilde{M}$  hurbilketa, modu eraginkorrean askatzen ikusi dugu, 
\begin{align*}
\tilde{M} \ =(I_s \otimes I_d - h \ BAB^{-1} \otimes J).
\end{align*}

Eta beraz,
\begin{algorithm}[h!]
 \BlankLine
  $\Delta L^{[0]}=0$\;
  \For{$l=0,1,2,\cdots$}
  {
    $\Delta L^{[l+1]} = \Delta L^{[l]}+ \tilde{M}^{-1} \ G^{[l]}$\;
  }
 \caption{.}
\end{algorithm}

\subsection*{Sasi-Newton.}


\begin{algorithm}[h!]
  $ \Delta L^{[k,1]} = \tilde{M}^{-1} \ g^{[k]}$\;
  \BlankLine
  \For{ (l=2,3,\dots)}
  {
   \BlankLine
   $G^{[k,l]} = g^{[k]}- M^{[k]} \Delta L^{[k,l-1]}$\;
   \BlankLine
   $\Delta L^{[k,l]}=\Delta L^{[k,l-1]}+ \tilde{M}^{-1} \ G^{[k,l]}$\;
  }
 \caption{Sasi-Newton Benetakoa.}
 \label{alg:901}
\end{algorithm}

Non
\begin{align*}
 G_i^{[k,l]} &=g_i^{[k]}- (\Delta L_i^{[k,l-1]}-hb_i J_i^{[k]} \sum_{j=1}^{s} \mu_{ij} \Delta L_{j}^{[k,l-1]}), \ \ i=1,\dots,s,\\
 G_i^{[k,l]} &=J_i^{[k]} \cdot (\sum_{j=1}^{s} hb_i\mu_{ij} \Delta L_{j}^{[k,l-1]})+(g_i^{[k]}-\Delta L_i^{[k,l-1]}), \ \ i=1,\dots,s.
\end{align*}

\paragraph*{} Proposamen berri honetan,
\begin{itemize}
\item Iterazioa=$1$: \emph{Newton Super-Sinplifikatua} (NSS-eraginkorra).
\item Iterazioa=$\infty$: \emph{Newton Benetako} (NB-eraginkorra).
\item Iterazioa=$2$ edo $3$: \emph{Sasi-Newton Benetakoa} (SNB-eraginkorra).
\end{itemize}

\section{Laburpena.}
\chapter{IRK: Eguzki-sistema.}


\section{Sarrera.}
  

Kapitulu honetan, eguzki-sistemaren ekuazio diferentzialei Kepler-en fluxuan oinarritutako aldagai aldaketa aplikatzea proposatuko dugu. \ref{chap:IRK-PF}~kapituluan puntu-finkoaren iterazioan oinarrituz eta \ref{chap:IRK-NEW}~kapituluan Newton sinplifikatuaren iterazioan oinarrituz, IRK inplementazioak garatu ditugu; eguzki-sistemaren problemaren integraziorako, bi inplementazioen artean, puntu-finkoarena nabarmen eraginkorragoa dela baieztatu dugu eta Newton sinplifikatuarena, erabat baztertu dugu. Ondorioz, puntu-finkoaren iterazioan oinarritutako IRK inplementazioa erabiliko dugu eta  ekuazio diferentzialetako aldagaiei eragingo diegun aldagai aldaketaren bidez, integrazio eraginkorra lortzea espero dugu.  

Aplikatzen dugun integrazio metodoa sinplektikoa eta simetrikoa da: neurri batean, splitting metodoen antzekoa. Aldagai berriekiko ekuazio diferentzialak, magnitude txikiko balioak hartzen dituzte eta honek, hiru abantaila eragingo ditu. Lehenik, eguzki-sistemaren problemaren trunkatze errore nagusiena ezabatzen dugunez, urrats luzera handiagoak erabili ahal izango ditugu. Bigarrenik, batura konpensatuaren konputazioan, informazio gutxiago galduko dugu. Hirugarrenik, puntu-finkoaren iterazioek konbergentzia azkarra izango dute. 

Lehenengo, Kepler-en fluxuaren inplementazioa azalduko dugu. Bigarrenik, aldagai aldaketa definitu eta metodoa integratzeko zehaztapenak emango ditugu. Hirugarrenik, eguzki-sistemaren problemaren zenbakizko integrazioak egingo ditugu: inplementazio honen eta doitasun altuko beste metodo sinplektikoen (konposizio eta splitting metodoak) eraginkortasunak, alderatuko ditugu.     

 

\section{Kepler-en fluxua.}
   
   
Kepler problema bi gorputzen problemaren kasu partikularra da eta  honako Hamiltondarra dagokio,
\begin{equation}
\label{eq: hamkepler}
H(q,p)=\frac{p^2}{2m}-\frac{\mu}{\|q\|},
\end{equation}
non $m$ eta $\mu$ konstanteen balioak  formulazioaren araberakoak diren.

Koordenatu sistema $q=q_2-q_1$ duen formulazioa aukeratzen badugu, konstanteen balioak hauek dira,  
\begin{equation*}
m=(1/m_1+1/m_2)^{-1},\ \ \mu=Gm_1m_2,
\end{equation*} 
%
eta ekuazio diferentzialak era honetan definitzen dira,
\begin{equation}
\label{eq:kode}
\dot{q}=p, \ \ \dot{p}= - \frac{k \ q}{\|q\|^3} ,
\end{equation}
non $k= \mu / m$ eta  $q,p \in \mathbb{R}^3$ diren.

Kepler problemaren soluzio zehatza kalkula daiteke: une bateko kokapen eta abiadurak emanik, $\Delta t$ denbora tarte bat igarotakoan (positiboa ala negatiboa), kokapen eta abiadura zehatzak konputatu daitezke. Eguzki-sistemaren integrazioetarako, Kepler problema doitasun handian eta era eraginkorrean kalkulatzea, funtsezkoa da. Kepler problemaren erreferentziazko inplementazioak, Danby \cite{Danby1992} eta J.Wisdom-enak  \cite{Wisdom2015} ditugu. 

Kepler-en fluxua, era honetan kalkulatzen da. Lehenik, koordenatu cartesiarretatik ($q,p\in \mathbb{R}^3$), koordenatu eliptikoetara $(a,e,i,\Omega,E)$ itzulpena egingo dugu. Koordenatu eliptikoetan, $E$ (\emph{eccentric anomaly}) aldagaia izan ezik, beste aldagaiak konstante mantentzen dira: beraz, $E_0$ balioa emanda, $\Delta t$ denbora tartea aurrera egin eta $E_1$ balio berria kalkulatuko dugu. Azkenik, koordenatu eliptikoetatik koordenatu cartesiarretara itzulpena eginez, kokapen eta abiadura berriak eskuratuko ditugu. 

\begin{equation*}
(q_0,v_0) \in \mathbb{R}^6 \ \ \ \longrightarrow \ \ \  (a,e,i,\Omega,E_0) \in \mathbb{R}^6 
\end{equation*}
\begin{equation*}
\quad \quad \quad \quad \quad \quad \quad \quad \downarrow \Delta t
\end{equation*}
\begin{equation*}
(q_1,v_1) \in \mathbb{R}^6 \ \ \ \longleftarrow \ \ \  (a,e,i,\Omega,E_1) \in \mathbb{R}^6 
\end{equation*}

Gorputz baten orbita Kepleriarra hiru motakoa izan daiteke: $H(q_0,p_0)<0$ denean orbita eliptikoa da, $H(q_0,p_0)>0$ orbita hiperbolikoa eta $H(q_0,p_0)=0$ orbita  parabolikoa. Kepler fluxuaren C inplementazioa, orbita eliptikoetarako garatu dugu eta zehaztasunak, \ref{erans:B1} eranskinean eman ditugu. (\ref{eq:kode}) problemari dagokion fluxua, era honetan defini daiteke,
\begin{align*}
\varphi_{\Delta t}^k:&  \quad \mathbb{R}^{6} \quad  \longrightarrow \quad \mathbb{R}^6,  \\
&  \quad u_0 \ \  \rightsquigarrow \ \ u_1. 
\end{align*} 
non $u=(q,v) \in \mathbb{R}^6$  den.

\section{Kepler Perturbatuaren problema.}

Kepler problemaren Hamiltondarra perturbatzen badugu, ezingo dugu aurreko atalean erabili dugun fluxua erabili problema ebazteko. Kasu honetan Hamiltondarra  bi zatitan banatuta egongo da;
\begin{align}
\begin{split}
\label{eq: hamkeplerpert}
&H(q,p,t)=H_K(q,p)+H_I(q,p,t)
\end{split}
\end{align} 
non $H_K$ mugimendu Kepleriarrari dagokion Hamiltondarraren aldea den, hau da, (\ref{eq: hamkepler}) ekuazioko eskuin aldea, eta $H_I$ perturbazioei dagokien Hamiltondarraren aldea den.

Problema berri honetan aldagai aldaketa bat egingo dugu, aldaketaren helburua da Keplerren fluxua erabili ahal izatea problemaren ebazpenean. 

\subsection*{Aldagai aldaketa.}

%Kontutan hartuko ditugun sistemak Hamiltondarrak dira, $H: \mathbb{R} \times \mathbb{R}^{2d} \longrightarrow \mathbb{R}$, gainera, Hamiltondarra bi zatitan bana dakieken sistemak hartuko ditugu kontuan:

(\ref{eq: hamkeplerpert}) problemari dagozkion ekuazioetan, Keplerren fluxuan oinarritutako aldagai aldaketa bat egingo dugu, baina horretarako notazioa finkatuko dugu: jatorrizko aldagaiak $u=(q,p) \in \mathbb{R}^{2d}$ izango dira eta aldagai berriak $U=(Q,P) \in \mathbb{R}^{2d}$ letra larriz adieraziko ditugu. Jatorrizko aldagaien bidez adierazitako problema, alegia, ebatzi beharreko hasierako baliodun problema, honakoa da:

\begin{align}
\begin{split}
\label{eq: HamEDA}
&\frac{du}{dt} = k(u) + g(u,t),\ \ \ u(0) = u_0
\end{split}
\end{align} 
non $k(u)$ (\ref{eq:kode}) ekuazioan adierazitako alde Kepleriarrari dagokion  eta $g(u,t)$ perturbazioari. 
Problema horretan honako aldagai aldaketa egingo dugu, kontuan izan urrats bakoitzean egingo dugula aldagai aldaketa, hau da $j=0, 1, 2 \ldots$ indizeak $j$. urratsean aplikatu beharreko aldaketa adierazten du:

\begin{align}\begin{split}
\label{eq: uUaldaketa}
&u(t) = \varphi_{t-(j+\frac{1}{2})h}^k\left(U_j^{j+\frac{1}{2}}(t)\right)
\end{split}
\end{align} 
$\varphi_{\Delta t}^k$ fluxua $\Delta t>0$ eta $\Delta t <0$ balioentzat definitzen da, eta  $u= \varphi_{-t}(\varphi_{t}(u))$ betetzen dela kontutan hartuz honako alderantzizko aldaketa ere egin dezakegu:
\begin{align}
\begin{split}
\label{eq: Uualdaketa}
U^{j+\frac{1}{2}}(t) = \varphi^k_{-t+(j+\frac{1}{2})h} \left( u(t) \right)
\end{split}
\end{align} 

Aldaketa hauekin asmoa da $i+1$ urratsa emateko $u_i \approx u(hi)$ zenbakizko soluzioan oinarrituz, aldagai aldaketaren bidez $U_i^{i+\frac{1}{2}}=\varphi^k_{\frac{h}{2}}(u_i)$ lortu, hau da, fluxuan $\frac{h}{2}$ aurrera egin aldagai berriak lortzeko, aldagai berri hauetan ebatzi jatorrizko problemaren urrats bati dagokion zenbakizko soluzioa (ikusiko dugun bezala, aldagai berrietan alde Kepleriarrari dagokion espresioak ez du eraginik eta, azken finean perturbazioari dagokion aldaketa da hemen kalkulatuko dena) eta azkenik, aldagai berri hauen balio berriak jatorrizko aldagaietara itzuli behar dira, baina $i+1$ urratsari dagozkion unera pasa behar dira aldagaiak, hau da, fluxuan aurrera $\frac{h}{2}$ egin behar da. Atzera egingo bagenu urratsaren hasierako balioei perturbazioak zein aldaketa eragiten dien kalkulatuko baikenuke, baina guk urratsaren bukaerako balioak nahi ditugu. Laburbilduz:


\begin{align*}
   &   \quad U_0^{\frac{1}{2}} \quad \quad \quad \quad \Longrightarrow  & U_1^{\frac{1}{2}}&  &\\
  & \nearrow \varphi_{\frac{h}{2}}(u_0) &            & \searrow \varphi_{\frac{h}{2}}(U_1)& \\
u_0 &                  &    &\quad \quad \quad  u_1
\end{align*}

Aldagai aldaketak fluxuan aurrera egiten du urratsaren luzeraren erdia. Hori horrela egiteak badu arrazoi bat: urratsa bere osotasunean simetrikoa da. Aurrera $h$ luzerako urratsa ematea $-h$ luzerako urratsa ematearekin desegiten baita. 

%\begin{figure}[h!]
%\centering
%\subfloat[Aldagai aldaketa.]{
%\includegraphics[width=.400\textwidth]{Aldagaialdaketa1}
%}
%\subfloat[Aldagai berrien integrazioa.]{
%\includegraphics[width=.400\textwidth]{Aldagaialdaketa2}
%}
%\caption[Atalen hasieraketa.]
%        {\small (a)irudian, aldagai aldaketa irudikatu dugu eta (b) irudian, perturbatutako gorputza baten orbitaren integrazioak erakutsi ditugu. Bi irudietan, $(Q,P)$ balioen aldaketa txikiak gorriz nabarmendu ditugu          
%        }
%\label{fig:Aldg}
%\end{figure}   

\subsection*{Aldagai berrietan ekuazio diferentzialak.}

(\ref{eq: uUaldaketa}) aldagai aldaketa abiapuntutzat hartuz, aldagai berriei dagozkien ekuazio diferentzialak lortu behar ditugu. Horrela, problemaren integrazioa aldagai berrien arabera egin ahal izango dugu. Irakur erraztasunagatik (\ref{eq: uUaldaketa}) ekuazioko $\varphi(U)$ indizerik gabe idatziko dugu, eta $\dot{U}$ri dagozkion ekuazioak lortze aldera bi aldeak $t$ aldagaiarekiko deribatuko ditugu:  
\begin{align}
\begin{split}
&\frac{d}{dt}u = \frac{d}{dt}\left(\varphi(U)\right),
\end{split}
\end{align}
Eskuin aldeari katearen erregla aplikatuz, 
\begin{align}
\begin{split}
&\dot{u} = \dot{\varphi}(U) + \varphi'(U) \frac{d}{dt}U.
\end{split}
\end{align}
$\varphi$ Kepler problemaren fluxua da, hau da, $\dot{u} = k(u)$ problemaren fluxua da, eta fluxuaren definizioz $\dot{\varphi}(U) = k(\varphi(U))$ da. Aldaketa horrekin,  eta (\ref{eq: HamEDA}) ekuazioarekin berdinduz,
\begin{align}
\begin{split}
&k(u) + g(u,t) = k(u) + \varphi'(U) \dot{U}.
\end{split}
\end{align}
Bi aldeetan $k(u)$  kenduz, $U$ aldagaiekiko ebatzi beharreko ekuazio diferentziala lortuko dugu:
\begin{align}
\begin{split}
\label{eq:hamEDAU}
&\dot{U} = \left(\varphi'(U)\right)^{-1} g(u,t).
\end{split}
\end{align}
Alderantzizko matrizeak kalkulatu beharrik gabe idatz ditzakegu (\ref{eq:hamEDAU}) ekuazioak. Horretarako $\varphi$ fluxuaren izaera sinplektikoa erabiliko dugu, hau da, $(\varphi')^tJ\varphi'= J$ propietatea betetzen du fluxuak,
non, 
\begin{equation*}
 J=\left(\begin{array}{cc}
   \ 0 & \ -I \\
     I & \ 0  \\
\end{array}\right),
\end{equation*}
ondorioz,
%
\begin{align}
\begin{split}
\label{eq:hamEDAU2}
&\dot{U} = J^{-1}(\varphi')^{t}(U)J g(u,t).
\end{split}
\end{align}
(\ref{eq:hamEDAU2}) ekuazioak $\varphi'(U)$ kalkulatzea eskatzen du, eta horretarako deribazio automatikoko teknikak erabil ditzakegu.


\paragraph*{Algoritmoa.}
$U$ aldagaietan oinarritutako ekuazio diferentzialen integraziorako (\ref{eq:hamEDAU2}) espresioaren konputazioa hiru urratsetan egingo dugu:
\begin{enumerate}
\item $\{u,aux\} \leftarrow KeplerFlowGen (t,U,mu)$.

Kepler-en fluxua $u= \varphi_t(U)$ aplikatuko dugu eta fluxuaren kalkulutarako erabilitako tarteko balioak, ~$aux\in \mathbb{R}^{16}$ aldagaian itzuliko ditugu. 

\item $g \leftarrow g(u,t)$.

Jatorrizko problemako ekuazio diferentzialetan perturbazioei dagokien espresioa kalkulatuko dugu.

\item $KeplerFlowGFcnaux(aux,U,t,g)$.

Urrats honetan $\varphi'_t()$ kalkulatu behar da. Deribazio automatikoaren tekniken bidez, Kepler fluxuaren $U$ aldagaiekiko deribatuaren konputazio eraginkorra definitu dugu. 

Hirugarren urratsak (\ref{eq:hamEDAU2}) espresioa konputatzeko behar dugun azken zatia kalkulatzen du, beraz, bere emaitza $\dot{U}$ren konputazioa izango da: 
\begin{align*}
\dot{U}&\leftarrow KeplerFlowGFcnaux(aux,U,t,g).
\end{align*}

\end{enumerate} 



\section{Alde Kepleriar bat baino gehiagoko sistemak.}

Alde Kepleriar bat baino gehiago dituzten problemetan ere Kepler perturbatuan egindako aldagai aldaketa egin dezakegu. Hainbat gorputzeko sisteman gorputz bakoitzari eragingo diogu aldagai aldaketa, bakoitzak bere fluxu Kepleriar perturbatua izango du, eta horretan oinarrituz egingo diogu aldaketa. Problemaren alde Kepleriarren kopurua $k$ bada, era honetako ekuazio diferentzialak ditugu,
\begin{equation}
\label{eq: n-pertEDA}
\frac{d}{dt}\left(\begin{array}{c}
                u  \\
                w  \\
\end{array}\right)=
\left(\begin{array}{c}
                \dot{u}_1  \\
                \dot{u}_2  \\
                \vdots \\
                \dot{u}_k    \\
                \dot{w}      \\
\end{array}\right)=
\left(\begin{array}{c}
                k^{\mu_1}(u_1)  \\
                k^{\mu_2}(u_2)  \\
                \vdots \\
                k^{\mu_k}(u_k)  \\
                0      \\
\end{array}\right)+
\left(\begin{array}{c}
      g_1(u_1, u_2\dots, u_k,w,t) \\
      g_2(u_1, u_2\dots, u_k,w,t) \\
                \vdots \\
      g_k(u_1, u_2\dots, u_k,w,t)\\
      g_{k+1}(u_1, u_2\dots, u_k,w,t)
\end{array}\right)
\end{equation} 
 
Gorputz bakoitzari dagokion aldagai aldaketa, bere $k^{\mu_j}(u_j)$ fluxu Kepleriarraren araberakoa da,
\begin{align}
\label{eq:aldfl2}
\begin{split}
u_j&= \varphi_t^{\mu_j}(U_j), \ \ \ j=1,\dots,k.
\end{split}
\end{align}
Bi aldeak $t$ aldagaiarekiko deribatuz eta katearen erregela aplikatuz,
\begin{equation*}
\left(\begin{array}{c}
                \dot{u}_1  \\
                \dot{u}_2  \\
                \vdots \\
                \dot{u}_k    \\
                \dot{w}      \\
\end{array}\right)=
\left(\begin{array}{c}
                \dot{\varphi}^{\mu_1}(U_1)  \\
                \dot{\varphi}^{\mu_2}(U_2)   \\
                \vdots \\
                \dot{\varphi}^{\mu_k}(U_k)   \\
                0      \\
\end{array}\right)+
\left(\begin{array}{c}
      (\varphi^{\mu_1})'(U_1) \frac{d}{dt}U_1 \\
      (\varphi^{\mu_2})'(U_2) \frac{d}{dt}U_2 \\
                \vdots \\
      (\varphi^{\mu_k})'(U_k) \frac{d}{dt}U_k\\
      g_{k+1}(u_1, u_2\dots, u_k,w,t)
\end{array}\right),
\end{equation*}
ekuazioetan fluxuen propietate eta definizioak erabiliz,
\begin{equation*}
\left(\begin{array}{c}
                \dot{u}_1  \\
                \dot{u}_2  \\
                \vdots \\
                \dot{u}_k    \\
                \dot{w}      \\
\end{array}\right)=
\left(\begin{array}{c}
                k^{\mu_1}(u_1)  \\
                k^{\mu_2}(u_2)  \\
                \vdots \\
                k^{\mu_k}(u_k)  \\
                0      \\
\end{array}\right)+
\left(\begin{array}{c}
      (\varphi^{\mu_1})'(U_1) \dot{U}_1 \\
      (\varphi^{\mu_2})'(U_2) \dot{U}_2 \\
                \vdots \\
      (\varphi^{\mu_k})'(U_k) \dot{U}_k\\
      g_{k+1}(u_1, u_2\dots, u_k,w,t)
\end{array}\right),
\end{equation*}
eta, azkenik, (\ref{eq: n-pertEDA}) ekuazioarekin berdinduz eta sinplifikatuz,
\begin{equation*}
\left(\begin{array}{c}
                \dot{U}_1  \\
                \dot{U}_2  \\
                \vdots \\
                \dot{U}_k    \\
                \dot{w}      \\
\end{array}\right)=
\left(\begin{array}{c}
      \left((\varphi^{\mu_1})'(U_1)\right)^{-1} g_1 \\
      \left((\varphi^{\mu_2})'(U_2)\right)^{-1} g_2 \\
                \vdots \\
     \left((\varphi^{\mu_k})'(U_k)\right)^{-1} g_k\\
      g_{k+1}
\end{array}\right),
\end{equation*}

Aldagai berriekiko ekuazio diferentzialak balioztatzeko, fluxuen propietateei esker, $(\varphi')^{-1}=J^{-1}(\varphi')^tJ$ kalkula dezakegu eta, Kepler perturbatuan bezalaxe, alderantzizko matrizerik kalkulatu beharrik ez dugu izango. Deribazio automatikoko teknikei esker, kalkulatu ahal izango ditugu. Bestalde, $\dot{w}$ aldagaien ekuazioak balioztatzeko $u_i$ aldagaiak behar ditugu, baina $U_i$ aldagaietatik lor ditzakegu, gainera, $g_i$ funtzioetarako ere behar ditugu. Ondorioz, Kepler perturbatuaren probleman bezala hiru urratsetan balioztatu ahal izango ditugu ekuazioak.



\subsection*{Metodo simetrikoa.}

Azpimarratu behar dugu aldagai aldaketa urratsero egiten dugula, eta ekuazio diferentziala aldagai berriekiko ebazten dugula. Aldagai berriak eta jatorrizko aldagaiak fluxuak erlazionatzen ditu: jatorrizko aldagaiak fluxuan $\frac{h}{2}$ aurrera eginez aldagai berriak lor ditzakegu. Ebatzi beharreko problema aldagai berrietan ebatziko dugu, eta $h$ luzerako urratsa emanez aldagai berriak aldatuko ditugu. Jatorrizko aldagaietara igarotzeko fluxuaren bidez mugitu behar ditugu balio horiek: $\frac{h}{2}$ atzera egiten badugu jatorrizko aldagaiak urratsaren hasieran kokatuko ditugu, baina perturbazioari dagokion aldaketa bere baitan daramate, izan ere, aldagai berriei metodoaren urratsa kalkulatu diegu, hau da, perturbazioari dagokion $h$ luzerako urratsa eman dugu. Bestalde, fluxuan $\frac{h}{2}$ aurrera egiten badugu hurrengo urratsaren hasierako egoerara eramango ditugu balioak.

Ondorioz integrazioko urrats bat hiru azpi-urratsen konbinazioa da:
\begin{enumerate}
\item $U_i^{i+\frac{1}{2}}=\varphi_{\frac{h}{2}}(u_i)$: fluxuaren arabera $\frac{h}{2}$ aurreratu.
\item Gauss metodoaren $h$ luzerako urratsa: $U_i^{i+\frac{1}{2}}$ aldagaietatik  $U_{i+1}^{i+\frac{1}{2}}$ balioetara pasako gara.
\item $u_{i+1}=\varphi_{\frac{h}{2}}(U_{i+1}^{i+\frac{1}{2}})$ fluxuaren araberako $\frac{h}{2}$ aurreratu.
\end{enumerate}

Hiru azpiurratsak simetrikoak dira, eta ondorioz, $u_{i+1}$ abiapuntutzat hartuz, $-h$ luzerako urratsa ematen badugu $u_i$ lortuko dugu:
\begin{enumerate}
\item $\varphi_{\frac{-h}{2}}(u_{i+1})$ kalkulatu behar da, baina $u_{i+1}$ hirugarren azpiurratsaren emaitza denez, bere espresioa jarriko dugu, eta ikusiko dugu $U_{i+1}^{i+\frac{1}{2}}$ aldagaietatik hasi eta fluxuan aurrera eta atzera egitearen parekoa dela, alegia, ez aldatzearen parekoa:
\[
U_{i+1}^{i+1+\frac{1}{2}}=\varphi_{\frac{-h}{2}}(u_{i+1})=\varphi_{\frac{-h}{2}}\left(\varphi_{\frac{h}{2}}(U_{i+1}^{i+\frac{1}{2}}) \right)
= U_{i+1}^{i+\frac{1}{2}}
\] 
\item Gauss metodoaren $-h$ luzerako urratsa: Gaussen metodoa simetrikoa denez, aurreko urratsean bukaerako egoera zenari $-h$ luzerako urratsa eragiteak hasierako egoerara itzultzen du, beraz, $U_{i+1}^{i+\frac{1}{2}}$ balioetatik abiatuz $U_i^{i+\frac{1}{2}}$ balioetara itzuliko gara.
\item Fluxuan urrats luzeraren erdia egin behar da: lehenengo azpiurratsean bezala, fluxuan $\frac{-h}{2}$ mugitu behar gara, baina fluxuan $\frac{h}{2}$ mugitutako balioekin egin behar dugu, hau da:
\[
\varphi_{\frac{-h}{2}}(U_i^{i+\frac{1}{2}}) = \varphi_{\frac{-h}{2}}\left(\varphi_{\frac{h}{2}}({u_i}) \right) = u_i
\]
\end{enumerate}

Metodoaren integrazio eskema orokorra \ref{fig:proiekzioa0}~irudian laburtu dugu, bere simetria ere bertan ikus daiteke,
\begin{figure} [h!]
{\includegraphics [width=16cm, height=4cm] {proiekzioa11}}
\caption[Aldagai aldaketa: metodo simetrikoa eta sinplektikoa]{\small Metodoaren integrazio eskema orokorra. Metodoa simetrikoa eta sinplektikoa da.}
\label{fig:proiekzioa0}
\end{figure} 

Integrazioaren urrats guztietan ez baditugu emaitzak itzuli behar, bi urratsen arteko, $\varphi_{h/2}$ fluxuaren bi konputazioak, $\varphi_{h}$ fluxuaren konputazio bakarrarekin konputatuko dugu. Horretarako, proiekzio kontzeptua sortuko dugu (\ref{fig:proiekzioa2}~irudia).

\begin{figure} [h!]
{\includegraphics [width=14cm, height=4cm] {proiekzioa12}}
\caption[Aldagai aldaketa: proiekzioa]{\small Proiekzioa: bi urratsen arteko, $\varphi_{h/2}$ fluxuaren bi konputazioak, $\varphi_{h}$ fluxuaren konputazio bakarrarekin konputatuko dugu}
\label{fig:proiekzioa2}
\end{figure} 


 Azkenik, emaitzak behar ditugun urratsetarako fluxua $\varphi_{-h/2}$ aplikatuko dugu (\ref{fig:proiekzioa1}~irudia). 

\begin{figure} [h!]
{\includegraphics [width=14cm, height=5cm] {proiekzioa1}}
\caption[Aldagai aldaketa: urratsak]{\small $u_i$ jatorrizko aldagaiak eta $U_i$ aldagai berriak dira. Lehenengo, $u_0$ jatorrizko aldagaien hasierako baliotik abiatuta, aldagai berriei dagokion $U_0^{1/2}$ hasierako balioa finkatuko dugu. Urrats bakoitza, integrazio eta proiekzioaren konposaketa da, \ref{fig:proiekzioa2}~irudian zehaztu dugun bezala. Erabiltzaileak definitutako urratsetarako, $u_n$ jatorrizko aldagaietan zenbakizko soluzioa itzuliko dugu}
\label{fig:proiekzioa1}
\end{figure} 


$u_i$ jatorrizko aldagaiak eta $U_i$ aldagai berriak adierazten duten notazioa erabiliko dugu. Hauek dira, integratzeko emango ditugun urratsak:
\begin{enumerate}
\item \emph{Startfun} funtzioa.

Lehenengo, $u_0$ jatorrizko aldagaien hasierako baliotik abiatuta, $\varphi_{h/2}$ fluxuaren konputazioaren bidez, aldagai berrietan dagokion hasierako balioa lortuko dugu.
\begin{align*}
u_0 \ \rightarrow \ U_0^{\frac{1}{2}}.
\end{align*}

\item \emph{Urratsa}.

Urratsa bi azpiurratsen konbinazio bezala ikusiko dugu: aldagai berriei Gaussen metodoaren bidezko integrazioaren urrats bat eta lortutako maitzei $\varphi_{h}$ bidez fluxuan $h$ aurrera egitea. Bigarren azpiurratsa fluxuaren bidez proiektatzea da. \ref{fig:proiekzioa2} irudian zehaztapenak eman ditugu. 
\begin{align*}
 U_0^{\frac{1}{2}} \ \rightarrow \ U_1^{\frac{1}{2}} \rightarrow \ U_1^{1+\frac{1}{2}}.
\end{align*}

Biribiltze errorea txikitzeko, proiekzioa doitasun altuan konputatzea garrantzitsua da. Modu honetan, batura konpensatua aplikatzerakoan zifra batzuk irabaziko ditugu. 

\item \emph{Outputfun} funtzioa.

Erabiltzaileak, $t$-ren balio jakin batzuetan $u(t)$ balioen zenbakizko soluzioak nahiko ditu, kasu horietan $\varphi_{-h/2}$ fluxuaren konputazioaren bidez, $U_i^{i+\frac{1}{2}}$ balioetatik $u_i$ jatorrizko aldagaien balioak lortuko dira:
\begin{align*}
U_n^{n+\frac{1}{2}} \ \rightarrow \ u_n.
\end{align*}


\end{enumerate}


Gauss metodoa, neurri batean  splitting eta konposizio metodoen antzekoak dira. 
\begin{align*}
\text{Konposizio metodoa} \ \ &\equiv \ \ \text{Gauss metodoa aldagai aldaketa gabe}.\\
\text{Splitting metodoa} \ \ &\equiv \ \  \text{Gauss metodoa aldagai aldaketarekin}.
\end{align*}

Splitting metodoekiko antzekotasuna azaltzeko, (\ref{eq:stverlet})~\emph{Störmer-Verlet} splitting metodoarekin konparatuko dugu. \emph{Störmer-Verlet} metodoa, era honetan aplikatzen da: $h/2$ fluxua aplikatu, perturbazioak kalkulatu eta berriz  $h/2$ fluxua aplikatu. Fluxuaren aldagai aldaketarekin, gauza bera egiten ari gara: $h/2$ fluxua aurreratu, perturbazioak kalkulatu (aldagai berrietan eta beraz, hobeto kalkulatzen dugu), $h/2$ fluxua aurreratu. 


\section{Zenbakizko esperimentuak.}
\label{s:7espmt}

Zenbakizko esperimentuetarako, puntu-finkoaren iterazioan oinarritutako Gauss metodoaren inplementazioa (\ref{chap:IRK-PF}~kapitulua) erabili dugu. Lehenengo, Gauss metodoaren bi bertsio konparatu ditugu: aldagai aldaketarik gabeko integrazioa eta aldagai aldaketa aplikatutako integrazioa. Kepler fluxuan oinarritutako aldagai aldaketarekin integrazioaren abantaila argia denez, hurrengo esperimentuetarako aldagai aldaketa aplikatutako integrazioa bakarrik hartu dugu kontutan.  
 
$s=6,8,9,16$ ataletako Kepler-en fluxuan oinarritutako aldagai aldaketa aplikatutako Gauss metodoen arteko metodo eraginkorrena aukeratu dugu, \emph{CO1035} konposizio eta \emph{ABAH1064} splitting  metodoekin konparatzeko.

Gauss metodoen konputaziorako, $64$-biteko (\emph{double}) eta $80$-biteko (\emph{long double}) doitasunak nahasi ditugu. Konputazioaren zati nagusiena, $64$-biteko doitasunean egin dugu eta proiekzioa kalkulatzeko, $80$-biteko doitasuna aplikatu dugu. Era honetan, modu merkean soluzioaren doitasuna hobetzea lortu dugu.

Gauss metodoaren exekuzio sekuentziala eta paraleloak egin ditugu. $s$ atalen funtzioen balioztapena, 
\begin{align*}
F_{n,i}=f(Y_{n,i}), \ i=1,\dots,s,
\end{align*}      
independenteak dira eta paraleloan kalkula daitezke. $s=8$ metodoaren integrazio paraleloak egin ditugu eta hari kopurua $2$ aplikatu dugu. 

\subsection{Problemak.}


9-planeten problema (\ref{sss:9body}~atala) erabili dugu integrazioetarako. Hasierako balioak \emph{DE-430} efemerideen artikulutik hartu ditugu: planeten masak  \ref{tab:9bodymas}~taulan laburtu ditugu; eta hasierako kokapen eta abiadurak \ref{tab:9bodyhas}~taulan aurki daitezke.

Koordenatu heliozentrikoei dagokien  Hamiltondar sistema (\ref{eq:nbodyHel}),
\begin{align*}
&H(q,p)=H_K(q,p)+H_I(q,p),
%&H(q,p)=\sum\limits_{i=1}^{N}\bigg(\frac{\|P_i\|^2}{2 \mu_i} -\frac{G m_0 m_i}{\|Q_i\|}\bigg)+H_I(q,p)
\end{align*}
integratu dugu. $H_K(q,p)$ mugimendu Kepleriarrari dagokion Hamiltondarraren aldea da  eta $H_I(q,p)$, perturbazioei dagokien Hamiltondarraren aldea da. 
%Kepler-en fluxuan oinarritutako aldagai aldaketaren bidez, alde Kepleriarra ekuazioetatik desagerrarazten dugu.    

Integrazioen tartea, $t_{end}=10^6$ egunetakoa da eta zenbakizko integrazioetan, $h$-ren balio ezberdinak erabili ditugu. $s=6$ metodoarentzat urrats luzerak aukeratu ditugu eta gainontzeko metodoentzat, $s$-atalen araberako urrats luzera proportzionalak finkatu ditugu:
\begin{align*}
&s=6: \quad  \ \ h=2^{k/4}, \ k=4,\dots,28, \\
&s=8: \quad  \ \ (8/6)h, \\
&s=9: \quad  \ \ (9/6)h, \\
&s=16: \quad (16/6)h. \\
\end{align*} 

Zenbakizko esperimentuetarako, aldagai aldaketa planeta guziei aplikatzea erabaki dugu. $9$-planeten probleman, gorputz kopurua txikia denez,  Kepler fluxuaren gainkarga esanguratsua da eta  barne-planetei bakarrik aplikatzea, eraginkorragoa izan daiteke. Baina, gorputz gehiago kontsideratzen baditugu (esaterako Ilargia eta asteroide nagusienak) edo eguzki-sistemaren eredu konplexuagoetan (esaterako erlatibitate efektua gehitzerakoan), perturbazio aldearen konputazioa nagusituko da eta Kepler fluxuaren kalkuluak pisua galduko luke. 


\subsection*{Gauss metodoen eraginkortasuna.}


\ref{fig:esp81a}~irudian, $s=6$ ataletako Gauss metodoaren bi bertsioen eraginkortasunak konparatu ditugu: aldagai aldaketarik gabeko integrazioa eta aldagai aldaketa aplikatutako integrazioa. Aldagai aldaketarik gabeko integraziorako, puntu-finkoaren iterazio partizionatua eta interpolazio bidezko hasieraketa aplikatu dugu. Ekuazio diferentzialen ebaluazioen konparaketan, aldagai aldaketa aplikatutako integrazioa, nabarmen eraginkorragoa da. Eredu sinplearen integrazioen exekuzio denboren konparaketan, aldea txikiago da, baina orduan eta eredu konplexuagoa izan abantaila handituz joango da.   

\begin{figure}[h!]
\centering
\begin{tabular}{c c}
\subfloat[ Gauss metodoak (FCN).]
{\includegraphics[width=.45\textwidth]{esperimentua801}}
&
\subfloat[ Gauss metodoak (CPU).]
{\includegraphics[width=.45\textwidth]{esperimentua802}}
\end{tabular}
\caption[Puntu-finkoaren eraginkortasun grafikoak]{\small 
Eraginkortasun grafikoak eskala logaritmiko bikoitzean irudikatu ditugu. Ardatz bertikalean, energiaren errore erlatibo maximoa eman dugu. (a) irudian, ekuazio diferentzialen ebaluazio kopuruarekiko (FCN) eraginkortasuna neurtu dugu. (b) irudian, CPU denborarekiko eraginkortasuna neurtu dugu. Irudi bakoitzean, Gauss metodoaren $s=6$ ataletako bi bertsio konparatu ditugu: aldagai aldaketa gabe berdez eta aldagai aldaketa aplikatuta grisez. }
\label{fig:esp81s}
\end{figure}

Jarraian, $s=6,8,9,16$ ataletako metodoen eraginkortasuna aztertu dugu. \ref{fig:esp81a}~irudian, diferentzialen ebaluazio kopuruarekiko (\emph{FCN}) eta \ref{fig:esp81s}~irudian, \emph{CPU}-denborarekiko (exekuzio paraleloetan \emph{Wall time}) erakutsi dugu. \emph{FCN}-rekiko eraginkortasuna, metodoak problema erreal batean nola jokatuko luke erakusten digu eta \emph{CPU}-rekiko  eraginkortasuna, problema zehatz honetarako gertatzen dena azaltzen digu.


\begin{figure} [h!]
\centerline{\includegraphics [width=8cm, height=6cm] {esperimentua812}}
\caption[Gauss metodoen eraginkortasun konparaketa (FCN)]{\small Eraginkortasun grafikoa eskala logaritmiko bikoitzean irudikatu dugu. Ardatz bertikalean, energiaren errore erlatibo maximoa eman dugu eta ardatz horizontalean, ekuazio diferentzialen ebaluazio kopurua (FCN). Gauss metodoaren $s$ ataletako lau integrazio konparatu ditugu: $s=6$  urdinez, $s=8$ gorriz, $s=9$ berdez, eta $s=16$ grisez}
\label{fig:esp81a}
\end{figure} 


\begin{figure}[h!]
\centering
\begin{tabular}{c c}
\subfloat[ Exekuzioa sekuentziala.]
{\includegraphics[width=.45\textwidth]{esperimentua811}}
&
\subfloat[ Exekuzio paraleloa.]
{\includegraphics[width=.45\textwidth]{esperimentua813}}
%\subfloat[Exekuzio paraleloa (hariak=$2$):Wall Time.]
%{\includegraphics[width=.5\textwidth]{esperimentua813}}
%&
%\subfloat[Exekuzio paraleloa (hariak=$4$): Wall Time.]
%{\includegraphics[width=.5\textwidth]{esperimentua814}}
\end{tabular}
\caption[Gauss metodoen eraginkortasun konparaketa (CPU Time)]{\small 
Eraginkortasun grafikoak eskala logaritmiko bikoitzean irudikatu ditugu. Ardatz bertikalean, energiaren errore erlatibo maximoa eman dugu eta ardatz horizontalean,  CPU denbora (exekuzio paraleloan Wall-Time). (a)  konputazioa modu sekuentzialean egin dugu eta (b) modu paraleloan hari kopurua $2$ izanik. Irudi bakoitzean, Gauss metodoaren $s$ ataletako lau integrazio konparatu ditugu: $s=6$  urdinez, $s=8$ gorriz, $s=9$ berdez, eta $s=16$ grisez. }
\label{fig:esp81s}
\end{figure}

Exekuzio sekuentzialak eta exekuzio paraleloak aztertuz,  Gauss metodo eraginkorrena aukeratu nahi dugu. Horretarako, biribiltze errorea nagusitzen hasten den inguruko unean gertatutakoa aztertu dugu: $s=8,9,16$ metodoak, $s=6$ metodoa baino eraginkorragoak azaldu zaizkigu. $s=8,9,16$ metodoak beraien artean oso antzekoak izanik, $s=8$ ataleko Gauss metodoa aukeratu dugu. Bestalde, exekuzio paraleloa, sekuentziala baino eraginkorragoa da.

\subsection*{Energia errorea eta errore globalak.}

%$s=8$ metodoarentzat, birbiltze errorea hasten den uneko urrats luzera hartu dut: $k=12, \ h=10,667$. Kokapen errore erlatiboaren estimazioa, $h/2$ integrazioarekiko diferentzia gisa kalkulatu ditugu.
Atal honetan, $s=8$ ataleko Gauss metodoaren, \emph{CO1035} konposizio metodoaren eta \emph{ABAH1064} splitting metodoaren erroreak konparatu ditugu.  Hiru errore mota ezberdin aztertu ditugu: energia errorea; kokapen eta abiaduren erroreen estimazioak; orbita eliptikoaren erdi-ardatz nagusiaren (\emph{semi-axis}) eta eszentrikotasunaren erroreen estimazioak. Gauss metodoa $h=10.63$ urrats luzerarekin integratu dugu eta  \emph{CO1035}, \emph{ABAH1064} metodoak $h=4.76$ urrats luzerarekin.

\subsubsection*{Energiaren eboluzioa}


\ref{fig:esp83}~irudian, lau integrazioen energiaren errore erlatiboaren eboluzioa erakutsi ditugu. Batetik, Gauss metodoa $64$-biteko (\emph{double}) proiekzioarekin eta $80$-biteko (\emph{double}) proiekzioarekin konputatutako integrazioak. Bestetik, \emph{CO1035} konposizio eta \emph{ABAH1064} splitting metodoen integrazioak. Proiekzioa $80$-biteko doitasunarekin kalkulatzerakoan, energiaren errorea modu esanguratsuan txikitzea lortu dugu. Aipagarria da ere, Gauss metodoa $80$-biteko proiekzioaren integrazioa, \emph{ABAH1064} splitting metodoarekin alderatuz, energia errorea txikiagoa dela.

\begin{figure}[h!]
\centering
\begin{tabular}{c c}
\subfloat[Gauss metodoa ($s=8$).]
{\includegraphics[width=.45\textwidth]{esperimentua831}}
&
\subfloat[ABAH1064 eta CO1035]
{\includegraphics[width=.45\textwidth]{esperimentua832}}
\end{tabular}
\caption[Energia errorea]{\small Energia errorearen eboluzioa lau integrazio metodoetarako erakutsi dugu. Ezkerreko irudian, $s=8$ ataletako Gauss metodoaren $h=10,667$ urrats luzerarekin egindako integrazioak erakutsi ditugu: proiekzioa $80$-biteko \emph{long double} doitasunarekin (laranjaz) eta proiekzioa $64$-biteko \emph{double} doitasunarekin (urdinez). Eskuineko irudian, splitting/konposizio metodoak erakutsi ditugu: ABAH1064 (laranjaz) eta CO1035(urdinez)}
\label{fig:esp83}
\end{figure}


\subsubsection*{Kokapen eta abiadura erroreen estimazioak}


\ref{fig:esp84}~irudian, Gauss ($s=8$) eta \emph{ABAH1064} metodoentzako, kokapen eta abiaduraren erroreen estimazioak erakutsi ditugu. Erroreak estimatzeko, urrats txikiagoako integrazioaren soluzioarekiko diferentzia gisa kalkulatu dugu. 
Gauss metodoaren integrazioan, urrats luzera handiago erabili arren, barne-planeten kokapen eta abiaduren errore estimazioak, \emph{ABAH1064} splitting metodoaren integrazioan baino txikiagoak izan dira. 

\begin{figure}[h!]
\centering
\begin{tabular}{c c}
\subfloat[Gauss metodoa (kokapen errorea)]
{\includegraphics[width=.45\textwidth]{esperimentua841}}
&
\subfloat[Gauss metodoa (abiadura errorea)]
{\includegraphics[width=.45\textwidth]{esperimentua842}}
\\
\subfloat[ABAH1064 (kokapen errorea)]
{\includegraphics[width=.45\textwidth]{esperimentua843}}
&
\subfloat[ABAH1064 (abiadura errorea)]
{\includegraphics[width=.45\textwidth]{esperimentua844}}
\end{tabular}
\caption[Kokapen eta abiadura erroreak]{\small Kokapen eta abiaduraren erroreen estimazioak erakutsi ditugu. (a) eta (b) irudietan, $s=8$ ataletako Gauss metodoaren errore estimazioak eman ditugu, $h=10,667$ urrats luzera aplikatutako integrazioarentzat. (c) eta (d) irudietan, \emph{ABAH1064} splitting metodoaren errore estimazioak eman ditugu, $h=4.76$ urrats luzera aplikatutako integraziorentzat. Kolore bakoitza planeta bakoitzari dagokion errorea da: Merkurio (urdin ilunez), Artizarra (marroi argiz), Lurra (berdez), Marte (gorriz), Jupiter (more argiz), Saturno (marroi ilunez), Urano (urdin argiz), Neptuno (laranja argiz), Pluto (morez)}
\label{fig:esp84}
\end{figure}


\subsubsection*{Eszentrikotasun eta erdi-ardatza nagusiaren erroreen estimazioak}


Planeten mugimendu orbitala, eliptikoa da. Orbitaren propietateak finkatzen dituzten bi konstante hauen erroreen estimazioa kalkulatuko dugu: 
\begin{enumerate}
\item $a$ erdi-ardatz nagusia (\emph{semi-axis}) izeneko konstantea, orbita eliptikoaren tamaina definitzen duena.
\item $e$ eszentrikotasuna konstantea, orbita eliptikoaren forma finkatzen duena. 
\end{enumerate} 

\ref{fig:esp87}~irudian, bai Gauss metodoarentzat, bai \emph{ABAH1064} splitting metodoarentzat, $a$ erdi-ardatz nagusiaren eta $e$ eszentrikotasun erroreak, kokapen eta abiaduren erroreak baino txikiagoak dira. Honek esan nahi du, integrazioaren orbitaren forma mantentzen dela eta errorea, orbitaren fasean nagusitzen dela. Esperimentu honetan ere, Gauss metodoaren erroreak, \emph{ABAH1064} spliting metodoarenak baina txikiagoak dira.   

\begin{figure}[h!]
\centering
\begin{tabular}{c c}
\subfloat[Gauss maetodoa (erdi-ardatz nagusiaren errorea)]
{\includegraphics[width=.45\textwidth]{esperimentua871}}
&
\subfloat[Gauss maetodoa (eszentrikotasunaren errorea)]
{\includegraphics[width=.45\textwidth]{esperimentua872}}\\
\subfloat[ABAH1064 (erdi-ardatz nagusiaren errorea)]
{\includegraphics[width=.45\textwidth]{esperimentua873}}
&
\subfloat[ABAH1064 (eszentrikotasunaren errorea)]
{\includegraphics[width=.45\textwidth]{esperimentua874}}
\end{tabular}
\caption[Erdi-ardatz nagusiaren eta eszentrikotasunaren errorea]{\small \small Orbita eliptikoaren $a$ erdi-ardatz nagusiaren  eta $e$ eszentrikotasunaren erroreen estimazioak erakutsi ditugu. (a) eta (b) irudietan, $s=8$ ataletako Gauss metodoaren errore estimazioak eman ditugu, $h=10,667$ urrats luzerarekin integratuz. (c) eta (d) irudietan, \emph{ABAH1064} Splitting metodoaren errore estimazioak eman ditugu, $h=4.76$ urrats luzerarekin integratuz. Kolore bakoitza planeta bakoitzari dagokion errorea da: Merkurio (urdin ilunez), Artizarra (marroi argiz), Lurra (berdez), Marte (gorriz), Jupiter (more argiz), Saturno (marroi ilunez), Urano (urdin argiz), Neptuno (laranja argiz), Pluto (morez)}
\label{fig:esp87}
\end{figure}

\subsection*{Biribiltze errorea.}


Hirugarren esperimentu honetan, biribiltze errorea ilustratzeko esperimentua egin dugu eta horretarako, momentu angeluarraren errore erlatiboaren eboluzioan oinarritu gara. % Momentu angeluarraren trunkatze errorea beti zero da eta metodo sinplektikoek inbariante koadratikoak zehazki mantentzen dituzte. 
Lau faktore hartu behar dira kontutan. Kepler-en fluxuak, momentu angeluarra zehazki mantentzen du eta jatorrizko ekuazio diferentzialen inbariante koadratikoa da. Hori dela-eta, Kepler-en fluxuan oinarritutako aldagai aldaketarekin lortutako ekuazio diferentzialen inbariante koadratikoa da. Integratzeko Runge-Kutta metodo sinplektikoa aplikatu dugunez, inbariante koadratikoak zehazki mantentzen ditu eta beraz,  ikusten duguna biribiltze errorea da.

\ref{fig:esp85}~irudian, Gaussen $s=8$ ataletako metodoarekin, $h=9.0$ eta $h=10.63$ urrats luzerarekin integrazioen soluzioen momentu angeluarraren errorea erakutsi dugu eta espero bezala, biribiltze errorea zein den erakusten dute.


\begin{figure}[h!]
\centering
\begin{tabular}{c c}
\subfloat[Momentu angeluarra $h=9.0$.]
{\includegraphics[width=.45\textwidth]{esperimentua851}}
&
\subfloat[Momentu angeluarra $h=10.63$]
{\includegraphics[width=.45\textwidth]{esperimentua852}}
\end{tabular}
\caption[Momentu angeluarra]{\small Momentu angeluarraren errore erlatiboaren eboluzioa erakutsi dugu. Gaussen $s=8$ ataletako metodoa integratu dugu $h=9.0$ eta $h=10.63$ urrats luzerarekin. }
\label{fig:esp85}
\end{figure}



\subsection*{Eraginkortasun konparaketa.}


Atal honetan, Gauss metodoa eta konposizio/splitting metodoen arteko eraginkortasunaren konparaketa egin dugu. \ref{fig:esp82a}~irudian, eraginkortasuna, ekuazio diferentzialaren ebaluazio kopuruaren arabera neurtu dugu. Gure inplementazio berriak, \emph{ABAH1064} splitting metodoak baino doitasun txikiagoa lortzen du. Etorkizuneko inplementazioaren konputazioa, $80$-biteko doitasunean eta proiekzioa, $128$-biteko doitasunean egin daiteke: era honetan, bien arteko koxka handiago izango da.
   

\begin{figure} [h!]
\centerline{\includegraphics [width=8cm, height=6cm] {esperimentua822}}
\caption[Metodo sinplektikoen eraginkortasun grafikoa (FCN)]{\small Eraginkortasun grafikoa, eskala logaritmiko bikoitzean irudikatu dugu. Ardatz bertikalean, energiaren errore erlatibo maximoa eman dugu eta ardatz horizontalean, ekuazio diferentzialen ebaluazio kopurua (FCN).  Hiru integrazio metodo konparatu ditugu: $s=6$ Gauss metodoa grisez, $ABAH1064$  urdinez eta $CO1035$ gorriz}
\label{fig:esp82a}
\end{figure} 

\ref{fig:esp82}~irudian, $s=8$ ataletako Gauss metodoa, modu sekuentzialean eta modu paraleloan exekutatu dugu. Eguzki-sistemaren eredu sinplearen integraziorako, splitting metodoak oso eraginkorrak azaldu zaizkigu. Gauss metodoaren exekuzioa paralelizatzeak abantaila erakusten du baina hala ere, splitting metodoak eraginkorragoak dira.

Dena den, eredu errealistagoak (gorputz kopurua handitzen delako edota erlatibitate efektua kontutan hartzen delako) integratzeko, Gauss metodoak eraginkorragoak bilakatzea espero da. Splitting metodoen konputazioak, modu trinkoan kalkulatu behar dira, hau da, atalen konputazioak sekuentzialki exekutatzen dira eta  ez ditu konputazio aldaerarik onartzen. Gauss metodoaren ekuazio inplizituak ebazteko, ordea, teknika ezberdinak konbina daitezke eta eraginkortasuna hobetzeko aukera asko eskaintzen dizkigu. Adibidez, iterazio gehienak problemaren eredu sinple batekin, doitasun baxuan kalkula daitezke  \cite{Beylkin2014} eta bukaerako iterazio pare bat eredu osoarekin, doitasun altuan. Gauss metodoaren $s$-ataletako funtzioen ebaluazioak  modu paraleloan exekutatu daitezke eta eguzki-sistemaren eredu konplexuagoa aplikatzen den neurrian, paralelizazioak abantaila handiagoa suposatuko du.



\begin{figure}[h!]
\centering
\begin{tabular}{c c}
\subfloat[Exekuzio sekuentziala.]
{\includegraphics[width=.45\textwidth]{esperimentua821}}
&
\subfloat[Exekuzio paraleloa.]
{\includegraphics[width=.45\textwidth]{esperimentua823}}
%\subfloat[Exekuzio paraleloa (hariak=$2$) Wall Time.]
%{\includegraphics[width=.5\textwidth]{esperimentua823}}
%&
%\subfloat[Exekuzio paraleloa (hariak=$4$) Wall Time.]
%{\includegraphics[width=.5\textwidth]{esperimentua824}}
\end{tabular}
\caption[Metodo sinplektikoen eraginkortasun grafikoa (CPU Time)]{\small 
Eraginkortasun grafikoak eskala logaritmiko bikoitzean irudikatu ditugu. Ardatz bertikalean, energiaren errore erlatibo maximoa eman dugu eta ardatz horizontalean,  CPU denbora (exekuzio paralelotan Wall-Time) erakutsi dugu. (a) konputazioa modu sekuentzialean egin dugu eta (b) modu paraleloan, hari kopurua $2$ izanik. Irudi bakoitzean,  hiru integrazio metodo konparatu ditugu: $s=6$ Gauss metodoa grisez, $ABAH1064$  urdinez eta $CO1035$ gorriz
}
\label{fig:esp82}
\end{figure}

%\begin{figure}[h!]
%\centering
%\begin{tabular}{c c}
%\subfloat[$s=16$ Exekuzio sekuentziala: CPU-denbora.]
%{\includegraphics[width=.5\textwidth]{esperimentua861}}
%&
%\subfloat[$s=16$ Exekuzio sekuentziala:: FCN.]
%{\includegraphics[width=.5\textwidth]{esperimentua862}}\\
%\subfloat[$s=16$ Exekuzio paraleloa: hariak=$2$.]
%{\includegraphics[width=.5\textwidth]{esperimentua863}}
%&
%\subfloat[$s=16$ Exekuzio paraleloa: hariak=$4$.]
%{\includegraphics[width=.5\textwidth]{esperimentua864}}
%\end{tabular}
%\caption{\small 
%Eraginkortasun grafikoak irudikatu ditugu: ezkerrean energiaren errore maximoa, CPU denborarekiko; eskuinean ekuazio diferentzialen ebaluazio kopuruarekiko (FCN). Lau integrazio metodo konparatu ditugu: $ABAH1064$  urdinez, $CO1035$ gorriz,  eta \emph{IRKFLUXU} grisez}
%\label{fig:esp82}
%\end{figure}


\section{Laburpena.}


Kapitulu honetan, eguzki-sistemaren integraziorako inplementazio berri bat aurkeztu dugu. Inplementazio berria, egungo metodo sinplektiko eraginkorrenekin alderatu dugu eta emaitzak, baikorrak izateko modukoak iruditu zaizkigu. Oinarrizko azterketa egin badugu ere, agerian geratu da, metodoak etorkizunean izan ditzakeen potentziala. Dudarik gabe, etorkizun hurbilean azterketa sakonagoa egin beharko litzateke, metodoaren propietate onak baieztatzeko. Horien artean, honako ideia aipatuko ditugu:
\begin{enumerate}
\item Biribiltze erroreen hedapenaren azterketa estatistikoa, puntu-finkoaren eta Newton sinplifikatuaren inplementazioetan egin genuen moduan.
\item $80$-biteko doitasuneko (\emph{long double}) integrazioaren konputazioa: kasu honetan, proiekzioa $128$-biteko doitasunean egitea komeniko litzateke. 
\item Kepler fluxuaren inplementazioaren hobekuntzak: oraingo inplementazioaren iterazio guztietan, ez dugu aurreko iterazioen informazio erabiltzen. Iterazio berri baten kalkuluan, aurreko iterazioren egoeretatik abiatuta, nahikoa izango litzateke fluxuaren iterazio bakarra egitea.  
\item Eguzki sistemaren eredu konplexuago batekin (Ilargia eta zenbait asteroide, erlatibitate efektua, eguzkiaren atxatamendua,\dots), Gauss metodoa eraginkorragoa bilakatzea espero da. Eguzki-sistemaren eredu konplexuago, aplikatzen den neurrian paralelizazioak abantaila handiagoa suposatuko du. Era berean, iterazio gehienak problemaren eredu sinple batekin, doitasun baxuan kalkula daitezke eta bukaerako iterazio pare bat eredu osoarekin, doitasun altuan.
\item Eraginkortasuna hobetzeko, Jacobiarraren hurbilpen sinple baten erabilera (puntu finkoaren eta Newton-en arteko algoritmo eraginkorra).  
\end{enumerate}    

Eguzki-sistema eredu konplexuetan, Kleperiarrak ez diren indarrak aplika daitezke. Aldagai hauen konbergentzia azkartzeko, Newton sinplifikatuaren iterazioan oinarritutako inplementazioa aplikatzea komeniko da.  

Azkenik, aipatu nahi dugu, inplementazioaren kodea, helbide honetan \url{https://github.com/mikelehu/IRK-SolarSystem} eskuragarri jarri dugula. 



\part{Eztabaida eta ondorioak.}
\chapter{Eztabaida.}

\section{Sarrera.}

\section{IRK Puntu finkoa.}

\section{IRK Newton.}

$(I_s \otimes I_d - h \ A \otimes J) \triangle Y = r$ ekuazio sistema askatzeko, Hairer-en inplementazio estandarrean, $A$ matrizearen,
\begin{equation*}
A=P^{-1}DP, \ \,
D=\begin{bmatrix}
\sigma_1 & 0            & \cdots & 0  \\
 0       & \sigma_2     & \cdots & 0   \\
 \cdots  & \cdots       & \cdots & \cdots  \\
 0       &  0           &        & \sigma_s \\
\end{bmatrix}
\end{equation*}
diagonalizazioa proposatzen da. $D$ matrizeak, balio propio konplexuak ditu. Zenbaki konplexuekin lana ez bada egin nahi, zenbaki errealeko deskonposaketa baliokidea,

\begin{equation*}
A=Q^{-1}RQ, \ \,
R=\begin{bmatrix}
\sigma_{1A} & \sigma_{1B}   &  0          &  0            & \cdots &  0           &    0       \\
\sigma_{1C} & \sigma_{1D}   & 0           &  0            & \cdots &  0           &    0       \\
 0          & 0             & \sigma_{2A} & \sigma_{2B}   & \cdots &  0           &    0       \\
 0          & 0             & \sigma_{2C} & \sigma_{2D}   & \cdots &  0           &    0       \\
 \cdots     & \cdots        &  \cdots     & \cdots        & \cdots & \cdots       &    \cdots   \\
 0          & 0             &  0          & 0             & \cdots & \sigma_{sA}  & \sigma_{sB} \\
 0          & 0             &  0          & 0             & \cdots & \sigma_{sC}  & \sigma_{sD} \\
\end{bmatrix}
\end{equation*}
Honen arabera, Hiarer-en inplementazioan,
\begin{itemize}
\item $s$ bikoitia $\rightarrow$ $(2d \times 2d)$ tamainako $[s/2]$  LU deskonposaketa.
\item $s$ bakoitia $\rightarrow$ $(2d \times 2d)$ tamainako $(s+1)/2$  LU deskonposaketa.
\end{itemize}

\paragraph*{}Gure inplementazioan, $\bar{A}$ matrizearen,
\begin{equation*}
\bar{A}=P^{-1}DP
\end{equation*}
diagonalizatzen dugu eta $D$ matrizeak, irudikari puruak ditu. Eta ondorioz, gure inplementazioaren bertsio errealean,
\begin{equation*}
\bar{A}=Q^{-1}RQ, \ \,
R=\begin{bmatrix}
0           & -\sigma_{1}   &  0          &  0            & \cdots &  0           &    0       \\
 \sigma_{1} & 0             & 0           &  0            & \cdots &  0           &    0       \\
 0          & 0             & 0           & -\sigma_{2}   & \cdots &  0           &    0       \\
 0          & 0             & \sigma_{2}  & 0             & \cdots &  0           &    0       \\
 \cdots     & \cdots        &  \cdots     & \cdots        & \cdots & \cdots       &    \cdots   \\
 0          & 0             &  0          & 0             & \cdots & 0            & -\sigma_{s} \\
 0          & 0             &  0          & 0             & \cdots & \sigma_{s}   & 0            \\
\end{bmatrix}
\end{equation*}
zeroak agertzen dira. Honen arabera, gure inplementazioan,
\begin{itemize}
\item $s$ bikoitia $\rightarrow$ $(d \times d)$ tamainako $[s/2]+1$  LU deskonposaketa.
\item $s$ bakoitia $\rightarrow$ $(d \times d)$ tamainako $(s+1)/2$  LU deskonposaketa.
\end{itemize}


\section{Laburpena.}
\chapter{Ondorioak.}


\section*{Sarrera}

Gure helburua, eguzki-sistemaren doitasun altuko eta epe luzeko integraziorako, Gaussen metodo inplizituen inplementazio eraginkorra lortzea da. Helburua lortu ote den ala ez galderari erantzuteko, balorazioa bi ikuspegietatik egin behar dela iruditzen zaigu. Batetik, lortutako inplementazioaren aldeko argumentuak azalduz, eraginkorra dela defenditu daiteke. Bestetik, tesian aurkeztutako ikerketa aurretiko lana da eta beraz, garrantzitsua da ere, inplementazioaren eraginkortasuna etorkizunean izan ditzakeen hobekuntzen arabera neurtzea. 

Gakoetako bat, integratu nahi den eguzki-sistemaren ereduaren konplexutasuna da. Eguzki-sistemaren ezagutza gero eta zehatzagoa denez \cite{Kaplan2015}, eredu gero eta konplexuagoak integratu nahi izango direla suposatu daiteke. Zentzu honetan, integrazio metodoa problema konplexuen integraziorako eraginkorra izan beharko luke. Edozein kasutan, argi dago, integrazio metodoak  ez duela aplikatu nahi den problemaren formulazio matematikoa mugatu behar eta askatasun osoa eduki behar dela, ekuazioetan nahi diren efektuak gehitzeko.   

Algoritmo baten eraginkortasuna, konputagailu hardware ingurune bati lotuta dago eta urteekin alda daiteke. Hau da,  konputagailu hardware jakin batean eraginkorra den algoritmoa, hardware berriagotan zaharkitua gera daiteke.  Ezaguna da, egungo konputagailuen ahalmena, konputazio paraleloan oinarritzen dela eta horregatik, idei berritzaileak aplikatu behar direla algoritmo eraginkorrak garatzeko. Gauss metodo inplizituak, inplementazio aukera asko eskaintzen dituenez, zenbakizko integrazioen arloan idei berriak garatzeko metodo egokiak direla \cite{Dongarra2017} esaten ausartuko gara.   


\section*{Eguzki-sistemaren integraziorako inplementazioa.}


Eguzki-sistemaren integraziorako, Runge-Kutta metodo inplizitu orokorra (IRK) aplikatu dugula azpimarratu behar dugu. Hori dela-eta, inplementazioa sinplea da eta inplementazioak, metodoaren ezaugarri on guztiak heredatzen ditu. Azken finean, puntu-finkoaren iterazioan oinarritutako IRK inplementazio estandarra aplikatu dugu, formulazio eta geratze irizpide bereziekin. Epe luzeko integrazioetarako, metodoa sinplektikoa izatea garrantzitsua da. Baina, IRK metodoaren formulazio estandarrak  sinplektikotasuna ez duenez ziurtatzen, IRK metodoa aplikatzeko formulazio berria proposatu dugu. Bestalde, inplementazio estandarraren geratze irizpidea hobetu dugu. Laburtuz, IRK inplementazio estandarrean oinarritzerakoan, inplementazio berri batek izan ditzakeen konplexutasunak ekidin ditugu. 

Kepler-en fluxuan oinarritutako aldagai aldaketa, ebatzi nahi dugun problemari lotuta dago. Aldagai aldaketa hau, teknika orokor baten aplikazioa da; hau da, ekuazio diferentzialetatik zati bat desagerrarazteko helburuarekin definitutako aldagai aldaketa aplikatzearena. Aldagai aldaketa honekin, ekuazio diferentzialetatik alde Kepleriarrari dagokion zatia desagertzen da eta mantso aldatzen diren ekuazio diferentzialak lortzen ditugu.  Aldagai berriekiko ekuazio diferentzialek, hiru abantaila dituzte. Lehenik, eguzki-sistemaren trunkatze errore nagusiena ezabatu dugunez, integrazioan urrats luzera handiagoak erabil daitezke. Bigarrenik, integrazioaren urratsaren konputazioaren baturan,
\begin{equation*}
% (\tilde{y}_{n+1}, e_{n+1}) \leftarrow \text{batura konpensatua}(\tilde{y}_n, e_n,\delta_n),
 y_{n+1}=y_{n}+\delta_n, \ \ n=1,2,\dots
\end{equation*}
$\delta_n$ gehikuntzak magnitude txikia du eta beraz, batuketa honetan informazio gutxiago galduko da. Gainera, $\delta_n$ doitasun altuan kalkulatzeak, abantaila areagotu egingo du. Hirugarrenik, puntu-finkoaren konbergentzia azkarra izango da.

Eguzki-sistema eredu konplexuetan, Kleperiarrak ez diren indarrak aplika daitezke. Aldagai hauen konbergentzia azkartzeko, Newton sinplifikatuaren iterazioan oinarritutako inplementazioa aplikatzea komeniko da.                  

Gauss metodoa, neurri batean  splitting eta konposizio metodoen antzekoak dira. 
\begin{align*}
\text{Konposizio metodoa} \ \ &\equiv \ \ \text{Gauss metodoa, aldagai aldaketa gabe}.\\
\text{Splitting metodoa}  \ \ &\equiv \ \  \text{Gauss metodoa, aldagai aldaketarekin}.
\end{align*}

Splitting metodoekiko antzekotasuna azaltzeko, (\ref{eq:stverlet})~\emph{Störmer-Verlet} splitting metodoarekin konparatuko dugu. \emph{Störmer-Verlet} metodoa, era honetan aplikatzen da:  fluxua $h/2$ aurreratu, perturbazioak kalkulatu eta berriz,  fluxua $h/2$ aurreratu. Kepler-en fluxuan oinarritutako aldagai aldaketa aplikatzen dugunean, gauza bera egiten ari gara:  fluxua $h/2$ aurreratu, perturbazioak kalkulatu (aldagai berrietan eta beraz, hobeto kalkulatzen dugu),  fluxua $h/2$ aurreratu (\ref{fig:urratsBat}~irudia).

%\begin{figure} [h!]
%{\includegraphics [width=16cm, height=4cm] {urratsBat}}
%\caption{\small Gauss metodoan, Kepler-en fluxuan oinarritutako aldagai aldaketa aplikatzen dugunean, urratsaren konputazioa}
%\label{fig:urratsBat}
%\end{figure} 

Splitting metodoak oso eraginkorrak dira eta eguzki-sistemaren eredu sinplearen integrazioen  esperimentuek, hori erakutsi digute. Baina eredu errealistagoak (gorputz kopurua handitzen delako edota erlatibitate efektua kontutan hartzen delako) integratzeko, Kepler fluxuan oinarritutako Gauss metodoak, splitting metodoekiko abantailak hauek izango dituela azpimarratuko dugu:  

\begin{enumerate}

\item Ordena altuko metodoak.

Konposizio eta  splitting metodo optimoenak, $p=10$ ordenakoak dira. Gauss metodoei dagokienez, edozein ordenako metodoak eraiki daitezke. Doitasun bikoitzeko konputazioetarako, $s=6$ ataletako metodoa eraginkorrena kontsideratzen da \cite{Hairer2008} baina doitasun altuagoko konputazioetarako, ordena altuagoko metodoak aplikatzea gomendagarria da.

\item Paralelizagarria.

Gauss metodoaren $s$-ataletako funtzioen ebaluazioak independenteak dira eta modu paraleloan exekutatu daitezke. Eguzki-sistemaren eredu konplexuagoa kontutan hartzen den neurrian, paralelizazioak abantaila handiagoa suposatuko du. Gauss metodoaren paralelizazio gaitasun hau azpimarratzekoa da, egungo algoritmo azkarren diseinua, kodearen paralelizazioan oinarritzen baita.

\item Hamiltondar orokorrak.

Gauss metodoa, edozein sistema Hamiltondarrari aplika dakioke. Splitting metodoa, ordea, sistema Hamiltondar banagarrietan bakarrik aplika daitezke; eguzki-sistemaren eredu errealistak integratzeko, Hamiltondarraren egitura mantendu behar da eta hainbat indar ez grabitazionalak  gehitzeko, zailtasunak izan ditzakegu. 

\item Malgutasuna.

Splitting metodoen konputazioak, modu trinkoan kalkulatu behar dira, hau da, atalen konputazioak sekuentzialki exekutatzen dira eta konputazioak ez ditu aldaerak onartzen. Gauss metodoaren ekuazio inplizituak ebazteko, ordea, teknika ezberdinak konbina daitezke eta eraginkortasuna hobetzeko aukera asko eskaintzen dizkigu. Adibidez, iterazio gehienak problemaren eredu sinple batekin, doitasun baxuan kalkula daitezke  \cite{Beylkin2014} eta bukaerako iterazio pare bat eredu osoarekin, doitasun altuan. 

\item Birparametrizazioa.

Birparametrizazio, eguzki-sistemaren integrazioetarako tresna baliagarria dela frogatu da \cite{Fukushima2007,Rauch1998} eta Gauss metodoan,  teknika hau zuzenean aplika daiteke. Splitting metodoetan, ordea, birparametrizazioa ezin daiteke erabili eta integrazioetan zailtasunak agertzen direnean (esaterako, planeta baten eszentrikotasuna handitzen denean), orduan integrazioaren tarte batean urratsaren luzera txikitu behar da soluzioaren doitasuna mantentzeko \cite{Laskar2009}.

\end{enumerate}


\section*{Etorkizuneko lanak.}


Eguzki-sistemaren integraziorako inplementazioa garatzeko aukerak zehaztuko ditugu. Lehen eginbeharra, eguzki-sistemaren eredu konplexuagoak integratzea da eta zehazki, arlo honetan  aplikatzen den eguzki-sistemaren ereduekin \cite{Laskar2011} gure inplementazioaren jokabidea aztertzea. Eredu sinplearekin lortutako emaitzekin baikorrak izateko arrazoiak baditugu eta eredu konplexuagoekin, gure inplementazioaren abantaila areagotu daitekeela pentsatzen dugu. Inplementazioaren eraginkortasuna hobetzeko tresna nagusienak hauek izan daitezke:


\begin{enumerate}
\item Eredu deskonposaketan oinarritutako teknikak.

Inplementazioa eraginkorragoa egiteko, lehen iterazioak eredu sinpleagoarekin kalkula daitezke eta bukaerako iterazioak, eredu konplexuagoarekin.

\item Denbora birparametrizazioak.

Gorputzen gerturatzeen tratamendurako, denbora birparametrizazio teknikak aplikatu daitezke. 

\item Problema oszilakorren teknikak.

Problema oszilakorren tratamendu aljebraikoan oinarritutako hobekuntza teknikak (metodo prozesatuak, promediatuen teknikak) aplika daitezke.
Wisdom-ek \cite{Wisdom2006} bere integrazio metodoen doitasuna hobetzeko teknikekin (\emph{symplectic correctors}) erlazionatutakoak dira.

\item Puntu-finkoaren eta Newton sinplifikatuaren iterazioen arteko algoritmo eraginkorra.

Eraginkortasuna hobetzeko Jacobiarraren hurbilpen sinple baten erabilera. 

\item Doitasun nahasia.

Eguzki-sistemaren integrazioetan, soluzioaren doitasuna hobetzeko ahalegin berezia egiten da \cite{Laskar2015}: $80$-biteko doitasuneko aritmetika eta batura konpensatua erabiltzen dira. Gure inplementazioarekin, $80$-biteko doitasuneko integrazioa, $128$-biteko doitasuneko konputazio batzuekin konbinatuz, emaitza onak lortuko ditugulakoan gaude.  

 
\item Paralelizazio eta bektorizazio teknika aurreratuak.

Kodeen eraginkortasuna hobetzeko, konputagailuen paralelizazio eta bektorizazio gaitasunak ondo erabili behar dira. Gauss metodoa azkartzeko, oinarrizko paralelizazioa aplikatu dugu eta honek, abantaila erakutsi du. Teknikak hauek modu aurreratuagoan aplikatzeko,  paralelizazio eta bektorizazio gaietan sakondu beharra ikusten dugu. 


\end{enumerate} 











% --------------------------------------------------------------
%:                  BACK MATTER: appendices, refs,..
% --------------------------------------------------------------

\part{Eranskinak}
\appendix
\chapter{Zientzia-konputazioa.}

%\epigraph{Optimizing a program can be a better solution than buying a faster computer!}
%{\textit {Performance optimization of Numerically Intensive Codes(2001)}}

\section{Sarrera.}

Azken hamarkadetan, zientzia konputazioaren hazkundea oso handia izan da eta bere erabilera ia zientzia arlo guztietara zabaldu da \cite{Goedecker2001}. Zientzialariek ahalmen handiko tresna berria (zenbakizko simulazioa) eskuragarri dute, neurri handi batean konputagailuen teknologiaren garapen handiari esker. Egungo oinarrizko konputagailuek, orain urte gutxitako superordenagailuen ahalmen berdina dute eta superordenagailuen konputazio gaitasuna ere, maila berdinean hazi da. Zenbakizko algoritmoek ere, garapen handia izan dute; algoritmo eraginkorragoak, idei berriak sortuz eta konputagailuen gaitasunei egokituz, garatu dira.

Konputazioaren alde garestiena, memoria eta prozesadorearen arteko datu mugimendua da. Prozesadoreak gero eta azkarragoak dira, baina memori atzipenaren abiaduraren hobekuntza mugatuagoa dago. Horregatik algoritmo eraginkorrak, prozesadorearen konputazio arik eta handiena,  memoria komunikazio arik eta txikienarekin, diseinatu behar dira.      
 
Inplementazio baten eraginkortasuna ez da exekuzio denboraren arabera bakarrik neurtu behar. Hori bezain garrantzitsua da kode ona idaztea \cite{Wilson2014} eta zentzu honetan hiru ezaugarri hauek bereziki zaindu behar dira:
\begin{enumerate}
\item Errorerik gabeko kodea.
\item Kode argia idaztea.
\item Etorkizunean erraz aldatu daitekeen kodea.
\end{enumerate}

Atal honetan, zientzia konputazioaren funtsezko osagaiak azalduko ditugu. Lehenengo, konputazioaren eraginkortasuna aztertzeko eta exekuzioak neurtzeko tresnen zehaztapenak emango ditugu. Ondoren, konputagailu hardware berriak eta paralelizaio gaitasunak laburtu dugu. Jarraian software ikuspegitik, programazio lengoaiak eta aljebra linealerako (LAPACK) liburutegia landu dugu. Azkenik, konpiladoreari buruzko argibide batzuk eman ditugu.

\section{Eraginkortasuna}

Zientzia konputazioaren inplementazio berri baten eraginkortasuna neurtzeko, koma-higikorreko eragiketa kopurua (\emph{flops}) erabili ohi da. Problema handia denean, datuen mugimendua koma-higikorreko eragiketak baino garestiagoa da eta eraginkortasuna, eragiketa kopuruaren arabera neurtzea okerra izan daiteke. 

Prozesadoreen maiztasun-abiadura hertzetan neurtzen da, hau da,  \emph{makina ziklo segundoko} kopuruaren arabera. Une honetako prozesadoreak gigahertz (Giga = $10^9$) mailakoak dira. Koma-higikorreko oinarrizko eragiketa bat ($\oplus,\ominus,\otimes,\oslash$) exekutatzeko  ziklo gutxi batzuk behar dira eta beraz, $1$ GHz-ko prozesadore batek
$>10^8$ koma-higikorreko eragiketa segundoko exekutatzen ditu ($>100$ megaflops) \cite{Pacheco2011}.

\paragraph*{\textbf{Adibidea}.} 
Demagun $A,B$ eta $C \ (n \times n)$ dimentsioko matrizeak ditugula eta $C=AB$ matrize arteko biderketa egiteko behar dugun denbora jakin nahi dugula.
\begin{equation*}
c_{ij}=\sum\limits_{i,j=1}^{n} a_{ij}*b_{ji}.
\end{equation*}

\begin{itemize}
\item $c_{ij}$ gai bakoitza kalkulatzeko $n$ biderketa eta ($n-1$) batura egin behar ditugu.
\item $C$ matrizeak $n^2$ osagaia ditu $\Rightarrow$ $\mathcal{O}(n^3)$ koma-higikorreko ariketak exekutatu behar dira.
\end{itemize}

Matrizearen tamaina $n=100$ bada,  orduan  $\mathcal{O}(n^3)=10^{6}$ eragiketa egin behar ditugu. $1$-GHz prozesadore batean exekutatzeko, $10^{-2}$ segundo baino gehiago beharko genituzke. Baina matrize honek,  $3.9$ \emph{MB} memoria beharrezkoa du eta  konputagailuaren Cache memoria baino handiagoa dela suposatuz, exekuzio denboran datuen mugimenduaren eragina nabarmena izango da. 

Konputazio gaitasuna (\emph{peak}), hardwareak fisikoki exekutatu dezakeen eragiketa kopuru maximoa bada, aplikazio gehienak, konputagailuaren konputazio gaitasunaren  $\%10$ baino gutxiagorekin exekutatzen dira.  Eraginkortasun horren txikia, memoria irakurketa/idazketetan galtzen da. Azpimarratu nahi dugu, $t_f$ eragiketa aritmetiko bat egiteko denbora bada eta $t_m$, datu bat memoria nagusitik cache memoriara mugitzeko denbora bada,
\begin{equation*}
 t_f \ll t_m,
\end{equation*}
eta etorkizunean, diferentzia hau handituz joango dela. Beraz, kodearen exekuzioa azkartzeko derrigorrezkoa da konputagailuan memorien arteko datuen mugimendua minimizatzea.
       
\subsubsection*{Exekuzio denboren neurketa.}

Unixeko \emph{time} agindua, konputazioen denborak ezagutzeko erabili daiteke \cite{Pacheco2011}:

\begin{lstlisting} 
S time ./a.out
<kodearen irteera>

real 0m38.856s
user 0m38.789s
sys  0m0.004s
\end{lstlisting}

Agindu honekin, \emph{./a.out} C programa exekutatuko da eta ondoren, programa exekutatzeko behar izan duen denboraren informazioa pantailaratuko du:
\begin{itemize}
\item  \emph{real}: hasi eta bukatu arteko denbora (\emph{wall-time} edo \emph{elapsed-time}).
\item \emph{user}:  prozesadoreak gure programa exekutatzen erabili duen denbora (\emph{CPU-time}).
\item \emph{sys}:  programa exekutatu ahal izateko, sistema eragile lanetan emandako denbora. 
\end{itemize}

Programa osoaren konputazio denborak ezagutu beharrean, kodearen zati bat neurtu nahi dugunean, C lengoaiaren bi funtzio hauek erabilgarriak ditugu:

\begin{enumerate}
\item clock().

Funtzioaren bi deien arteko CPU denbora neurtzeko erabiliko dugu (\textbf{CPU time}).

\begin{lstlisting}[language=C]

#include <time.h>

    clock_t clock0, clock1; 
    double elapsed_cpu_time;

    clock0= clock();

    <neurtu nahi den kodea>

    clock1=clock();
    
    elapsed_cpu_time=(clock1 - clock0)/CLOCKS_PER_SEC);

\end{lstlisting}

\item time().

Funtzioaren bi deien artean igarotako denbora neurtzeko erabiliko dugu (\textbf{elapsed-time}).

\begin{lstlisting}[language=C]

#include <time.h>

    time_t  time0,time1;
    double elapsed_time;

    time(&time0);

    <neurtu nahi den kodea>

    time(&time0);
    
    elapsed_time=difftime(time1,time0);

\end{lstlisting}

\end{enumerate}

CPU denborari buruzko argibide bat ematea komeni da. Neurtzen ari garen kodea sekuentzialki exekutatzen bada, hau da, hari (\emph{thread}) bakarrekoa, orduan kode zati hori exekutatzeko erabili duen CPU denbora itzuliko du. Aldiz, kodea paraleloan exekutatzen bada, orduan , hari guztien CPU denboren batura itzuliko ditu. 


\paragraph*{Adibidea.} Argiago azaltzeko, $C=AB$ matrize biderketaren kodearen bi exekuzioen denboren neurketak zehaztuko ditugu:

\begin{enumerate}
\item $200 \times 200$ tamainako, bi matrizeen biderketa sekuentzialaren denborak hauek izan dira:
\begin{lstlisting}
elapsed-time=2.1s
elapsed-cpu-time=2.07s
\end{lstlisting}

\item $200 \times 200$ tamainako, bi matrizeen biderketa paraleloaren (hariak=$2$) denborak hauek izan dira:
\begin{lstlisting}
elapsed-time=1.0s
elapsed-cpu-time=2.35s
\end{lstlisting}
 
 
\end{enumerate}
 
\emph{Elapsed-time} deiturikoa, kode paraleloen denborak neurtzeko irizpidea da. Programazio paraleloan, algoritmoen exekuzio denborak egokien neurtzen duen aldagaia da baina, une berean programa bakarra exekutatzea behartuta gaude.       


\section{Hardwarea.}


Orokorrean, gaur-egungo konputagailuak (super-konputagailu, eramangarri,...) paraleloak dira. $1986-2002$ urteen artean, txip barruko transistore dentsitatea handitzen zen heinean, prozesadore bakarreko konputagailuen eraginkortasuna hobetuz joan zen. Baina teknologi honen garapena muga fisikoetara iritsi zenean, bide honetatik konputagailuen abiadura hobetzea ezinezkoa bilakatu zen. Horrela, 2005.urtetik aurrera fabrikatzaileek konputagailuen gaitasuna hobetzeko, txipan prozesadore bat baino gehiago erabiltzea erabaki zuten.  

%\begin{itemize}
%\item Moore's Law (1965). Processor speed doubles every 18 months.
%\item Moore's Law Reinterpreter (2006). Number of cores per chip can double every two years.
%\end{itemize}    

Konputagailuen eredu aldaketa honen ondorioz, algoritmo azkarrak garatzeko kodearen paralelizazio gaitasunari heldu behar zaio. Programazio paralelo teknikak inplementatzeko, beharrezko da prozesadore berrien hardware arkitektura berriak ulertzea. Gaia nahiko konplexua izanik, ikuspegi orokor bat ematera mugatuko gara.

\subsection*{Memori hierarkia.}

Memorien arteko datuen komunikazioak, algoritmoaren eraginkortasuna baldintzatuko du eta zentzu honetan, konputagailuaren memoria hierarkiaren kudeaketa egokia egitea funtsezkoa da. Konputagailuaren memoria mota ezberdinen hierarkia (\ref{fig:memhier}irudia) eta funtzionamendua deskribatuko dugu. 
\begin{figure}[h]
\centerline{\includegraphics[width=10cm, height=4cm] {MemoryHierarchy}}
\caption{Memoria hierarkia.}
\label{fig:memhier}
\end{figure} 

CPU-k, koma-higikorreko eragiketak exekutatzen ditu: erregistroetatik datuak irakurri, eragiketak kalkulatu eta emaitza erregistroetan idazten ditu. Memoria nagusia eta erregistroen artean, $2$ edo $3$ mailako cache memoria dugu: lehen cache memoria ($L1$) txikiena eta azkarrena da, eta beste mailak ($L2$, $L3$,\dots), handiagoak eta motelagoak dira. Memoria nagusian, exekutatzen diren programak eta datuak gordetzen dira ($1-4$ GB artekoa). Azkenik, disko gogorrean konputagailuko datu (argazki, bideo,...) eta erabilgarri ditugun programa guztiak gordetzen dira.  

CPU-k datu bat behar duenean, memoria hierarkian zehar bilatuko du: lehenik $L1$ cachean, ondoren $L2$ cachean,...eta hauetan ez badago, memoria nagusira joko du. Memoria nagusi eta cache memoria arteko irakurketa eta idazketa guzti hauetan,  informazio kontsistentzia mantentzeko hainbat arau aurrera ematen dira. 

Cache memoria lerroka egituratuta dago eta lerro bakoitza $64$ edo $128$ bytez ($8$ edo $16$ doitasun bikoitzeko zenbaki) osatuta dago. Programa batek datu bat behar duenean, memoria nagusitik lerro tamainako datu taldea (memorian jarraian gordetako datuak) irakurriko du eta cachean idatziko du. Programatzaileak, algoritmo eraginkorrak inplementatzeko memorien arteko komunikazio hau minimizatzen saiatu behar du eta horretarako, inplementazioaren diseinua datuen memoria atzipen jarraian oinarritu behar du. Ezaugarri hau, \emph{spatial/data locality} izenaz ezaguna da eta helburua, cachera ekartzen diren datuak, memoria nagusian idatzi aurretik gutxienez behin erabiltzea da. 

\paragraph*{Adibidea.} Adibide honetan, $A=(a_{ij})_{i,j}^{n,m}$ matrize baten osagaien batura ($sum=\sum_{i,j=0}^{n,m} a_{ij}$) kalkulatzeko bi inplementazio aztertuko ditugu. C lengoaian matrizeak lerroka gordetzen dira ($n=m=100$),
\begin{equation*}
A=\left(\begin{array}{ccccc}
  1    & 2    & 3    & \dots & 100 \\
  101 & 102 & 103 &\dots & 200 \\
  201 & 202 & 203 &\dots & 300 \\
  \dots & \dots & \dots & \dots & \dots \\
  9.901 & 9.902 & 9.903 &\dots & 10.000 \\
  \end{array}\right).  
\end{equation*}
eta horregatik, lehen aukera bigarrena baino eraginkorragoa izango da. Lehen inplementazioan, kanpo iterazioa lerroka (Algoritmoa \ref{algo:lerroka}):  matrizearen lehen osagaia $a(1,1)$ behar dugunean, memoria nagusitik Cachera osagai honetaz gain, jarraiko 16 osagaiak ekarriko dira ($a(1,1),a(1,2),\dots,a(1,16)$). Honela, hurrengo $15$ batura egiteko behar ditugun datuak Cachean eskura izango ditugu memoria irakurketa berririk egin gabe. Bigarren inplementazioan, kanpo iterazioa zutabeka (Algoritmoa \ref{algo:zutabeka}): bigarren osagaia ($a(2,1)$) gehitzeko memoria irakurketa berri bat egin behar dugu. 

\begin{algorithm}[h]
 \BlankLine
  $int \ n$\;
  $double \ a[n][m]$\;
  \BlankLine
  $sum=0$\;
  \For{$i\leftarrow 1$ \KwTo $n$}
  {
   \BlankLine
    \For{$j\leftarrow 1$ \KwTo $m$}
   {
    \BlankLine 
    $sum+=a(i,j)$\;
   }
 }
 \caption{Memoria atzipena eraginkorra.}
 \label{algo:lerroka}
\end{algorithm} 
\begin{algorithm}[h]
 \BlankLine
  $int \ n$\;
  $double \ a[n][m]$\;
  \BlankLine
  $sum=0$\;
  \For{$j\leftarrow 1$ \KwTo $m$}
  {
   \BlankLine
    \For{$i\leftarrow 1$ \KwTo $n$}
   {
    \BlankLine 
    $sum+=a(i,j)$\;
   }
 }
 \caption{Memoria atzipena ez-eraginkorra.}
 \label{algo:zutabeka}
\end{algorithm} 


\subsection*{Arkitektura motak.}

Prozesadore anitzeko konputagailu hardware berriak, konplexuak eta heterogeneoak dira. Egoera honetan, programatzaileak zailtasun handiak ditu arkitektura berriek eskaintzen dituzten gaitasunak ondo kudeatzeko \cite{Luszczek2014}.   

Lehen hurbilpen modura bi sistema nagusi bereiziko ditugu:  memoria konpartitutako eta memoria banatutako sistemak. Memoria konpartitutako sistemetan, prozesadore guztiek memoria osoa konpartitzen dute eta inplizituki konpartitutako datuen atzipenaren bidez komunikatzen dira (\ref{fig:mks} irudia). Memoria banatutakotako sistemetan aldiz, prozesadore bakoitzak bere memoria pribatua du eta esplizituki bidalitako mezuen bidez komunikatzen dira (\ref{fig:mbs} irudia). Aipatzekoa da, sistema handietan bi memoria motak nahasten direla, hau da, batetik goiko mailan memoria banatuta alde bat eta bestetik, konputazio unitate bakoitzak memoria konpartitutako aldea.    

Hirugarren sistema osagarria ere aipatuko dugu, GPU (Graphical Processor Unit) unitateetan oinarritutako konputazioa.
Jokoen eta animazio industrian, grafiko oso azkarrak beharrak bultzatuta  sortutako teknologia da. Oinarrian, imajinak pantailaratzeko prozesagailu asko paraleloan lan egiten dute eta azken hamarkadan, \emph{GPU} unitate hauek zientzia konputaziora zabaldu dira.  

\begin{figure}[h]
\centerline{\includegraphics[width=12cm, height=6cm] {Arkitektura1}}
\caption{Memoria konpartitutako sistemak.}
\label{fig:mks}
\end{figure}  

\begin{figure}[h]
\centerline{\includegraphics[width=12cm, height=4cm] {Arkitektura2}}
\caption{Memoria banatutako sistemak.}
\label{fig:mbs}
\end{figure}  

\subsubsection*{Oinarrizko konputagailu paraleloa.}

Gure lanerako, oinarrizko konputagailu paraleloak kontsideratuko ditugu: memoria konpartitutako eta prozesadore anitzeko unitate bat edo gehiagoz osatutako sistemak. Prozesadore anitzeko unitate bakoitzak txipean CPU bat baino gehiago ditu. Normalean CPU bakoitzak $L1$ bere cache memoria du. Aipatzeko da, era honetako sistemetan prozesadore kopurua mugatua dela (normalean $\leq 32$ ) \cite{Pacheco2011,EijkhoutHPC}.

Memoria konpartitutako sistemen artean bi mota bereiziko ditugu:

\begin{enumerate}
\item UMA sistemak (uniform memory access). 
Txip prozesadore guztiak zuzenean memoria konektatuta daude eta guztiak atzipen denbora berdina dute (\ref{fig:UMA}.irudia).

 \begin{figure}[h]
 \centerline{\includegraphics[width=12cm, height=5cm] {ArkitekturaUMA}}
 \caption{Memoria konpartitutako sistemak (UMA).}
 \label{fig:UMA}
 \end{figure}  

\item NUMA sistemak (nonuniform memory access).
Txip prozesadore bakoitza, hardware berezi baten bidez zuzenean memoria bloke bati konektatuta dago. Zuzenean konektatuta dagoen memori blokearen atzipen denbora, beste txipan zehar konektatutako memoriaren atzipena baina azkarragoa da ( \ref{fig:NUMA}.irudia).

 \begin{figure}[h]
 \centerline{\includegraphics[width=12cm, height=4cm] {ArkitekturaNUMA}}
 \caption{Memoria konpartitutako sistemak (NUMA).}
 \label{fig:NUMA}
 \end{figure}  
\end{enumerate}  

\subsubsection*{Bektorizazioa (SIMD).}

Prozesadore berriei hardware bektore unitateak gehitu zaizkie, eta  (AVX) bektore instrukzioetan oinarritutako paralelizazioa eskaintzen dute \cite{Muller2009,Andrey2013}. CPU-ak bektore erregistroan gorde diren zenbaki multzo bati, eragiketa berdina aplikatzen die (\emph{Single Instruction Multiple Data}, SIMD). Bektore erregistro hauek  $512$-biteko tamainakoak ($512/64=8$ doitasun bikoitzeko zenbaki) izan daitezke eta oro har, oinarrizko eragiketa aritmetikoak (batuketa, kenketa, biderketa eta zatiketa) aplika daitezke. Zentzu honetan ulertu behar da hardware bidezko bektorizazioak,  doitasun bikoitzeko konputazioen eraginkortasuna $x8$ hobetzen duela  eta doitasun arrunteko konputazioetan, berriz, $x16$.     

Adibide moduan, hurrengo pseudokodearen bidez, iterazio bakoitzean 8 osagaien bektore baten batura erakutsiko nahi dugu,

\begin{algorithm}[H]
 \BlankLine
  \For{$i=0$; $i<n$ ; $i+=8$}
  {
   \BlankLine
    $A[i:(i+8)]=A[i:(i+8)]+B[i:(i+8)]$\;
   \BlankLine
  }
 \caption{SIMD (bektorizazioa).}
\end{algorithm}


\section{Programazio lengoaiak.}

Fortran eta C, aplikazio zientifikoetan gehien erabiltzen diren programazio lengoaiak dira \cite{Higham2015}. Fortran (formula translation) 1950 hamarkadan garatutako goi-mailako lehen lengoaia izan zen eta oraindik ere, oso zabaldua da. Fortran estandarraren hainbat bertsio 
sortu dira: Fortran 66, 77, 90, 95, 2003 eta 2008. Hauetako bertsio bakoitzean funtzionalitate berriak  eta C lengoaiarekin lan egiteko bateragarritasuna gehitu zaizkio.  C lengoaia, 1970 hamarkadan jaio zen eta hardwarearekiko hurbiltasun" ezaugarri nagusiak, konpiladoreari kode eraginkorra sortzeko aukera ematen dio. C lengoaia \emph{UNIX} sistema eragileari lotuta jaio zen eta hurrengo garapenetan bere izaera askea mantendu du. C lengoaiaren estandarrak 1989, 1999 eta 2011 dira.   

Fortran eta C lengoaiek, ez dituzte kodea paraleloan exekutatzeko tresnarik, hau da, ez dago konputazio banatu, eta prozesadore ezberdinen artean aldi berean exekuzioak zehazteko modurik. Konputazioa paraleloa inplementatzeko, bi dira interfaze aplikazio programa (\emph{API}) moduan inplementatuta dauden sistema nagusiak \cite{Pacheco2011}:

\begin{enumerate}
\item  \emph{MPI} (Message Passing Interface).
Erabiliena da, memoria banatutako sistemetarako pentsatua baina memoria konpartitutako sistemetan ere aplika daitekeena.

\item  \emph{OpenMP} (Open Specifications for MultiProcessing).
Erabiltzeko errazagoa eta memoria konpartitutako sistemetan bakarrik aplika daitekeena.

\end{enumerate}

Eraginkortasun altuko konputazioaren (\emph{high performance computing}) programazioa konplexua da: espezializazio handikoa  eta konputagailuen hardware jakin baterako egokitutakoa. Zaitasun hauek, proiektu zientifikoak aurrera ateratzeko eta mantentzeko arazo asko eragiten ditu. Azken urteotan, eragozpen hau gainditzeko, programazio lengoaia interesgarriak sortu dira (adibidez, Julia \cite{Bezanson2014} edo Chapel \cite{Balaji2015}) baina oraindik, hauen arrakasta ikustekoa da.

Azkenik, zientzia konputazioan \emph{problemak ebazteko inguruneak} (Problem Solving Enviroments) deituriko softwareak aipatuko ditugu. Ingurune hauek, programazio leiho interaktibo batean, goi-mailako lengoai batean inplementazioen garapena eta emaitzen azterketa egiteko aukera eskaintzen dute. Matlab eta Mathematica \cite{WolframResearch} programazio ingurune nagusienak dira. Guk Mathematica bi modutara erabili dugu. Lehenik, prototipoak garatzeko tresna gisa: idei berriak garatu eta probatu, inplementazioa C lengoaian egin aurretik. Bigarrenik, gure C inplementazioen esperimentuak Mathematica ingurunetik exekutatu eta emaitzak grafikoki aztertu ditugu. Era honetan, irakurleari Mathematicako dokumentuetan esperimentuen zehaztasun guztiak eta esperimentu berdina errepikatzeko aukera ematen diogu.       

\subsection*{OpenMP}   

\emph{OpenMP} \cite{OpenMP} memoria konpartitutako sistemetan programazio paraleloa exekutatzeko interfaze aplikazio programa (\emph{API}) da. \emph{OpenMP} programazioan, memoria osoaren atzipena duten prozesadore multzo batek osatzen du konputazio sistema.

Hasieran programaren hari bakarra prozesadore batean exekutatuko da, kode-paraleloko unera iritsi arte. Orduan, hari multzo independenteak exekutatuko dira une paraleloa bukaera iritsi arte. Exekuzio kontrolari, \emph{fork-join} eredua deitzen zaio eta grafikoki (\ref{fig:forkjoin} irudian ) adierazi dugu.
 
\begin{figure}[h]
\centerline{\includegraphics[width=12cm, height=4cm] {ForkJoin}}
\caption[OpenMP programazio modeloa.]{OpenMp programazio eredua.}
\label{fig:forkjoin}
\end{figure}  
 
\begin{itemize}
\item OpenMP programen hasieran prozesu bakarra dago, hari (thread) nagusia. 
\item FORK: hari nagusiak, hari talde paraleloa sortzen du.
\item JOIN: kode-paraleloko hari guztiak bukatzen dutenean (sinkronizazioa), hari nagusiak soilik jarraitzen du.
 \end{itemize}

Paralelizazioan hari kopurua zehaztu behar da, eta ohikoa izaten da hari bat prozesadore bakoitzeko sortzea. Konpilazio direktiben bidez (C kodean \emph{pragma} izeneko preprozesadore aginduak),  paralelizazioa nola exekutatu behar den zehazten da.
\begin{itemize}
\item Kode paralelizagarria adierazi.
\item Hariaren datu pribatuak zehaztu.
\item Harien arteko sinkronizazioa.
\end{itemize}


\paragraph*{\textbf{Adibidea1.}} C lengoaian, \emph{OpenMP} konpilazio direktibak adierazteko \emph{pragma} hitza lerroaren hasieran idatziko dugu. Adibide honetan, programaren \emph{for-iterazioa} paraleloan exekutatu daitekeela  eta hari kopurua bi dela zehaztu dugu. 

\begin{lstlisting}[language=C]
#    include <omp.h>

     int thread_count=2;

#    pragma omp parallel for num_threads(thread_count) 
     for (i = 0; i<n; i++)
     {
       ! Aginduak 
     }
\end{lstlisting}

Konpilazioan, \emph{-fopenmp} aukera zehaztu behar dugu,
\begin{lstlisting}[language=C]
$ gcc -g -Wall -fopenmp adibidea.c -o adibidea.o.
\end{lstlisting}
Orokorrean, defektuz aldagaia guztiak harien artean konpartituta daude eta aldagaiak pribatuak direla zehazteko, esplizituki adierazi behar da. Goiko adibidea salbuespena da; \emph{for} iterazioaren kontagailua  (adibideko $i$ aldagaia) pribatua da.

Algoritmo batean, kodearen zati bat  da paralelizagarria \cite{Pacheco2011}. Suposa dezagun, konputazioaren $\%50$ sekuentziala dela eta beste $\%50$ paraleloan exekutatu daitekeela. Zati sekuentzialak, lortu daitekeen konputazio optimoena (zati paralelizagarriaren exekuzio denbora zero dela kontsideratzea) mugatuko du eta beraz, gehienez exekuzio sekuentziala baino bi aldiz azkarrago izango da. 
Kontzeptu hau orokortzen badugu, konputazioaren $(1/S)$ sekuentziala eta gainontzekoa, $(1-1/S)$ paralizagarria kontsideratuz, orduan kode optimoena prozesadore kopurua edozein delarik, $S$ faktorea hobea izango da. $T_s$ makina sekuentzial batean exekuzio denbora izendatzen badugu, $P$ prozesadore kopurua erabiliz lortuko den konputazio denbora $T_p$,
\begin{align*}
T_p &=(1/S)T_s + (1-1/S)T_s/P,
\end{align*}
eta prozesadore kopuru oso handia kontsideratuko bagenu,
\begin{align*}
T_p &\rightarrow (1/S) T_s ,  \ \ P \rightarrow \infty.
\end{align*} 

Konputazio paraleloan, $T_p$ denborari paralelizazioak duen gainkarga gehitu behar zaio. Gainkarga hau, kontzeptu ezberdinez osatuta dago eta milisegunduko mailako eragiketak izan daitezke.  
\begin{itemize}
\item Prozesuak edo hariak sortzeko denbora.
\item Sinkronizazio denbora.
\item Datu konpartitutako komunikazioa.
\end{itemize}

   
\section{Aljebra lineal dentsorako liburutegiak.}


Zenbakizko integrazioen aljebra lineala, konputazioaren alde konplexua da. Aljebra linealeko eragiketak inplementatzen dituzten kalitate handiko liburutegiak daude eta inplementazio berriak, liburutegi hauetan oinarritzea gomendagarria da \cite{Hogben2013}. Liburutegi hauek, ondo probatutako softwareak dira, konplexutasun handikoak, modu seguruan eta azkarrean exekutatzeko diseinatu dira.

Hauek dira, aljebra linealerako liburutegi aipagarrienak:
\begin{enumerate}
\item BLAS (Basic Linear Algebra Subroutines): matrize eta bektoreen arteko eragiketa aritmetikoak biltzen dituen liburutegia. 
\item LAPACK (Linear Algebra Package): aljebra linealaren problemak ebazteko liburutegia.
%\item ScaLAPACK (Parallel Distributed Dense Linear Algebra): arkitektura berriei egokitutako LAPACK liburutegiaren garapena.
\end{enumerate}

%Inplementazio hauen funtzioak \emph{Fortran} lengoaian garatuta daude eta beste lengoaietatik (C, C++, Java, Python) deitzeko interfazeak daude. 

Inplementazio hauen funtzioak \emph{Fortran} lengoaian garatuta daude eta ezaugarri hauek dituzte:
\begin{enumerate}
\item Datu-mota hauetarako aplika daitezke:
\begin{enumerate}
\item S: doitasun arrunta (\emph{float}, $32$-bit).
\item D: doitasun bikoitza (\emph{double}, $64$-bit).
\item C: zenbaki konplexua doitasun arruntean (complex).
\item Z: zenbaki konplexua doitasun bikoitzean (complex double).
\end{enumerate}  

\item Matrize dentsoetarako liburutegiak dira. Matrize egitura hauek zehaztu daitezke.
\begin{enumerate}
\item Matrize orokorrak.

 GE=General; GB=General Band.
\item Matrize simetrikoak.

 SY=SYmmetric ; SB=Symmetric Band; SP=Symmetric Packed.
\item Hermitiar matrizeak.

 HE=HErmitian ; HB=Hermitian Band; HP=Hermitian Packed.
\item Matrize triangualarrak.

 TR=TRiangular ; TB=TRiangular Band; TP=Triangular Packed.
\end{enumerate}

\end{enumerate}   

\subsection*{BLAS.}

\href{http://www.netlib.org/blas}{BLAS} liburutegiak \cite{Intel2015}, bektore eta matrizeen arteko funtzio estandarrak biltzen ditu. Liburutegia, $142$ errutinaz osatuta dago eta hauek, hiru taldeetan sailkatzen dira: 
\begin{enumerate}
\item BLAS-1: $\mathcal{O}(n)$ bektore-bektore eragiketak.

 Adibidea.
 $y=\alpha*x+y, \ \text{non} \ \alpha \in \mathbb{R}, \ \text{eta} \  x,y \in \mathbb{R}^n$. \\
 $2n$ eragiketa aritmetiko eta $3n$ irakurketa/idazketa.
 
 Konputazio intentsitatea: $2n/3n=2/3$. 

\item BLAS-2: $\mathcal{O}(n^2)$ matrize-bektore eragiketak.

 Adibidea.
 $y=\alpha*A*x+\beta*y, \ \text{non} \ \alpha,\beta \in \mathbb{R},\ x,y \in \mathbb{R}^n \text{eta} \ A \in \mathbb{R}^{n \times n}$. \\ 
 $\mathcal{O}(n^2)$ eragiketa aritmetiko eta $\mathcal{O}(n^2)$ irakurketa/idazketa.
 
 Konputazio intentsitatea: $\approx {2n^2}/{n^2}=2$. 
 
\item BLAS-3: $\mathcal{O}(n^3)$ matrize-matrize eragiketak.

 Adibidea.
 $C=\alpha*A*B+\beta*C, \ \text{non} \ \alpha,\beta \in \mathbb{R} \ \text{eta} \ A,B \in \mathbb{R}^{n \times n}$. \\ 
 $\mathcal{O}(n^3)$ eragiketa aritmetiko eta $\mathcal{O}(n^2)$ irakurketa/idazketa.
 
 Konputazio intentsitatea: $\approx {2n^3}/{4n^2}={n}/{2}$. 

\end{enumerate}

BLAS-1 eta BLAS-2 funtzioen konputazio intentsitatea txikia da eta beraz, talde hauetako funtzioetan, datuen komunikazioa nagusia da. BLAS-3 funtzioetan aldiz, konputazio intentsitatea handiagoa da eta ezaugarri honi esker, tamaina handiko matrizeen kalkuluetan, konputagailuaren konputazio gaitasuna ondo aprobetxatu ahal izango da 

Aljebra linealeko aplikazioen exekuzio denboraren zati garrantzitsuena, behe-mailako eragiketa hauen konputazioak ematen du. Behe-mailako eragiketen hauen optimizazioak, konputagailu bakoitzaren araberakoak dira eta espezializazio handia eskatzen du. Fabrikatzaile bakoitzak optimizatutako bere BLAS liburutegia du (AMD-ACML,Intel-MKL). Bestalde, optimizatutako BLAS instalazioa, \emph{ATLAS} (Automatically Tuned Linear Algebra Software) izeneko aplikazioaren bidez ere egin daiteke. 

Inplementazio guztiek, interfaze berdina erabiltzen dute eta beraz, BLAS-en oinarritutako garapena edozein konputagailuan erabili daiteke (portabilitatea). \emph{BLAS} liburutegia, Fortran lengoaian inplementatuta dago eta C lengoaiatik \emph{BLAS} funtzioen erabilpena errazteko, \emph{cblas} interfazea erabiltzea gomendagarria da. 

\paragraph{Adibidea.}\emph{BLAS} liburutegiaren eraginkortasuna, \emph{cblas\_dgemm()} matrizeen biderkadura funtzioaren  bidez aztertu dugu eta gure inplementazio arrunta baino $10\times$ azkarragoa dela baieztatu dugu (Taula \ref{tab:blas}). 

\begin{table}[h]
\caption[BLAS liburutegiaren eraginkortasuna.] 
{\small{BLAS liburutegiaren eraginkortasuna. C lengoaiako gure garapena (C-arrunta) eta BLAS liburutegiaren cblas\_dgemm() inplementazioak konparatu ditugu. $n$ tamainako ezberdineko matrizeak biderkatu ditugu, eta biderketa bakoitza $ntest$ alditan errepikatu dugu, kasu guztiek eragiketa aritmetiko kopuru berdina izan ditzaten}}
\label{tab:blas}      
\centering
{%
\begin{tabular}{l c c c c c } 
 \hline
  $n$     & ntests        &  \multicolumn{2}{c}{C-Arrunta}  & \multicolumn{2}{c}{cblas\_dgemm} \\
 \\
          &                   & Wall T. & CPU T. &  Wall T. & CPU T. \\
 \hline         
 $10$  &   $5.00\times10^8$   & $478.$   & $478.$  & $205.$  & $206.$   \\ 
 $20$  &   $6.25\times10^7$   & $491.$   &  $491.$  & $92.$   & $91.$   \\ 
 $30$  &   $1.85\times10^7$   & $474.$    & $474.$ & $78.$   & $78.$   \\ 
 $40$  &   $7.81\times10^6$   & $523.$    & $523.$ & $66.$   & $66.$   \\ 
 $50$  &   $4.00\times10^6$   & $493.$    & $493.$ & $64.$   & $64.$   \\ 
 $60$  &   $2.31\times10^6$   & $479.$    & $479.$ & $58.$   & $58.$   \\ 
 $70$  &   $1.45\times10^6$   & $475.$    & $475.$ & $43.$   & $170.$   \\ 
 $80$  &   $9.76\times10^5$   & $469.$    & $469.$ & $45.$   & $177.$   \\ 
 $90$  &   $6.85\times10^5$   & $491.$    & $491.$ & $47.$   & $186.$   \\ 
 $100$ &   $5.00\times10^5$   & $466.$    & $466.$ & $39.$   & $156.$   \\ 
 $200$ &   $6.25\times10^4$   & $504.$    & $504.$ & $34.$   & $138.$   \\ 
 $400$ &   $7.81\times10^3$   & $657.$    & $657.$ & $35.$   & $140.$   \\ 
   \hline
 \end{tabular}}
\end{table}   

\subsection*{LAPACK.}


\href{http://www.netlib.org/lapack/}{LAPACK}, $1992.$ urtean garatu zen \cite{Anderson1999,Higham2002} eta aljebra linealaren problemak ebazteko funtzioen liburutegia da. Jatorrizko bertsioa \emph{Fortran 77} lengoaian inplementatuta dago eta liburutegiaren dokumentazioa nahiz kodea \emph{Netlib} software bilgunean eskuratu daiteke. Matrize dentsoetarako garatuta dago eta problema hauetarako errutinak biltzen ditu: 
\begin{enumerate}
\item Ekuazio-sistema linealen ebazpena: $AX=b$.
\item Linear least square problems: $\|Ax-b\|$ minimizatzen duen $x$ balioa bilatu.
\item Eigenvalues problems.
\item Balio singularren deskonposaketa (SVD).
\end{enumerate}

LAPACK liburutegia, konputagailu sekuentzial eta memoria konpartitutako konputagailuetan erabilgarria izateko diseinatuta dago. Eraginkortasuna \emph{BLAS} funtzio optimizatuen menpe dago eta funtzioen inplementazioa, gehien bat BLAS-3 taldeko funtzioetan oinarritzen da. 

\emph{LAPACKE} liburutegia C-lengoaiatik LAPACK funtzioei deitzeko interfazea da. LAPACK liburutegiaren funtzioak, hiru taldeetan banatzen dira \cite{Anderson1999,Intel2015}:
\begin{enumerate}
\item Problema osoa ebazten dituzten errutinak (drivers). Talde honetan funtzio arruntak eta funtzio espezializatuak daude.

\textbf{Adibidea}.

LAPACKE-dgelsv (LAPACK-ROW-MAJOR, n, nrhs, A, lda, ipiv, B, ldb);

Matrize orokorren (GE), $A * X = B$ sistema lineala ebazten du,


\item Konputazio errutinak. Lan zehatz bat exekutatzen duen errutinak.

\textbf{Adibidea}.

LAPACKE-dgetrf (LAPACK-ROW-MAJOR, n, m, A, lda, ipiv);

Errutina honek $A$ ($n \times m$) tamainako matrizearen $LU$ faktorizazioa kalkulatzen du, $A=P*L*U$.

LAPACKE-dgetrs (LAPACK-ROW-MAJOR,trans,n,nrhs,A,lda,ipiv,B,ldb);

Errutina honek $A*X=B$ ekuazio sistemaren $X$ soluzioa kalkulatzen du.
   

\item Errutina laguntzaileak.
\end{enumerate}


\subsection*{Matrize bakanak.}


$A \in \mathbb{R}^{m \times n}$ matrizeari bakana esaten zaio, baldin abantaila atera daitekeen zero osagai kopuru adina baditu. Honek esan nahi du, matrizearen zero ez diren osagaien kopurua $n_{nz}$,
\begin{equation*}
n_{nz} \ll mn.
\end{equation*}

\begin{figure}[h]
\centerline{\includegraphics[width=6cm, height=6cm] {SparseMatrizeak}}
\caption{Matrize bakanak.}
\label{fig:61}
\end{figure}    

Matrizearen bakantasuna, konputazioaren memoria eta exekuzio denbora gutxitzeko erabil daiteke.
\begin{enumerate}
\item Doitasun bikoitzeko $A \in \mathbb{R}^{m \times n}$ matrizea,
\begin{enumerate}
\item Dentsoa: $8mn$ byte.
\item Bakana: $\approx16n_{nz}$ (gordetzeko teknikaren arabera).
\end{enumerate} 
\item $y=y+Ax, \ y,x \in \mathbb{R}^n \text{eta} \ A \in \mathbb{R}^{m \times n}$,
\begin{enumerate}
\item Dentsoa: $\mathcal{O}(mn)$, eragiketa aritmetikoak.
\item Bakana: $\mathcal{O}(n_{nz})$, eragiketa aritmetikoak.
\end{enumerate}
\end{enumerate}

%Oraindik, matrize bakanentzako  \emph{BLAS} liburutegia estandarra ez dagoela \url{http://www.netlib.org/blas/blast-forum} konprobatu dugu. Aldiz, Intelen \emph{Math Kernel Library} izeneko liburutegian matrize bakanentzako \emph{BLAS} eta \emph{LAPACK} funtzioak garatuta dituzte \url{https://software.intel.com/en-us/intel-mkl-support/documentation}.

\section{Konpiladorea.}


Konpiladorearen zeregina konplexua da, goi mailan idatzitako programari dagokion makina kodea sortzea (konputagailuaren errekurtsoak modu eraginkorrean erabiltzen dituenak) \cite{EijkhoutHPC}. Konpiladoreak heuristikotan oinarritutako kode aldaketak eragiten ditu, eraginkortasuna hobetzeko asmoarekin. Horregatik, programatzaileak konpiladorearen optimizazio automatiko hauek kontutan hartu behar ditu eta ahal duen neurrian, bere kodean konpiladorearen optimizazioak erraztu.      

\paragraph{Konpiladoreak.} Konpiladore ezberdinak daude:
\begin{enumerate}
\item \emph{gcc} (\emph{GNU} open source compiler).
\begin{lstlisting}
$ gcc -v
$ gcc version 5.4.0 20160609 (Ubuntu 5.4.0-6ubuntu1~16.04.2) 
\end{lstlisting}

\item Konpiladore komertzialak: Intel (\emph{icc}),\dots

\end{enumerate}

\paragraph{Optimizazioak.} Konpiladoreek, optimizazio maila estandarrak eskaintzen dituzte. Orokorrean, hurrengo kode optimizazioak izango ditugu:
\begin{enumerate}
\item \emph{-O0}.
Kode optimizazio nagusienak aplikatzeko aukera da. Kodea \emph{debugger} moduan aztertzen ari garenean gomendatzen da.
\item \emph{-O2}.
Kode eraginkorra sortzeko aukera gomendagarriena. 
\end{enumerate}  

Maila altuagoko optimizazioak aplika daitezke, baina optimizazio hauek arriskutsuak izan daitezke. Kodearen exekuzio denboraren analisia egiteko tresnak (\emph{gprof}) daude. Algoritmoaren funtzio bakoitzaren exekuzio denborari buruzko informazio erabilgarria lortuko dugu.   

\paragraph{Konpilazio aginduak}.
\begin{enumerate}
\item Konpilazioa, esteka egin eta adibidea.exe exekutagarria sortzeko
\begin{lstlisting}[language=C]
$ gcc adibidea.c -o adibidea.exe
\end{lstlisting}

\item Konpilazio eta esteka urratsak banatuta.
\begin{lstlisting}[language=C]
$ gcc adibidea.c  # creates adibidea.o
$ gcc adibidea.o -o adibidea.exe
\end{lstlisting}

\end{enumerate}

Gure esperimentuetarako era honetan burutu dugu konpilazioa,
\begin{lstlisting}[language=C]
gcc -O2 -Wall -std=c99 -fno-common adibidea.c
\end{lstlisting}

\paragraph*{Makefile.}

Normalean, aplikazioaren kodea fitxategi ezberdinetan egituratzen da eta konpilazio prozesua konplexua izan daiteke. \emph{Makefile} fitxategia, lengoaia berezi bat erabiliz, konpilazio prozesua automatizatzeko programa moduko bat dugu \cite{EijkhoutHPC}. \ref{sec:C1} eranskinean, \emph{Makefile} lengoaiaren oinarrizko adibideak eman ditugu.

\section{Laburpena.}

Algoritmo bat inplementatzen dugunean kontutan hartu beharrekoa:

\begin{enumerate}

\item Lerro edo zutabe araberako iterazioak exekuzio denboran eragin handia du.

\item Kodea garbia eta ulergarria mantendu behar da.

\item LAPACK eta BLAS liburutegiak oso eraginkorrak dira, eta inplementazio berrietan erabiltzea komenigarria da.


\end{enumerate}

Atal honi dagokion gomendatutako bibliografia:  
"An introduction to parallel programming", P. Pacheco \cite{Pacheco2011},"Introduction to High Performance Scientific Computing" ,V. Eijkhout \cite{EijkhoutHPC}, "Performance optimization of numerically intensive codes" , Goedecker \cite{Goedecker2001}, "Handbook of linear algebra" ,Leslie Hogben \cite{Hogben2013}.

LAPACK eta BLAS liburutegiak erabiltzeko informazio interesgarria "Intel Math Kernel Library. Refrence Manual" \cite{Intel2015} dokumentuan aurki daiteke.

\chapter{Kodea}.

Kodeari buruzko hainbat ohar emateko.

\section{Gauss metodoa}
\label{sec:B1}

\subsection*{Metodoaren koefizienteak.}
GaussCollocationCoefficients.nb funtzioa.

\begin{figure}[h!]
\centerline{\includegraphics[width=14cm, height=18cm] {Code_Gauss}}
\caption{Code Gauss.}
\label{fig:bost}
\end{figure}

\subsection*{Atalen hasieraketa koefizienteak.}

\section{Zientzia konputazioa.}

\subsection*{FMA.}

Nola jakin gure \emph{linux} konputagailu batek \emph{FMA} instrukzioak dituen ala ez ? Honako agindua exekutatu eta \emph{fma} flaga azaltzen den ala ez begiratu behar dugu.
\begin{lstlisting}
$ grep fma < /proc/cpuinfo
\end{lstlisting}
eta konpilatzeko \emph{-mfma} flag-a zehaztu behar da,
\begin{lstlisting}
$ gcc -O2 -Wall -std=c99 -fno-common -mfma adibidea.c
\end{lstlisting}

\subsection*{OpenMP}

\paragraph*{\textbf{Adibidea2.}} \emph{Reduction} direktiba $+,-,*,min,max$ funtzioekin erabili daiteke. Direktiba honen adibidea emateko, trapezio erregelaren bidezko zenbakizko integrazioaren inplementazioa erakutsiko dugu.
\begin{align*}
approx=h*(f(x_0)/2+f(x_1)+f(x_2)+\dots+f(x_{n-1})+f(x_{n}))
\end{align*} 

\begin{lstlisting}[language=C]
#    include <omp.h>

     int thread_count=2;
     
     h= (b-a)/n;
     approx = (f(a)+f(b))/2.0;
     
#    pragma omp parallel for num_threads(thread_count) \
     reduction(+: approx)
      for (i = 0; i<n; i++) approx+= f(a+i*h);
      
     approx=h*approx;
\end{lstlisting}

\subsection*{Makefile adibideak.}

 
\paragraph*{Adibidea1.}
Adibide sinple baten bidez azalduko dugu bere erabilpena. Hauxe da, \emph{Makefile}-aren oinarrizko elementua,
\begin{lstlisting}
helburua: dependentziak
>TAB>  helburua lortzeko aginduak
\end{lstlisting}

Demagun aplikazioa bat hiru fitxategietan banatuta dugula,
\begin{lstlisting}[language=C]
/*file: main.c*/
void main()
{
    printf("Main program");
    sub1();
    sub2();
}
\end{lstlisting}

\begin{lstlisting}[language=C]
/*file: sub1.c*/
void sub1()
{
    printf("sub1");
}
\end{lstlisting}

\begin{lstlisting}[language=C]
/*file: sub2.c*/
void sub2()
{
    printf("sub2");
}
\end{lstlisting}

Konpilazioa automatizatzeko \emph{make} fitxategia,

\begin{lstlisting} [language=C]
main.exe: main.o sub1.o sub2.o
	      gcc main.o sub1.o sub2.o -o main.exe
main.o: main.c
        gcc -c main.c
sub1.o: sub1.c
        gcc -c sub1.c        
sub2.o: sub2.c
        gcc -c sub2.c        
\end{lstlisting}

Eta erabilpena,

\begin{lstlisting}
$ make main.exe
  gcc -c main.c
  gcc -c sub1.c
  gcc -c sub2.c
  gcc main.o sub1.o sub2.o -o main.exe
\end{lstlisting} 

\paragraph*{Adibidea2.} 
Aurreko adibidea, modu egokiagoan idatzitako \emph{make} fitxategia,

\begin{lstlisting} [language=C]

CC = /usr/bin/gcc
FLAGS=-O2 -Wall -std=c99 -fno-common 
OBJECTS=  main.o sub1.o sub2.o
.PHONY: clean help

main.exe: $(OBJECTS)
	      ${CC} $(OBJECTS) -o main.exe
	      
%.o: %.c
     ${CC} ${FLAGS} -c $<	      
	      
clean:
     rm -f $(OBJECTS) main.exe

help:
    @echo "Valid targets;"
    @echo " main.exe"
    @echo " main.o"
    @echo " sub1.o"
    @echo " sub2.o"
    @echo " clean"
             
\end{lstlisting}



\section{Problemak}

\subsection*{Pendulu bikoitza}

Mathematican,  DoublePendulum.m eta DoublePendulumSTIFF.m paketetan, hurrenez-hurren pendulu bikoitz arruntaren eta pendulu bikoitz zurrunaren honako funtzioen inplementazioak garatu ditugu:
\begin{enumerate}
   \item Hamiltondarra: DoublePendulumHam eta DoublePendulumSTIFFHam.
   \item EDA: DoublePendulumODE eta DoublePendulumSTIFFODE.
   \item Jakobiarra: DoublePendulumJAC eta DoublePendulumSTIFFJAC.
\end{enumerate}

\paragraph*{}C-lengoaiako inplementazio baliokidea, GaussUserProblem.c fitxategian inplementatu ditugu. Pendulu bikoitz arruntaren problema $problem=5$ gisa  (ode5, jac5, ham5) eta pendulu bikoitz zurrunaren problema $problem=6$ gisa (ode6, jac6, ham6) izendatu dugu.

\subsection*{N-body problema}

\paragraph*{} Mathematicako NBodyProblem.m paketean honako funtzioak garatu ditugu.

\begin{enumerate}
   \item Hamiltondarra: NBodyHAM.
   \item EDA: NBodyODE.
   \item Jakobiarra: Ez dut garatu.
\end{enumerate}

-\paragraph*{} C-lengoaiako inplementazioa:

\begin{enumerate}
   \item Hamiltondarra: HamNBody().
   \item EDA: OdeNbody().
   \item Jakobiarra: JacNBody().
\end{enumerate}

\section{Fortran kodeak}.

\paragraph*{} Haireren konposizio metodoaren Fortran kodearen  C-lengoaiako itzulpena egin dugu. Konposizio metodoaren azalpenak liburuko \cite{Hairer2006}  \emph{II.4} eta \emph{V.3} ataletan ematen dira. \emph{GNI-Comp} izeneko kodea eskuragarri dago \cite{HairerGniComp} helbidean.  

\section{IRK-Puntu finkoa}

\section{IRK-Newton}


\subsection{IRK-Newton koefizienteak.}

GaussCoefficients(s).nb mathematicako fntzioaren bidez aurrekalkulatzen ditugu koefizienteak.

\subsection{IRK-Newton eraginkorra.}

\paragraph*{} Algoritmoaren hainbat zehaztapen emango ditugu,

\begin{enumerate}

\item LAPACK.

\emph{LU deskonposaketa} egiteko funtzioa,
\begin{lstlisting} [language=C]
GETRF(LAPACK_ROW_MAJOR, n, m, MM, lda, ipiv);
\end{lstlisting}

Ekuazio sistemaren ebazpena (\emph{Solve}) egiteko,
\begin{lstlisting} [language=C]
GETRS(LAPACK_ROW_MAJOR,trans,n,nrhs,MM,lda,ipiv,fl,ldb);
\end{lstlisting}

\end{enumerate}

\section{Eguzki-sistema}

   
\chapter{Notazioa.}

\section{Notazioa.}

\begin{table}
\caption{Notazioa.}
\label{tab:21}       % Give a unique label
\begin{tabular}{ l l c c c } 
 \hline
 Notazioa                                    &  Esanahia                                    & Adibidea \\
 \hline
 $\mathbb{R}$                                &  Zenbaki errealen multzoa                    &  \\
 $\mathbb{F}$                                &  Koma-higorreko zenbakien multzoa            &  \\
 $\mathbb{R}^{n},\mathbb{R}^{m \times n}$    &  $n$ dimentsioko bektore eta $m \times n$ dimentsioko matrizeak   & \\
 $\approx$                                   &  Hurbilpena                                  & $y_n \approx y(t_n)$ \\
 $fl()$                                      &  Koma-higikorreko balioa esleitzen duen funtzioa & $\tilde{y}=fl(y)$ \\
 $\otimes$                                   &  Biderketa tensoriala                        &                       \\
 $\lceil \cdots \rceil$                      &  Round up to nearest integer                 & $\lceil 2.4 \rceil=3$ \\
 $\lfloor \cdots \rfloor$                    &  Round down to the nearest integer          & $\lfloor 2.6 \rfloor=2$ \\
 $Const$                                     &  Balio konstantea                           & $H(p(t),q(t))=Cosnt$ \\
 \hline
 \end{tabular}
\end{table}

\section{Hitz-zerrenda.}

\begin{table}
\caption{Hitz-zerrenda.}
\label{tab:21}       % Give a unique label
\begin{tabular}{ l l c c c } 
 \hline
 Euskaraz                                &  Ingelesez                           & Laburdura    & Adibidea \\
 \hline
 Ekuazio diferentzial arrunta            &  Ordinary Differential equation      & \emph{ODE}   & $\dot{y}=f(t,y)$ \\
 Hasierako baliodun problema             &  Initial value problem               & \emph{IVP}   &                  \\
 Zurruna								 &  Stiff                               &              &                  \\
 Sinplektikoa                            &  Symplectic                          &              &                  \\
 Desplazamendu                           &  Drift                               &              &                  \\
                                         &  Splitting methods                   &              &                  \\
                                         &  Conposition methods                 &              &                  \\
 Runge-Kutta Esplizitua                  &  Explicit Runge-Kutta                &  \emph{ERK}  &                  \\
 Runge-Kutta Inplizitua                  &  Inplicit Runge-Kutta                &  \emph{IRK}  &                  \\
                                         &  A-stability, B-stability            &              &                  \\
                                         &  Implicit Midpoint method            &              &                  \\
                                         &  Adjoint                             &              &                  \\
                                         &  bias                                &              &                  \\
                                         &  compensated summation               &              &                  \\
 Biribiltzea                             &  roundoff                            &              &                  \\
 Portabilitate                           &  portable                            &              &                  \\
 Doitasun arrunta                        &  Single precision                    &              &                  \\
 Doitasun bikoitza                       &  Double precision                    &              &                  \\
 Doitasun laukoitza                      &  Quadruple precision                 &              &                  \\
 Multiple-digit representation           &  Digito-anitzeko adierazpena         &              &                  \\
 Multiple-term representation            &  Termino-anitzeko adierazpena        &              &                  \\  
 Haria                                   &  Thread                              &              &                  \\
                                         &  Graphical Processor Unit            &  \emph{GPU}  &                  \\
                                         &  Least Square                        &              &                  \\
                                         &  Least Eigenvalues problems          &              &                  \\
 Balio singularren deskonposaketa        &  Singular values descomposition      & \emph{SVD}   &                  \\
 Ezentrizidadea                          &  Eccentricity                        &              &                  \\
                                         &  Eccentric anomaly                   &              &                  \\
 Astronomical unit (AU)                  &  Unitate astronomikoa                &              &                  \\
 Julian date                             &  Data juliotar                       &              &                  \\
 LU Decomposition  (low, up)             &  LU-deskonposaketa                   &              &                  \\ 
 Flops                                   &  Koma-higikorreko eragiketa segunduko &             &                  \\   
 Peak                                    &  Exekuzio gaitasuna                   &             &                  \\
 Wall-time, elapsed-time                 &                                       &             &                  \\
 CPU-time                                &                                       &             &                  \\
 Cache memoria                           &                                       &             &                  \\ 
 spatial/data locality                   &                                       &             &                  \\
 Single Instruction Multiple Data        &                                       & SIMD        &                  \\
 Fork-join                               &                                       &             &                  \\
 Application programming interface       & Interfaze aplikazio programa          & API         &                  \\
 Eraginkortasun altuko konputazioa       & High performance computing            & HPC         &                  \\
 Problem solving enviroments             & Problemak ebazteko inguruneak         & PSE         &                  \\
 Portable                                &                                       &             &                  \\
 Linear least square problems            &                                       &             &                  \\
 Eigenvalues problems                    &                                       &             &                  \\
 Sparse matrices                         & Matrize bakanak                       &             &                  \\
 
                                                             
 
                                
 \hline
 \end{tabular}
\end{table}
  
\chapter{Nire-Argibideak}.

Niretzako dokumentatutako argibide batzuk. Ez ditut tesian sartu behar.

\section{IRK-Newton.}

\subsection*{Newton sinplifikatuaren iterazioa.}
Newton sinplifikatuaren iterazioan formulazio berrian ekuazio honen jatorria:

\begin{align*}
\mbox{Askatu} \ \ & \triangle L_i^{[k]} \\
\ \ &\triangle L_i^{[k]} - h b_i \ J_i \sum_{j=1}^{s} \mu_{ij}  \ \triangle L_j^{[k]} = g_i^{[k]}  , \ \ i=1,\dots,s.
\end{align*}

\paragraph*{Frogapena.}

Abiapuntua.
\begin{align*}
& \triangle L_i=L_i-L_i^{[k]},\\
& L_i-h b_i f(y_n+\sum_{j=1}^{s} \mu_{ij}L_j)=0
\end{align*}

Honako garapena egingo dugu.
\begin{align*}
L_i-h b_i f(y_n+\sum_{j=1}^{s} \mu_{ij}L_j)= L_i^{[k]}+\triangle L_i-h b_i f(y_n+\sum_{j=1}^{s} \mu_{ij} (L_j^{[k]}+\triangle L_j)).
\end{align*}
Eta $f(y_n+\sum_{j=1}^{s} \mu_{ij} (L_j^{[k]}+\triangle L_j))$ linealizatuz,
\begin{align*}
& \approxeq L_i^{[k]}+\triangle L_i-h b_i f(y_n+\sum_{j=1}^{s} \mu_{ij}L_j^{[k]})-h b_i f'(y_n+\sum_{j=1}^{s} \mu_{ij} L_j^{[k]}) (\sum_{j=1}^{s} \mu_{ij} \triangle L_j))= \\
& \triangle L_i- hb_i J_i \ (\sum_{j=1}^{s} \mu_{ij} \triangle L_j))= -L_i^{[k]}+h b_i f(y_n+\sum_{j=1}^{s} \mu_{ij}L_j)
\end{align*}
non $J_i=f'(y_n+\sum_{j=1}^{s} \mu_{ij} L_j^{[k]})=f'(Y_i)$.

%\include{Appendix}


%\begin{acknowledgements}
%If you'd like to thank anyone, place your comments here
%and remove the percent signs.
%\end{acknowledgements}

%\listoffigures

%\newpage
%\section*{Taulen zerrenda.}

%\listoftables

\backmatter
\nocite{*}
\addcontentsline{toc}{chapter}{Bibliografia}
\bibliography{./bibliografia/mybib}
\bibliographystyle{myplain2-doi}

\printindex

\end{document}


