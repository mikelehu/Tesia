\chapter{Ondorioak.}


\section*{Ideiak}


Lehenengo puntu-finkoaren iterazioan oinarritutako eta Newton sinplifikatuaren iterazioan oinarritutako IRK inplementazioak jorratu dugu. Bi inplementazio hauetan, biribiltze errorearen ikuspegitik optimoa izatea lehenetsi dugu. Newton sinplifikatuaren IRK inplementazio estandarra konplexua eta konputazionalki garestia denez, modu eraginkorrean kalkulatzeko proposamen egin dugu.

Eguzki-sistema integratzeko, IRK metodo partizionatuak Newton sinplifikatuaren IRK metodoak baino egokiagoak direla baieztatu dugu. Dena den, puntu-finkoaren inplementazioak energia drift-a azaltzen du eta hau garrantzitsua balitz, Newton-en inplementazioa aplikatu beharko litzateke.        

\section*{Laburpena}

Discovery Through the Power of Mathematics, Physics, and the Imagination
'New Worlds, New Horizons in Astronomy and Astrophysics' liburuko, ondorio hau lapurtu dugu,
ISBN 978-0-309-15802-2 | DOI 10.17226/12951

\begin{displayquote}
\textbf{Discovery Through the Power of Mathematics, Physics, and the Imagination.}

Finally, it is important to remember that many of the most far-reaching and revolutionary discoveries in astronomy were not solely the direct result of observations with telescopes or numerical simulations with computers. Rather, they also sprang from the imagination of inspired theorists thinking in deep and original ways about how to understand the data, and making testable predictions about new ideas.

In the coming decade, major challenges loom that require the development of fundamental new theories. Observations and computer simulations are necessary components, but to complete the path from discovery to understanding, theorists will need to freely exercise their imaginations.
\end{displayquote}



