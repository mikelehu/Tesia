\chapter{Ondorioak.}


\section*{Ideiak}

\begin{enumerate}
\item 1-ideia.

Lehenengo puntu-finkoaren iterazioan oinarritutako eta Newton sinplifikatuaren iterazioan oinarritutako IRK inplementazioak jorratu dugu. Bi inplementazio hauetan, biribiltze errorearen ikuspegitik optimoa izatea lehenetsi dugu. Newton sinplifikatuaren IRK inplementazio estandarra konplexua eta konputazionalki garestia denez, modu eraginkorrean kalkulatzeko proposamen egin dugu.

Eguzki-sistema integratzeko, IRK metodo partizionatuak Newton sinplifikatuaren IRK metodoak baino egokiagoak direla baieztatu dugu. Dena den, puntu-finkoaren inplementazioak energia drift-a azaltzen du eta hau garrantzitsua balitz, Newton-en inplementazioa aplikatu beharko litzateke.        

\item 2-ideia.

Splitting metodo oinarrizkoan (verlet), zera egiten dugu: $h/2$ fluxua aplikatu , perturbazioak kalkulatu eta gero berriz  $h/2$ fluxua aplikatu. Fluxuaren aldagai aldaketarekin, gauza bera egiten ari gara: $h/2$ fluxua aurreratu, perturbazioak kalkulatu (aldagai berrietan eta hobeto kalkulatzen dugu), $h/2$ fluxua aurreratu.

Paralelizazioaz, IRK metodoak errazten du: funtzio ebaluaztapenean bektorizazioa aplikatzea eta ataletan paralelizazioa aplikatzea.

\item 3-ideia.

Gauss metodoak hiru abantaila azaltzen dituzte:

\begin{enumerate}
\item Paraleligarritasuna.

\item Iterazioak.

Iterazio gehienak eredu sinple batekin kalkula daitezke \cite{Beylkin2014} eta eta bukaeran iterazio pare bat eredu osoarekin. 

\item Sinplektikoa.

Metodo sinplektikoa denez, edozein problema Hmiltondarrari aplika daiteke. 

\item Birparametrizazioa.

Integratzaile sinplektikoak luzera finkoko urratsa eduki behar du eta zentzu honetan, birparametrizazioa eraginkortasuna hobetzeko beste bide bat da. Arazorik gabe aplikatu daiteke.


\end{enumerate}



\end{enumerate}




\section*{Laburpena}

Discovery Through the Power of Mathematics, Physics, and the Imagination
'New Worlds, New Horizons in Astronomy and Astrophysics' liburuko, ondorio hau lapurtu dugu,
ISBN 978-0-309-15802-2 | DOI 10.17226/12951

\begin{displayquote}
\textbf{Discovery Through the Power of Mathematics, Physics, and the Imagination.}

Finally, it is important to remember that many of the most far-reaching and revolutionary discoveries in astronomy were not solely the direct result of observations with telescopes or numerical simulations with computers. Rather, they also sprang from the imagination of inspired theorists thinking in deep and original ways about how to understand the data, and making testable predictions about new ideas.

In the coming decade, major challenges loom that require the development of fundamental new theories. Observations and computer simulations are necessary components, but to complete the path from discovery to understanding, theorists will need to freely exercise their imaginations.
\end{displayquote}


\section{Etorkizuneko lanak.}

\subsection*{Kepler fluxuaren aldagai aldaketa.}

Orain Fluxuaren aldagai aldaketa aplikatu dugunean, IRK puntu-finkoan oinarritu gara. Esperimentuetan, oso azkar konbergitzen duela ikusi dugu . Bestalde, guzki-sistemaren eredutan kepleriarrak ez diren hainbat perturbazio izan daitezkeeela; aldagai hauen (W), konbergentzia
azkartzeko, Newton sinplifikatua erabiltzea komeniko zaigula uste dugu.

\subsection*{Zenbakizko integrazio metodo inplizituak.}

Gauss, Chebichev, \dots 

\subsection*{Denbora birparametrizazioa.}

Exzentrizitate handiko orbitadun sistementzako hainbat “erregularizazio simpletiko integratzaileak“ aztertu dira:
Levison and Duncan (1994). “RMVS”: Regularized mixed variable sympletic integrator”, Mikkola (1997), Fukushima (2001), Beus(2003).

Erregularizazioa aplikatzen dituzte metodo simpletiko ezbedinen “review” honakoa dugu: “Rauch
and Holman (1999). Dynamical chaos in the Wisdom-Holman integrator”.


